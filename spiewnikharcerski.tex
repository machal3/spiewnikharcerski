% Ten plik został wygenerowany automatycznie.

% --- Źródło: 000poczontek.tex ---
% --- Źródło: 000poczontek.tex ---
\documentclass[14pt, a4paper]{extarticle}

\usepackage{gchords}
\usepackage{changepage}
\usepackage[T1]{fontenc}
\usepackage{lmodern}

% --- KODOWANIE I JĘZYK ---
\usepackage{polski}         % Polskie reguły typograficzne
\usepackage{setspace}
% --- WYMIARY STRONY I MARGINESY ---
\usepackage[left=2cm, top=2cm, right=2cm, bottom=1cm]{geometry}

% --- TABELE ---
\usepackage{longtable} % Obsługa długich tabel
\usepackage{array}     % Rozszerzone opcje tabel
\usepackage{multirow}  % Łączenie wierszy (opcjonalnie)

% --- GRAFIKA I IKONY ---
\usepackage{graphicx}
\usepackage{fontawesome5} % Nowsza wersja ikon (np. GitHub)
\usepackage{tikz}         % Do rysowania (jeśli używasz)
\usepackage{tkz-euclide}  % Geometria euklidesowa (jeśli używasz)
\setstretch{0.9}
% --- NAGŁÓWKI I SEKCJE ---
\usepackage{titlesec}

% Formatowanie tytułów sekcji (piosenek):
% \bfseries - pogrubienie, \Large - duży tekst, brak numeracji
\titleformat{\section}{\large\bfseries}{}{0pt}{}
% Usunięcie numeracji sekcji z całego dokumentu (żeby nie było "1. Tytuł"):
\setcounter{secnumdepth}{0} 
\setlength{\LTpre}{0pt}
\setlength\LTleft{0pt}
\setlength\LTright\fill

% --- SPIS TREŚCI ---
\usepackage{tocloft}
% Dodanie kropek w spisie treści między tytułem a numerem strony:
\renewcommand{\cftsecleader}{\cftdotfill{\cftdotsep}}
\usepackage{tocloft}

% Zmniejszenie odstępów przed konkretnymi elementami:
\setlength{\cftbeforesecskip}{1pt}    % Odstęp przed sekcją
\setlength{\cftbeforesubsecskip}{0pt} % Odstęp przed podsekcją
\renewcommand{\cftsecfont}{\normalfont}

% --- NAGŁÓWEK I STOPKA ---
\usepackage{fancyhdr}
\pagestyle{fancy}
\fancyhf{} % Wyczyść domyślne ustawienia
\fancyfoot[C]{} % Numer strony na środku stopki
\renewcommand{\headrulewidth}{0pt} % Usuń linię w nagłówku
\renewcommand{\footrulewidth}{0pt} % Usuń linię w stopce

% --- KOLORY ---
\usepackage{xcolor}
\definecolor{Zloty}{HTML}{FFD700} % Twój zdefiniowany kolor
\definecolor{CiemnyNiebieski}{RGB}{0,0,139} % Dodatkowy kolor do linków

% --- LINKI (musi być na końcu preambuły) ---
\usepackage{hyperref}
\hypersetup{
	colorlinks=true,       % Kolorowe linki zamiast ramek
	linkcolor=black,       % Spis treści czarny
	urlcolor=CiemnyNiebieski, % Linki do stron (np. YouTube) niebieskie
	pdftitle={Śpiewnik},   % Tytuł w metadanych PDF
	pdfauthor={Autor}      % Autor w metadanych
}

% --- START DOKUMENTU ---
\begin{document}
	
	\vspace*{3cm}
	\begin{center}
		{\fontsize{80}{20}
			\textbf{\color{Zloty}{ZŁOTY}}}\\
		\vspace{0.25cm}
		{\fontsize{49}{20} \textbf{ŚPIEWNIK}}\\
		\vspace{5cm}
		\textbf{\Large	Wersja harcerska} \\
		\vspace{0.5cm}
		\faTree
		
	\end{center}
	
	\newpage
	\faSmile[regular]
	\vfill
	\noindent \textbf{Wydanie nr. 3.5 (przed zimowiskiem 2026)}
	\\\\
	\textbf{Sklejenie wszystkiego w całość:}\\
	Michał Nowacki
	\\\\
	\textbf{Wykonanie:} Śpiewnik zrobiony na pomocą \LaTeX
	\\\\
	\faGithub
	\\\\
	\textbf{Część chwytów i piosenek została zapożyczona od:}\\
	Michał F. Rozbiewski, p. Jakub Pawlikowski, Szymon Warda, Wojciech Ambroziak,
	Bartosz Jażdżyk, jakieś śpiewniki z internetu, w szczególności szesnastki.
	\newpage
	% Przykładowy spis treści:
	\begin{adjustwidth}{1cm}{1cm} % Zwiększa lewy i prawy margines o 2 cm
		\tableofcontents
	\end{adjustwidth}
	\newpage
	
	% --- TUTAJ ZACZYNAJĄ SIĘ PIOSENKI --
	\clearpage

\clearpage

% --- Źródło: A_my_nie_chcemy_uciekać_stąd.tex ---
\section{\textbf{A my nie chcemy uciekać stąd}}
\vspace{-\baselineskip}
\textit{Przemysław Gintrowski, Jacek Kaczmarski}\\
\begin{longtable}{ll}
Stanął w ogniu nasz wielki dom & \textbf{d} \\
Dym w korytarzach kręci sznury & \textbf{d} \\
Jest głęboka, naprawdę czarna noc & \textbf{d} \\
Z piwnic płonące uciekają szczury & \textbf{d} \\
& \\
Krzyczę przez okno czoło w szybę wgniatam & \textbf{a B} \\
Haustem powietrza robię w żarze wyłom & \textbf{C d} \\
Ten co mnie słyszy ma mnie za wariata & \textbf{a B} \\
Woła - co jeszcze świrze ci się śniło & \textbf{C d} \\
& \\
Więc chwytam kraty rozgrzane do białości & \textbf{d B} \\
Twarz swoją widzę, twarz w przekleństwach & \textbf{C d} \\
A obok sąsiad patrzy z ciekawością & \textbf{d B} \\
Jak płonie na nim kaftan bezpieczeństwa & \textbf{C d} \\
& \\
Dym w dziurce od klucza a drzwi bez klamek & \textbf{a B} \\
Pękają tynki wzdłuż spoconej ściany & \textbf{C d} \\
Wsuwam swój język w rozpalony zamek & \textbf{a B} \\
Śmieje się za mną ktoś jak obłąkany & \textbf{C d} \\
& \\
Lecz większość śpi nadal, przez sen się uśmiecha & \textbf{d a C d} \\
A kto się zbudzi nie wierzy w przebudzenie & \textbf{B C d} \\
Krzyk w wytłumionych salach nie zna echa & \textbf{d a C d} \\
Na rusztach łóżek milczy przerażenie & \textbf{B C d} \\
& \\
Ci przywiązani dymem materaców & \textbf{d a C d} \\
Przepowiadają życia swego słowa & \textbf{B C d} \\
Nam pod nogami żarzą się posadzki & \textbf{d a C d} \\
Deszcz iskier czerwonych osiada na głowach & \textbf{B C d} \\
& \\
Dym coraz gęstszy, obcy ktoś się wdziera & \textbf{d a C d} \\
A My wciśnięci w najdalszy sali kąt & \textbf{B C d} \\
„Tędy!” - wrzeszczy – „Niech was jasna cholera!” & \textbf{d a C d} \\
A my nie chcemy uciekać stąd & \textbf{B C d} \\
& \\
A my nie chcemy uciekać stąd! & \textbf{a C d} \\
Krzyczymy w szale wściekłości i pokory & \textbf{B C d} \\
Stanął w ogniu nasz wielki dom! & \textbf{d a C d} \\
Dom dla psychicznie i nerwowo chorych! & \textbf{B C d} \\
\end{longtable}
\clearpage

% --- Źródło: Ajrisz.tex ---
\section{\textbf{Ajrisz}}
\vspace{-\baselineskip}
\textit{T.Love}\\
\begin{longtable}{ll}
Zabieram cię do baru & \textbf{D} \\
Będzie 8:0 dla mnie & \textbf{A} \\
Zakładamy się że Polska pokona Anglię & \textbf{G A D} \\
Drugi strong, trzeci strong & \textbf{D} \\
Rozmawiamy bez wytchnienia & \textbf{D G} \\
O uczuciu które jest & \textbf{D} \\
Najlepsze bez wątpienia & \textbf{G A D} \\
Ty mnie chyba nie znasz  & \\
I nie rozumiesz nic  & \\
Bo ty nie wiesz jak się tutaj pije  & \\
Kolejnej wiosny łyk  & \\
& \\
Czwarty strong, piąty strong  & \\
Coraz bliżej twego ciała  & \\
Oczy moje lewitują  & \\
Odległość jest już mała  & \\
Dotknij mojej dłoni  & \\
I na zewnątrz wyjdźmy stąd  & \\
Ten spacer przeznaczeniem naszym  & \\
Mocno czuję to  & \\
& \\
Prostych słów się boi & \textbf{D G} \\
Największy nawet twardziel & \textbf{D A} \\
Proste słowa z gardła & \textbf{D G} \\
Nie chcą wyjść najbardziej & \textbf{G A D} \\
Mówią że mnie kochasz & \textbf{h G} \\
I że mną nie wzgardzisz & \textbf{D A} \\
Prawdziwa moja miłość & \textbf{h G} \\
Nazywa się ajrisz & \textbf{G A D} \\
& \\
Już powinniśmy skończyć  & \\
Do domu już czas  & \\
Bo tak lubię z tobą pić  & \\
Kolejny raz  & \\
& \\
& \\
\end{longtable}
\newpage
\begin{longtable}{ll}
Szósty strong, siódmy strong  & \\
Rozmawiamy bez wytchnienia  & \\
O uczuciu które jest  & \\
Najlepsze bez wątpienia  & \\
Czujesz jak tu pachnie  & \\
Tak wygląda chyba raj  & \\
Najlepsze miesiące  & \\
To kwiecień, czerwiec, maj  & \\
& \\
Prostych słów się boi  & \\
Największy nawet twardziel  & \\
Proste słowa z gardła  & \\
Nie chcą wyjść najbardziej  & \\
Mówią że mnie kochasz  & \\
I że mną nie wzgardzisz  & \\
Prawdziwa moja miłość  & \\
Nazywa się ajrisz  & \\
& \\
\end{longtable}
\clearpage

% --- Źródło: Aletojuzbylo.tex ---
\section{Ale to już było}
\vspace{-\baselineskip}
\textit{Maryla Rodowicz}\\
\begin{longtable}{ll}
Z wielu pieców się jadło chleb & \textbf{C G C} \\
Bo od lat przyglądam się światu & \textbf{d G} \\
Nieraz rano zabolał łeb & \textbf{C G C} \\
I mówili zmiana klimatu & \textbf{d G} \\
Czasem trafił się wielki raut & \textbf{e d} \\
Albo feta proletariatu & \textbf{F G} \\
Czasem podróż w najlepszym z aut & \textbf{e d} \\
Częściej szare drogi powiatu & \textbf{F G} \\
& \\
\hspace*{2em}\textit{Ale to już było i nie wróci więcej} & \textbf{F G C} \\
\hspace*{2em}\textit{I choć tyle się zdarzyło} & \textbf{e F} \\
\hspace*{2em}\textit{To do przodu wciąż wyrywa głupie serce} & \textbf{C} \\
\hspace*{2em}\textit{Ale to już było znikło gdzieś za nami} & \textbf{F G C} \\
\hspace*{2em}\textit{Choć w papierach lat przybyło} & \textbf{e F} \\
\hspace*{2em}\textit{To naprawdę wciąż jesteśmy tacy sami} & \textbf{C} \\
& \\
Na regale kolekcja płyt &  \\
I wywiadów pełne gazety & \\
Za oknami kolejny świt & \\
I w sypialni dzieci oddechy & \\
One lecą drogą do gwiazd & \\
Przez niebieski ocean nieba & \\
Ale przecież za jakiś czas & \\
Będą mogły same zaśpiewać & \\
& \\
\hspace*{2em}\textit{Ale to już było i nie wróci więcej} & \\
\hspace*{2em}\textit{I choć tyle się zdarzyło} & \\
\hspace*{2em}\textit{To do przodu wciąż wyrywa głupie serce} & \\
\hspace*{2em}\textit{Ale to już było znikło gdzieś za nami} & \\
\hspace*{2em}\textit{Choć w papierach lat przybyło} & \\
\hspace*{2em}\textit{To naprawdę wciąż jesteśmy tacy sami} & \\
\end{longtable}
\clearpage

% --- Źródło: Alexander.tex ---
\section{\textbf{Alexander}}
\vspace{-\baselineskip}
\textit{Myslovitz}\\
\begin{longtable}{ll}
Już nie będę z Tobą kłócił się & \textbf{e C G} \\
I tak nigdy nie mam racji & \textbf{D e} \\
Wydawać by się mogło, że & \textbf{C G} \\
Jesteśmy źle dobrani & \textbf{D e} \\
Najgorsze jest jednak to & \textbf{C G} \\
Twoje rozczarowanie & \textbf{D e} \\
Wiem, zapomniałem Ci powiedzieć, że & \textbf{C G} \\
Jestem zakochany & \textbf{D e} \\
& \\
\hspace*{2em}\textit{Więc lepiej mnie zabij, wyrzuć z pamięci} & \textbf{e C} \\
\hspace*{2em}\textit{Lepiej odejdź, pozwól mi odejść} & \textbf{G D} \\
\hspace*{2em}\textit{Lepiej zapomnij, pozwól zapomnieć} & \textbf{e C} \\
\hspace*{2em}\textit{Lepiej daj mi następną szansę} & \textbf{G D} \\
& \\
Wiem, potrzebujesz tego, czego ja & \\
Nigdy mogę Ci nie dać & \\
Nie dlatego, że nie chcę Ci dać & \\
A dlatego, że sam tego nie mam & \\
Najgorsze jest jednak to & \\
Twoje rozczarowanie & \\
Wiem, zapomniałem Ci powiedzieć, że & \\
Jestem zakochany & \\
& \\
\hspace*{2em}\textit{Więc lepiej mnie zabij, wyrzuć z pamięci...} & \\
\end{longtable}
\clearpage

% --- Źródło: Alibi.tex ---
\section{Alibi}
\begin{longtable}{ll}
\textbf{a F C G} \\ & \\
\hspace*{2em}\textit{Nie będę tłumaczył się jak winny ktoś, co złego w tym że kocham cię}  & \\
\hspace*{2em}\textit{Mam alibi co dzień pytaj o co chcesz, co złego w tym jest}  & \\
\hspace*{2em}\textit{Nie będę tłumaczył się jak winny ktoś co złego w tym że kocham cię}  & \\
\hspace*{2em}\textit{Mam alibi pytaj o co chcesz co złego w tym jest}  & \\
& \\
Nie będę tłumaczył się że lato nie nadchodzi a zima wciąż jest  & \\
Nie będę tłumaczył się że budzik rano dzwoni i zawsze jest za wcześnie  & \\
Nie czuję się winny ni trochę że raz wypada reszka a raz wypada orzeł  & \\
Nie odpowiadam za to wszystko co dzieje się beze mnie co dzieje się za szybko  & \\
& \\
\hspace*{2em}\textit{Nie będę tłumaczył się jak winny ktoś co złego w tym że kocham cię…}  & \\
& \\
Co jest nie tak że jest nie tak że koniec szach mat robota wciąż nie idzie  & \\
Ze wszystkich czterech stron fiasko i nic się nie udaje od początku do końca  & \\
Nic nigdy tu nie będzie działać i zasady BHP nie pomogą ni trochę  & \\
I nigdy tu nie będzie dobrze gdy serce stoi w miejscu gdy serce wciąż nie działa  & \\
& \\
\hspace*{2em}\textit{Nie będę tłumaczył się jak winny ktoś co złego w tym że kocham cię…}  & \\
& \\
To pytanie zwykłe znam kto odpowiedź prostą ma  & \\
A niby mówisz o tym samym co ja  & \\
A niby czekasz tak samo jak ja  & \\
Oddychaj głęboko tak  & \\
Bo kiedy przyjdzie nie wiesz sam gdzie powieje wiatr  & \\
Nie ma chwili by zawahać się więc na pewno  & \\
& \\
\hspace*{2em}\textit{Nie będę tłumaczył się jak winny ktoś co złego w tym że kocham cię…}  & \\
\hspace*{2em}\textit{|| x2}  & \\
\end{longtable}
\clearpage

% --- Źródło: Alleluja.tex ---
\section{Alleluja}
\begin{longtable}{ll}
Tajemny akord kiedyś brzmiał & \textbf{G e} \\
Pan cieszył się, gdy Dawid grał & \textbf{G e} \\
Ale muzyki dziś tak nikt nie czuje & \textbf{C D G D} \\
Kwarta i kwinta, tak to szło & \textbf{G C D} \\
Raz wyżej w dur, raz niżej w moll & \textbf{e C} \\
Nieszczęsny król ułożył Alleluja & \textbf{D h e} \\
& \\
\hspace*{2em}\textit{Alleluja, Alleluja} & \textbf{C e} \\
\hspace*{2em}\textit{Alleluja, Alleluja.…} & \textbf{C G D G D} \\
& \\
Na wiarę nic nie chciałeś brać &  \\
Lecz sprawił to księżyca blask & \\
Że piękność jej na zawsze Cię podbiła & \\
Kuchenne krzesło tronem twym & \\
Ostrzygła Cię, już nie masz sił & \\
I z gardła ci wydarła Alleluja & \\
& \\
\hspace*{2em}\textit{Alleluja, Alleluja} & \\
\hspace*{2em}\textit{Alleluja, Alleluja.....} & \\
& \\
Dlaczego mi zarzucasz wciąż & \\
Że nadaremno wzywam Go & \\
Ja przecież nawet nie znam Go z imienia & \\
Jest w każdym słowie światła błysk & \\
Nieważne, czy usłyszy dziś & \\
Najświętsze, czy nieczyste Alleluja & \\
& \\
\hspace*{2em}\textit{Alleluja, Alleluja} & \\
\hspace*{2em}\textit{Alleluja, Alleluja..} & \\
\hspace*{2em}\textit{Alleluja, Alleluja} & \\
\hspace*{2em}\textit{Alleluja, Alleluja.....} & \\
& \\
Tak się starałem, ale cóż & \\
Dotykam tylko, zamiast czuć & \\
Lecz mówię prawdę, nie chcę was oszukać & \\
I chociaż wszystko poszło źle & \\
Przed Panem Pieśni stawię się & \\
Na ustach mając tylko Alleluja & \\
& \\
\hspace*{2em}\textit{Alleluja, Alleluja} & \\
\hspace*{2em}\textit{Alleluja, Alleluja..} & \\
\hspace*{2em}\textit{Alleluja, Alleluja} & \\
\hspace*{2em}\textit{Alleluja, Alleluja.....} & \\
\end{longtable}
\clearpage

% --- Źródło: Archanioły_Śląskiej_Ziemi.tex ---
\section{Archanioły \smash{Śląskiej} Ziemi}
\begin{longtable}{ll}
Katowickie słońce spowił czarnej wrony cień & \textbf{e C G D} \\
Nadciągnęły chmury zła, zaczął padać krwisty deszcz  & \\
Młodzieńczy wiatr się zerwał, stawił opór siłom tym  & \\
Ich ołtarzem Wieża, pomnikiem jesteśmy my  & \\
& \\
\hspace*{2em}\textit{Archanioły śląskiej ziemi}  & \\
\hspace*{2em}\textit{W naszych sercach lilia lśni}  & \\
\hspace*{2em}\textit{Archanioły śląskiej ziemi}  & \\
\hspace*{2em}\textit{Wierzymy też w lepsze dni}  & \\
& \\
\hspace*{2em}\textit{Archanioły śląskiej ziemi}  & \\
\hspace*{2em}\textit{Weźcie nas pod skrzydła swe}  & \\
\hspace*{2em}\textit{Archanioły śląskiej ziemi}  & \\
\hspace*{2em}\textit{Nie zmienię się!}  & \\
& \\
Strąceni niczym kamień w zapomnienia morze śmierć  & \\
Nie upadli wcale, wciąż śpiewają pieśń  & \\
Hymn młodości szepce też bieszczadzki wiatr  & \\
Śląskie archanioły przemierzają świat  & \\
& \\
\hspace*{2em}\textit{Archanioły śląskiej ziemi...}  & \\
& \\
Choć ślady stóp zatarte przez historii wiatr  & \\
Nie jesteśmy sami, wciąż wspierają nas  & \\
Płomienie gwiazd na niebie rozpalają gdy  & \\
Mrok ogarnia Ciebie, strach rozwiewa sny  & \\
& \\
\hspace*{2em}\textit{Archanioły śląskiej ziemi...}  & \\
& \\
\end{longtable}
\clearpage

% --- Źródło: Arka_Noego.tex ---
\section{\textbf{Arka Noego}}
\vspace{-\baselineskip}
\textit{Jacek Kaczmarski, Przemysław Gintrowski}\\
\begin{longtable}{ll}
W pełnym słońcu w środku lata & \textbf{e D} \\
Wśród łagodnych fal zieleni & \textbf{e D} \\
Wre zapamiętała praca & \textbf{e D} \\
Stawiam łódź na suchej ziemi & \textbf{e D h a e (a e a e)} \\
Owad w pąku drży kwitnącym & \textbf{e D} \\
Chłop po barki brodzi w życie & \textbf{e D} \\
Ja pracując w dzień i w nocy & \textbf{e D} \\
Mam już burty i poszycie & \textbf{e D h a e (a e a e)} \\
& \\
\hspace*{2em}\textit{Budujcie Arkę przed potopem} & \textbf{e a e} \\
\hspace*{2em}\textit{Dobądźcie na to swych wszystkich sił!} & \textbf{D a h e (a e)} \\
\hspace*{2em}\textit{Budujcie Arkę przed potopem} & \textbf{e a e} \\
\hspace*{2em}\textit{Choćby tłum z waszej pracy kpił!} & \textbf{D a h e (a e)} \\
\hspace*{2em}\textit{Ocalić trzeba co najdroższe} & \textbf{e a e} \\
\hspace*{2em}\textit{A przecież tyle już tego jest!} & \textbf{D a h e (a e)} \\
\hspace*{2em}\textit{Budujcie Arkę przed potopem} & \textbf{e a e} \\
\hspace*{2em}\textit{Odrzućcie dziś każdy zbędny gest} & \textbf{D a h e a (a e)} \\
& \\
Muszę taką łódź zbudować & \textbf{e D} \\
By w niej całe życie zmieścić & \textbf{e D} \\
Nikt nie wierzy w moje słowa & \textbf{e D} \\
Wszyscy mają ważne wieści & \textbf{e D h a e (a e a e)} \\
Ktoś się o majątek kłóci & \textbf{e D} \\
Albo łatwy węszy żer & \textbf{e D} \\
Zanim się ze snu obudzi & \textbf{e D} \\
Będę miał już maszt i ster! & \textbf{e D h a e (a e a e)} \\
& \\
\hspace*{2em}\textit{Budujcie Arkę przed potopem} & \textbf{e a e} \\
\hspace*{2em}\textit{Niech was nie mami głupców chór!} & \textbf{D a h e (a e)} \\
\hspace*{2em}\textit{Budujcie Arkę przed potopem} & \textbf{e a e} \\
\hspace*{2em}\textit{Słychać już grzmot burzowych chmur!} & \textbf{D a h e (a e)} \\
\hspace*{2em}\textit{Zostawcie kłótnie swe na potem} & \textbf{e a e} \\
\hspace*{2em}\textit{Wiarę przeczuciom dajcie raz!} & \textbf{D a h e (a e)} \\
\hspace*{2em}\textit{Budujcie Arkę przed potopem} & \textbf{e a e} \\
\hspace*{2em}\textit{Zanim w końcu pochłonie was!} & \textbf{D a h e (a e)} \\
\end{longtable}
\newpage
\begin{longtable}{ll}
\textbf{Każdy z was jest łodzią w której!} & \textbf{e D} \\
Może się z potopem mierzyć & \textbf{e D} \\
Cało wyjść z burzowej chmury & \textbf{e D} \\
Musi tylko w to uwierzyć! & \textbf{e D h a e (a e a e)} \\
Lecz w ulewie grzmot za grzmotem & \textbf{e D} \\
I za późno krzyk na trwogę & \textbf{e D} \\
I za późno usta z błotem & \textbf{e D} \\
Wypluwają mą przestrogę! & \textbf{e D h a e (a e a e)} \\
& \\
\hspace*{2em}\textit{Budujcie Arkę przed potopem} & \textbf{fis h fis} \\
\hspace*{2em}\textit{Słyszę sterując w serce fal!} & \textbf{E h cis fis (h fis)} \\
\hspace*{2em}\textit{Budujcie Arkę przed potopem} & \textbf{fis h fis} \\
\hspace*{2em}\textit{Krzyczy ten co się przedtem śmiał!} & \textbf{E h cis fis (h fis)} \\
\hspace*{2em}\textit{Budujcie Arkę przed potopem} & \textbf{fis h fis} \\
\hspace*{2em}\textit{Naszych nad własnym losem łez!} & \textbf{E h cis fis (h fis)} \\
\hspace*{2em}\textit{Budujcie Arkę przed potopem} & \textbf{fis h fis} \\
\hspace*{2em}\textit{Na pierwszy i na ostatni chrzest!} & \textbf{E h cis fis} \\
& \\
& \\
\end{longtable}
\clearpage

% --- Źródło: Autobiografia.tex ---
\section{\textbf{Autobiografia}}
\vspace{-\baselineskip}
\textit{Perfect}\\
\begin{longtable}{ll}
Miałem dziesięć lat, gdy usłyszał o nim świat & \textbf{e} \\
W mej piwnicy był nasz klub & \textbf{a7 D} \\
Kumpel radio zniósł, usłyszałem „Blue Suede Shoes” & \textbf{e} \\
I nie mogłem w nocy spać & \textbf{a7 D} \\
Wiatr odnowy wiał, darowano reszty kar & \textbf{e} \\
Znów się można było śmiać & \textbf{a7 D} \\
W kawiarniany gwar, jak tornado jazz się wdarł & \textbf{e} \\
I ja też, chciałem grać & \textbf{a7 D} \\
& \\
Ojciec, Bóg wie gdzie Martenowski stawiał piec & \textbf{e} \\
Mnie paznokieć z palca zszedł & \textbf{a7 D} \\
Z gryfu został wiór, grałem milion różnych bzdur & \textbf{e} \\
I poznałem co to seks & \textbf{a7 D} \\
Pocztówkowy szał, każdy z nas ich pięćset miał & \textbf{e} \\
Zamiast nowej pary jeans & \textbf{a7 D} \\
A w sobotnią noc, był Luksemburg, chata, szkło & \textbf{e} \\
Jakże się, chciało żyć! & \textbf{A e} \\
& \\
\hspace*{2em}\textit{Było nas trzech} & \textbf{C} \\
\hspace*{2em}\textit{W każdym z nas inna krew} & \textbf{D} \\
\hspace*{2em}\textit{Ale jeden przyświecał nam cel} & \textbf{G C} \\
\hspace*{2em}\textit{Za kilka lat} & \textbf{a} \\
\hspace*{2em}\textit{Mieć u stóp cały świat} & \textbf{F} \\
\hspace*{2em}\textit{Wszystkiego w bród} & \textbf{C} \\
\hspace*{2em}\textit{Alpagi łyk} & \textbf{C} \\
\hspace*{2em}\textit{I dyskusje po świt} & \textbf{D} \\
\hspace*{2em}\textit{Niecierpliwy w nas ciskał się duch} & \textbf{G C} \\
\hspace*{2em}\textit{Ktoś dostał w nos} & \textbf{a} \\
\hspace*{2em}\textit{To popłakał się ktoś} & \textbf{F} \\
\hspace*{2em}\textit{Coś działo się} & \textbf{C} \\
& \\
Poróżniła nas, za jej Poli Raksy twarz & \textbf{a} \\
Każdy by się zabić dał & \textbf{a7 D} \\
W pewną letnią noc, gdzieś na dach wyniosłem koc & \textbf{e} \\
I dostałem to, com chciał & \textbf{a7 D} \\
Powiedziała mi, że kłopoty mogą być & \textbf{e} \\
Ja jej, że egzamin mam & \textbf{a7 D} \\
Odkręciła gaz, nie zapukał nikt na czas & \textbf{e} \\
Znów jak pies, byłem sam & \textbf{a e} \\
\end{longtable}
\newpage
\begin{longtable}{ll}
\hspace*{2em}\textit{Stu różnych ról} & \textbf{C} \\
\hspace*{2em}\textit{Czym ugasić mój ból} & \textbf{D} \\
\hspace*{2em}\textit{Nauczyło mnie życie jak nikt} & \textbf{G C} \\
\hspace*{2em}\textit{W wyrku na wznak} & \textbf{a} \\
\hspace*{2em}\textit{Przechlapałem swój czas} & \textbf{F} \\
\hspace*{2em}\textit{Najlepszy czas} & \textbf{C} \\
\hspace*{2em}\textit{W knajpie dla braw} & \textbf{C} \\
\hspace*{2em}\textit{Klezmer kazał mi grać} & \textbf{D} \\
\hspace*{2em}\textit{Takie rzeczy, że jeszcze mi wstyd} & \textbf{G C} \\
\hspace*{2em}\textit{Pewnego dnia} & \textbf{a} \\
\hspace*{2em}\textit{Zrozumiałem, że ja} & \textbf{F} \\
\hspace*{2em}\textit{Nie umiem nic} & \textbf{C} \\
& \\
\hspace*{2em}\textit{Słuchaj mnie tam!} & \textbf{C} \\
\hspace*{2em}\textit{Pokonałem się sam} & \textbf{D} \\
\hspace*{2em}\textit{Oto wyśnił się wielki mój sen} & \textbf{G C} \\
\hspace*{2em}\textit{Tysięczny tłum} & \textbf{a} \\
\hspace*{2em}\textit{Spija słowa z mych ust} & \textbf{F} \\
\hspace*{2em}\textit{Kochają mnie} & \textbf{C} \\
\hspace*{2em}\textit{W hotelu fan} & \textbf{C} \\
\hspace*{2em}\textit{mówi: „Na taśmie mam”} & \textbf{D} \\
\hspace*{2em}\textit{„To, jak w gardłach im rodzi się śpiew”} & \textbf{G C} \\
\hspace*{2em}\textit{Otwieram drzwi} & \textbf{a} \\
\hspace*{2em}\textit{I nie mówię już nic} & \textbf{F} \\
\hspace*{2em}\textit{Do czterech ścian} & \textbf{C} \\
& \\
& \\
& \\
\end{longtable}
\clearpage

% --- Źródło: Autobus_z_piekieł_bram.tex ---
\section{Autobus z piekieł bram}
\begin{longtable}{ll}
\textbf{Kowalski:}  & \\
Wsiadajcie już, & \textbf{a} \\
Zapnijcie pas, & \textbf{C} \\
Melodię z dreszczem, & \textbf{G} \\
Dla was gram. & \textbf{a} \\
Cztery koła czarne, & \textbf{a} \\
Siedzeń rząd, & \textbf{E} \\
Już gęsią skórkę mam… & \textbf{E7 a} \\
Autobus z piekieł bram. & \textbf{E7 a} \\
& \\
\textbf{Rico: } & \\
Autobus z piekieł bram! & \textbf{E7 a} \\
& \\
\textbf{Kowalski:}  & \\
Od zmierzchu po świt & \textbf{a} \\
Kierowcy w nim brak, & \textbf{C} \\
Zamiast ropy żłopie zło. & \textbf{G} \\
Asfalt groźnie tnie, & \textbf{a} \\
Wciąż do przodu rwie & \textbf{E} \\
I zwierzęta rozgniata na proch. & \textbf{E7} \\
& \\
Oto autobus z piekieł bram. & \textbf{E7 a} \\
& \\
\textbf{Rico:}  & \\
Autobus z piekieł bram! & \textbf{E7 a} \\
& \\
\textbf{Kowalski:}  & \\
Pociechy swe & \textbf{a} \\
Ukryjcie, bo gdzieś & \textbf{C} \\
Potworny bus żreć wyruszył już… & \textbf{G} \\
& \\
\textbf{Fred:}  & \\
Rozjechał dwóch moich wujków  & \\
I jeszcze sześciu kuzynów.  & \\
Podobno nie czuli bólu...  & \\
Tak zgaduję.  & \\
& \\
\textbf{Kowalski i Rico:}  & \\
Autobus z piekieł bram! & \textbf{E7 a} \\
& \\
\end{longtable}
\clearpage

% --- Źródło: Ballada_Wrześniowa.tex ---
\section{Ballada Wrześniowa}
\vspace{-\baselineskip}
\textit{Jacek Kaczmarski}\\
\begin{longtable}{ll}
Długośmy na ten dzień czekali  & \\
Z nadzieją niecierpliwą w duszy,  & \\
Kiedy bez słów Towarzysz Stalin  & \\
Na mapie fajką strzałki ruszy.  & \\
& \\
Krzyk jeden pomknął wzdłuż granicy  & \\
I zanim zmilkł, zagrzmiały działa –  & \\
To w bój z szybkością nawałnicy  & \\
Armia Czerwona wyruszała.  & \\
& \\
– A cóż to za historia nowa? –  & \\
Zdumiona spyta Europa.  & \\
– Jak to? To chłopcy Mołotowa  & \\
I sojusznicy Ribbentropa.  & \\
& \\
Zwycięstw się szlak ich serią znaczy,  & \\
Sztandar wolności okrył chwałą;  & \\
Głowami polskich posiadaczy  & \\
Brukują Ukrainę całą.  & \\
& \\
Pada Podole, w hołdach Wołyń,  & \\
Lud pieśnią wita ustrój nowy,  & \\
Płoną majątki i kościoły  & \\
I Chrystus – z kulą w tyle głowy.  & \\
& \\
Nad polem bitwy dłonie wzniosą  & \\
We wspólną pięść, co dech zapiera –  & \\
Nieprzeliczone dzieci Soso,  & \\
Niezwyciężony miot Hitlera.  & \\
& \\
Już starty z map wersalski bękart,  & \\
Już wolny Żyd i Białorusin,  & \\
Już nigdy więcej polska ręka  & \\
Ich do niczego nie przymusi.  & \\
& \\
Nową im wolność głosi „Prawda”,  & \\
Świat cały wieść obiega w lot,  & \\
Że jeden odtąd łączy sztandar  & \\
Gwiazdę, sierp, hakenkreuz i młot.  & \\
& \\
Tych dni historia nie zapomni,  & \\
Gdy stary ląd w zdumieniu zastygł  & \\
I święcić będą nam potomni  & \\
Po pierwszym września – siedemnasty.  & \\
\end{longtable}
\clearpage

% --- Źródło: Ballada_majowa.tex ---
\section{Ballada majowa}
\begin{longtable}{ll}
Brnąłem do ciebie maju & \textbf{D A} \\
Przez mrozy i biele & \textbf{G0 h} \\
Przez śnieżyce i zaspy & \textbf{G fis} \\
I lute zawieje & \textbf{G A} \\
Przez bezbarwne szpitalne & \textbf{D A} \\
Korytarze stycznia & \textbf{G0 h} \\
W tych korytarzach słońce & \textbf{G fis} \\
Gasło ustawicznie & \textbf{G A D} \\
& \\
\hspace*{2em}\textit{A teraz maj dookoła maj} & \textbf{D A} \\
\hspace*{2em}\textit{Wyświęca ogrody} & \textbf{G0 h} \\
\hspace*{2em}\textit{I cały ja i cały ja} & \textbf{G D} \\
\hspace*{2em}\textit{Zanurzony w jordanie pogody} & \textbf{G A} \\
\hspace*{2em}\textit{A teraz maj i maj i maj} & \textbf{D A} \\
\hspace*{2em}\textit{Dokoła się święci} & \textbf{G0 h} \\
\hspace*{2em}\textit{Od wonnych bzów szalonych bzów} & \textbf{G D} \\
\hspace*{2em}\textit{Wprost w głowie się kręci} & \textbf{G A D} \\
& \\
I płyną przeze mnie dmuchawce  & \\
Jak dzieciństwa echa  & \\
I wielka jest majowa moc  & \\
Kiedy niebo się do ziemi uśmiecha  & \\
Śpi w twoim wnętrzu chłopiec  & \\
W chłopcu pierwszy zachwyt poznaję  & \\
Z twoich ziaren wyrosną sady  & \\
Strudzonemu pielgrzymką ulżyj dodaj wiary  & \\
& \\
\hspace*{2em}\textit{A teraz maj dookoła maj...}  & \\
& \\
\end{longtable}
\clearpage

% --- Źródło: Ballada_o_dziewczynie_co_piła_gorące_mleko.tex ---
\section{Ballada o dziewczynie, co piła gorące mleko}
\begin{longtable}{ll}
Są małe stacje wielkich kolei & \textbf{C D} \\
Nieznane jak obce imiona, & \textbf{e} \\
Są małe stacje wielkich kolei & \textbf{C D} \\
Jakiś napis i lampa zielona. & \textbf{e} \\
& \\
Na takiej stacji dawno już temu  & \\
Z daleka jadąc, z daleka  & \\
Widziałem dziewczynę w niebieskim szaliku,  & \\
Jak piła gorące mleko.  & \\
& \\
Teraz tamtędy już nigdy nie jeżdżę, & \textbf{F C} \\
A miasto moje daleko. & \textbf{F C} \\
Lecz myślę czasem o tamtej dziewczynie, & \textbf{F C} \\
Jak piła gorące mleko. & \textbf{G C} \\
& \\
\hspace*{2em}\textit{I nieraz chciałbym aby tu była,} & \textbf{c9 D} \\
\hspace*{2em}\textit{A może to miałoby sens.} & \textbf{G e} \\
\hspace*{2em}\textit{Jak ona śmiesznie to mleko piła,} & \textbf{c9 D} \\
\hspace*{2em}\textit{Gapiąc się na mnie spod rzęs.} & \textbf{G e} \\
& \\
Mam swoje sprawy, inne podróże  & \\
I nie tamtędy mi droga.  & \\
Lubię ulice wesołe i duże  & \\
I kolorowe światła na rogach.  & \\
& \\
Pewnie ma chłopca tamta dziewczyna,  & \\
A może wybrała się w świat.  & \\
Albo po prostu może jest głupia  & \\
Jak jej siedemnaście lat.  & \\
& \\
Zresztą to przecież nie ma znaczenia,  & \\
Mieszkam naprawdę daleko.  & \\
Lecz myślę czasem o tamtej dziewczynie,  & \\
Jak piła gorące mleko.  & \\
& \\
\hspace*{2em}\textit{I nieraz chciałbym aby tu była,}  & \\
\hspace*{2em}\textit{A może to miałoby sens.}  & \\
\hspace*{2em}\textit{Jak ona śmiesznie to mleko piła,}  & \\
\hspace*{2em}\textit{Gapiąc się na mnie spod rzęs.}  & \\
& \\
& \\
& \\
\end{longtable}
\clearpage

% --- Źródło: Ballada_o_dzikim_zachodzie.tex ---
\section{Ballada o dzikim zachodzie}
\begin{longtable}{ll}
Potwierdzają to setne przykłady, & \textbf{D G D G D} \\
Że westerny wciąż jeszcze są w modzie, & \textbf{h A7 D G D} \\
Wysłuchajcie więc, proszę, ballady & \textbf{G D G D} \\
O tak zwanym najdzikszym zachodzie & \textbf{h A7 D G D7} \\
& \\
Miasto było tam, jakich tysiące, & \textbf{G g D D7} \\
Wokół preria i skały naprzeciw, & \textbf{G D} \\
Jak gdzie indziej, świeciło tam słońce, & \textbf{G D G D} \\
Marli starcy, rodziły się dzieci, & \textbf{h A7 D G D7} \\
& \\
\hspace*{2em}\textit{I tym tylko od innych różni się ta ballada,} & \textbf{G D A7 D} \\
\hspace*{2em}\textit{Że w tym mieście gdzieś na prerii krańcach} & \textbf{G D D7} \\
\hspace*{2em}\textit{Na jednego mieszkańca jeden szeryf przypadał,} & \textbf{G D A7 D} \\
\hspace*{2em}\textit{Jeden szeryf na jednego mieszkańca} & \textbf{h A7 D G D A7} \\
& \\
Konsekwencje ten fakt miał ogromne,  & \\
Bo nikt w mieście za spluwę nie chwytał,  & \\
I od dawna już każdy zapomniał,  & \\
Jak wygląda prawdziwy bandyta.  & \\
& \\
Choć finanse poniekąd leżały,  & \\
Gospodarka i przemysł był na nic,  & \\
Ale każdy, czy duży czy mały,  & \\
Czuł się za to bezpieczny bez granic  & \\
& \\
\hspace*{2em}\textit{I tym tylko od innych różni się ta ballada...}  & \\
& \\
Jeśli państwa historia ta nudzi,  & \\
To pocieszcie się tym, że nareszcie  & \\
Którejś nocy krzyk ludzi obudził,  & \\
Bank rozbity! bandyci są w mieście  & \\
& \\
Dobrzy ludzie, na próżno wołacie,  & \\
Nikt nie wstanie, za spluwę nie chwyci,  & \\
Skoro każdy świadomość zatracił,  & \\
Czym się różnią od ludzi bandyci,  & \\
& \\
\hspace*{2em}\textit{I tym tylko od innych różni się ta ballada...}  & \\
\end{longtable}
\newpage
\begin{longtable}{ll}
Potwierdzają to setne przykłady, & \textbf{D G D G D} \\
Że westerny wciąż jeszcze są w modzie, & \textbf{h A7 D G D} \\
Wysłuchaliście, państwo, ballady & \textbf{G D G D} \\
O tzw. najdzikszym zachodzie, & \textbf{h A7 D G D7} \\
& \\
Miasto było tam, jakich tysiące, & \textbf{G g D D7} \\
Ludzkie w nim krzyżowały się drogi, & \textbf{G D} \\
Lecz nie wszystkim świeciło tam słońce, & \textbf{G D G D} \\
Bo bandyci krążyli bez trwogi & \textbf{h A7 D G D7} \\
& \\
\hspace*{2em}\textit{Wyciągnijmy więc morał w tej balladzie ukryty,} & \textbf{G D A7 D} \\
\hspace*{2em}\textit{Gdy nie grozi nam żadne riffifi,} & \textbf{G D D7} \\
\hspace*{2em}\textit{Że czasami najtrudniej jest rozpoznać bandytę,} & \textbf{G D A7 D} \\
\hspace*{2em}\textit{Gdy dokoła są sami szeryfi} & \textbf{h A7 D G D A7} \\
& \\
& \\
& \\
\end{longtable}
\clearpage

% --- Źródło: Ballada_o_harcerskiej_miłości.tex ---
\section{Ballada o harcerskiej miłości}
\begin{longtable}{ll}
\textbf{e C G D}  & \\
& \\
Historia ta która bez wątpienia  & \\
Jest warta opowiedzenia był raz sobie młody harcerz  & \\
Który żył samotnie z siwym gołąbkiem  & \\
Historie znam która bez wątpienia jest warta opowiedzenia  & \\
Był raz sobie młody harcerz który żył samotnie z siwym gołąbkiem  & \\
& \\
Pewnego razu poznał on dziewczynę  & \\
Przechadzała się samotnie po ulicach w Lublinie  & \\
Kochali się nawzajem każdą chwilą byli razem i  & \\
przyrzekli sobie razem że nie zginą na zawsze  & \\
& \\
Na wieki na wieki razem || x3  & \\
Na wieki!  & \\
& \\
On nie mógł zrozumieć dlaczego tak się stało  & \\
Ukochana jego wyjechała Jeden tylko ślad  & \\
Został po niej Był to mały liścik na jego stole  & \\
& \\
Przepraszam Cię kochany to się stać musiało  & \\
Muszę się zajmować moją chorą mamą  & \\
Tymczasem mój drogi zapomnij mnie na zawsze  & \\
Bo już nigdy przenigdy nie będziemy razem  & \\
& \\
Mijały dni całe tygodnie a harcerz nie mógł  & \\
Zapomnieć o niej Po swej wybrance był zdruzgotany  & \\
Nie wychodził z domu i płakał nocami  & \\
Wpadł mu do głowy pomysł doskonały  & \\
& \\
Posłał więc gołąbka do swej ukochanej  & \\
Silny gołąbek jak grotu strzała pomknął w daleką  & \\
Daleką krainę by zaspokoić serce harcerza  & \\
Przyniósł mu taką taką nowinę:  & \\
& \\
Drogi harcerzu ulżyj cierpieniu twoja dziewczyna nie żyje  & \\
Ten czarny krzyżyk ci zostawiła abyś go włożył na szyje  & \\
Ten czarny krzyżyk ci zostawiła abyś go włożył na szyję  & \\
& \\
Zdruzgany harcerz tymi słowami  & \\
Jeszcze się bardziej zasmucił  & \\
Wyciągnął nóż serce swe przebił i do kochanej powrócił || x5  & \\
Na wieki na wieki razem || x3  & \\
Na wieki  & \\
\end{longtable}
\clearpage

% --- Źródło: Ballada_o_jednej_pani_Wisniewskiej.tex ---
\section{Ballada o jednej (pani) Wisniewskiej}
\begin{longtable}{ll}
Żyli w pałacu hrabia z hrabiną, & \textbf{a} \\
On zwał się Rodryg, ona Francesca, & \textbf{d a} \\
A w drugim domku za ich meliną & \textbf{d a} \\
Mieszkała sobie jedna Wiśniewska. & \textbf{a E a} \\
& \\
Niewinne serce miała hrabina  & \\
I takąż duszę pieską, niebieską,  & \\
A on był gałgan i straszna świnia,  & \\
Bo pitigrilił się z tą Wiśniewską.  & \\
& \\
Biedna hrabina łzami płakała,  & \\
Z ciągłej żałoby wyschła na deskę  & \\
I na kolanach męża błagała:  & \\
Odczep się, draniu, od tej Wiśniewskiej.  & \\
& \\
Próżno chodziła z hrabią na udry,  & \\
Na próżno klęła swą dolę pieską,  & \\
On ciągle ganiał do tej łachudry  & \\
I szeptał czule: "O, ty Wiśniewsko!"  & \\
& \\
Aż raz hrabina miecz zdjęła z ściany,  & \\
Zmierzyła hrabię okiem królewskim.  & \\
Siedź tu, powiada, ty - w herb drapany,  & \\
Dzisiaj nie pójdziesz do tej Wiśniewskiej.  & \\
& \\
On zaś będący pod alkoholem,  & \\
Czyli, jak mówią - zalany w pestkę,  & \\
Wyrżnął hrabinę łbem w antresolę  & \\
I dawaj, gazu! Do tej Wiśniewskiej,  & \\
& \\
Biedna hrabina padła na dywan,  & \\
Cała zalała się krwią niebieską,  & \\
A gdy poczuła, że dogorywa,  & \\
Rzekła: poczekaj, o ty, Wiśniewsko.  & \\
& \\
& \\
\end{longtable}
\newpage
\begin{longtable}{ll}
Potem się odbył pogrzeb wspaniały, & \textbf{a} \\
Hrabia nad grobem uronił łezkę, & \textbf{d a} \\
Strasznie się martwił przez dzionek cały, & \textbf{d a} \\
A na noc poszedł ... do tej Wiśniewskiej. & \textbf{a E a} \\
& \\
Wtedy hrabina z mogiły wstała,  & \\
Wyrwała z trumny sękatą deskę,  & \\
Poszła za hrabią, na śmierć go sprała  & \\
I rozwaliła łeb tej Wiśniewskiej.  & \\
& \\
Chociaż lebiegi grzeszyli tyle  & \\
I na nich w końcu też przyszła kreska.  & \\
Dziś sobie leżą w jednej mogile:  & \\
Hrabia, hrabina i... ta Wiśniewska.  & \\
\end{longtable}
\clearpage

% --- Źródło: Ballada_o_krzyżowcu.tex ---
\section{\textbf{Ballada o krzyżowcu}}
\begin{longtable}{ll}
Wolniej, wolniej, wstrzymaj konia! & \textbf{e} \\
Dokąd pędzisz w stal odziany? & \textbf{A} \\
Pewnie tam, gdzie w słońcu błyszczą & \textbf{C} \\
Jeruzalem białe ściany. & \textbf{D} \\
& \\
Pewnie myślisz, że w świątyni & \textbf{e} \\
Zniewolony pan twój czeka, & \textbf{A} \\
Abyś przybył go ocalić, & \textbf{C} \\
Abyś przybył doń z daleka. & \textbf{D} \\
& \\
\hspace*{2em}\textit{(Na na na naj…)} & \textbf{e A C D} \\
& \\
Wolniej, wolniej, wstrzymaj konia! & \textbf{e} \\
Byłem wczoraj w Jeruzalem, & \textbf{A} \\
przemierzałem puste sale - & \textbf{C} \\
Pana twego nie widziałem. & \textbf{D} \\
& \\
Pan opuścił święte miasto & \textbf{e} \\
Przed minutą, przed godziną, & \textbf{A} \\
W chłodnym gaju na pustyni & \textbf{C} \\
Z Mahometem pije wino. & \textbf{D} \\
& \\
\hspace*{2em}\textit{(Na na na naj…)}  & \\
& \\
Wolniej, wolniej, wstrzymaj konia! & \textbf{e} \\
Chcesz oblegać Jeruzalem? & \textbf{A} \\
Strzegą go wysokie wieże, & \textbf{C} \\
Strzegą go mahometanie. & \textbf{D} \\
& \\
Pan opuścił Święte Miasto, & \textbf{e} \\
Na nic poświęcenie twoje, & \textbf{A} \\
Po co niszczyć białe mury, & \textbf{C} \\
Po co ludzi niepokoić. & \textbf{D} \\
& \\
\hspace*{2em}\textit{(Na na na naj…)}  & \\
& \\
Wolniej, wolniej, wstrzymaj konia & \textbf{e} \\
Porzuć walkę niepotrzebną, & \textbf{A} \\
Porzuć miecz i włócznię swoją & \textbf{C} \\
I jedź ze mną, i jedź ze mną. & \textbf{D} \\
& \\
Bo, gdy szlakiem ku północy & \textbf{e} \\
Podążają hufce ludne, & \textbf{A} \\
Ja unoszę dumnie głowę & \textbf{C} \\
I odjeżdżam na południe.  & \\
\end{longtable}
\clearpage

% --- Źródło: Ballada_o_smutnym_programiście.tex ---
\section{Ballada o smutnym programiście}
\begin{longtable}{ll}
Po co mi był ZX Spectrum? & \textbf{D A} \\
Na co mi była Amiga? & \textbf{E A E} \\
Wybrałem sobie karierę & \textbf{D A} \\
starego nudnego grzyba. & \textbf{D E A} \\
& \\
A mogłem być hydraulikiem, & \textbf{D A} \\
klientów miałbym bez liku, & \textbf{E A E} \\
ale mnie się zachciało & \textbf{D A D} \\
pisać programy w BASIC-u. & \textbf{E A} \\
& \\
\hspace*{2em}\textit{Programuję w .Necie} & \textbf{A} \\
\hspace*{2em}\textit{już trzecie stulecie,} & \textbf{E} \\
\hspace*{2em}\textit{bo kto się w PHP-ie połapie?} & \textbf{D E} \\
\hspace*{2em}\textit{I zmierzam do celu} & \textbf{A} \\
\hspace*{2em}\textit{z użyciem SQL-u} & \textbf{E} \\
\hspace*{2em}\textit{pod Microsoft Windows Vista.} & \textbf{D E A E} \\
& \\
\hspace*{2em}\textit{A kiedy mi smutno,} & \textbf{A} \\
\hspace*{2em}\textit{statyczny konstruktor} & \textbf{E} \\
\hspace*{2em}\textit{utworzę} & \textbf{D} \\
\hspace*{2em}\textit{w edytorze.} & \textbf{E} \\
\hspace*{2em}\textit{Bo wewnątrz mej głowy} & \textbf{A} \\
\hspace*{2em}\textit{mam świat obiektowy,} & \textbf{E} \\
\hspace*{2em}\textit{ja, smutny programista.} & \textbf{D A E} \\
& \\
Za oknem ptaki śpiewają  & \\
i tyle jest piękna na świecie,  & \\
lecz nic nie widzę pięknego  & \\
w pisaniu programów w .Necie.  & \\
& \\
Ach, gdzie te wózki widłowe,  & \\
gdzie te miotły i szmaty?  & \\
Ja nie chcę być programistą,  & \\
ja chcę iść do łopaty!  & \\
& \\
\hspace*{2em}\textit{Programuję w .Necie...}  & \\
\end{longtable}
\newpage
\begin{longtable}{ll}
Gdy widzę szczęśliwych roboli,  & \\
pijanych w cztery litery,  & \\
wiem wtedy, że wybrałem  & \\
błędną ścieżkę kariery.  & \\
& \\
Też chcę być na budowie,  & \\
tam pić mi nikt nie zabroni.  & \\
Chcę tyle co oni pracować,  & \\
chcę tyle zarabiać co oni!  & \\
& \\
\hspace*{2em}\textit{Programuję w .Necie...}  & \\
& \\
& \\
& \\
\end{longtable}
\clearpage

% --- Źródło: Ballada_o_spalonej_synagodze.tex ---
\section{Ballada o spalonej synagodze}
\vspace{-\baselineskip}
\textit{Jacek Kaczmarski}\\
\begin{longtable}{ll}
Żydzi, Żydzi wstawajcie płonie synagoga! & \textbf{a G a} \\
Cała Wschodnia Ściana porosła już żarem! & \textbf{a G a} \\
Powietrze drży, skacze po suchych belkach ogień! & \textbf{a G a} \\
Wstawajcie, chrońcie święte księgi wiary! & \textbf{a G e a} \\
 & \\
Gromadzą się ciemni w krąg płonącego gmachu, & \textbf{a G a} \\
Biegają iskry po pejsach, tlą się długie brody, & \textbf{a G C} \\
W oczach blask pożaru, rozpaczy i strachu – & \textbf{d C d a} \\
Gdzie będą teraz wznosić swoje próżne modły?! & \textbf{a G e a} \\
 & \\
Płonie synagoga! Trzask i krzyk gardłowy! & \textbf{a G a} \\
Belki w żar się sypią, iskry w górę lecą! & \textbf{a G a} \\
Tam, w środku Żyd został! Jeszcze widać ręce! & \textbf{a G a} \\
Już czarne! O, Jehowo, za co? O, Jehowo?! & \textbf{a G e a} \\
 & \\
Noc zapadła. Stoją wszyscy wielkim kołem, & \textbf{a G a} \\
Chmury nisko, w ciemnościach gwiazda się dopala. & \textbf{a G C} \\
Patrzą na swe chałaty porosłe popiołem, & \textbf{d C d a} \\
Rząd rąk i twarzy w mrok się powoli oddala… & \textbf{a G e a} \\
 & \\
Stali tak do rana, deszcz spadł, ciągle stali & \textbf{a d} \\
Aż popiół się zmieszał ze szlamem. & \textbf{d a} \\
– Jeden Żyd się spalił! Jeden Żyd się spalił! & \textbf{a} \\
Śpiewał w knajpie nad wódką włóczęga pijany. & \textbf{d e a} \\
\end{longtable}
\clearpage

% --- Źródło: Ballada_o_zamku.tex ---
\section{\textbf{Ballada o zamku}}
\begin{longtable}{ll}
Stał kiedyś zamek, który twierdzą niezdobytą był & \textbf{e e G D e} \\
Potężne mury, czarna fosa, stalą kute drzwi &  \\
łuczników wielka, zbrojna moc strzegła zamku dzień i noc &  \\
By żaden nieproszony człek przez bramy nie-nie mógł przejść &  \\
& \\
A żył tam pewien stary mag co włosy białe miał jak śnieg &  \\
I pewien bardzo młody bard co z myśli splatał pieśń &  \\
I chociaż czas rozdzielił ich dekada długich lat &  \\
Z jednego dzbana pili wciąż: młody bard i stary mag &  \\
& \\
Najechał kiedyś zamek ten pewien bardzo możny pan &  \\
Najemnych setki zebrał dwie by wznieść podwoje bram &  \\
I rozgorzał wielski boj, z ran toczyła się krew &  \\
Na wieży biało-włosy mag słał swój magi-magiczny zew &  \\
& \\
I spłynął z rozwścieczonych chmur pan huraganów – wiatrów król &  \\
I na lawinie ludzkich ciał błyskawic swoje armie słał &  \\
I wygrał bitwę zamku pan, ostały się podwoje bram &  \\
A ten kto żyw ku wieży biegł, aby magowi pokłon nieść &  \\
& \\
Lecz przerwał tupot kroków ich czerwonej strzały wściekły świst &  \\
Mag zachwiał się a potem zbladł, szepcząc zaklęcie z wieży spadł  & \\
Nim rozbił się o kamieni brzeg zaklęcia błysk rozjaśnił dzień &  \\
Wieczornym niebem przemknął ptak ciągnąc za sobą nocy cień  & \\
& \\
Zostało po nim kilka ksiąg, niedopitego miodu dzban &  \\
Nikt nigdy już nie widział go i młody bard pozostał sam &  \\
Gdy nucił smutnej pieśni ton, gdy skuwał mróz gałęzie drzew  & \\
To blady świt rozjaśniał mrok, białego ptaka niosąc śpiew.  & \\
& \\
& \\
\end{longtable}
\clearpage

% --- Źródło: Ballada_o_św_Mikołaju.tex ---
\section{Ballada o św. Mikołaju}
\begin{longtable}{ll}
W rozstrzelanej chacie & \textbf{a G E} \\
Rozpaliłem ogień, & \textbf{a G a} \\
Z rozwalonych pieców & \textbf{a G E} \\
Pieśni wyniosłem węgle. & \textbf{F E} \\
& \\
Naciagnałem na drzazgi gontów & \textbf{a C} \\
Błękitną płachtę nieba & \textbf{d E} \\
Będę malować od nowa & \textbf{a d C E} \\
wioskę w dolinie. & \textbf{d E a G} \\
& \\
\hspace*{2em}\textit{Święty Mikołaju, opowiedz jak tu było,} & \textbf{C G C E} \\
\hspace*{2em}\textit{Jakie pieśni śpiewano?} & \textbf{a d C E} \\
\hspace*{2em}\textit{Gdzie się pasły konie.} & \textbf{d E a} \\
& \\
A on nie chce gadać  & \\
Ze mną po polsku  & \\
Z wypalonych źrenic  & \\
Tylko deszcze płyną.  & \\
& \\
Hej ślepcze, nauczę  & \\
Swoje dziecko po łemkowsku  & \\
Będziecie razem żebrać  & \\
W malowanych wioskach.  & \\
& \\
\hspace*{2em}\textit{Święty Mikołaju, opowiedz jak tu było...}  & \\
\end{longtable}
\clearpage

% --- Źródło: Ballada_rajdowa.tex ---
\section{Ballada rajdowa}
\begin{longtable}{ll}
Właśnie tu, na tej ziemi, młody harcerz meldował & \textbf{G D} \\
Swą gotowość umierać za Polskę & \textbf{C G} \\
Tak jak ty niesiesz plecak on niósł w ręku karabin & \textbf{G D} \\
W sercu miłość nadzieję i troskę & \textbf{C D G} \\
Może tu w Nowej Słupi Daleszycach Bielicach & \textbf{G D} \\
Brzozowymi krzyżami znaczone & \textbf{C G} \\
Swą dziewczynę pożegnał nic nie wiedząc że tylko & \textbf{G D} \\
Kilka dni życia mu przeznaczonych & \textbf{C G} \\
& \\
\hspace*{2em}\textit{Naszej ziemi śpiewajmy ziemi pokłon składajmy} & \textbf{G D} \\
\hspace*{2em}\textit{Taki prosty serdeczny harcerski} & \textbf{C D G} \\
\hspace*{2em}\textit{Niechaj echo poniesie tę balladę rajdową} & \textbf{G D} \\
\hspace*{2em}\textit{W nowe jutro i przyszłość nową} & \textbf{C D G} \\
& \\
Na pomniku wyryto że szesnaście miał wiosen & \textbf{G D} \\
Że był śmiały odważny radosny & \textbf{C G} \\
Kiedy padał płakała cała puszcza jodłowa & \textbf{G D} \\
Nie doczekał czekanej tak wiosny & \textbf{C G} \\
I choć on nie doczekał to nie zginął tak sobie & \textbf{G D} \\
Przetarł szlak którym dzisiaj wędrujesz & \textbf{C G} \\
Kiedy tak przy ognisku śpiewasz sobię balladę & \textbf{G D} \\
W sercu tak jak on ojczyznę czujesz & \textbf{C G} \\
& \\
\hspace*{2em}\textit{Naszej ziemi śpiewajmy ziemi pokłon składajmy...}  & \\
& \\
& \\
& \\
\end{longtable}
\clearpage

% --- Źródło: Ballada_z_gór.tex ---
\section{Ballada z gór}
\begin{longtable}{ll}
Tu króluje zeszłoroczny czas & \textbf{G C D} \\
Na posłaniu z liści buczynowych & \textbf{B F C} \\
Stąd do ziemi dalej niż do gwiazd & \textbf{a h e} \\
Zachwytu swego nie wysłowisz & \textbf{C D G} \\
& \\
Rosną skrzydła u ramion & \textbf{e} \\
Czas się w wieczność przemienia & \textbf{C7+} \\
Obłocznieją wszystkie ziemskie sprawy & \textbf{a} \\
Gdy zbliżamy się do szczytu po kamieniach & \textbf{H7} \\
& \\
Rosną skrzydła u ramion & \textbf{e} \\
Czas się w wieczność przemienia & \textbf{C7+} \\
Góry i wolność dokoła & \textbf{a} \\
Chyba dostąpimy tu wniebowstąpienia & \textbf{H7} \\
& \\
A w schronisku święty Piotr z herbatą & \textbf{G C D} \\
I widoki nieziemskie na świat & \textbf{B F C} \\
Przy ognisku rozłożymy się z gitarą & \textbf{a h e} \\
Posłuchamy co nam w duszy gra & \textbf{C D G} \\
& \\
Z pleców góry zrzucimy do stóp & \textbf{G C D} \\
I zmęczenie rozzujemy z nóg & \textbf{B F C} \\
To schronisko to prawdziwy raj & \textbf{a h e} \\
Niechaj wiecznie odpoczynek trwa & \textbf{C D G} \\
& \\
Rosną skrzydła u ramion  & \\
Rosną przepastne błękity  & \\
\smash{Życie} pełne olśnień i zachwytów  & \\
Tyś wędrówką najwytrwalszą ku szczytom  & \\
& \\
Góry tu wszystko jest święte  & \\
Tu wspinaczki nasze wniebowzięte  & \\
W górach los ma \smash{Światowida} twarz  & \\
Od ogniska bije jeszcze baśni blask  & \\
\end{longtable}
\clearpage

% --- Źródło: Barwy_Piosenki.tex ---
\section{Barwy Piosenki}
\vspace{-\baselineskip}
\textit{Skaldowie}\\
\begin{longtable}{ll}
Piosenka, której nie ma & \textbf{C a} \\
Niebieska jest jak świt & \textbf{d G7} \\
Nic nie ma, ledwie temat & \textbf{C a} \\
Rym jakiś, jakiś rym & \textbf{e A7} \\
Nieśmiała, nieforemna & \textbf{d G7} \\
Szara tak jak nocy brzeg & \textbf{e a} \\
Gdy świt z ptaków snem miesza się & \textbf{d G7 C a d G7} \\
& \\
Piosenka, której nie ma  & \\
Nie zna jeszcze barw  & \\
Gdzieś tam jest zieleń ciemna  & \\
I sjeny ciepły czar  & \\
Lecz pierwszy cień piosenki  & \\
Nie ma nic z palety tej  & \\
Jest ledwie cień lub jeszcze mniej  & \\
Johnny Tomala - ty nadasz barwę jej  & \\
Ty nadasz barwę jej, ty nadasz barwę jej  & \\
& \\
A później jest zielona  & \\
Jakby zerwał ktoś  & \\
Zielone winogrona  & \\
Bo miał czekania dość  & \\
Już struna uderzona  & \\
Nutki ślad, zielony liść  & \\
Już ton, pierwszy ton musi przyjść  & \\
& \\
Piosenka pierwszych spotkań  & \\
W cieniu bladych bzów  & \\
Dziewczyna taka wiotka  & \\
I tak jej braknie słów  & \\
Piosenka pierwszych liści  & \\
Pierwszych traw z zielonych łąk  & \\
Ten świat chce przyjść  & \\
Przyjść nam do rąk  & \\
Johnny Tomala - ty nam załatwisz to  & \\
Ty nam załatwisz to, ty nam załatwisz to  & \\
\end{longtable}
\newpage
\begin{longtable}{ll}

A potem przyjdą słowa  & \\
I to będzie już  & \\
Piosenka kolorowa  & \\
Z kwitnącą burzą róż  & \\
Brązowa bossa nova  & \\
Złoty skwar upalnych dni  & \\
To wszystko jest w niej, jest od dziś  & \\
& \\
Swą wartość będzie znała & \textbf{C a} \\
Jak dziewczyna zna & \textbf{d G7} \\
Piosenka już dojrzała & \textbf{C a} \\
Sierpniowych pełna barw & \textbf{e A7} \\
Piosenka pełni lata & \textbf{d G7} \\
Ciężki od owoców sad  & \\
Tych kilka nut to jest ten świat  & \\
Johnny Tomala - zagraj nam właśnie tak  & \\
Zagraj nam właśnie tak, zagraj nam właśnie tak & \textbf{d G7 C a d G7} \\
& \\
Piosenka już gotowa  & \\
Jeszcze tylko rytm  & \\
Ta barwa gitarowa  & \\
I to już jest big beat  & \\
Piosenka kolorowa  & \\
Nagle ma szesnaście lat  & \\
To jest nasz rytm, to nasz świat  & \\
& \\
Piosenka kolorowa  & \\
\smash{Śpiewa} o tym, że  & \\
\smash{Świat} tworzy się od nowa  & \\
Od nowa zacznie się  & \\
Piosenka bigbitowa  & \\
Taki jest jej cały świat  & \\
Świat, co wciąż ma tak mało lat  & \\
Johnny Tomala - ty nam otworzysz świat  & \\
Ty nam otworzysz świat, ty nam otworzysz świat  & \\
\end{longtable}
\clearpage

% --- Źródło: Bez_słów.tex ---
\section{Bez słów}
\begin{longtable}{ll}
Chodzą ulicami ludzie & \textbf{G D} \\
Maj przechodzą lipiec, grudzień & \textbf{e h} \\
Zagubieni wśród ulic bram & \textbf{C G D} \\
Przemarznięte grzeją dłonie & \textbf{G D} \\
Dokądś pędzą, za czymś gonią & \textbf{e h} \\
I budują wciąż domki z kart. & \textbf{C G D} \\
& \\
\hspace*{2em}\textit{A tam w mech odziany w kamień} & \textbf{C G} \\
\hspace*{2em}\textit{Tak zaduma w wiatru graniu} & \textbf{C G} \\
\hspace*{2em}\textit{Tam powietrze ma inny smak.} & \textbf{C G D} \\
\hspace*{2em}\textit{Porzuć kroków rytm na bruku} & \textbf{C G} \\
\hspace*{2em}\textit{Spróbuj znajdziesz jeśli szukasz} & \textbf{C G} \\
\hspace*{2em}\textit{Zechcesz nowy świat, własny świat.} & \textbf{C G D} \\
& \\
Płyną ludzie miastem szarzy & \textbf{G D} \\
Pozbawieni złudzeń, marzeń & \textbf{e h} \\
Omijając wciąż główny nurt. & \textbf{C G D} \\
Kryją się w swych norach krecich & \textbf{G D} \\
I śnić nawet o karecie & \textbf{e h} \\
Co lśni złotem nie potrafią już. & \textbf{C G D} \\
& \\
\hspace*{2em}\textit{A tam w mech odziany w kamień...}  & \\
& \\
Żyją ludzie asfalt depczą & \textbf{G D} \\
Nikt nie krzyknie, każdy szepce & \textbf{e h} \\
Drzwi zamknięte zaklepany krąg. & \textbf{C G D} \\
Tylko czasem kropla z oczu & \textbf{G D} \\
Po policzku w dół się stoczy & \textbf{e h} \\
I to dziwne drżenie rąk. & \textbf{C G D} \\
& \\
\hspace*{2em}\textit{A tam w mech odziany w kamień...}  & \\
\end{longtable}
\clearpage

% --- Źródło: Biała_Sukienka.tex ---
\section{\textbf{Biała Sukienka}}
\begin{longtable}{ll}
Czasami gdy mam chandrę i jestem sam & \textbf{a a F C} \\
Kieruję wzrok za okno wysoko tam & \textbf{a e F G C} \\
Gdzie nad dachami domów i w noc i dniem & \textbf{E a D7 G} \\
Nadpływa kołysząca marzeniem snem & \textbf{a e F G C} \\
& \\
\hspace*{2em}\textit{I ona taka w tej białej sukience} & \textbf{C G} \\
\hspace*{2em}\textit{Jak piękny ptak który zapiera w piersi dech} & \textbf{C F C} \\
\hspace*{2em}\textit{Chwyciłem mocno jej obie ręce} & \textbf{G C F} \\
\hspace*{2em}\textit{Oczarowany zasłuchany w słodki śmiech} & \textbf{C D7 G G7} \\
& \\
\hspace*{2em}\textit{I cała w żaglach jak w białej sukience} & \textbf{C G} \\
\hspace*{2em}\textit{Jak piękny ptak który zapiera w piersi dech} & \textbf{C F C} \\
\hspace*{2em}\textit{Chwyciłem mocno ster w obie ręce} & \textbf{G C F} \\
\hspace*{2em}\textit{I żeglowałem zasłuchany w fali śpiew} & \textbf{C G C} \\
& \\
Wspomnienia przemijają a w sercu żal & \textbf{a a F C} \\
Wciąż w łajbę się przemienia dziewczęcy czar & \textbf{a e F G C} \\
Jeżeli mi nie wierzysz to gnaj co tchu & \textbf{E a D7 G} \\
Tam z kei możesz ujrzeć coś z mego snu & \textbf{a e F G C} \\
& \\
\hspace*{2em}\textit{I ona taka w tej białej sukience...}  & \\
& \\
Nie wiem czy jeszcze kiedyś zobaczę ją & \textbf{a a F C} \\
Czy tylko w moich myślach jej oczy lśnią & \textbf{a e F G C} \\
Gdy pochylona ostro do wiatru szła & \textbf{E a D7 G} \\
Znowu się przeplatają obrazy dwa & \textbf{a e F G C} \\
& \\
\hspace*{2em}\textit{I ona taka w tej białej sukience...}  & \\
& \\
\end{longtable}
\clearpage

% --- Źródło: Bieszczady.tex ---
\section{\textbf{Bieszczady}}
\begin{longtable}{ll}
Tu w dolinach wstaje mgłą wilgotny dzień, & \textbf{e a} \\
Szczyty ogniem płoną, stoki kryje cień, & \textbf{D7 G H7} \\
Mokre rosą trawy wypatrują dnia, & \textbf{e a} \\
Ciepła, które pierwszy słońca promień da. & \textbf{D7 G H7} \\
& \\
\hspace*{2em}\textit{Cicho potok gada (nanana), gwarzy pośród skał} & \textbf{G C D G} \\
\hspace*{2em}\textit{O tym deszczu, co chmury trochę wody dał,} & \textbf{G C D G} \\
\hspace*{2em}\textit{\smash{Świerki} zapatrzone w horyzontu kres} & \textbf{G C D G} \\
\hspace*{2em}\textit{Głowy pragną wysoko, jak najwyżej wznieść.} & \textbf{G C D G} \\
& \\
Tęczą kwiatów barwny połoniny łan, & \textbf{e a} \\
Słońcem wypełniony jagodowy dzban, & \textbf{D7 G H7} \\
Pachnie świeżym sianem pokos pysznych traw, & \textbf{e a} \\
Owies dzwoneczkami cisza niebu gra. & \textbf{D7 G H7} \\
& \\
\hspace*{2em}\textit{ Cicho potok gada (nanana), gwarzy pośród skał...}  & \\
& \\
Serenadą świerszczy, kaskadami gwiazd & \textbf{e a} \\
Noc w zadumie kroczy mroku ścieląc płaszcz, & \textbf{D7 G H7} \\
Wielkim wozem księżyc rusza na swój szlak, & \textbf{e a} \\
Pozłocistym sierpem gasi lampy dnia. & \textbf{D7 G H7} \\
& \\
\hspace*{2em}\textit{ Cicho potok gada (nanana), gwarzy pośród skał...}  & \\
\end{longtable}
\clearpage

% --- Źródło: Bieszczadzki_Rock_n_Roll.tex ---
\section{Bieszczadzki Rock ‘n’ Roll}
\begin{longtable}{ll}
Miały już Bieszczady swoje tango, & \textbf{G} \\
miały także taniec zwany sambą. & \textbf{C7 G} \\
Miały też poleczkę prosto z pola, & \textbf{G} \\
lecz nie miały jeszcze rock`n`rolla. & \textbf{D C G} \\
& \\
\hspace*{2em}\textit{Bieszczady rock`n`roll,} & \textbf{G} \\
\hspace*{2em}\textit{połonina woogie-boogie,} & \textbf{G} \\
\hspace*{2em}\textit{gdy jesteś tylko sam,} & \textbf{G C7} \\
\hspace*{2em}\textit{dzień się staje taki długi.} & \textbf{C7 G} \\
\hspace*{2em}\textit{Gdy jesteś z nami wraz,} & \textbf{G D C} \\
\hspace*{2em}\textit{bardzo szybko mija czas.} & \textbf{C G} \\
& \\
Na stanicy błoto po kolana, & \textbf{G} \\
a deszcz pada od samego rana. & \textbf{C7 G} \\
Przemoczone wszystko do niteczki, & \textbf{G} \\
chciałbyś zmienić buty i majteczki. & \textbf{D C G} \\
& \\
\hspace*{2em}\textit{Bieszczady rock`n`roll...}  & \\
& \\
Na obozie od samego rana & \textbf{G} \\
druh komendant ciągle krzyczy na nas. & \textbf{C7 G} \\
Przemoczone buty i namioty, & \textbf{G} \\
wszystkim nam się zbiera na wymioty. & \textbf{D C G} \\
& \\
\hspace*{2em}\textit{Bieszczady rock`n`roll...}  & \\
& \\
\end{longtable}
\clearpage

% --- Źródło: Bieszczadzki_trakt.tex ---
\section{\textbf{Bieszczadzki trakt}}
\begin{longtable}{ll}
Kiedy nadejdzie czas, wabi nas ognia blask, & \textbf{G D C G} \\
Na polanie gdzie króluje zły (oboźny). & \textbf{D C G} \\
Gwiezdny pył w ogniu tym, łzy wyciśnie nam dym, & \textbf{G D C G} \\
Tańczą iskry z gwiazdami, a my & \textbf{D C G} \\
& \\
\hspace*{2em}\textit{Śpiewajmy wszyscy w ten radosny czas,} & \textbf{C D G} \\
\hspace*{2em}\textit{Śpiewajmy razem ilu jest tu nas.} & \textbf{C D e} \\
\hspace*{2em}\textit{Choć lata młode szybko płyną, wiemy że} & \textbf{C D G e} \\
\hspace*{2em}\textit{Nie starzejemy się*.} & \textbf{C D G} \\
& \\
W lesie gdzie licho śpi, ma przygoda swe drzwi. & \textbf{G D C G} \\
Chodźmy tam, gdzie na ścianie lasu lśnią, & \textbf{D C G} \\
oczy sów, wilcze kły, rykiem powietrze drży & \textbf{G D C G} \\
tylko gwiazdy przyjazne dziś są. & \textbf{D C G} \\
& \\
\hspace*{2em}\textit{Śpiewajmy wszyscy w ten radosny czas…}  & \\
& \\
Dorzuć do ognia drew, w górę niech płynie śpiew, & \textbf{G D C G} \\
wiatr poniesie go w wilgotny świat & \textbf{D C G} \\
Każdy z nas o tym wie, znowu spotkamy się, & \textbf{G D C G} \\
a połączy nas bieszczadzki trakt. & \textbf{D C G} \\
& \\
\hspace*{2em}\textit{Śpiewajmy wszyscy w ten radosny czas…}  & \\
& \\
\end{longtable}
\clearpage

% --- Źródło: Bieszczadzkie_Reggae.tex ---
\section{Bieszczadzkie Reggae}
\begin{longtable}{ll}
Porannej mgły snuje się dym & \textbf{d C d C} \\
Jutrzenki szal na stokach gór & \textbf{d C d C} \\
Nowy dzień budzi się, budzi się & \textbf{F C d C} \\
Melodię dnia już rosa gra & \textbf{d C d C} \\
& \\
\hspace*{2em}\textit{Reggae, bieszczadzkie reggae} & \textbf{d C d} \\
\hspace*{2em}\textit{Słońcem pachnące ma jagód smak} & \textbf{C d C d C} \\
\hspace*{2em}\textit{Reggae, bieszczadzkie reggae} & \textbf{d C d} \\
\hspace*{2em}\textit{Jak potok rwący przed siebie gna} & \textbf{C d C d C} \\
& \\
Połonin czar ma taką moc & \textbf{d C d C} \\
Że gdy je ujrzysz pierwszy raz & \textbf{d C d C} \\
Wrócić chcesz, wrócisz chcesz znów za rok & \textbf{F C d C} \\
Z poranną rosą czekać dnia & \textbf{d C d C} \\
& \\
\hspace*{2em}\textit{Reggae, bieszczadzkie reggae} & \textbf{d C d} \\
\hspace*{2em}\textit{Słońcem pachnące ma jagód smak} & \textbf{C d C d C} \\
\hspace*{2em}\textit{Reggae, bieszczadzkie reggae} & \textbf{d C d} \\
\hspace*{2em}\textit{Jak potok rwący przed siebie gna} & \textbf{C d C d C} \\
& \\
\end{longtable}
\clearpage

% --- Źródło: Bieszczadzkie_anioły.tex ---
\section{\textbf{Bieszczadzkie anioły}}
\begin{longtable}{ll}
Anioły są takie ciche, zwłaszcza te w Bieszczadach & \textbf{a G} \\
Gdy spotkasz takiego w górach, wiele z nim nie pogadasz & \textbf{a e} \\
Najwyżej na ucho ci powie, gdy będzie w dobrym humorze & \textbf{C G C F} \\
Że skrzydła nosi w plecaku, nawet przy dobrej pogodzie & \textbf{C G a e a} \\
& \\
Anioły są całe zielone, zwłaszcza te w Bieszczadach & \textbf{a G} \\
Łatwo w trawie się kryją, i w opuszczonych sadach & \textbf{a e} \\
W zielone grają ukradkiem, nawet karty mają zielone & \textbf{C G C F} \\
Zielone mają pojęcie, a nawet zielony kielonek & \textbf{C G a e a} \\
& \\
\hspace*{2em}\textit{Anioły bieszczadzkie, bieszczadzkie anioły} & \textbf{C G a} \\
\hspace*{2em}\textit{Dużo w was radości i dobrej pogody} & \textbf{C G a} \\
\hspace*{2em}\textit{Bieszczadzkie anioły, anioły bieszczadzkie} & \textbf{C G a} \\
\hspace*{2em}\textit{Gdy skrzydłem cię dotkną już jesteś ich bratem} & \textbf{C G a} \\
& \\
Anioły są całkiem samotne, zwłaszcza te w Bieszczadach & \textbf{a G} \\
W kapliczkach zimą drzemią, choć może im nie wypada & \textbf{a e} \\
Czasem taki anioł samotny, zapomni dokąd ma lecieć & \textbf{C G C F} \\
I wtedy całe Bieszczady, mają szaloną uciechę & \textbf{C G a e a} \\
& \\
\hspace*{2em}\textit{Anioły bieszczadzkie, bieszczadzkie anioły...}  & \\
& \\
Anioły są wiecznie ulotne, zwłaszcza te w Bieszczadach & \textbf{a G} \\
Nas też czasami nosi, po ich anielskich śladach & \textbf{a e} \\
One nam przyzwalają i skrzydłem wskazują drogę & \textbf{C G C F} \\
I wtedy w nas się zapala, wieczny bieszczadzki ogień & \textbf{C G a e a} \\
& \\
\hspace*{2em}\textit{Anioły bieszczadzkie, bieszczadzkie anioły...}  & \\
\end{longtable}
\clearpage

% --- Źródło: Bieszczadzkie_tango.tex ---
\section{Bieszczadzkie tango}
\begin{longtable}{ll}
Piękna jak Bieszczady młodość nasza jest, & \textbf{A E A} \\
Jak dobrze po pracy radość w sercu mieć. & \textbf{A E D A} \\
Gdy masz dzielne ręce, oczu młody blask, & \textbf{D A D A} \\
Razem z nami pracuj, śpiewaj z nami tak. & \textbf{A E A} \\
& \\
\hspace*{2em}\textit{Bieszczadzkie tango, połonina je zna,} & \textbf{A} \\
\hspace*{2em}\textit{Bieszczadzkie tango nuci potok las gra,} & \textbf{D A} \\
\hspace*{2em}\textit{Bieszczadzkie tango, tango tych gór,} & \textbf{D A} \\
\hspace*{2em}\textit{Bieszczadzkie tango, śpiewa echo do chmur.} & \textbf{E A} \\
& \\
Pragniesz iść w nieznane dalej bracie bież & \textbf{A E A} \\
W lasy pełne jagód no i zwierza też & \textbf{A E D A} \\
Gdy na szlaku spotkasz ślady wielkich łap & \textbf{D A D A} \\
Nie krzycz nie uciekaj lecz zaśpiewaj tak & \textbf{A E A} \\
& \\
\hspace*{2em}\textit{Bieszczadzkie tango, połonina je zna,} & \textbf{A} \\
\hspace*{2em}\textit{Bieszczadzkie tango nuci potok las gra,} & \textbf{D A} \\
\hspace*{2em}\textit{Bieszczadzkie tango, tango tych gór,} & \textbf{D A} \\
\hspace*{2em}\textit{Bieszczadzkie tango, śpiewa echo do chmur.} & \textbf{E A} \\
& \\
& \\
\end{longtable}
\clearpage

% --- Źródło: Bitwa.tex ---
\section{\textbf{Bitwa}}
\begin{longtable}{ll}
Okręt nasz wpłynął w mgłę i fregaty dwie & \textbf{e D C a} \\
Popłynęły naszym kursem by nie zgubić się. & \textbf{e D G H} \\
Potem szkwał wypchnął nas poza mleczny pas & \textbf{e D C a} \\
I nikt wtedy nie przypuszczał, że fregaty śmierć nam niosą. & \textbf{e D G H} \\
& \\
\hspace*{2em}\textit{Ciepła krew poleje się strugami,} & \textbf{G D e h} \\
\hspace*{2em}\textit{Wygra ten kto utrzyma ship.} & \textbf{C D e} \\
\hspace*{2em}\textit{W huku dział ktoś przykryje się falami,} & \textbf{G D e h} \\
\hspace*{2em}\textit{Jak da Bóg ocalimy bryg.} & \textbf{C D e} \\
& \\
Nagły huk w uszach grał i już atak trwał, & \textbf{e D C a} \\
To fregaty uzbrojone rzędem w setkę dział. & \textbf{e D G H} \\
Czarny dym spowił nas, przyszedł śmierci czas. & \textbf{e D C a} \\
Krzyk i lament mych kamratów, przerywany ogniem katów. & \textbf{e D G H} \\
& \\
\hspace*{2em}\textit{Ciepła krew poleje się strugami...}  & \\
& \\
Pocisk nasz trafił w maszt, usłyszałem trzask, & \textbf{e D C a} \\
To sterburtę rozwaliła jedna z naszych salw. & \textbf{e D G H} \\
„Żagiel staw” krzyknął ktoś, znów piratów złość, & \textbf{e D C a} \\
Bo od rufy nam powiało, a piratom w mordę wiało. & \textbf{e D G H} \\
& \\
\hspace*{2em}\textit{Ciepła krew poleje się strugami...}  & \\
& \\
Z fregat dwóch tylko ta pierwsza w pogoń szła, & \textbf{e D C a} \\
Wnet abordaż rozpoczęli gdy dopadli nas. & \textbf{e D G H} \\
Szyper ich dziury dwie zrobił w swoim dnie, & \textbf{e D C a} \\
Nie pomogło to psubratom, reszta z rei zwisa za to. & \textbf{e D G H} \\
& \\
\hspace*{2em}\textit{Ciepła krew poleje się strugami...}  & \\
& \\
Po dziś dzień tamtą mgłę i fregaty dwie, & \textbf{e D C a} \\
Kiedy noc zamyka oczy widzę w swoim śnie. & \textbf{e D G H} \\
Tamci co śpią na dnie, uśmiechają się, & \textbf{e D C a} \\
Że ich straszną śmierć pomścili bracia, którzy zwyciężyli. & \textbf{e D G H} \\
& \\
\hspace*{2em}\textit{Ciepła krew poleje się strugami...}  & \\
\end{longtable}
\clearpage

% --- Źródło: Bolero.tex ---
\section{Bolero}
\begin{longtable}{ll}
W małym miasteczku & \textbf{a} \\
Gdzieś na krańcach Hiszpanii & \textbf{G} \\
Stary krawiec Augusto & \textbf{F} \\
Szył bolera najtaniej & \textbf{E} \\
I czy pan był bogaty & \textbf{a} \\
Pan był biedny czy kmieć & \textbf{G} \\
Każdy takie bolero & \textbf{F} \\
Chciał mieć & \textbf{E} \\
& \\
\hspace*{2em}\textit{To bolero} & \textbf{a} \\
\hspace*{2em}\textit{Dla bogatych cavaleros} & \textbf{G} \\
\hspace*{2em}\textit{W tym bolero będziesz senior} & \textbf{G} \\
\hspace*{2em}\textit{Prezentował się jak struś} & \textbf{E} \\
\hspace*{2em}\textit{Na bolero cavaleros ty się skuś} & \textbf{F E} \\
& \\
Jakie chcesz pan bolero & \textbf{a} \\
Białe, czarne, różowe & \textbf{G} \\
Zapinane od przodu & \textbf{F} \\
Czy wkładane przez głowę & \textbf{E} \\
Z przodu czarne guziki & \textbf{a} \\
Z tyłu patka czy nie & \textbf{G} \\
Jakie chcesz pan bolero OLE! & \textbf{F E} \\
& \\
\hspace*{2em}\textit{To bolero…}  & \\
& \\
Na corridę gdy pójdziesz & \textbf{a} \\
Tym bolero okryty & \textbf{G} \\
O biust karter zabije & \textbf{F} \\
Serce twej seniority & \textbf{E} \\
No i ona zemdlona & \textbf{a} \\
Na twe łono bez sił & \textbf{G} \\
Padnie, szepcąc „Amigo! & \textbf{F} \\
Kto to szył?” & \textbf{E} \\
& \\
\hspace*{2em}\textit{To bolero…}  & \\
\end{longtable}
\clearpage

% --- Źródło: Carpe_diem.tex ---
\section{\textbf{Carpe diem}}
\begin{longtable}{ll}
Wcześnie rano dzisiaj wstałem & \textbf{e G} \\
Zimną wodą zmyłem twarz & \textbf{D e} \\
Gdzieś zerwałem jagód parę & \textbf{G} \\
Wokół mnie szumiący las & \textbf{D e} \\
Założyłem lekki plecak & \textbf{D} \\
Wziąłem w dłonie szary płaszcz & \textbf{e} \\
Powiedziałem sobie: stary & \textbf{G} \\
Komu w drogę, temu czas! & \textbf{D e} \\
& \\
\hspace*{2em}\textit{Dla mnie dzień, dla mnie noc}  & \\
\hspace*{2em}\textit{Ja po prostu kocham życie}  & \\
\hspace*{2em}\textit{Życie płynie jak we śnie}  & \\
\hspace*{2em}\textit{Raz jest dobrze, a raz źle}  & \\
\hspace*{2em}\textit{Słońce siostrą, księżyc bratem}  & \\
\hspace*{2em}\textit{Droga - azymutem życia}  & \\
\hspace*{2em}\textit{Życie chwilą, Ty miłością}  & \\
\hspace*{2em}\textit{Dla mnie to największy skarb}  & \\
& \\
Nocowałem gdzie się dało  & \\
Do snu grywał tylko wiatr  & \\
Ptaki mnie budziły rano  & \\
Bym wyruszał dalej w świat  & \\
Ktoś zapytał dokąd idę  & \\
Czy mam jakoś własny szlak  & \\
Carpe diem mu odrzekłem  & \\
I dodałem i jeszcze tak  & \\
& \\
\hspace*{2em}\textit{Dla mnie dzień, dla mnie noc}  & \\
\hspace*{2em}\textit{Ja po prostu kocham życie}  & \\
\hspace*{2em}\textit{Życie płynie jak we śnie}  & \\
\hspace*{2em}\textit{Raz jest dobrze, a raz źle}  & \\
\hspace*{2em}\textit{Słońce siostrą, księżyc bratem}  & \\
\hspace*{2em}\textit{Droga - azymutem życia}  & \\
\hspace*{2em}\textit{Życie chwilą, Ty miłością}  & \\
\hspace*{2em}\textit{Dla mnie to największy skarb}  & \\
\end{longtable}
\clearpage

% --- Źródło: Cerkiew_w_ogniu.tex ---
\section{Cerkiew w ogniu}
\begin{longtable}{ll}
\hspace*{2em}\textit{Choć przetrwała gorący czas} & \textbf{gis Fis gis} \\
\hspace*{2em}\textit{Cerkiew w ogniu stanęła jesienią} & \textbf{H E Fis} \\
\hspace*{2em}\textit{Nie boskiej chwały to blask} & \textbf{E Fis gis} \\
\hspace*{2em}\textit{Nie wiara w niej wstała płomieniem} & \textbf{E Fis gis} \\
\hspace*{2em}\textit{Nie boskiej chwały to blask} & \textbf{E Fis gis} \\
\hspace*{2em}\textit{Nie wiara w niej wstała płomieniem} & \textbf{E Fis gis} \\
& \textbf{cis A fis H} \\
Jak sam Bóg w mroku lśni ikonostas & \textbf{cis A} \\
Z nocy nagle rozbłysły ikony & \textbf{fis gis} \\
Złotem spływa z ich oczu rozpacz & \textbf{cis A} \\
Nim się żywot świętych dokona & \textbf{fis gis} \\
& \\
Bizantyjskie czernieją twarze & \textbf{cis E} \\
Otwierają oczy szeroko & \textbf{A H} \\
Święty Michał przegrywa z Szatanem & \textbf{cis gis} \\
Święty Jerzy pada przed smokiem & \textbf{fis gis} \\
& \\
\hspace*{2em}\textit{Choć przetrwała gorący czas...}  & \\
& \\
Bóg wszechmocny patrzy bezsilnie & \textbf{cis A} \\
Jakby tylko z obrazu był Bogiem & \textbf{fis gis} \\
Nie z kadzideł dym kryje mandylion & \textbf{cis gis} \\
To gniewu trawi go ogień & \textbf{fis gis} \\
& \\
Iskry z oczu aniołów lecą & \textbf{cis E} \\
Załamują prorocy dłonie & \textbf{A H} \\
Na wszystkie świętości złorzecząc & \textbf{cis A} \\
A niech was piekło pochłonie & \textbf{fis gis} \\
& \\
\hspace*{2em}\textit{Choć przetrwała gorący czas...}  & \\
& \\
W proch obrócą się czarne zgliszcza & \textbf{gis Fis gis} \\
Lecz co rok drzewa wokół goreją & \textbf{H E Fis} \\
Pożar przecież pamięci nie zniszczy & \textbf{E Fis gis} \\
O tej cerkwi co zgasła jesienią & \textbf{E Fis gis} \\
Pożar przecież pamięci nie zniszczy & \textbf{E Fis gis} \\
O tej cerkwi co zgasła jesienią & \textbf{A H Fis} \\
& \\
\end{longtable}
\clearpage

% --- Źródło: Chodź_pomaluj_mój_świat.tex ---
\section{Chodź, pomaluj mój świat}
\vspace{-\baselineskip}
\textit{Dwa plus jeden}\\
\begin{longtable}{ll}
Piszesz mi w liście, że kiedy pada, & \textbf{a d} \\
Kiedy nasturcje na deszczu mokną, & \textbf{G a} \\
Siadasz przy stole, wyjmujesz farby & \textbf{C G} \\
I kolorowe otwierasz okno. & \textbf{D E a} \\
& \\
Trawy i drzewa są takie szare, & \textbf{a d} \\
Barwę popiołu przybrały nieba. & \textbf{G a} \\
W ciszy tak smutno, szepce zegarek & \textbf{C G} \\
O czasie, co mi go nie potrzeba. & \textbf{D E a} \\
& \\
\hspace*{2em}\textit{Więc chodź, pomaluj mój świat} & \textbf{C d} \\
\hspace*{2em}\textit{Na żółto i na niebiesko,} & \textbf{F C} \\
\hspace*{2em}\textit{Niech na niebie stanie tęcza} & \textbf{C d} \\
\hspace*{2em}\textit{Malowana twoją kredką.} & \textbf{F G} \\
\hspace*{2em}\textit{Więc chodź, pomaluj mi życie,} & \textbf{C d} \\
\hspace*{2em}\textit{Niech świat mój się zarumieni,} & \textbf{F C} \\
\hspace*{2em}\textit{Niech mi zalśni w pełnym słońcu,} & \textbf{C d} \\
\hspace*{2em}\textit{Kolorami całej ziemi.} & \textbf{F G} \\
& \\
Za siódmą górą, za siódmą rzeką,  & \\
Twoje sny zamieniasz na pejzaże.  & \\
Niebem się wlecze wyblakłe słońce,  & \\
Oświetla ludzkie wyblakłe twarze.  & \\
& \\
\hspace*{2em}\textit{Więc chodź, pomaluj mój świat}  & \\
\hspace*{2em}\textit{Na żółto i na niebiesko,}  & \\
\hspace*{2em}\textit{Niech na niebie stanie tęcza}  & \\
\hspace*{2em}\textit{Malowana twoją kredką.}  & \\
\hspace*{2em}\textit{Więc chodź, pomaluj mi życie,}  & \\
\hspace*{2em}\textit{Niech świat mój się zarumieni,}  & \\
\hspace*{2em}\textit{Niech mi zalśni w pełnym słońcu,}  & \\
\hspace*{2em}\textit{Kolorami całej ziemi.}  & \\
\end{longtable}
\clearpage

% --- Źródło: Chłopcy_z_Botany_Bay.tex ---
\section{Chłopcy z Botany Bay}
\begin{longtable}{ll}
Już nad Hornem zapada noc & \textbf{h A7 h} \\
Wiatr na żaglach położył się & \textbf{h A7 D} \\
A tam jeszcze korsarze na Botany Bay & \textbf{G A7 h A h} \\
Upychają zdobycze swe & \textbf{G a h} \\
& \\
Jolly Roger na maszcie już śpi  & \\
Jutro przyjdzie z Hiszpanem się bić  & \\
A korsarze znużeni na Botany Bay  & \\
Za zwycięstwo dziś będą swe pić  & \\
& \\
Śniady Clark puchar wznosi do ust  & \\
„Bracia niech toast idzie na dno!”  & \\
Tylko Johnny nie pije bo kilka mil stąd  & \\
Otuliło złe morze go  & \\
& \\
Nie podnosi kielicha do ust  & \\
Zawsze on tu najgłośniej się śmiał  & \\
Mistrz fechtunku z Florencji ugodził go  & \\
Już nie będzie za szoty się brał  & \\
& \\
W starym porcie zapłacze Margot  & \\
Jej kochanek nie wróci już  & \\
Za dezercję do panny na kei w Brisbane  & \\
Oddać musiał swą głowę pod nóż  & \\
& \\
Tak niewielu zostało dziś ich  & \\
Resztę zabrał Neptun pod dach  & \\
Choć na ustach wciąż uśmiech to w sercach lód  & \\
W kuflu miesza się rum i strach  & \\
& \\
To ostatni chyba już rejs  & \\
Cios sztyletem lub kula w pierś  & \\
Bóg na szkuner w niebiosach zabierze ich  & \\
Wszystkich chłopców z Botany Bay  & \\
& \\
Już nad Hornem zapada noc  & \\
Wiatr na żaglach położył się  & \\
A tam jeszcze korsarze na Botany Bay  & \\
Upychają zdobycze swe  & \\
& \\
\end{longtable}
\clearpage

% --- Źródło: Ciągle_pada.tex ---
\section{Ciągle pada}
\vspace{-\baselineskip}
\textit{Czerwone Gitary}\\
\begin{longtable}{ll}
Ciągle pada, asfalt ulic jest dziś śliski jak brzuch ryby & \textbf{G e} \\
Mokre niebo się opuszcza coraz niżej & \textbf{C} \\
Żeby przejrzeć się w marszczonej deszczem wodzie & \textbf{D7} \\
& \\
A ja? A ja chodzę desperacko i na przekór wszystkim moknę & \textbf{G e} \\
Patrzę w niebo chwytam w usta deszczu krople & \textbf{C} \\
Patrzą na mnie rozpłaszczone twarze w oknie to nic & \textbf{D7} \\
& \\
Ciągle pada, ludzie biegną bo się bardzo boją deszczu & \textbf{e C} \\
Stoją w bramie ledwie się w tej bramie mieszcząc & \textbf{A7} \\
Ludzie skaczą przez kałuże na swej drodze & \textbf{D7} \\
& \\
A ja? A ja chodzę nie przejmując się ulewą ani śpiesząc & \textbf{e C} \\
Czując jak mi krople deszczu usta pieszczą & \textbf{A7} \\
Ze złożonym parasolem idę pieszo o tak & \textbf{D7} \\
& \\
Ciągle pada, alejkami już strumienie wody płyną & \textbf{G e} \\
Jakaś para się okryła peleryną & \textbf{C} \\
Przyglądając się jak mokną bzy w ogrodzie & \textbf{D7} \\
& \\
A ja? A ja chodzę w strugach wody ale z czołem podniesionym & \textbf{G e} \\
Żadna siła mnie nie zmusza i nie goni & \textbf{C} \\
Idę niby zwiastun burzy z kwiatkiem w dłoni o tak & \textbf{D7} \\
& \\
Ciągle pada, nagle ogniem otworzyły się niebiosa & \textbf{e C} \\
Potem zaczął deszcz ulewny sieć z ukosa & \textbf{A7} \\
Liście klonu się zatrzęsły w wielkiej trwodze & \textbf{D7} \\
& \\
A ja? A ja chodzę i niestraszna mi wichura ani ulewa & \textbf{e C} \\
Ani piorun który trafił obok drzewa & \textbf{A7} \\
Słucham wiatru który wciąż inaczej śpiewa & \textbf{D7} \\
& \\
Ciągle pada, nagle ogniem otworzyły się niebiosa & \textbf{e C} \\
Potem zaczął deszcz ulewny siać z ukosa & \textbf{A7} \\
Liście klonu się zatrzęsły w wielkiej trwodze & \textbf{D7} \\
& \\
A ja? A ja chodzę i niestraszna mi wichura ani ulewa & \textbf{e C} \\
Ani piorun który trafił obok drzewa & \textbf{A7} \\
Słucham wiatru który wciąż inaczej śpiewa & \textbf{D7} \\
& \\
A ja? A ja chodzę desperacko i na przekór wszystkim moknę & \textbf{G e} \\
Patrzę w niebo chwytam w usta deszczu krople & \textbf{C} \\
Patrzą na mnie rozpłaszczone twarze w oknie to nic & \textbf{D7} \\
\end{longtable}
\clearpage

% --- Źródło: Czarny_blues_o_czwartej_nad_ranem.tex ---
\section{\textbf{Czarny blues o czwartej nad ranem}}
\vspace{-\baselineskip}
\textit{Stare Dobre Małżeństwo}\\
\begin{longtable}{ll}
Czwarta nad ranem - może sen przyjdzie & \textbf{A cis} \\
Może mnie odwiedzisz & \textbf{D A} \\
Czwarta nad ranem - może sen przyjdzie & \textbf{E fis} \\
Może mnie odwiedzisz & \textbf{D E A} \\
& \\
Czemu cię nie ma na odległość ręki? & \textbf{A E} \\
Czemu mówimy do siebie listami? & \textbf{fis cis} \\
Gdy ci to śpiewam, u mnie pełnia lata & \textbf{D A} \\
Gdy to usłyszysz, będzie środek zimy & \textbf{D E} \\
& \\
Czemu się budzę o czwartej nad ranem & \textbf{A E} \\
I włosy twoje próbuję ugłaskać & \textbf{fis cis} \\
Lecz nigdzie nie ma twoich włosów & \textbf{D A} \\
Jest tylko blada nocna lampka & \textbf{D E} \\
Łysa śpiewaczka & \textbf{fis} \\
& \\
Śpiewamy bluesa, bo czwarta nad ranem & \textbf{A E} \\
Tak cicho, żeby nie zbudzić sąsiadów & \textbf{fis cis} \\
Czajnik z gwizdkiem świruje na gazie & \textbf{D A} \\
Myślałby kto, że rodem z Manhattanu & \textbf{D E} \\
& \\
Czwarta nad ranem - może sen przyjdzie & \textbf{A cis} \\
Może mnie odwiedzisz & \textbf{D A} \\
Czwarta nad ranem - może sen przyjdzie & \textbf{E fis} \\
Może mnie odwiedzisz & \textbf{D E A} \\
& \\
Herbata czarna myśli rozjaśnia & \textbf{A E} \\
A list twój sam się czyta & \textbf{fis cis} \\
\smash{Że} można go śpiewać - za oknem mruczą bluesa & \textbf{D A} \\
Topole z Krupniczej & \textbf{D E} \\
& \\
I jeszcze strażak wszedł na solo & \textbf{A E} \\
Ten z Mariackiej Wieży & \textbf{fis cis} \\
Jego trąbka jak księżyc  błyszczy nad topolą & \textbf{D A} \\
Nigdzie się jej nie spieszy & \textbf{D E} \\
& \\
Już piąta - może sen przyjdzie  & \\
Może mnie odwiedzisz  & \\
Już piąta - może sen przyjdzie  & \\
Może mnie odwiedzisz  & \\
\end{longtable}
\clearpage

% --- Źródło: Czarny_chleb_i_czarna_kawa.tex ---
\section{\textbf{Czarny chleb i czarna kawa}}
\vspace{-\baselineskip}
\textit{Strachy na Lachy}\\
\begin{longtable}{ll}
Jedzie pociąg, złe wagony, & \textbf{a C G a} \\
Do więzienia wiozą mnie.  & \\
Świat ma tylko cztery strony,  & \\
A w tym świecie nie ma mnie.  & \\
& \\
Gdy swe oczy otworzyłem  & \\
Wielki żal ogarnął mnie.  & \\
Po policzkach łzy spłynęły,  & \\
Zrozumiałem wtedy, że...  & \\
& \\
\hspace*{2em}\textit{Czarny chleb i czarna kawa,}  & \\
\hspace*{2em}\textit{Opętani samotnością,}  & \\
\hspace*{2em}\textit{Myślą swą szukają szczęścia,}  & \\
\hspace*{2em}\textit{Które zwie się wolnością... (x2)}  & \\
& \\
Młodsza siostra zapytała:  & \\
„Mamo, gdzie braciszek mój?”  & \\
Brat Twój w ciemnej celi siedzi!  & \\
Odsiaduje wyrok swój.  & \\
& \\
\hspace*{2em}\textit{Czarny chleb i czarna kawa...}  & \\
& \\
Wtem do celi klawisz wpada,  & \\
I zaczyna więźnia bić.  & \\
Młody więzień na twarz pada,  & \\
Serce mu przestaje bić.  & \\
& \\
I nadejdzie chwila błoga  & \\
Śmierć zabierze oddech mój,  & \\
Moje ciało stąd wyniosą  & \\
A pod celą będą znów  & \\
& \\
Inny czarny chleb i czarna kawa,  & \\
Opętani samotnością,  & \\
Myślą swą szukają szczęścia,  & \\
Które zwie się wolnością...  & \\
& \\
\hspace*{2em}\textit{Czarny chleb i czarna kawa...}  & \\
\end{longtable}
\clearpage

% --- Źródło: Czeska_szanta.tex ---
\section{Czeska szanta}
\begin{longtable}{ll}
Śmiali się ze mnie sąsiedzi i śmiała się ze mnie matka & \textbf{d C d F G A} \\
Że chcę być marynarzem, co pływa na czeskich statkach & \textbf{B F E d} \\
Lecz uczy tego historia, śpiewają o tym rybitwy & \textbf{d C d F G A} \\
Że czeska marynarka nie przegrała żadnej bitwy! & \textbf{B F E d} \\
& \\
\hspace*{2em}\textit{Niech czeski naród powstanie, chłopaki liny w dłoń!} & \textbf{A d A} \\
\hspace*{2em}\textit{Jesteśmy morskim krajem, mówimy do siebie ahoj! || x2} & \textbf{B F F E d} \\
& \\
Znalazłem czeską załogę, brakuje czeskiego portu  & \\
Nie chciałem by moi ludzie byli gorszego sortu  & \\
Jest ze mną Karel i Jożin, myśleliśmy cały rok  & \\
Aż wpadliśmy na pomysł – nazywa się "suchy dok"  & \\
& \\
\hspace*{2em}\textit{Niech czeski naród powstanie, chłopaki liny w dłoń...  || x2}  & \\
& \\
Oprócz czeskiego portu przyda się czeskie morze  & \\
Ale tu już sam Pan Bóg nam raczej nie pomoże  & \\
Ale nam to nie przeszkadza i zaczyna się przygoda  & \\
Za statek będzie nam służyć napędzana żaglem skoda  & \\
& \\
\hspace*{2em}\textit{Niech czeski naród powstanie, chłopaki liny w dłoń...  || x2}  & \\
& \\
Tak z wiatrem do Bałtyku, pędzimy ile sił  & \\
Chciałem być marynarzem, stworzyłem czeski film  & \\
I biorąc za przykład Putina, wciągamy banderę w przestworza  & \\
Tak czeska republika zyskała dostęp do morza!  & \\
& \\
\hspace*{2em}\textit{Niech czeski naród powstanie, chłopaki liny w dłoń...  || x2}  & \\
\end{longtable}
\clearpage

% --- Źródło: Deesis.tex ---
\section{Deesis}
\vspace{-\baselineskip}
\textit{Na Bani}\\
\begin{longtable}{ll}
Tam las się pochyla prastarym chojarem & \textbf{a a7 a} \\
Nad ciemnym strumieniem niebo się zamyka & \textbf{a a7 a} \\
Tam buk rosochatym zakrywa konarem & \textbf{a a7 a} \\
Mogiły, których nawet wiatr unika & \textbf{F d G a} \\
& \\
Dla oczu ukryty- niedostrzegły wzrokiem & \textbf{a a7 a} \\
Jedyny ślad wymarłej sadyby & \textbf{B} \\
Gasnąca drożyna spleciona z potokiem & \textbf{a a7 a} \\
Jakby nie była- odeszła jak gdyby & \textbf{B} \\
& \\
Porosły drzewa, gdzie umarły chaty & \textbf{a G D a} \\
Zamknęły się cerkwi widmowe wierzeje & \textbf{e F d E7} \\
Opuścił sioło Chrystus Pantokrator & \textbf{a G D a} \\
Zgięty w pół krzyż samotnie niszczeje & \textbf{e F d E7} \\
& \\
\hspace*{2em}\textit{Na przekór podnieśmy ku niebu ektenie} & \textbf{a C} \\
\hspace*{2em}\textit{Na próżno błagajmy błogosławienia} & \textbf{G D} \\
\hspace*{2em}\textit{Za Łemkowynę módlmy się daremnie} & \textbf{a C} \\
\hspace*{2em}\textit{Może dostąpimy jeszcze Przebóstwienia} & \textbf{G F} \\
& \\
\hspace*{2em}\textit{Mgły poruszymy świętym wozduchem} & \textbf{a C} \\
\hspace*{2em}\textit{Chramowe ikony podniesiemy z pyłu} & \textbf{G D} \\
\hspace*{2em}\textit{Obudzimy chóry naszej wiary kruchej} & \textbf{a C} \\
\hspace*{2em}\textit{Hospody pomyłuj! Hospody pomyłuj! Hospody pomyłuj!} & \textbf{G D F} \\
& \\
Dlaczego przestałem być Rusnakom bratem & \textbf{a a7 a} \\
Chociaż wzorem dla nas te same hermeneje & \textbf{a a7 a} \\
Dokąd uleciały cheruby skrzydlate & \textbf{a a7 a} \\
Czemu płomień w oczach już duszy nie grzeje? & \textbf{F d G} \\
& \\
Słupy dymu z nagła podparły ciężkie chmury & \textbf{a a7 a} \\
Gdy od nieba dzielił nas tylko ikonostas & \textbf{B} \\
Umilkły w bólu niewzruszone góry & \textbf{a a7 a} \\
Jedyne, które miały tu pozostać & \textbf{B} \\
& \\
Ku czyjej chwale wzniosły się pochodnie & \textbf{a G D a} \\
Całopalne ofiary dla którego Boga & \textbf{e F d E7} \\
Jacyż to święci tej krwi byli głodni & \textbf{a G D a} \\
Wjakim obrządku ojców kto pochował? & \textbf{e F d E7} \\
& \\
\hspace*{2em}\textit{Na przekór podnieśmy ku niebu ektenie... }  & \\
& \\
& \\
\end{longtable}
\newpage
\begin{longtable}{ll}
Na plecach wysoko ponieśmy ektenie & \textbf{a C} \\
Chociaż niegodniśmy błogosławienia & \textbf{G D} \\
Odprawmy na szczytach pokutę zmęczeniem & \textbf{a C} \\
Może doprosimy się tym przebaczenia & \textbf{G F} \\
& \\
Mgły poruszymy świętym wozduchem & \textbf{a C} \\
Chramowe ikony podniesiemy z pyłu & \textbf{G D} \\
Obudzimy chóry naszej wiary kruchej & \textbf{a C} \\
Hospody pomyłuj! Hospody pomyłuj! Hospody pomyłuj! & \textbf{G D F} \\
\end{longtable}
\clearpage

% --- Źródło: Diabeł_i_Anioł.tex ---
\section{\textbf{Diabeł i Anioł}}
\begin{longtable}{ll}
Idzie diabeł ścieżką krzywą pełen myśli złych & \textbf{d B C F} \\
Nie pożyczył mu na piwo, nie pożyczył nikt. & \textbf{d B C d} \\
Słońce pali go od rana, wiatr gorący dmucha, & \textbf{d B C F} \\
Diabeł się z pragnienia słania w ten piekielny upał… & \textbf{d B C d} \\
& \\
Idzie anioł wśród zieleni, dobrze mu się wiedzie. & \textbf{F F C F A7} \\
Pełno drobnych ma w kieszeni i przyjaciół wszędzie… & \textbf{d B C d} \\
Nagle przystanęli obaj na drodze pod śliwką. & \textbf{d B C F} \\
Zobaczyli, że im browar wyszedł naprzeciwko… & \textbf{d B C d} \\
& \\
Nie ma szczęścia na tym świecie, ni sprawiedliwości: & \textbf{F F C F A7} \\
Anioł pije piwko trzecie, diabeł mu zazdrości… & \textbf{d B C d} \\
Postaw kufel, rzecze diabeł, Bóg ci wynagrodzi, & \textbf{d B C F} \\
My artyści w taki upał żyć musimy w zgodzie… & \textbf{d B C d} \\
& \\
Na to anioł zatrzepotał skrzydeł pióropuszem & \textbf{F F C F A7} \\
powiada: Dam ci dychę w zamian za twą duszę… & \textbf{d B C d} \\
Musiał diabeł aniołowi swoją duszę sprzedać & \textbf{d B C F} \\
I stworzyli sobie piekło z odrobiną nieba. & \textbf{d B C d} \\
\end{longtable}
\clearpage

% --- Źródło: Dni_których_nie_znamy.tex ---
\section{\textbf{Dni, których nie znamy}}
\vspace{-\baselineskip}
\textit{Marek Grechuta}\\
\begin{longtable}{ll}
Tyle było dni, do utraty sił, & \textbf{a C G C} \\
Do utraty tchu, tyle było chwil, & \textbf{d A C G} \\
Gdy żałujesz tych, z których nie masz nic, & \textbf{a C G C} \\
Jedno warto znać, jedno tylko wiedz: & \textbf{d A C G} \\
& \\
\hspace*{2em}\textit{Że ważne są tylko te dni, których jeszcze nie znamy,} & \textbf{F d G} \\
\hspace*{2em}\textit{Ważnych jest kilka tych chwil, tych, na które czekamy,} & \textbf{a F G C} \\
\hspace*{2em}\textit{Ważne są tylko te dni, których jeszcze nie znamy,} & \textbf{F d G} \\
\hspace*{2em}\textit{Ważnych jest kilka tych chwil, tych na które czekamy.} & \textbf{a F G C} \\
& \\
Pewien znany ktoś, kto miał dom i sad,  & \\
Zgubił nagle sens i w złe kręgi wpadł,  & \\
Choć majątek prysł, on nie stoczył się,  & \\
Wytłumaczyć umiał sobie wtedy właśnie, że...  & \\
& \\
\hspace*{2em}\textit{Że, ważne są tylko te dni, których jeszcze nie znamy,}  & \\
\hspace*{2em}\textit{Ważnych jest kilka tych chwil, tych na które czekamy,}  & \\
\hspace*{2em}\textit{Ważne są tylko te dni, których jeszcze nie znamy,}  & \\
\hspace*{2em}\textit{Ważnych jest kilka tych chwil, tych na które czekamy}  & \\
& \\
Jak rozpoznać ludzi, których już nie znamy?  & \\
Jak pozbierać myśli z tych nieposkładanych?  & \\
Jak oddzielić nagle rozum swój od serca?  & \\
Jak usłyszysz siebie w takim szumnym scherzu?  & \\
& \\
Jak rozpoznać ludzi, których już nie znamy?  & \\
Jak pozbierać myśli z tych nieposkładanych?  & \\
Jak odnaleźć nagle radość i nadzieję?  & \\
Odpowiedzi szukaj, czasu jest niewiele.  & \\
& \\
\hspace*{2em}\textit{Ważne są tylko te dni, których jeszcze nie znamy,}  & \\
\hspace*{2em}\textit{Ważnych jest kilka tych chwil, tych na które czekamy,}  & \\
\hspace*{2em}\textit{Ważne są tylko te dni, których jeszcze nie znamy,}  & \\
\hspace*{2em}\textit{Ważnych jest kilka tych chwil, na które czekamy.}  & \\
& \\
Na na na na na na na na na...  & \\
\end{longtable}
\clearpage

% --- Źródło: Dwudziesty_lutego.tex ---
\section{24 lutego}
\begin{longtable}{ll}
To 24 był lutego, & \textbf{C} \\
poranna zrzedła mgła & \textbf{G} \\
A wyszło z niej siedem uzbrojonych kryp, & \textbf{a} \\
Turecki niosły znak & \textbf{F G a} \\
No i…  & \\
& \\
\hspace*{2em}\textit{Znów bijatyka, no} & \textbf{C} \\
\hspace*{2em}\textit{i znów bijatyka, no} & \textbf{C} \\
\hspace*{2em}\textit{i bijatyka cały dzień.} & \textbf{C G} \\
\hspace*{2em}\textit{I porąbany dzień i porąbany łeb,} & \textbf{a C} \\
\hspace*{2em}\textit{razem Bracia, aż po zmierzch, no} & \textbf{F G a} \\
& \\
I już pierwszy zbliża się do burta,  & \\
zwie się Goździk Li  & \\
Z Algieru Pasza wysłał go,  & \\
żeby nam upuścił krwi  & \\
No i…  & \\
& \\
\hspace*{2em}\textit{Znów bijatyka, no...}  & \\
& \\
Już następny zbliża się do burta,  & \\
zwie się Róży Pąk  & \\
Plunęliśmy ze wszystkich luf,  & \\
bardzo szybko szedł na dno  & \\
No i…  & \\
& \\
\hspace*{2em}\textit{Znów bijatyka, no...}  & \\
& \\
W naszych rękach dwa i dwa na dnie,  & \\
cała reszta zwiała gdzieś  & \\
A jeden z nich zabraliśmy,  & \\
aż na Starej Anglii brzeg  & \\
No i…  & \\
& \\
\hspace*{2em}\textit{Znów bijatyka, no...}  & \\
\end{longtable}
\clearpage

% --- Źródło: Dym_z_jałowca.tex ---
\section{\textbf{Dym z jałowca}}
\begin{longtable}{ll}
Dym z jałowca łzy wyciska & \textbf{C a} \\
Noc się coraz wyżej wznosi & \textbf{F G} \\
Strumień srebrną falą błyska & \textbf{C a} \\
Czyjś głos w leśnej ciszy prosi & \textbf{F G} \\
& \\
\hspace*{2em}\textit{Żeby była taka noc,} & \textbf{C a} \\
\hspace*{2em}\textit{kiedy myśli mkną do Boga} & \textbf{F G} \\
\hspace*{2em}\textit{Żeby były takie dni,} & \textbf{C a} \\
\hspace*{2em}\textit{że się przy nim ciągle jest} & \textbf{F G} \\
\hspace*{2em}\textit{Żeby był przy tobie ktoś,} & \textbf{C a} \\
\hspace*{2em}\textit{kogo nie zniechęci droga} & \textbf{F G} \\
\hspace*{2em}\textit{Abyś plecak swoich dni,} & \textbf{C a} \\
\hspace*{2em}\textit{stromą ścieżką umiał nieść} & \textbf{F G} \\
\hspace*{2em}\textit{do Boga.} & \textbf{C} \\
& \\
Ogrzej dłonie przy ognisku  & \\
Płomień twarz ci zarumieni  & \\
Usiądziemy razem blisko  & \\
Jedną myślą połączeni.  & \\
& \\
\hspace*{2em}\textit{Żeby była taka noc...}  & \\
& \\
Tuż pod szczytem się zatrzymaj &  \\
Spójrz jak gwiazdy w dół spadają  & \\
Spójrz jak droga kosodrzewina  & \\
Góry wraz z tobą wołają  & \\
& \\
\hspace*{2em}\textit{Żeby była taka noc...}  & \\
& \\
\end{longtable}
\clearpage

% --- Źródło: Dzieci.tex ---
\section{Dzieci}
\vspace{-\baselineskip}
\textit{Elektryczne Gitary}\\
\begin{longtable}{ll}
Dzieci wesoło wybiegły ze szkoły  & \\
Zapaliły papierosy, wyciągnęły flaszki  & \\
Chodnik zapluły, ludzi przepędziły  & \\
Siedzą na ławeczkach i ryczą do siebie  & \\
& \\
\hspace*{2em}\textit{Wszyscy mamy źle w głowach, że żyjemy}  & \\
\hspace*{2em}\textit{Hej, hej, la, la, la, la, hej, hej, hej, hej}  & \\
\hspace*{2em}\textit{Wszyscy mamy źle w głowach, że żyjemy}  & \\
\hspace*{2em}\textit{Hej, hej, la, la, la, la, hej, hej, hej, hej}  & \\
& \\
Tony papieru, tomy analiz  & \\
Genialne myśli, tłumy na sali  & \\
Godziny modlitw, lata nauki  & \\
Przysięgi, plany, podpisy, druki  & \\
& \\
\hspace*{2em}\textit{Wszyscy mamy źle w głowach, że żyjemy...}  & \\
& \\
Wzorce, przykłady, szlachetne zabiegi  & \\
Łańcuchy dłoni, zwarte szeregi  & \\
Warstwy tradycji, wieki kultury  & \\
Tydzień dobroci, ręce do góry  & \\
& \\
\hspace*{2em}\textit{Wszyscy mamy źle w głowach, że żyjemy...}  & \\
& \\
Dzieci wesoło wybiegły ze szkoły  & \\
Zapaliły papierosy, wyciągnęły flaszki  & \\
Chodnik zapluły, ludzi przepędziły  & \\
Siedzą na ławeczkach i ryczą do siebie  & \\
& \\
\hspace*{2em}\textit{Wszyscy mamy źle w głowach, że żyjemy...}  & \\
& \\
\end{longtable}
\clearpage

% --- Źródło: Dziesięć_w_skali_Beauforta.tex ---
\section{10 w skali Beauforta}
\begin{longtable}{ll}
Kołysał nas zachodni wiatr, & \textbf{a d} \\
Brzeg gdzieś za rufą został. & \textbf{E7 a} \\
I nagle ktoś jak papier zbladł: & \textbf{d a} \\
Sztorm idzie, panie bosman! & \textbf{H7 E7} \\
& \\
\hspace*{2em}\textit{A bosman tylko zapiął płaszcz} & \textbf{F C F C} \\
\hspace*{2em}\textit{I zaklął: - Ech, do czorta!} & \textbf{F E7 a} \\
\hspace*{2em}\textit{Nie daję łajbie żadnych szans!} & \textbf{F G a E7 a} \\
\hspace*{2em}\textit{Dziesięć w skali Beauforta!} & \textbf{F E7 a} \\
& \\
Z zasłony ołowianych chmur & \textbf{a d} \\
Ulewa spadła nagle. & \textbf{E7 a} \\
Rzucało nami w górę, w dół, & \textbf{d a} \\
I fala zmyła żagle. & \textbf{H7 E7} \\
& \\
\hspace*{2em}\textit{A bosman tylko zapiął płaszcz...}  & \\
& \\
Gdzie został ciepły, cichy kąt & \textbf{a d} \\
I brzegu kształt znajomy? & \textbf{E7 a} \\
Zasnuły mgły daleki ląd & \textbf{d a} \\
Dokładnie, z każdej strony. & \textbf{H7 E7} \\
& \\
\hspace*{2em}\textit{A bosman tylko zapiął płaszcz...}  & \\
& \\
O pokład znów uderzył deszcz & \textbf{a d} \\
I padał już do rana. & \textbf{E7 a} \\
Piekielnie ciężki to był rejs, & \textbf{d a} \\
Szczególnie dla bosmana. & \textbf{H7 E7} \\
& \\
\hspace*{2em}\textit{A bosman tylko zapiął płaszcz} & \textbf{F C F C} \\
\hspace*{2em}\textit{I zaklął: - Ech, do czorta!} & \textbf{F E7 a} \\
\hspace*{2em}\textit{Przedziwne czasem sny się ma!} & \textbf{F G a E7 a} \\
\hspace*{2em}\textit{Dziesięć w skali Beauforta!} & \textbf{F E7 a} \\
\hspace*{2em}\textit{Dziesięć w skali Beauforta!} & \textbf{F E7 a} \\
\hspace*{2em}\textit{Dziesięć w skali Beauforta!}  & \\
\end{longtable}
\clearpage

% --- Źródło: Dziewczyna_rumiankowa.tex ---
\section{Dziewczyna rumiankowa}
\begin{longtable}{ll}
Przez okno wbiegłaś tu,  & \\
prosto do mojego snu, &  \\
Gdzie rośnie tyle traw, &  \\
żeby mogły ukryć nas. &  \\
Dziewczyna rumiankowa &  \\
wśród białych kwiatów tańczy,  & \\
O mnie mało sobie dba,  & \\
nie wie jeszcze, że to ja.  & \\
& \\
\hspace*{2em}\textit{Ile jabłek na jabłoni,}  & \\
\hspace*{2em}\textit{tyle lat (tyle lat) cię będę gonił}  & \\
\hspace*{2em}\textit{W mysiej dziurze, czy na chmurze}  & \\
\hspace*{2em}\textit{nie ukryjesz się na dłużej.}  & \\
\hspace*{2em}\textit{Ile jabłek na jabłoni,}  & \\
\hspace*{2em}\textit{tyle lat (tyle lat) cię będę gonił}  & \\
\hspace*{2em}\textit{W mysiej dziurze, czy na chmurze}  & \\
\hspace*{2em}\textit{zawsze znajdę Cię}  & \\
& \\
Z łąk jasnych zbiegłaś tu,  & \\
gdzie sosnowy szumi bór,  & \\
Gdzie rośnie taki las,  & \\
w którym nikt nie znajdzie nas.  & \\
Herbaty zaparzone,  & \\
nad kubkiem twoje oczy,  & \\
Co nie widzą jeszcze mnie  & \\
i nie wiedzą, że to my.  & \\
& \\
\hspace*{2em}\textit{Ile jabłek na jabłoni...}  & \\
& \\
& \\
\end{longtable}
\clearpage

% --- Źródło: Dziewczyna_z_granatem.tex ---
\section{Dziewczyna z granatem}
\begin{longtable}{ll}
Żoliborz, Ochota, Wola & \textbf{a e} \\
Różaną po schodkach w dół & \textbf{a F} \\
Dziewczęta spod Parasola & \textbf{a C} \\
Żołnierski włożyły strój & \textbf{F E} \\
Błękitna chustka marzenie & \textbf{a G} \\
W koszyku granaty dwa & \textbf{F F} \\
I zdjęcie chłopaka w kieszeni & \textbf{a E} \\
Pamiątka letniego dnia & \textbf{F G} \\
& \\
\hspace*{2em}\textit{Dziewczyno z granatem w ręce} & \textbf{C d} \\
\hspace*{2em}\textit{Dziewczyno w zielonej sukience!} & \textbf{F E} \\
\hspace*{2em}\textit{Warszawa się broni, walka trwa} & \textbf{C d} \\
\hspace*{2em}\textit{O wolność, o honor, o kraj!} & \textbf{C G} \\
\hspace*{2em}\textit{Do broni dziewczyno kochana!} & \textbf{C d} \\
\hspace*{2em}\textit{Do broni chłopaku mój!} & \textbf{F G} \\
\hspace*{2em}\textit{Wnet wolna będzie Warszawa!} & \textbf{C d} \\
\hspace*{2em}\textit{Do broni, po wolność, na bój!} & \textbf{F F E} \\
& \\
Powstańcy na Starówce  & \\
Tam poczta broni się  & \\
List pisze sanitariuszka  & \\
Walczymy. Kocham cię!  & \\
I tylko czasem dłonie  & \\
Słowa jakiegoś strzęp  & \\
On z batalionu „Zośka”, a ona kto to wie?  & \\
& \\
\hspace*{2em}\textit{Dziewczyno z granatem w ręce...}  & \\
& \\
Zza rogu seria z kaemu  & \\
Ciszę przerywa: ta, ta...  & \\
Niewolę rozrywa granat  & \\
I z Błyskawicy strzał  & \\
Warszawa jeszcze się broni  & \\
Weź miła dwie Filipinki  & \\
Zawleczka jest tutaj zobacz  & \\
Podobna do twojej szminki!  & \\
& \\
\hspace*{2em}\textit{Dziewczyno z granatem w ręce...}  & \\
\end{longtable}
\clearpage

% --- Źródło: Dziewięćset_mil.tex ---
\section{900 mil}
\begin{longtable}{ll}
Tyle już minęło dni, & \textbf{a C} \\
Czas wysuszył z oczu łzy, & \textbf{d a} \\
Żaden list nie czeka na mnie już & \textbf{F C a} \\
& \\
\hspace*{2em}\textit{Gdyby pociąg szybciej biegł} & \textbf{a C} \\
\hspace*{2em}\textit{Byłbym w domu jeszcze dziś,} & \textbf{d a} \\
\hspace*{2em}\textit{Jeszcze 900 mil, tam gdzie mój dom.} & \textbf{C E a} \\
& \\
Oddam wszystko to, co mam, & \textbf{a C} \\
Oddam wam pierścionek swój & \textbf{d a} \\
I walizkę swoją oddam wam. & \textbf{F C a} \\
& \\
\hspace*{2em}\textit{Gdyby pociąg szybciej biegł} & \textbf{a C} \\
\hspace*{2em}\textit{Byłbym w domu jeszcze dziś,} & \textbf{d a} \\
\hspace*{2em}\textit{Jeszcze 900 mil, tam gdzie mój dom.} & \textbf{C E a} \\
& \\
Pociąg, który wiezie mnie, & \textbf{a C} \\
Ma wagonów chyba sto, & \textbf{d a} \\
Stukot kół już słychać na sto mil. & \textbf{F C a} \\
& \\
Moja miła czeka mnie & \textbf{a C} \\
Żebym wrócił do niej znów & \textbf{d a} \\
Jeszcze 900 mil tam, gdzie mój dom & \textbf{C E a} \\
& \\
Gdy dziewczyna powie - nie, & \textbf{a C} \\
Nie wyruszę nigdzie już. & \textbf{d a} \\
Wrócę do rodzinnych moich stron & \textbf{F C a} \\
& \\
\hspace*{2em}\textit{Gdyby pociąg szybciej biegł} & \textbf{a C} \\
\hspace*{2em}\textit{Byłbym w domu jeszcze dziś,} & \textbf{d a} \\
\hspace*{2em}\textit{Jeszcze 900 mil, tam gdzie mój dom.} & \textbf{C E a} \\
& \\
Już z wagonu widać ja, & \textbf{a C} \\
Już z peronu widać ją, & \textbf{d a} \\
Stoi, trzyma w dłoniach bukiet róż & \textbf{F C a} \\
& \\
Lecz nie dla mnie kwiaty te, & \textbf{a C} \\
Inny chłopak dostał je, & \textbf{d a} \\
jeszcze 900 mil tam, gdzie mój dom & \textbf{C E a} \\
& \\
\end{longtable}
\clearpage

% --- Źródło: Dziki_włóczęga.tex ---
\section{Dziki włóczęga}
\begin{longtable}{ll}
Byłem dzikim włóczęgą przez tak wiele dni,  & \\
Przepuściłem pieniądze na dziwki i gin.  & \\
Dzisiaj wracam do domu, pełny złota mam trzos  & \\
I zapomnieć chcę wreszcie jak podły był los.  & \\
& \\
\hspace*{2em}\textit{Już nie wrócę na morze,}  & \\
\hspace*{2em}\textit{Nigdy więcej o nie.}  & \\
\hspace*{2em}\textit{Wreszcie koniec włóczęgi,}  & \\
\hspace*{2em}\textit{Na pewno to wiem.}  & \\
& \\
I poszedłem do baru, gdzie bywałem nie raz,  & \\
Powiedziałem barmance, że forsy mi brak.  & \\
Poprosiłem o kredyt, powiedziała idź precz,  & \\
Mogę mieć tu stu takich na skinienie co dzień.  & \\
& \\
\hspace*{2em}\textit{Już nie wrócę na morze...}  & \\
& \\
Gdy błysnąłem dziesiątką, to skoczyła jak kot  & \\
I butelkę najlepszą przysunęła pod nos.  & \\
Powiedziała zalotnie, co chcesz mogę ci dać,  & \\
Ja jej na to ty flądro, spadaj znam inny bar.  & \\
& \\
\hspace*{2em}\textit{Już nie wrócę na morze...}  & \\
& \\
Gdy stanąłem przed domem, przez otwarte drzwi,  & \\
Zobaczyłem rodziców, czy przebaczą mi.  & \\
Matka pierwsza spostrzegła jak w sieni wciąż tkwię.  & \\
Zobaczyłem ich radość i przyrzekłem, że …  & \\
& \\
\hspace*{2em}\textit{Już nie wrócę na morze...}  & \\
& \\
\end{longtable}
\clearpage

% --- Źródło: Dłonie_plecione.tex ---
\section{Dłonie plecione}
\begin{longtable}{ll}
Idę sama, cały świat przede mną & \textbf{e D C9 G} \\
Kocham Cię to ja, nikt inny wiesz  & \\
To ja i tylko ja.  & \\
& \\
\hspace*{2em}\textit{Dłonie plecione}  & \\
\hspace*{2em}\textit{Maków szkarłatem}  & \\
\hspace*{2em}\textit{Ja byłam zakochana}  & \\
\hspace*{2em}\textit{A Ty pachniałeś latem}  & \\
\hspace*{2em}\textit{Płomień gwiazd}  & \\
\hspace*{2em}\textit{Blaskiem usypiał nas!}  & \\
& \\
Twoje imię w uszach brzmi  & \\
Jeśli myślisz, że zapomnę, mylisz się  & \\
Choć czas upływa nigdy nic  & \\
nie zmieni się.  & \\
& \\
\hspace*{2em}\textit{Dłonie plecione...}  & \\
& \\
Tylko Ty znikasz kiedy widzę Cię  & \\
Gdy idę do Ciebie  & \\
Udajesz, że nie znasz mnie.  & \\
& \\
\hspace*{2em}\textit{Dłonie plecione...}  & \\
& \\
\end{longtable}
\clearpage

% --- Źródło: Długość_dźwięku_w_samotności.tex ---
\section{\textbf{Długość dźwięku w samotności}}
\vspace{-\baselineskip}
\textit{Myslovitz}\\
\begin{longtable}{ll}
\hspace*{2em}\textit{I nawet kiedy będę sam} & \textbf{F d / C a} \\
\hspace*{2em}\textit{Nie zmienię się to nie mój świat} & \textbf{a G / e D} \\
\hspace*{2em}\textit{Przede mną droga którą znam} & \textbf{F d / C a} \\
\hspace*{2em}\textit{Którą ja wybrałem sam} & \textbf{a G / e D} \\
& \\
Tak zawsze genialny & \textbf{B F / F C} \\
Idealny muszę być & \textbf{d C / a G} \\
I muszę chcieć super luz i już & \textbf{B F / F C} \\
Setki bzdur i już to nie ja & \textbf{a C / a G} \\
& \\
\hspace*{2em}\textit{I nawet kiedy będę sam...}  & \\
& \\
Wiesz lubię wieczory & \textbf{B F} \\
Lubię się schować na jakiś czas & \textbf{d C} \\
I jakoś tak nienaturalnie & \textbf{B F} \\
Trochę przesadnie pobyć sam & \textbf{a C} \\
& \\
Wejść na drzewo i patrzeć w niebo & \textbf{B F} \\
Tak zwyczajnie tylko że & \textbf{d C} \\
Tutaj też wiem kolejny raz & \textbf{B F} \\
Nie mam szans być kim chcę & \textbf{a C} \\
& \\
\hspace*{2em}\textit{I nawet kiedy będę sam...}  & \\
& \\
Noc a nocą gdy nie śpię & \textbf{B F} \\
Wychodzę choć nie chcę spojrzeć na & \textbf{d C} \\
Chemiczny świat pachnący szarością & \textbf{B F} \\
Z papieru miłością gdzie ty i ja & \textbf{a C} \\
& \\
I jeszcze ktoś, nie wiem kto & \textbf{B F} \\
Chciałby tak przez kilka lat & \textbf{d C} \\
Zbyt zachłannie i trochę przesadnie & \textbf{B F} \\
Pobyć chwilę sam chyba go znam & \textbf{a C} \\
& \\
\hspace*{2em}\textit{I nawet kiedy będę sam...}  & \\
\end{longtable}
\clearpage

% --- Źródło: Emeryt.tex ---
\section{\textbf{Emeryt}}
\vspace{-\baselineskip}
\textit{EKT Gdynia}\\
\begin{longtable}{ll}
Leżysz wtulona w pościel, coś cichutko mruczysz przez sen & \textbf{d a} \\
Łóżko szerokie a pościel świeża za oknem prawie dzień & \textbf{F A} \\
A jeszcze niedawno koja w niej pachnący rybą koc & \textbf{d C} \\
Fale bijące o pokład i bosmana zdarty głos & \textbf{F A} \\
& \\
\hspace*{2em}\textit{To wszystko było, minęło zostało tylko wspomnienie} & \textbf{d C G d} \\
\hspace*{2em}\textit{Już nie poczuję wibracji pokładu gdy kable grają} & \textbf{d C G d} \\
\hspace*{2em}\textit{Już tylko dom i ogródek i tak aż do śmierci} & \textbf{d C G d} \\
\hspace*{2em}\textit{A przecież stare żaglowce po morzach jeszcze pływają} & \textbf{d C G d} \\
& \\
Nie gniewaj się kochanie, że trudno ze mną żyć  & \\
Że zapomniałem kupić mleko i gary zmyć  & \\
Lecz jeszcze niedawno okręt mym drugim domem był  & \\
Tam nie stało się w kolejkach, tam nie było miejsca dla złych  & \\
& \\
\hspace*{2em}\textit{To wszystko było, minęło zostało tylko wspomnienie...}  & \\
& \\
Upłynie sporo czasu nim przyzwyczaję się  & \\
Czterdzieści lat na morzu zamknięte w jeden dzień  & \\
Skąd lekarz może wiedzieć, że za morzem tęskno mi  & \\
Że duszę się na lądzie, śni mi się pokład pełen ryb  & \\
& \\
\hspace*{2em}\textit{To wszystko było, minęło zostało tylko wspomnienie...}  & \\
& \\
Wiem, masz do mnie żal, mieliśmy do przyjaciół iść  & \\
Spotkałem kolegę z rejsu, on w morze idzie dziś  & \\
Siedziałem potem na kei, ze łzami patrzyłem na port  & \\
Jeszcze przyjdzie taki dzień kiedy opuszczę go...  & \\
& \\
\hspace*{2em}\textit{To wszystko było, minęło zostało tylko wspomnienie...}  & \\
\end{longtable}
\clearpage

% --- Źródło: Epitafium_dla_Majora_Ognia.tex ---
\section{\textbf{Epitafium dla Majora Ognia}}
\begin{longtable}{ll}
Jeśli umrzeć mam za ciebie… & \textbf{C a} \\
Jeśli sztandar zwinąć mam… & \textbf{C a} \\
Na każdego przecież przyjdzie kiedyś pora. & \textbf{F G a} \\
Srebrny orzeł z mojej czapki & \textbf{C a} \\
Na urwiskach w sercu Tatr & \textbf{C a} \\
Znajdzie gniazdo & \textbf{G} \\
& \\
\hspace*{2em}\textit{Ty mnie ukryj moja ziemio podhalańska} & \textbf{F G a} \\
\hspace*{2em}\textit{Górski lesie kołysz mnie do snu} & \textbf{F G a} \\
\hspace*{2em}\textit{Tyle razy mnie chroniła Ręka Pańska} & \textbf{F G a} \\
\hspace*{2em}\textit{Dziś mnie do raportu wezwał Bóg.} & \textbf{F G a} \\
& \\
Wilki mają swoje ścieżki & \textbf{C a} \\
Podążałem szlakiem wilczym & \textbf{C a} \\
Śmierć ścigała mnie po lasach & \textbf{F G} \\
Jak pies gończy & \textbf{a} \\
Przystanęła nad potokiem & \textbf{F G} \\
Krótki błysk nad górskim stokiem & \textbf{C F} \\
Wola Boża - moja walka dziś się kończy. & \textbf{C G a} \\
& \\
\hspace*{2em}\textit{Ty mnie ukryj moja ziemio podhalańska}  & \\
\hspace*{2em}\textit{Górski lesie kołysz mnie do snu}  & \\
\hspace*{2em}\textit{Tyle razy mnie chroniła Ręka Pańska}  & \\
\hspace*{2em}\textit{Dziś mnie do raportu wezwał Bóg.}  & \\
& \\
Polski orzeł srebrnopióry & \textbf{F G} \\
W nasze dusze wbił pazury & \textbf{C F} \\
I legendę partyzancką ponad szczyty wzniósł & \textbf{C G E a} \\
Na Podhalu, pod Turbaczem & \textbf{F G} \\
Partyzancka wierzba płacze & \textbf{C F} \\
Zgasł już „Ogień” & \textbf{C G} \\
Ale pamięć po nim wciąż się tli & \textbf{E a} \\
& \\
\hspace*{2em}\textit{Ty mnie ukryj moja ziemio podhalańska}  & \\
\hspace*{2em}\textit{Górski lesie kołysz mnie do snu}  & \\
\hspace*{2em}\textit{Tyle razy mnie chroniła Ręka Pańska}  & \\
\hspace*{2em}\textit{Dziś mnie do raportu wezwał Bóg.}  & \\
\end{longtable}
\clearpage

% --- Źródło: Epitafium_dla_Majora_Ponurego.tex ---
\section{\textbf{Epitafium dla Majora Ponurego}}
\begin{longtable}{ll}
Mgła schodzi z lasu Panie Majorze & \textbf{e D} \\
Wiatr się po lesie chaszcze jak ptak & \textbf{C G D} \\
Już się szkopy nie tułają po borze & \textbf{e h} \\
Niejednego przez nas trafił szlag. & \textbf{C D} \\
Jutro do wsi pewnie zajdziemy & \textbf{e D} \\
Pies nie szczeknie - przecież my swoi & \textbf{C G D} \\
U mej matuli cokolwiek zjemy & \textbf{e h} \\
Potem śpiewaniem do snu ukoi... & \textbf{C D} \\
& \\
\hspace*{2em}\textit{I dobrze odpoczniem nim odejdziem w góry} & \textbf{C D} \\
\hspace*{2em}\textit{lecz co Pan Major taki ponury? (x2)} & \textbf{e D} \\
& \\
Do diabła ze śmiercią Panie Majorze  & \\
Pan szedł z nią razem w 39-tym  & \\
Potem trza było się z wojskiem łączyć  & \\
I miecze ostrzyć daleko za morzem  & \\
Myśmy czekali, bo wodza brakło  & \\
Lichy to zwierz co walczy bez oka  & \\
Wieści przysłali słowo się rzekło  & \\
I biały orzeł z góry spikował...  & \\
& \\
\hspace*{2em}\textit{I w piersi wroga wbił swe pazury}  & \\
\hspace*{2em}\textit{Lecz co Pan Major taki ponury? (x2)}  & \\
& \\
To nie był taki zwyczajny bój  & \\
Lufa się zgrzała jak klucze od piekła  & \\
Mocno się wrzynał w kieszeni nabój  & \\
I każda chwila jak wieczność się wlekła  & \\
Strasznie Pan dostał Panie Majorze  & \\
Jak mi Bóg miły nie mogło być gorzej  & \\
Krew się przelała przez głębokie rany  & \\
Archanioł Michał otworzył bramy  & \\
& \\
\hspace*{2em}\textit{Pozdrówcie ode mnie Świętokrzyskie Góry}  & \\
\hspace*{2em}\textit{szepnął i skonał Major Ponury (x2)}  & \\
& \\
\hspace*{2em}\textit{Skonał i odszedł odnaleźć swe góry}  & \\
\hspace*{2em}\textit{Serca bohater Major Ponury... (x2)}  & \\
\end{longtable}
\clearpage

% --- Źródło: Epitafium_dla_S_Jesienina.tex ---
\section{Epitafium dla S. Jesienina}
\vspace{-\baselineskip}
\textit{Jacek Kaczmarski, Przemysław Gintroski, Zbigniew Łapiński}\\
\begin{longtable}{ll}
Wściekła się wielka niedźwiedzica & \textbf{d C} \\
I lecą z pyska płaty piany & \textbf{A7 d} \\
Samotny szczeniak do księżyca & \textbf{d C} \\
Zanosi jazgot obłąkany & \textbf{A7 d} \\
& \\
Po mlecznej drodze wóz się toczy & \textbf{c Bb} \\
Turkocze za nim piąte koło & \textbf{G7 c} \\
Sponad zodiaku o północy  & \\
Ze mną ptak ryba i dziwoląg  & \\
& \\
I Ruś cerkiewna Ruś dawnej wiary  & \\
Szumi pomoriem w riazańskiej duszy  & \\
Huczy po karczmach krwawym pożarem  & \\
I Twoje brzozy pali Sergiuszu  & \\
& \\
Brzozo wędrowna czemu się śnisz  & \\
I tak cię zrąbią dla mnie na krzyż  & \\
Chato rodzinna płyń tam gdzie kres  & \\
Niech się dzwoneczek śmieje do łez  & \\
& \\
Ryczy lew ranny ponad głową  & \\
Bliźnięta płyną rzeką modrą  & \\
Byk galopuje łąką płową  & \\
I pręży się na wadze skorpion  & \\
& \\
Tętni przez łąki koziorożec  & \\
Strzelec go tropi nieustannie  & \\
I płynie wodnik po jeziorze  & \\
Po utopionej płacze pannie  & \\
& \\
I Ruś jak panna niech płacze po nim  & \\
Święta Łagoda zejdzie do ziemi  & \\
Na bruku Moskwy klon oszroniony  & \\
Biblijny prorok Sergiusz Jesienin  & \\
& \\
Riazańska matko skąd w oczach łzy  & \\
Karczemne szczęście samogon dym  & \\
Moskwo karczemna płyń za mną płyń  & \\
Pokochał zodiak riazański syn  & \\
& \\
Otwórzcie mi stróże anieli  & \\
Błękitne podwoje dni  & \\
O północy anioł w bieli  & \\
Z moim wiernym koniem znikł  & \\
& \\
Koń mój Bogu niepotrzebny  & \\
Koń mój siła ma i moc  & \\
Słyszę gryzie łańcuch srebrny  & \\
Rży żałośnie w głuchą noc  & \\
& \\
Widzę pędzi wśród zamieci  & \\
Targa gniewnie gruby sznur  & \\
Jak z miesiąca z niego leci  & \\
Sierść bułana w kłęby chmur  & \\
& \\
Tutaj Jesienin w najśmieszniejszej z gier  & \\
Wybiegał w błękit zza karczemnej Moskwy  & \\
Riazańską łąką zakwitł w Angleterre  & \\
Sen pożegnalny ostatni oktostych  & \\
& \\
Pomódl się pomódl za Jesienina  & \\
Przeżegnaj wszystkie dalekie drogi  & \\
Wychyl wieczorem czareczkę wina  & \\
Ostatnie grosze rozdaj ubogim  & \\
& \\
Nie pragnął krzyku odpoczynek snił  & \\
W sekundzie się przeżywał od nowa  & \\
A potem długo waliła do drzwi  & \\
Zniecierpliwiona służba hotelowa  & \\
& \\
Podróżną sakwę zarzuć na ramię  & \\
Wyjdź na gościniec do bramy nieba  & \\
Słyszysz jak tętni przez śnieżną zamieć  & \\
Księżyc na nowiu bułany źrebak  & \\
& \\
Weszli krzyknęli a jeden się bał  & \\
Bo spoza okien milczące niebiosa  & \\
A tam na stole gdzie Jesienin stał  & \\
Snuł się powoli dymek z papierosa  & \\
& \\
Słyszysz jak woła każdego ranka  & \\
Wiatr myszkujący po połoninach  & \\
Leci w dal z wiatrem rżenie bułanka  & \\
Pomódl się pomódl za Jesienina  & \\
\end{longtable}
\clearpage

% --- Źródło: Epitafium_dla_Wysockiego.tex ---
\section{Epitafium dla Wysockiego}
\vspace{-\baselineskip}
\textit{Jacek Kaczmarski}\\
\begin{longtable}{ll}
To moja droga z piekła do piekła  & \\
W dół na złamanie karku gnam!  & \\
Nikt mnie nie trzyma, nikt nie prześwietla  & \\
Nie zrywa mostów, nie stawia bram!  & \\
& \\
Po grani! Po grani!  & \\
Nad przepaścią bez łańcuchów, bez wahania!  & \\
Tu na trzeźwo diabli wezmą  & \\
Zdradzi mnie rozsądek - drań  & \\
W wilczy dół wspomnienia zmienią  & \\
Ostrą grań!  & \\
& \\
Po grani! Po grani! Po grani!  & \\
Tu mi drogi nie zastąpią pokonani!  & \\
Tylko łapią mnie za nogi  & \\
Krzyczą - nie idź! Krzyczą - stań!  & \\
Ci, co w pół stanęli drogi  & \\
I zębami, pazurami kruszą grań!  & \\
& \\
To moja droga z piekła do piekła  & \\
W przepaść na łeb na szyję skok!  & \\
„Boskiej Komedii” nowy przekład  & \\
I w pierwszy krąg piekła mój pierwszy krok!  & \\
& \\
Tu do mnie! Tu do mnie!  & \\
Ruda chwyta mnie dziewczyna swymi dłońmi  & \\
I do końskiej grzywy wiąże  & \\
Szarpię grzywę - rumak rży!  & \\
Ona - co ci jest mój książę? -  & \\
Szepce mi...  & \\
& \\
Do piekła! Do piekła! Do piekła!  & \\
Nie mam czasu na przejażdżki wiedźmo wściekła!  & \\
- Nie wiesz ty co cię tam czeka -  & \\
Mówi sine tocząc łzy  & \\
- Piekło też jest dla człowieka!  & \\
Nie strasz, nie kuś i odchodząc zabierz sny!  & \\
& \\
To moja droga z piekła do piekła  & \\
Wokół postaci bladych tłok  & \\
Koń mnie nad nimi unosi z lekka  & \\
I w drugi krąg kieruje krok!  & \\
\end{longtable}
\newpage
\begin{longtable}{ll}
Zesłani! Zesłani!  & \\
Naznaczeni, potępieni i sprzedani!  & \\
Co robicie w piekła sztolniach  & \\
Brodząc w błocie, depcząc lód!  & \\
Czy śmierć daje ludzi wolnych  & \\
Znów pod knut!?  & \\
- To nie tak! To nie tak! To nie tak!  & \\
Nie użalaj się nad nami - tyś poeta!  & \\
Myśmy raju znieść nie mogli  & \\
Tu nasz żywioł, tu nasz dom!  & \\
Tu nie wejdą ludzie podli  & \\
Tutaj żaden nas nie zdziesiątkuje grom!  & \\
& \\
- Pani bagien, mokradeł i śnieżnych pól  & \\
Rozpal w łaźni kamienie na biel!  & \\
Z ciał rozgrzanych niech się wytopi ból  & \\
Tatuaże weźmiemy na cel!  & \\
Bo na sercu, po lewej, tam Stalin drży  & \\
Pot zalewa mu oczy i wąs!  & \\
Jego profil specjalnie tam kłuli my  & \\
Żeby słyszał jak serca się rwą!  & \\
& \\
To moja droga z piekła do piekła  & \\
Lampy naftowe wabią wzrok  & \\
Podmiejska chata, mała izdebka  & \\
I w trzeci krąg kieruję krok:  & \\
& \\
- Wchodź śmiało! Wchodź śmiało!  & \\
Nie wiem jak ci trafić tutaj się udało!  & \\
Ot jak raz samowar kipi, pij herbatę  & \\
Synu, pij!  & \\
Samogonu z nami wypij!  & \\
Zdrowy żyj!  & \\
& \\
Nam znośnie! Nam znośnie!  & \\
Tak żyjemy niewidocznie i bezgłośnie!  & \\
Pożyjemy i pomrzemy  & \\
Nie usłyszy o nas świat  & \\
A po śmierci wypijemy  & \\
Za przeżytych w dobrej wierze parę lat!  & \\
& \\
To moja droga z piekła do piekła  & \\
Miasto a w Mieście przy bloku blok  & \\
Wciągam powietrze i chwiejny z lekka  & \\
Już w czwarty krąg kieruję krok!  & \\
& \\
Do cyrku! Do cyrku! Do kina!  & \\
Telewizor włączyć - bajka się zaczyna!  & \\
Mama w sklepie, tata w barze  & \\
Syn z pepeszy tnie aż gra!  & \\
Na pionierskiej chuście marzeń  & \\
Gwiazdę ma!  & \\
& \\
Na mecze! Na mecze! Na wiece!  & \\
Swoje znać, nie rzucać w oczy się bezpiece!  & \\
Sąsiad - owszem, wypić można  & \\
Lecz to sąsiad, brat - to brat  & \\
Jak świat światem do ostrożnych  & \\
Zwykł należeć i uśmiechać się ten świat!  & \\
& \\
To moja droga z piekła do piekła  & \\
Na scenie Hamlet, skłuty bok  & \\
Z którego właśnie krew wyciekła -  & \\
To w piąty krąg kolejny krok!  & \\
& \\
O Matko! O Matko!  & \\
Jakże mogłaś jemu sprzedać się tak łatwo!  & \\
Wszak on męża twego zabił  & \\
Zgładzi mnie, splugawi tron  & \\
Zniszczy Danię, lud ograbi  & \\
Bijcie w dzwon!  & \\
& \\
Na trwogę! Na trwogę! Na trwogę!  & \\
Nie wybieraj między żądzą swą a Bogiem!  & \\
Póki czas naprawić błędy  & \\
Matko, nie rób tego - stój!  & \\
Cenzor z dziewiątego rzędu:  & \\
- Nie, w tej formie to nie może wcale pójść!  & \\
& \\
To moja droga z piekła do piekła  & \\
Wódka i piwo, koniak, grog  & \\
Najlepszych z nas ostatnia Mekka  & \\
I w szósty krąg kolejny krok!  & \\
& \\
Na górze! Na górze! Na górze!  & \\
Chciałoby się żyć najpełniej i najdłużej!  & \\
O to warto się postarać!  & \\
To jest nałóg, zrozum to!  & \\
Tam się żyje jak za cara!  & \\
I ot co!  & \\
& \\
Na dole, na dole, na dole  & \\
Szklanka wódki i razowy chleb na stole!  & \\
I my wszyscy tam - i tutaj  & \\
Tłum rozdartych dusz na pół  & \\
Po huśtawce mdłość i smutek  & \\
Choćbyś nawet co dzień walił głową w stół!  & \\
& \\
To moja droga z piekła do piekła  & \\
Z wolna zapada nade mną mrok  & \\
Więc biesów szpaler szlak mi oświetla  & \\
Bo w siódmy krąg kieruję krok!  & \\
Tam milczą i siedzą  & \\
I na moją twarz nie spojrzą - wszystko wiedzą  & \\
Siedzą, ale nie gadają  & \\
Mętny wzrok spod powiek lśni  & \\
Żują coś, bo im wypadły  & \\
Dawno kły!  & \\
& \\
Więc stoję! Więc stoję! Więc stoję!  & \\
A przed nimi leży w teczce życie moje!  & \\
Nie czytają, nie pytają -  & \\
Milczą, siedzą - kaszle ktoś  & \\
A za oknem werble grają -  & \\
Znów parada, święto albo jeszcze coś...  & \\
& \\
I pojąłem co chcą ze mną zrobić tu  & \\
I za gardło porywa mnie strach!  & \\
Koń mój zniknął a wy siedmiu kręgów tłum  & \\
Macie w uszach i w oczach piach!  & \\
Po mnie nikt nie wyciągnie okrutnych rąk  & \\
Mnie nie będą katować i strzyc!  & \\
Dla mnie mają tu jeszcze ósmy krąg!  & \\
Ósmy krąg, w którym nie ma już nic  & \\
& \\
Pamiętajcie wy o mnie co sił! Co sił!  & \\
Choć przemknąłem przed wami jak cień!  & \\
Palcie w łaźni, aż kamień się zmieni w pył -  & \\
Przecież wrócę, gdy zacznie się dzień!  & \\
\end{longtable}
\clearpage

% --- Źródło: Every_breath_you_take.tex ---
\section{Every breath you take}
\vspace{-\baselineskip}
\textit{The Police}\\
\begin{longtable}{ll}
Every breath you take, & \textbf{G} \\
And every move you make & \textbf{e} \\
Every bond you break, every step you take, & \textbf{C D} \\
I'll be watching you & \textbf{e} \\
& \\
Every single day, & \textbf{G} \\
Every word you say & \textbf{e} \\
Every game you play, every night you stay, & \textbf{C D} \\
I'll be watching you & \textbf{e} \\
& \\
Oh can't you see, & \textbf{C} \\
You belong to me & \textbf{G} \\
How my poor heart aches & \textbf{A7} \\
With every step you take & \textbf{D} \\
& \\
Every move you make & \textbf{G} \\
And every vow you break & \textbf{e} \\
Every smile you fake, every claim you stake, & \textbf{C D} \\
I'll be watching you & \textbf{e} \\
& \\
Since you've gone I've been lost without a trace & \textbf{Es} \\
I dream at night I can only see your face & \textbf{F} \\
I look around, but it's you I can't replace & \textbf{Es} \\
I feel so cold and I long for your embrace & \textbf{F} \\
I keep crying baby, baby, please & \textbf{Es G} \\
& \\
Oh can't you see...  & \\
& \\
Every move you make...  & \\
& \\
\end{longtable}
\clearpage

% --- Źródło: Few_Days.tex ---
\section{Few Days}
\begin{longtable}{ll}
O Panie, czemu w ziemi tkwię & \textbf{F C F} \\
Hej raz hej raz & \textbf{F B} \\
I macham szuflą cały dzień & \textbf{F C F} \\
Hej na morze czas & \textbf{d A7 d} \\
& \\
\hspace*{2em}\textit{Mogę kopać tu dalej} & \textbf{d A7 d} \\
\hspace*{2em}\textit{Few days few days} & \textbf{F B} \\
\hspace*{2em}\textit{Mogę kopać przez dni parę} & \textbf{F A7 d} \\
\hspace*{2em}\textit{Ale wracać chcę} & \textbf{d A7 d} \\
& \\
Tam każdy takie bajdy plótł  & \\
Nie raz nie raz  & \\
Przekroczysz Jukon złota w bród  & \\
Hej na morze czas  & \\
& \\
\hspace*{2em}\textit{Mogę kopać tu dalej...}  & \\
& \\
Wykopię jeszcze parę dziur  & \\
Hej raz hej raz  & \\
Wtoczę płonnej skały wór  & \\
Hej na morze czas  & \\
& \\
\hspace*{2em}\textit{Mogę kopać tu dalej...}  & \\
& \\
Za żonę tu łopatę mam  & \\
Już dość już dość  & \\
A zysk że jej używam sam  & \\
Hej na morze czas  & \\
& \\
\hspace*{2em}\textit{Mogę kopać tu dalej...}  & \\
& \\
O Panie nie jest to Twój raj  & \\
O nie o nie  & \\
Nadzieję innym głupcom daj  & \\
Ja na morze chcę  & \\
& \\
\hspace*{2em}\textit{Mogę kochać się dalej}  & \\
\hspace*{2em}\textit{Few days few days}  & \\
\hspace*{2em}\textit{Mogę kochać przez dni parę}  & \\
\hspace*{2em}\textit{Ale wracać chcę}  & \\
\end{longtable}
\clearpage

% --- Źródło: Gdy_szedłem_na_wartę.tex ---
\section{Gdy szedłem na wartę}
\begin{longtable}{ll}
Gdy szedłem raz od Warty, & \textbf{G C G} \\
Sam jeden w ciemną noc, & \textbf{D G} \\
Stał o swą broń oparty & \textbf{G C G} \\
Wielkopolski harcerz nasz. & \textbf{D G} \\
& \\
Ty, młody skaucie, powiedz nam, & \textbf{D G} \\
Co ty tu robisz w nocy sam? & \textbf{D G} \\
Ha stoję dla Ojczyzny mej, & \textbf{G C G} \\
Ojczyzno moja, żyj! & \textbf{D G} \\
& \\
Co ty tu robisz w późny czas,  & \\
sam jeden w ciemną noc.  & \\
Na niebie śliczne gwiazdy,  & \\
promienista jest ich moc.  & \\
& \\
Gdy wspomnę o rodzinie mej  & \\
i o mej lubej, kochanej,  & \\
ja stoję dla Ojczyzny mej,  & \\
Ojczyzno moja, żyj!  & \\
\end{longtable}
\clearpage

% --- Źródło: Gdzie_jesteś_Rudy_Alku_i_Zośko.tex ---
\section{Gdzie jesteś Rudy, Alku i Zośko?}
\begin{longtable}{ll}
O czym marzyłeś druhu młody depcząc stopami granie Tatr?  & \\
Jakie marzenia i przygody wyśpiewał Ci tatrzański wiatr?  & \\
Jak piłeś radość kroplą rosy, jak ogarniałeś sercem świat?  & \\
I jak na przyszłe swoje losy rzuciłeś czynu trwały ślad?  & \\
& \\
\hspace*{2em}\textit{Gdzie jesteście,} &  \\
\hspace*{2em}\textit{Rudy, Alku, Zośko?}  & \\
\hspace*{2em}\textit{Gdzie jesteście?}  & \\
\hspace*{2em}\textit{Gdzie Twe dzieci, Polsko?}  & \\
& \\
O czym marzyłeś druhu młody, gdy z nagła pękł Twój marzeń świat?  & \\
Jakie tęsknoty w chwili trwogi rozwiał okrutny dziejów wiatr?  & \\
Jak wybrnąłeś w toni wojny i jak uciekłeś od swych snów?  & \\
Jak stało się że tak spokojny odszedłeś aby wrócić znów ?  & \\
& \\
\hspace*{2em}\textit{Gdzie jesteście,}  & \\
\hspace*{2em}\textit{Rudy, Alku, Zośko?}  & \\
\hspace*{2em}\textit{Gdzie jesteście?}  & \\
\hspace*{2em}\textit{Na Twych szańcach, Polsko!}  & \\
& \\
O czym dziś myślisz druhu młody stojąc gdzie krzyży białych ślad?  & \\
O czym dziś myślisz i co chciałbyś zachować w sercu z tamtych lat?  & \\
Jak to się dzieje, że pamiętasz, jak to się dzieje, że ich znasz?  & \\
A może Tobie o nich śpiewa odwieczne pieśni echo Tatr?  & \\
& \\
\hspace*{2em}\textit{Gdzie jesteście,}  & \\
\hspace*{2em}\textit{Rudy, Alku, Zośko?}  & \\
\hspace*{2em}\textit{Gdzie jesteście?}  & \\
\hspace*{2em}\textit{Zawsze z Tobą, Polsko}  & \\
& \\
\end{longtable}
\clearpage

% --- Źródło: Gdzie_ta_keja.tex ---
\section{\textbf{Gdzie ta keja}}
\begin{longtable}{ll}
Gdyby tak ktoś przyszedł i powiedział: & \textbf{a} \\
– Stary, czy masz czas? & \textbf{G a} \\
Potrzebuję do załogi jakąś nową twarz, & \textbf{C G G7 C} \\
Amazonka, Wielka Rafa, oceany trzy, & \textbf{C C7 F d} \\
Rejs na całość, rok, dwa lata – to powiedziałbym: & \textbf{a E E7 a} \\
& \\
\hspace*{2em}\textit{Gdzie ta keja, a przy niej ten jacht?} & \textbf{a E7 a} \\
\hspace*{2em}\textit{Gdzie ta koja wymarzona w snach?} & \textbf{C G C} \\
\hspace*{2em}\textit{Gdzie te wszystkie sznurki od tych szmat?} & \textbf{G A7 d A7 d} \\
\hspace*{2em}\textit{Gdzie ta brama na szeroki świat?} & \textbf{a E7 a} \\
& \\
Gdzie ta keja, a przy niej ten jacht? & \textbf{a E7 a} \\
Gdzie ta koja wymarzona w snach? & \textbf{C G C} \\
W każdej chwili płynę w taki rejs, & \textbf{G A7 d A7 d} \\
Tylko gdzie to jest? No gdzie to jest? & \textbf{a E7 a} \\
& \\
Gdzieś na dnie wielkiej szafy, leży ostry nóż, &  \\
Stare dżinsy wystrzępione impregnuje kurz, &  \\
W kompasie igła zardzewiała, lecz kierunek znam,  & \\
Biorę wór na plecy i przed siebie gnam.  & \\
& \\
\hspace*{2em}\textit{Gdzie ta keja, a przy niej ten jacht...}  & \\
& \\
Przeszły lata zapyziałe, rzęsą porósł staw,  & \\
Na przystani czółno stało – kolorowy paw.  & \\
Zaokrągliły się marzenia, wyjałowiał step,  & \\
Lecz wciąż marzy o załodze ten samotny łeb.  & \\
& \\
\hspace*{2em}\textit{Gdzie ta keja, a przy niej ten jacht...}  & \\
\end{longtable}
\clearpage

% --- Źródło: Green_Horn.tex ---
\section{Green Horn}
\begin{longtable}{ll}
Miałem wtedy ech szesnaście te lat & \textbf{C F C} \\
Swoją drogą szedłem przez świat & \textbf{F G} \\
W sercu zawsze tkwiła uparta myśl & \textbf{C F C a} \\
By za głosem morza iść & \textbf{C G} \\
& \\
\hspace*{2em}\textit{A więc śpiewam „Żegnaj Carling Ford} & \textbf{C F C} \\
\hspace*{2em}\textit{I żegnaj nam Green Horn} & \textbf{C G7} \\
\hspace*{2em}\textit{Będę myślał o tobie w dzień i noc} & \textbf{C C7 F C} \\
\hspace*{2em}\textit{Aby kiedyś wrócić tu} & \textbf{F G} \\
\hspace*{2em}\textit{Aby kiedyś wrócić znów”} & \textbf{F G C} \\
& \\
W każde z wielkich siedmiu sztormowych mórz  & \\
Wychodziłem bo zmuszał mnie los  & \\
Każdy w porcie ode mnie usłyszeć mógł  & \\
„Nigdy więcej morza już”  & \\
& \\
\hspace*{2em}\textit{A więc śpiewam „Żegnaj Carling Ford...}  & \\
& \\
Mam dziewczynę słodką Mary Doyle  & \\
Dom swój również ma na Green Horn  & \\
Sercem duszą pragnie zatrzymać mnie  & \\
Żebym w morze nie zwiał jej  & \\
& \\
\hspace*{2em}\textit{A więc śpiewam „Żegnaj Carling Ford...}  & \\
& \\
Życiem szczur lądowy kieruje sam  & \\
Chce zostanie lub pójdzie w dal  & \\
Ale kiedy morze wejdzie w krew  & \\
Gdy zawoła pójdziesz wnet  & \\
& \\
\hspace*{2em}\textit{A więc śpiewam „Żegnaj Carling Ford...}  & \\
\end{longtable}
\clearpage

% --- Źródło: Grosza_daj_wiedźminowi.tex ---
\section{Grosza daj wiedźminowi}
\begin{longtable}{ll}
Tę balladę wam, & \textbf{a D} \\
śpiewa skromny bard, & \textbf{d} \\
co z Geraltem z Rivii & \textbf{G} \\
wyruszył na szlak. & \textbf{D G E} \\
& \\
Diaboła spotkał tam,  & \\
co szukał z nim zwady,  & \\
z elfów hufcami  & \\
urządzał biesiady.  & \\
& \\
Pochwycili mnie,  & \\
podstępem, no bo jak?!  & \\
Zniszczyli mi lutnię,  & \\
skopali jak psa!  & \\
& \\
Ciała nasze dźgał  & \\
ten rogaty stwór,  & \\
zapłakał nasz wiedźmin,  & \\
„Mam dosyć już!”  & \\
& \\
\hspace*{2em}\textit{Grosza daj Wiedźminowi,} & \textbf{a E C} \\
\hspace*{2em}\textit{sakiewką potrząśnij,} & \textbf{D a} \\
\hspace*{2em}\textit{sakiewką potrząśnij, (ło, o, o!)} & \textbf{D a} \\
\hspace*{2em}\textit{Grosza daj Wiedźminowi,} & \textbf{a E C} \\
\hspace*{2em}\textit{sakiewką potrząśnij!} & \textbf{D E E7 a} \\
& \\
& \\
\end{longtable}
\newpage
\begin{longtable}{ll}
Lecz chwycił Biały Wilk, & \textbf{a D} \\
za morderczy róg, & \textbf{d} \\
co tylu już przed nim & \textbf{G} \\
obalił był z nóg. & \textbf{D G E} \\
& \\
Elfy cisnął precz,  & \\
aż na górski szczyt,  & \\
daleko od ludzi,  & \\
gdzie miejsce ich.  & \\
& \\
Choć oberwał sam  & \\
zmiażdżył bestii kark,  & \\
ten obrońca ludzkości,  & \\
toastu jest wart.  & \\
& \\
Oto moja pieśń,  & \\
to wasz bohater jest,  & \\
on wrogów pokonał,  & \\
nalejcie mu więc!  & \\
& \\
\hspace*{2em}\textit{Grosza daj Wiedźminowi... || x3}  & \\
& \\
\end{longtable}
\clearpage

% --- Źródło: Gór_mi_mało.tex ---
\section{\textbf{Gór mi mało}}
\begin{longtable}{ll}
Drogi Mistrzu Mistrzu mojej drogi & \textbf{C G} \\
Mistrzu Jerzy i Mistrzu Wojciechu & \textbf{d G} \\
Przez was w górach schodziłem nogi  & \\
Nie mogąc złapać oddechu  & \\
& \\
Gór co stoją nigdy nie dogonię   & \\
Znikających punktów na mapie   & \\
Jakie miejsce nazwę swym domem  & \\
Jakim dotrę do niego szlakiem  & \\
& \\
\hspace*{2em}\textit{Gór mi mało i trzeba mi więcej} & \textbf{C G} \\
\hspace*{2em}\textit{Żeby przetrwać od zimy do zimy}& \textbf{a e} \\
\hspace*{2em}\textit{Ktoś mnie skazał na wieczną wędrówkę} & \textbf{F C} \\
\hspace*{2em}\textit{Po śladach które sam zostawiłem} & \textbf{d G} \\
& \\
\hspace*{2em}\textit{Góry góry i ciągle mi nie dość}  & \\
\hspace*{2em}\textit{Skazanemu na gór dożywocie}  & \\
\hspace*{2em}\textit{Świat na dobre mi zbieszczadział}  & \\
\hspace*{2em}\textit{Szczyty wolnym mijają mnie krokiem}  & \\
& \\
Pańscy święci święci bezpańscy  & \\
\smash{Święty} Jerzy Mikołaju Michale  & \\
Starodawni gór świętych mieszkańcy  & \\
Imię wasze pieśniami wychwalam  & \\
& \\
Gór co stoją nigdy nie dogonię  & \\
Znikających punktów na mapie  & \\
I chaty by nazwać ją swym domem  & \\
Do której żaden szlak by nie trafił  & \\
& \\
\hspace*{2em}\textit{Gór mi mało i trzeba mi więcej...}  & \\
& \\
\end{longtable}
\clearpage

% --- Źródło: Hakuna_Matata.tex ---
\section{Hakuna Matata}
\begin{longtable}{ll}
\hspace*{2em}\textit{Hakuna matata, jak cudownie to brzmi} & \textbf{F C} \\
\hspace*{2em}\textit{Hakuna matata, to nie byle bzik!} & \textbf{F D G} \\
\hspace*{2em}\textit{Już się nie martw, aż do końca twych dni!} & \textbf{a F D7} \\
\hspace*{2em}\textit{Naucz się tych dwóch radosnych słów} & \textbf{C G7} \\
\hspace*{2em}\textit{Hakuna Matata!} & \textbf{C} \\
& \\
Otóż, gdy był z niego mały wieprz & \textbf{B F C} \\
Gdy był ze mnie mały wieprz & \textbf{B F C} \\
(No pięknie - Dzięki)  & \\
& \\
Woń przykrą rozsiewał kiedy kończył jeść, & \textbf{Dis F} \\
innym jego kąpanie ciężko było znieść & \textbf{C G} \\
Mówią o mnie żem cham, jam subtelny gość & \textbf{B F C} \\
Przykre że, przy mnie ktoś wciąż zatykał nos & \textbf{Dis F G7} \\
& \\
Och! Co za wstyd! -było mu wstyd! & \textbf{C} \\
Cały świat mi zbrzydł! - był brzydszy niż ty! & \textbf{G7} \\
Czułem się paskudnie! - a może cudnie? & \textbf{B7} \\
& \\
Zawsze gdy chciałem...  & \\
(No no Pumba nie przy dzieciach...)  & \\
(O sorry)  & \\
& \\
\hspace*{2em}\textit{Hakuna matata, jak cudownie to brzmi...}  & \\
& \\
I już się nie martw, aż do końca twych dni! & \textbf{a F D} \\
Naucz się tych dwóch radosnych słów! & \textbf{C G7} \\
Hakuna matata & \textbf{a F} \\
Hakuna matata & \textbf{G} \\
Hakuna matata & \textbf{a F} \\
Hakuuuna matata & \textbf{G} \\
Matata ha ha ha  & \\
\end{longtable}
\clearpage

% --- Źródło: Hej_Sokoły.tex ---
\section{Hej Sokoły}
\begin{longtable}{ll}
Hej, tam gdzieś znad Czarnej Wody & \textbf{a} \\
Wsiada na koń Kozak młody & \textbf{E E7} \\
Czule żegna się z dziewczyną & \textbf{a} \\
Jeszcze czulej z Ukrainą & \textbf{E7 a G} \\
& \\
\hspace*{2em}\textit{Hej, hej, hej Sokoły} & \textbf{C} \\
\hspace*{2em}\textit{Omijajcie góry, lasy, pola, doły} & \textbf{G G7} \\
\hspace*{2em}\textit{Dzwoń, dzwoń, dzwoń dzwoneczku} & \textbf{a} \\
\hspace*{2em}\textit{Mój stepowy skowroneczku} & \textbf{E7 a G7} \\
& \\
\hspace*{2em}\textit{Hej, hej, hej Sokoły} & \textbf{C} \\
\hspace*{2em}\textit{Omijajcie góry, lasy, pola, doły} & \textbf{G G7} \\
\hspace*{2em}\textit{Dzwoń, dzwoń, dzwoń dzwoneczku} & \textbf{a} \\
\hspace*{2em}\textit{Mój stepowy, dzwoń, dzwoń, dzwoń} & \textbf{E7 a E7} \\
& \\
Pięknych dziewcząt jest nie mało  & \\
Lecz najwięcej w Ukrainie  & \\
Tam me serce pozostało  & \\
Przy kochanej mej dziewczynie  & \\
& \\
\hspace*{2em}\textit{Hej, hej, hej Sokoły...}  & \\
& \\
Ona biedna tam została,  & \\
Przepióreczka moja mała,  & \\
A ja tutaj w obcej stronie  & \\
Dniem i nocą tęsknię do niej.  & \\
& \\
\hspace*{2em}\textit{Hej, hej, hej Sokoły...}  & \\
& \\
Żal, żal za dziewczyną  & \\
Za zieloną Ukrainą  & \\
Żal, żal, serce płacze  & \\
Już jej więcej nie zobaczę  & \\
& \\
\hspace*{2em}\textit{Hej, hej, hej Sokoły...}  & \\
& \\
Wina, wina, wina dajcie  & \\
A jak umrę, pochowajcie  & \\
Na zielonej Ukrainie  & \\
Przy kochanej mej dziewczynie  & \\
& \\
\end{longtable}
\clearpage

% --- Źródło: Hej_morze_moje_morze.tex ---
\section{Hej morze, moje morze}
\begin{longtable}{ll}
Hej me Bałtyckie Morze  & \\
Wdzięczny ci jestem bardzo  & \\
Toś ty mnie wychowało  & \\
Toś ty mnie wychowało  & \\
Szkołeś mi dało twardą  & \\
Toś ty mnie wychowało  & \\
Toś ty mnie wychowało  & \\
Szkołeś mi dało twardą  & \\
& \\
Szkołeś mi dało twardą  & \\
Uczyłoś łodzią pływać  & \\
Żagle pięknie cerować  & \\
Żagle pięknie cerować  & \\
Codziennie pokład zmywać  & \\
Żagle pięknie cerować  & \\
Żagle pięknie cerować  & \\
Codziennie pokład zmywać  & \\
& \\
Codziennie pokład zmywać  & \\
Od soli i od kurzy  & \\
Mosiądze wyglansować  & \\
Mosiądze wyglansować  & \\
W ciszy, czy w czasie burzy  & \\
Mosiądze wyglansować  & \\
Mosiądze wyglansować  & \\
W ciszy, czy w czasie burzy  & \\
& \\
W ciszy, czy w czasie burzy  & \\
Trzeba przy pacy śpiewać  & \\
Bo kiedy śpiewu nie ma  & \\
Bo kiedy śpiewu nie ma  & \\
Neptun się będzie gniewać  & \\
Bo kiedy śpiewu nie ma  & \\
Bo kiedy śpiewu nie ma  & \\
Neptun się będzie gniewać  & \\
& \\
Neptun się będzie gniewać  & \\
I klątwę brzydką rzuci  & \\
Wpakuje na mieliznę  & \\
Wpakuje na mieliznę  & \\
Albo nam łódź wywróci  & \\
Wpakuje na mieliznę  & \\
Wpakuje na mieliznę  & \\
Albo nam łódź wywróci  & \\
\end{longtable}
\newpage
\begin{longtable}{ll}
Albo nam łódź wywróci  & \\
I krzyknie, „Hej partacze”  & \\
Nakarmię wami rybki  & \\
Nakarmię wami rybki  & \\
Nikt po was nie zapłacze  & \\
Nakarmię wami rybki  & \\
Nakarmię wami rybki  & \\
Nikt po was nie zapłacze  & \\
& \\
Nikt po nas nie zapłacze  & \\
Nikt nam nie dopomoże  & \\
Za wszystkie miłe rady  & \\
Za wszystkie miłe rady  & \\
Dziękuję tobie morze  & \\
Za wszystkie miłe rady  & \\
Za wszystkie miłe rady  & \\
Dziękuję tobie morze  & \\
& \\
Hej morze, moje morze  & \\
Wdzięczny ci jestem bardzo  & \\
Toś ty mnie wychowało  & \\
Toś ty mnie wychowało  & \\
Szkołeś mi dało twardą  & \\
Toś ty mnie wychowało  & \\
Toś ty mnie wychowało  & \\
Szkołeś mi dało twardą  & \\
& \\
\end{longtable}
\clearpage

% --- Źródło: Hej_przyjaciele.tex ---
\section{Hej przyjaciele}
\begin{longtable}{ll}
Tam dokąd chciałem już nie dojdę & \textbf{C G} \\
Szkoda zdzierać nóg & \textbf{F C} \\
Już wędrówki naszej wspólnej & \textbf{C G} \\
Nadchodzi kres & \textbf{F C} \\
Wy pójdziecie inną drogą & \textbf{C G} \\
Zostawcie mnie & \textbf{F C} \\
Odejdziecie sam zostanę & \textbf{C G} \\
Na rozstaju dróg & \textbf{F C} \\
& \\
\hspace*{2em}\textit{Hej przyjaciele zostańcie ze mną} & \textbf{C G F C} \\
\hspace*{2em}\textit{Przecież wszystko to co miałem, oddałem wam} & \textbf{C G F C} \\
\hspace*{2em}\textit{Hej przyjaciele choć chwilę jedną} & \textbf{C G F C} \\
\hspace*{2em}\textit{Znów w życiu mi nie wyszło, znowu będę sam} & \textbf{C G F C} \\
& \\
Znów spóźniłem się na pociąg  & \\
I odjechał już  & \\
Tylko jego mglisty koniec  & \\
Zamajaczył mi  & \\
Stoję smutny na peronie  & \\
Z tą walizką jedną  & \\
Tak jak człowiek który zgubił  & \\
Do domu swego klucz.  & \\
& \\
\hspace*{2em}\textit{Hej przyjaciele zostańcie ze mną...}  & \\
& \\
Tam dokąd chciałem już nie mogę  & \\
Szkoda zdzierać butów  & \\
Już wędrówki naszej wspólnej  & \\
Nadchodzi kres  & \\
Wy pójdziecie inną drogą  & \\
Zostawcie mnie  & \\
Zamazanych drogowskazów  & \\
Nie odczyta już.  & \\
& \\
\hspace*{2em}\textit{Hej przyjaciele zostańcie ze mną...}  & \\
& \\
& \\
\end{longtable}
\newpage
\begin{longtable}{ll}
\end{longtable}
\clearpage

% --- Źródło: Hej_w_góry.tex ---
\section{\textbf{Hej w góry}}
\begin{longtable}{ll}
\hspace*{2em}\textit{Hej, w góry, w góry} & \textbf{C} \\
\hspace*{2em}\textit{Popatrz tam wstaje blady świt (tam wstaje)} & \textbf{d} \\
\hspace*{2em}\textit{Jeszcze tak nieporadnie chce ominąć szczyt} & \textbf{F f C G} \\
\hspace*{2em}\textit{Hej, miły panie, czekaj} & \textbf{C} \\
\hspace*{2em}\textit{Zaraz my też będziemy tam (będziemy)} & \textbf{d} \\
\hspace*{2em}\textit{Nie będziesz musiał schodzić z połoniny Gam (z połoniny sam)} & \textbf{F f C G C} \\
& \\
Bywały dni, że słońca złoty blask  & \\
Wzawody szedł z sennym brzaskiem  & \\
To dziwne więc, że skoro świt  & \\
Wiatr i deszcz razem tańczą  & \\
& \\
\hspace*{2em}\textit{Hej, w góry, w góry...}  & \\
& \\
Zagrajcie nam, może się cofnie czas  & \\
Do tamtych dni naszych marzeń  & \\
Do dni spędzonych pośród sennych skał  & \\
Gdy czas umykał w pełni zdarzeń  & \\
& \\
\hspace*{2em}\textit{Hej, w góry, w góry..}  & \\
\end{longtable}
\clearpage

% --- Źródło: Hiszpańskie_dziewczyny.tex ---
\section{\textbf{Hiszpańskie dziewczyny}}
\begin{longtable}{ll}
Żegnajcie nam dziś, hiszpańskie dziewczyny, & \textbf{e C H7} \\
Żegnajcie nam dziś, marzenia ze snów, & \textbf{e G D} \\
Ku brzegom angielskim już ruszać nam pora, & \textbf{C D e} \\
Lecz kiedyś na pewno wrócimy tu znów. & \textbf{C H7 e} \\
& \\
\hspace*{2em}\textit{I smak waszych ust, hiszpańskie dziewczyny,} & \textbf{e C H7} \\
\hspace*{2em}\textit{W noc ciemną i złą nam będzie się śnił.} & \textbf{e G D} \\
\hspace*{2em}\textit{Leniwie popłyną znów rejsu godziny,} & \textbf{C D e} \\
\hspace*{2em}\textit{Wspomnienie ust waszych przysporzy nam sił.} & \textbf{C H7 e} \\
& \\
Niedługo ujrzymy znów w dali Cape Deadman  & \\
I Głowę Baranią sterczącą wśród wzgórz,  & \\
I statki stojące na redzie przed Plymouth.  & \\
Klarować kotwicę najwyższy czas już.  & \\
& \\
\hspace*{2em}\textit{I smak waszych ust, hiszpańskie dziewczyny...}  & \\
& \\
A potem znów żagle na masztach rozkwitną,  & \\
Kurs szyper wyznaczy do Portland i Wight,  & \\
I znów stara łajba potoczy się ciężko  & \\
Przez fale w kierunku na Beachie, Fairlee Light.  & \\
& \\
\hspace*{2em}\textit{I smak waszych ust, hiszpańskie dziewczyny...}  & \\
& \\
Zabłysną nam bielą skał zęby pod Dover  & \\
I znów noc w kubryku wśród legend i bajd.  & \\
Powoli i znojnie tak płynie nam życie  & \\
Na wodach i w portach przy South Foreland Light.  & \\
& \\
\hspace*{2em}\textit{I smak waszych ust, hiszpańskie dziewczyny...}  & \\
& \\
& \\
\end{longtable}
\newpage
\begin{longtable}{ll}
\end{longtable}
\clearpage

% --- Źródło: Hotel_California.tex ---
\section{Hotel California}
\vspace{-\baselineskip}
\textit{Eagles}\\
\begin{longtable}{ll}
On a dark desert highway, cool wind in my hair & \textbf{h Fis} \\
Warm smell of colitas rising up through the air & \textbf{A E} \\
Up ahead in the distance, I saw a shimmering light & \textbf{G D} \\
My head grew heavy and my sight grew dim, & \textbf{e} \\
I had to stop for the night & \textbf{Fis} \\
& \\
There she stood in the doorway, I heard the mission bell &  \\
And I was thinkin' to myself, ,,This could be heaven or this could be hell''  & \\
Then she lit up a candle and she showed me the way &  \\
There were voices down the corridor, &  \\
I thought I heard them say  & \\
& \\
\hspace*{2em}\textit{“Welcome to the Hotel California} & \textbf{G D} \\
\hspace*{2em}\textit{Such a lovely place (such a lovely place), such a lovely face} & \textbf{e h7} \\
\hspace*{2em}\textit{Plenty of room at the Hotel California} & \textbf{G D} \\
\hspace*{2em}\textit{Any time of year (any time of year), you can find it here”} & \textbf{e Fis} \\
& \\
Her mind is Tiffany-twisted, she got the Mercedes-Benz, uh  & \\
She got a lot of pretty, pretty boys that she calls friends  & \\
How they dance in the courtyard, sweet summer sweat  & \\
Some dance to remember,  & \\
Some dance to forget  & \\
& \\
So I called up the Captain, “Please bring me my wine”  & \\
He said, "We haven't had that spirit here since 1969"  & \\
And still, those voices are calling from far away  & \\
Wake you up in the middle of the night  & \\
Just to hear them say  & \\
& \\
\hspace*{2em}\textit{,,Welcome to the Hotel California}  & \\
\hspace*{2em}\textit{Such a lovely place (such a lovely place), such a lovely face}  & \\
\hspace*{2em}\textit{They’ re livin’ it up at the Hotel California}  & \\
\hspace*{2em}\textit{What a nice surprise (what a nice surprise), bring your alibis''}  & \\
& \\
Mirrors on the ceiling, the pink champagne on ice  & \\
And she said, “We are all just prisoners here of our own device”  & \\
And in the master's chambers, they gathered for the feast  & \\
They stab it with their steely knives,  & \\
But they just can't kill the beast  & \\
& \\
Last thing I remember, I was running for the door  & \\
I had to find the passage back to the place I was before  & \\
“Relax,”  said the night man, “We are programmed to receive  & \\
You can check out any time you like,  & \\
But you can never leave”  & \\
\end{longtable}
\clearpage

% --- Źródło: Idziemy_w_jasną.tex ---
\section{Idziemy w jasną}
\begin{longtable}{ll}
Idziemy w jasną z błękitu utkaną dal, & \textbf{d A A7 d} \\
Drogą wśród łąk, pól bezkresnych & \textbf{A} \\
I wśród mórz szumiących fal. & \textbf{A7 d} \\
& \\
Cicho, szeroko, jak okiem spojrzenia ślę  & \\
Jakieś się snują marzenia,  & \\
W wieczornej spowite mgle.  & \\
& \\
Idziemy naprzód i ciągle pniemy się wzwyż,  & \\
By zdobyć szczyt ideałów,  & \\
Świetlany, harcerski krzyż.  & \\
\end{longtable}
\clearpage

% --- Źródło: Irlandzki_Żeglarz.tex ---
\section{\textbf{Irlandzki Żeglarz}}
\begin{longtable}{ll}
Wśród zielonych wzgórz ojciec mój zbudował dom & \textbf{C G C} \\
Wokół rosły krzewy bzu & \textbf{F C G} \\
Całe lata tam spędzaliśmy pośród łąk & \textbf{F C G a} \\
I torfowych grząskich pól & \textbf{F G} \\
Miałem tego dość chciałem uciec chciałem biec & \textbf{C G C} \\
I jak ptaki z wiatrem gnać & \textbf{F C G} \\
Ludzie drzewom przecież nie podobni są & \textbf{F C G a} \\
Żeby w jednym miejscu stać & \textbf{F G} \\
& \\
\hspace*{2em}\textit{A ta łajba jest całym domem mym} & \textbf{d a} \\
\hspace*{2em}\textit{Gdy znika ląd} & \textbf{C G} \\
\hspace*{2em}\textit{Ona serce ma które bije w nim} & \textbf{d a} \\
\hspace*{2em}\textit{Ding dong ding dong} & \textbf{F C G} \\
\hspace*{2em}\textit{Jak wolności łyk tak jak wiatru szept} & \textbf{d a} \\
\hspace*{2em}\textit{Szczęśliwy ton} & \textbf{C G} \\
\hspace*{2em}\textit{Morze wzywa mnie z całych swoich sił} & \textbf{d a} \\
\hspace*{2em}\textit{Sercem jak dzwon} & \textbf{F C G} \\
& \\
Tak z zielonych łąk los na morze rzucił mnie &  \\
Na cedrowy stary jacht  & \\
Trzeba było sił trzeba było wielu lat  & \\
By się albatrosem stać  & \\
Ten cedrowy ship już nie jeden przeżył sztorm  & \\
Bawełnianą wożąc nić  & \\
Teraz mimo lat wciąż gotowy jest jak ja  & \\
W każdej chwili w morze iść  & \\
& \\
\hspace*{2em}\textit{A ta łajba jest całym domem mym...}  & \\
\end{longtable}
\clearpage

% --- Źródło: Jadę_ulicą.tex ---
\section{Jadę ulicą}
\begin{longtable}{ll}
\textbf{Julian:}  & \\
Jadem ulicą.  & \\
Opony asfalt drą  & \\
Pingwiny jak zobaczą drzewo,  & \\
na zawał umrą. Hej!  & \\
& \\
\textbf{Ref:}  & \\
Julian, Julian  & \\
on twardy jest jak głaz.  & \\
& \\
\textbf{Mort:}  & \\
Jedziemy panie ciut za wolno,  & \\
trzeba wcisnąć gaazz!  & \\
& \\
\textbf{Julian:}  & \\
O Gonią nas z dużą prędkością!  & \\
& \\
\textbf{Mort:}  & \\
O poważnie to wcisnę sprzęgło!  & \\
& \\
\end{longtable}
\clearpage

% --- Źródło: Jagnięcym_futerkiem_wałek_pokryty.tex ---
\section{Jagnięcym futerkiem wałek pokryty}
\begin{longtable}{ll}
\hspace*{2em}\textit{Jagnięcym futerkiem wałek pokryty} & \textbf{d d4 d C d} \\
\hspace*{2em}\textit{Na metalowym leży regale} & \textbf{d F a d} \\
\hspace*{2em}\textit{Trochę widoczny, choć trochę skryty} & \textbf{d d4 d C d} \\
\hspace*{2em}\textit{Przywodzi na myśl tatrzańskie hale} & \textbf{d F a d} \\
& \\
Czyją to była owa owieczka & \textbf{d F C d} \\
Której futerkiem malujesz krużganek? & \textbf{d F C d} \\
Hasała po łąkach jak tancereczka & \textbf{d F C d} \\
A teraz po niej pozostał ten wałek… & \textbf{d F C d} \\
& \\
\hspace*{2em}\textit{Jagnięcym futerkiem wałek pokryty…}  & \\
& \\
Owieczko pogodna, gdzie masz Pasterza? & \textbf{d F C d} \\
czy nie wiesz, że wilk przerobi Cię w wałek? & \textbf{d F C d} \\
Pasterz nakarmi, gdy będziesz głodna, & \textbf{d F C d} \\
nie sprzeda nikomu – On Cię ocali. & \textbf{d F C d} \\
& \\
Włoży w Twe usta słowa kwieciste,  & \\
przestaniesz beczeć tym głosem baranim…  & \\
Zaśpiewasz wtedy pieśni wieczyste,  & \\
na chmurce odtańczysz niebiański balet.  & \\
& \\
\hspace*{2em}\textit{Jagnięcym futerkiem wałek pokryty…}  & \\
\end{longtable}
\clearpage

% --- Źródło: Jagnięcym_futerkiem_wałek_pokryty_2.tex ---
\section{Jagnięcym futerkiem wałek pokryty 2}
\begin{longtable}{ll}
\hspace*{2em}\textit{Jagnięcym futerkiem wałek pokryty} & \textbf{d d4 d C d} \\
\hspace*{2em}\textit{Na metalowym leży regale} & \textbf{d F a d} \\
\hspace*{2em}\textit{Trochę widoczny, choć trochę skryty} & \textbf{d d4 d C d} \\
\hspace*{2em}\textit{Przywodzi na myśl tatrzańskie hale} & \textbf{d F a d} \\
& \\
Czyją to była owa owieczka & \textbf{d F C d} \\
Której futerkiem malujesz krużganek? & \textbf{d F C d} \\
Hasała po łąkach jak tancereczka & \textbf{d F C d} \\
A teraz po niej pozostał ten wałek… & \textbf{d F C d} \\
& \\
Już nie zabeczy głosikiem drżącym  & \\
Skacząc po trawce na dworze  & \\
Pozostał po niej wałek kapiący  & \\
Farbą w zielonym kolorze…  & \\
& \\
Po cóż by miała skakać po dworze?  & \\
Jeszcze by dachówki zbiła  & \\
Toć po polu hasała, nieboże  & \\
Zanim na wałku skończyła  & \\
& \\
\hspace*{2em}\textit{Jagnięcym futerkiem wałek pokryty} & \textbf{d d4 d C d} \\
\hspace*{2em}\textit{W dodatku w przystępnej cenie} & \textbf{d F a d} \\
\hspace*{2em}\textit{Oto co zostało – myślisz przybity} & \textbf{d d4 d C d} \\
\hspace*{2em}\textit{Ten wałek i jagniąt milczenie…} & \textbf{d F a d} \\
& \\
\end{longtable}
\clearpage

% --- Źródło: Jak.tex ---
\section{Jak}
\vspace{-\baselineskip}
\textit{Stare Dobre Małżeństwo}\\
\begin{longtable}{ll}
Jak po nocnym niebie sunące & \textbf{A E} \\
Białe ob loki nad lasem & \textbf{D A} \\
Jak na szyi wędrowca apaszka szamotana wiatrem & \textbf{h D A A4} \\
& \\
Jak wyciągnięte tam powyżej & \textbf{A E} \\
Gwiaździste ramiona wasze & \textbf{D A} \\
A tu są nasze, a tu są nasze & \textbf{h D A A4} \\
& \\
Jak suchy szloch w te dżdżysta noc & \textbf{A E} \\
Jak winny - li  - niewinny sumienia wyrzut & \textbf{D A} \\
Ze się żyje, gdy umarło tylu, tylu, tylu & \textbf{h D A A4} \\
& \\
Jak suchy szloch w te dżdżystą noc & \textbf{A E} \\
Jak lizać rany celnie zadane & \textbf{D A} \\
Jak lepić serce w proch potrzaskane & \textbf{h D A A4} \\
& \\
Jak suchy szloch w te dżdżysta noc & \textbf{A E} \\
Pudowy kamień, pudowy kamień & \textbf{D A} \\
Jak na nim stanę, on na mnie stanie & \textbf{h D} \\
On na mnie stanie, spod niego wstanę & \textbf{A A4} \\
& \\
Jak suchy szloch w te dżdżystą noc & \textbf{A E} \\
Jak z lota kula nad wodami & \textbf{D A} \\
Jak świt pod spuchniętymi powiekami & \textbf{h D A A4} \\
& \\
Jak zorze miłe, śliczne polany & \textbf{A E} \\
Jak słońca pierś, jak garb swój nieść & \textbf{D A} \\
Jak do was, siostry mgławicowe & \textbf{h D} \\
Ten zawodzący śpiew & \textbf{A A4} \\
& \\
Jak biec do końca, potem odpoczniesz & \textbf{A E} \\
Potem odpoczniesz, cudne manowce & \textbf{D A} \\
Cudne manowce, cudne, cudne manowce & \textbf{h D A A4} \\
& \\
& \\
\end{longtable}
\newpage
\begin{longtable}{ll}
\end{longtable}
\clearpage

% --- Źródło: Jaki_był_ten_dzień.tex ---
\section{Jaki był ten dzień}
\begin{longtable}{ll}
Późno już otwiera się noc & \textbf{e a D h} \\
Sen podchodzi do drzwi na palcach jak kot & \textbf{C G a H} \\
Nadchodzi czas ucieczki na out &  \\
Gdy kolejny mój dzień wspomnieniem się stał &  \\
& \\
\hspace*{2em}\textit{Jaki był ten dzień co darował co wziął}  & \\
\hspace*{2em}\textit{Czy mnie wyniósł pod niebo czy rzucił na dno}  & \\
\hspace*{2em}\textit{Jaki był ten dzień czy coś zmienił czy nie}  & \\
\hspace*{2em}\textit{Czy był tylko nadzieją na dobre i złe}  & \\
& \\
Łagodny mrok zasłania mi twarz  & \\
Jakby przeczuł że chcę być sobą chociaż raz  & \\
Nie skarżę się że mam to co mam  & \\
Że przegrałem coś znów i jestem tu sam  & \\
& \\
\hspace*{2em}\textit{Jaki był ten dzień co darował co wziął}  & \\
\hspace*{2em}\textit{Czy mnie wyniósł pod niebo czy rzucił na dno}  & \\
\hspace*{2em}\textit{Jaki był ten dzień czy coś zmienił czy nie}  & \\
\hspace*{2em}\textit{Czy był tylko nadzieją na dobre i złe}  & \\
& \\
Miliony gwiazd ze snu budzi cię  & \\
Swe promienie Ci ślą, więc chciej przyjąć je  & \\
Miniony dzień złóż u nieba wrót  & \\
Niech popłynie melodia z księżycowych nut  & \\
& \\
\hspace*{2em}\textit{Jaki był ten dzień co darował co wziął}  & \\
\hspace*{2em}\textit{Czy mnie wyniósł pod niebo czy rzucił na dno}  & \\
\hspace*{2em}\textit{Jaki był ten dzień czy coś zmienił czy nie}  & \\
\hspace*{2em}\textit{Czy był tylko nadzieją na dobre i złe}  & \\
& \\
\end{longtable}
\clearpage

% --- Źródło: Jasnowłosa.tex ---
\section{Jasnowłosa}
\begin{longtable}{ll}
Na tańcach ją poznałem, długowłosą blond & \textbf{G C D G} \\
Dziewczynę moich marzeń nie wiadomo skąd & \textbf{G e C D7} \\
Ona się tam wzięła piękna niczym kwiat & \textbf{G e C F D7} \\
Czy jak syrena wyszła z morza, czy ją przywiał wiatr & \textbf{G C D7 G} \\
& \\
\hspace*{2em}\textit{Żegnaj, Irlandio, czas w drogę mi już} &  \\
\hspace*{2em}\textit{W porcie gotowa stoi moja łódź}  & \\
\hspace*{2em}\textit{Na wielki ocean przyjdzie mi zaraz wyjść}  & \\
\hspace*{2em}\textit{I pożegnać się z dziewczyną na Long Sherry}  & \\
& \\
Ująłem ją za rękę delikatną jak  & \\
Latem mały motyl albo róży kwiat  & \\
Poszedłem z nią na plażę wsłuchać się w szum fal  & \\
Pokazałem jasnowłosej wielki morza czar  & \\
& \\
\hspace*{2em}\textit{Żegnaj, Irlandio, czas w drogę mi już...}  & \\
& \\
Za moment wypływamy w długi, trudny rejs  & \\
I z piękną mą dziewczyną przyjdzie rozstać się  & \\
Żagle pójdą w górę, wiatr je pogna w przód  & \\
I przez morza mie uniesie, czy zostaniesz tu  & \\
& \\
\hspace*{2em}\textit{Żegnaj, Irlandio, czas w drogę mi już...}  & \\
\end{longtable}
\clearpage

% --- Źródło: Jałta.tex ---
\section{Jałta}
\vspace{-\baselineskip}
\textit{Jacek Kaczmarski}\\
\begin{longtable}{ll}
Jak nowa – rezydencja carów, & \textbf{D G D} \\
Służba swe obowiązki zna; & \textbf{D A D} \\
Precz wysiedlono stąd Tatarów, & \textbf{D G D} \\
Gdzie na świat wyrok zapaść ma. & \textbf{D A D} \\
& \\
Okna już widzą, słyszą ściany, & \textbf{D G D} \\
Jak kaszle nad cygarem Lew; & \textbf{D G D} \\
Jak skrzypi wózek popychany & \textbf{D G D} \\
Z kalekim Demokratą w tle. & \textbf{D A D} \\
& \\
Lecz nikt nie widzi i nie słyszy, & \textbf{A Fis h} \\
Co robi Góral w krymską noc, & \textbf{D G A D} \\
Gdy gestem w wiernych towarzyszy & \textbf{A Fis h} \\
Wpaja swą legendarną moc. & \textbf{G D A D} \\
& \\
Nie miejcie żalu do Stalina, & \textbf{d C d} \\
Nie on się za tym wszystkim krył; & \textbf{d C d} \\
Przecież to nie jest jego wina, & \textbf{d C d} \\
Że Roosevelt w Jałcie nie miał sił. & \textbf{d C d} \\
& \\
Gdy się triumwirat wspólnie brał & \textbf{d C d} \\
Za świata historyczne kształty – & \textbf{d C d} \\
Wiadomo kto Cezara grał – & \textbf{d C d C} \\
I tak rozumieć trzeba Jałtę. & \textbf{D A D} \\
& \\
W resztce cygara mdłym ogniku & \textbf{D G D} \\
Pływała Lwa Albionu twarz: & \textbf{D A D} \\
Nie rozmawiajmy o Bałtyku, & \textbf{D G D} \\
Po co w Europie tyle państw? & \textbf{D A D} \\
& \\
Polacy? – chodzi tylko o to, & \textbf{D G D} \\
Żeby gdzieś w końcu mogli żyć… & \textbf{D G D} \\
Z tą Polską zawsze są kłopoty – & \textbf{D G D} \\
Kaleka troszczy się i drży. & \textbf{D A D} \\
& \\
Lecz uspokaja ich gospodarz, & \textbf{A Fis h} \\
Pożółkły dłonią głaszcząc wąs: & \textbf{D G A D} \\
Mój kraj pomocną dłoń im poda, & \textbf{A Fis h} \\
Potem niech rządzą się jak chcą. & \textbf{G D A D} \\
& \\
Nie miejcie żalu do Churchilla, & \textbf{d C d} \\
Nie on wszak za tym wszystkim stał; & \textbf{d C d} \\
Wszak po to tylko był triumwirat, & \textbf{d C d} \\
By Stalin dostał to, co chciał. & \textbf{d C d} \\
& \\
& \\
Komu zależy na pokoju, & \textbf{d C d} \\
Ten zawsze cofnie się przed gwałtem; & \textbf{d C d} \\
Wygra, kto się nie boi wojen – & \textbf{d C d C} \\
I tak rozumieć trzeba Jałtę. & \textbf{D A D} \\
& \\
Ściana pałacu słuch napina, & \textbf{D G D} \\
Gdy do Kaleki mówi Lew – & \textbf{D A D} \\
Ja wierzę w szczerość słów Stalina, & \textbf{D G D} \\
Dba chyba o radziecką krew. & \textbf{D A D} \\
& \\
I potakuje mu Kaleka, & \textbf{D G D} \\
Niezłomny demokracji stróż: & \textbf{D G D} \\
Stalin to ktoś na miarę wieku, & \textbf{D G D} \\
Oto mąż stanu, oto wódz! & \textbf{D A D} \\
& \\
Bo sojusz wielkich to nie zmowa, & \textbf{A Fis h} \\
To przyszłość świata – wolność, ład! & \textbf{D G A D} \\
Przy nim i słaby się uchowa & \textbf{a Fis h} \\
I swoją część otrzyma – strat! & \textbf{D G A D} \\
& \\
Nie miejcie żalu do Roosevelta, & \textbf{d C d} \\
Pomyślcie ile musiał znieść! & \textbf{d C d} \\
Fajka, dym cygar i butelka – & \textbf{d C d} \\
Churchill, co miał sojusze gdzieś. & \textbf{d C d} \\
& \\
Wszakże radziły trzy imperia & \textbf{d C d} \\
Nad granicami, co zatarte; & \textbf{d C d} \\
W szczegółach zaś już siedział Beria – & \textbf{d C d C} \\
I tak rozumieć trzeba Jałtę! & \textbf{D A D} \\
& \\
Więc delegacje odleciały, & \textbf{D G D} \\
Ucichł na Krymie carski gród. & \textbf{D A D} \\
Gdy na Zachodzie działa grzmiały, & \textbf{D G D} \\
Transporty ludzi szły na Wschód. & \textbf{D A D} \\
& \\
Świat wolny święcił potem tryumf, & \textbf{D G D} \\
Opustoszały nagle fronty; & \textbf{D G D} \\
W kwiatach już prezydenta grób, & \textbf{D G D} \\
A tam transporty i transporty. & \textbf{D A G} \\
\end{longtable}
\newpage
\begin{longtable}{ll}
Czerwony świt się z nocy budzi, & \textbf{A Fis h} \\
Z woli wyborców odszedł Churchill, & \textbf{D G A D} \\
A tam transporty żywych ludzi, & \textbf{A Fis h} \\
A tam obozy długiej śmierci. & \textbf{D G A D} \\
& \\
Nie miejcie więc do Trójcy żalu, & \textbf{d C d} \\
Wyrok historii za nią stał & \textbf{d C d} \\
Opracowany w każdym calu – & \textbf{d C d} \\
Każdy z nich chronił, co już miał. & \textbf{d C d} \\
& \\
Mógł mylić się zwiedziony chwilą – & \textbf{d C d} \\
Nie był Polakiem ani Bałtem… & \textbf{d C d} \\
Tylko ofiary się nie mylą – & \textbf{d C d C} \\
I tak rozumieć trzeba Jałtę! & \textbf{D A D} \\
& \\
\end{longtable}
\clearpage

% --- Źródło: Jesienne_wino.tex ---
\section{Jesienne wino}
\begin{longtable}{ll}
Z brzękiem ostróg wjechałem do miasta & \textbf{a G a G} \\
Pod jesień było, czas złotych liści nastał & \textbf{a F C C} \\
W kieszeni worek srebra, czas do domu & \textbf{d C G a} \\
Wtem za plecami woła głoss: & \textbf{G F a G a G} \\
& \\
\hspace*{2em}\textit{Usiądź razem ze mną, spróbuj mego wina} & \textbf{a C G C} \\
\hspace*{2em}\textit{Z czereśni, wiśni, resztek lata} & \textbf{d C} \\
\hspace*{2em}\textit{Choć jesień się zaczyna} & \textbf{G F} \\
\hspace*{2em}\textit{Tyle tej jesieni jeszcze jest przed nami} & \textbf{a C G C} \\
\hspace*{2em}\textit{Zdążysz wrócić do domu} & \textbf{d C} \\
\hspace*{2em}\textit{Nim noc zawita nad drogami ... hey} & \textbf{G F} \\
& \textbf{a G a G} \\
& \\
Słońce stało w zenicie, bił południowy żar  & \\
A w gardle kurz przebytych dróg  & \\
No co tam, spocznę sobie, przecież nie zaszkodzi  & \\
Do przejścia niedaleką jeszcze drogę mam (a ona kusi)  & \\
& \\
\hspace*{2em}\textit{Usiądź razem ze mną, spróbuj mego wina...}  & \\
& \\
Zbudziłem się w promieniach zachodu  & \\
Pod starą karczmą, co rynek zamyka  & \\
Zabrała moje srebro, duszę i ostrogi  & \\
Zostało pragnienie i tępy głowy ból (i pamięć jej słów) & \textbf{G F F a G a G} \\
& \\
\hspace*{2em}\textit{Usiądź razem ze mną, spróbuj mego wina... || x2}  & \\
\end{longtable}
\clearpage

% --- Źródło: Jest_już_za_późno_nie_jest_za_późno.tex ---
\section{Jest już za późno, nie jest za późno}
\begin{longtable}{ll}
Jeszcze zdążymy w dżungli ludzkości siebie odnaleźć & \textbf{C d C} \\
Tęskność zawrotna przybliża nas & \textbf{F C d G} \\
Zbiegną się wreszcie tory sieroce naszych dwu planet & \textbf{C d C} \\
Cudnie spokrewnią się ciała nam & \textbf{F C d G} \\
& \\
\hspace*{2em}\textit{Jest już za późno! - Nie jest za późno!} & \textbf{e F} \\
\hspace*{2em}\textit{Jest już za późno! - Nie jest za późno!} & \textbf{e F} \\
\hspace*{2em}\textit{Jest już za późno! - Nie jest za późno!} & \textbf{e F d G} \\
& \\
Jeszcze zdążymy tanio wynająć małą mansardę  & \\
Z oknem na rzekę lub też na park  & \\
Z łożem szerokim, piecem wysokim, ściennym zegarem;  & \\
Schodzić będziemy codziennie w świat  & \\
& \\
\hspace*{2em}\textit{Jest już za późno! - Nie jest za późno!} & \textbf{e F} \\
\hspace*{2em}\textit{Jest już za późno! - Nie jest za późno!} & \textbf{e F} \\
\hspace*{2em}\textit{Jest już za późno! - Nie jest za późno!} & \textbf{e F d G} \\
& \\
Jeszcze zdążymy naszą miłością siebie zachwycić  & \\
Siebie zachwycić i wszystko w krąg  & \\
Wojna to będzie straszna, bo czas nas będzie chciał zniszczyć,  & \\
Lecz nam się uda zachwycić go.  & \\
& \\
\hspace*{2em}\textit{Jest już za późno! - Nie jest za późno!} & \textbf{e F} \\
\hspace*{2em}\textit{Jest już za późno! - Nie jest za późno!} & \textbf{e F} \\
\hspace*{2em}\textit{Jest już za późno! - Nie jest za późno!} & \textbf{e F d G} \\
& \\
\end{longtable}
\clearpage

% --- Źródło: Jesteśmy_Jagódki.tex ---
\section{Jesteśmy Jagódki}
\begin{longtable}{ll}
Jesteśmy jagódki, czarne jagódki & \textbf{a d} \\
Mieszkamy w lasach zielonych & \textbf{G C} \\
Oczka mamy czarne, buźki granatowe & \textbf{a d} \\
A raczki zielone i seledynowe & \textbf{E a} \\
& \\
\hspace*{2em}\textit{A kiedy dzień nadchodzi} & \textbf{d G} \\
\hspace*{2em}\textit{Idziemy na jagody} & \textbf{C a} \\
\hspace*{2em}\textit{A nasze czarne serca} & \textbf{d E} \\
\hspace*{2em}\textit{Bija nam radośnie bum ta ra ra bum} & \textbf{E a} \\
& \\
Pójdziemy na jagódki, wysmarujemy brodki & \textbf{a d} \\
Od pasa polowe, trochę na głowę & \textbf{G C} \\
Resztę sobie zjemy, się wysmarujemy & \textbf{a d} \\
I będziemy tańczyć taniec jagodowy & \textbf{E a} \\
& \\
\hspace*{2em}\textit{A kiedy dzień nadchodzi} & \textbf{d G} \\
\hspace*{2em}\textit{Idziemy na jagody} & \textbf{C a} \\
\hspace*{2em}\textit{A nasze czarne serca} & \textbf{d E} \\
\hspace*{2em}\textit{Bija nam radośnie bum ta ra ra bum} & \textbf{E a} \\
& \\
\end{longtable}
\clearpage

% --- Źródło: Jolka_Jolka_pamiętasz.tex ---
\section{\textbf{Jolka, Jolka, pamiętasz}}
\vspace{-\baselineskip}
\textit{Budka Suflera}\\
\begin{longtable}{ll}
Jolka, Jolka pamiętasz lato ze snu, & \textbf{C G a} \\
Gdy pisałaś: Tak mi źle, & \textbf{C G a} \\
Urwij się choćby zaraz, coś ze mną zrób, & \textbf{C G a} \\
Nie zostawiaj tu samej, o nie. & \textbf{C G F} \\
& \\
Żebrząc wciąż o benzynę gnałem przez noc.  & \\
Silnik rzęził ostatkiem sił,  & \\
Aby być znowu w tobie, śmiać się i kląć.  & \\
Wszystko było tak proste w te dni.  & \\
& \\
Dziecko spało za ścianą, czujne jak ptak.  & \\
Niechaj Bóg wyprostuje mu sny.  & \\
Powiedziałaś, że nigdy, że nigdy aż tak,  & \\
Słodkie były jak krew twoje łzy.  & \\
& \\
\hspace*{2em}\textit{Emigrowałem, z \textbf{objęć} twych nad ranem,} & \textbf{e D} \\
\hspace*{2em}\textit{Dzień mnie wyganiał, nocą znów wracałem.} & \textbf{e D} \\
\hspace*{2em}\textit{Dane nam było słońca zaćmienie,} & \textbf{e D} \\
\hspace*{2em}\textit{Następne będzie może za sto lat.} & \textbf{h G A} \\
& \\
Plażą szły zakonnice, a słońce w dół,  & \\
Wciąż spadało nie mogąc spaść.  & \\
Mąż tam w świecie za funtem  & \\
Odkładał funt, na toyotę przepiękną, aż strach. &  \\
& \\
Mąż twój wielbił porządek i pełne szkło,  & \\
Narzeczoną miał kiedyś jak sen.  & \\
Z autobusem Arabów zdradziła go,  & \\
Nigdy już nie był sobą, o nie.  & \\
& \\
\hspace*{2em}\textit{Emigrowałem, z \textbf{ramion} twych nad ranem...} & \\
& \\
W wielkiej żyliśmy wannie i rzadko tak,  & \\
Wypełzaliśmy na suchy ląd.  & \\
Czarodziejka gorzałka  & \\
Tańczyła w nas, meta była o dwa kroki stąd.  & \\
& \\
Nie wiem ciągle dlaczego zaczęło się tak,  & \\
Czemu zgasło, też nie wie nikt.  & \\
Są wciąż różne koło mnie, nie budzę się sam,  & \\
Ale nic nie jest proste w te dni.  & \\
& \\
\hspace*{2em}\textit{Emigrowałem, z \textbf{objęć} twych nad ranem...}  & \\
\end{longtable}
\clearpage

% --- Źródło: Już_rozpaliło_się_ognisko.tex ---
\section{\textbf{Już rozpaliło się ognisko}}
\begin{longtable}{ll}
Już rozpaliło się ognisko, & \textbf{C} \\
Dając nam dobrej wróżby znak. & \textbf{G G7} \\
Siedliśmy wszyscy przy nim blisko, & \textbf{G} \\
Bo w całej Polsce siedzą tak. & \textbf{G7 C} \\
& \\
Siedzą harcerze przy płomieniach, & \textbf{C F} \\
Ciepły blask ognia skupia ich. & \textbf{G} \\
Wszystko co złe to szuka cienia, & \textbf{C} \\
Do światła dobro garnie się. & \textbf{G C} \\
& \\
Mówiłeś druhu komendancie,  & \\
Że zaufanie do nas masz,  & \\
Że wierzysz w nasze szczere chęci,  & \\
Wszak ty harcerskie serca znasz.  & \\
& \\
Warunki tylko warunkami,  & \\
Od dawna wszak słyszymy to.  & \\
Lecz my jesteśmy harcerzami,  & \\
I zwyciężymy wszelkie zło.  & \\
& \\
& \\
\end{longtable}
\newpage
\begin{longtable}{ll}
\end{longtable}
\clearpage

% --- Źródło: Kantyczka_z_lotu_ptaka.tex ---
\section{\textbf{Kantyczka z lotu ptaka}}
\vspace{-\baselineskip}
\textit{Jacek Kaczmarski}\\
\begin{longtable}{ll}
Patrz mój dobrotliwy Boże & \textbf{a} \\
Na swój ulubiony ludek & \textbf{C} \\
Jak wychodzi rano w zboże & \textbf{G} \\
Zginać harde karki z trudem & \textbf{E7} \\
& \\
Patrz, jak schyla się nad prac & \textbf{a} \\
Jak pokornie klęski znosi & \textbf{C} \\
I nie pyta – Po co? Za co? & \textbf{G} \\
Czasem o coś Cię poprosi: & \textbf{E7} \\
& \\
\hspace*{2em}\textit{Ujmij trochę łaski nieba!} & \textbf{a E7 a} \\
\hspace*{2em}\textit{Daj spokoju w zamian, chleba!} & \textbf{C G} \\
\hspace*{2em}\textit{Innym udziel swej miłości!} & \textbf{E7 a} \\
\hspace*{2em}\textit{Nam – sprawiedliwości!} & \textbf{F a E7} \\
& \\
Smuć się, Chryste Panie w chmurze & \textbf{a} \\
Widząc, jak się naród bawi & \textbf{C} \\
Znowu chciałby być przedmurzem & \textbf{G} \\
I w pogańskiej krwi się pławić & \textbf{E7} \\
& \\
Dymią kuźnie i warsztaty & \textbf{a} \\
Lecz nie pracą a – skargami & \textbf{C} \\
Że nie taka, jak przed laty & \textbf{G} \\
Łaska Twoja nad hufcami: & \textbf{E7} \\
& \\
\hspace*{2em}\textit{Siły grożą Ci nieczyste} & \textbf{a E7 a} \\
\hspace*{2em}\textit{Daj nam wsławić się, o Chryste!} & \textbf{C G} \\
\hspace*{2em}\textit{Kalwin, Litwin nam ubliża!} & \textbf{E7 a} \\
\hspace*{2em}\textit{Dźwigniem ciężar Krzyża!} & \textbf{F a E7} \\
& \\
Załam ręce Matko Boska; & \textbf{a} \\
Upadają obyczaje & \textbf{C} \\
Nie pomogła modłom chłosta – & \textbf{G} \\
Młodzież w szranki ciała staje & \textbf{E7} \\
& \\
W nędzy gzi się krew gorąca & \textbf{a} \\
Bez sumienia, bez oddechu & \textbf{C} \\
Po czym z własnych trzewi strząsa & \textbf{G} \\
Niedojrzały owoc grzechu & \textbf{E7} \\
& \\
& \\
\end{longtable}
\newpage
\begin{longtable}{ll}
\hspace*{2em}\textit{Co zbawienie nam, czy piekło!} & \textbf{A E7 a} \\
\hspace*{2em}\textit{Byle życie nie uciekło!} & \textbf{C G} \\
\hspace*{2em}\textit{Jeszcze będzie czas umierać!} & \textbf{E7 a} \\
\hspace*{2em}\textit{Żyjmy tu i teraz!} & \textbf{F a E7} \\
& \\
Grzmijcie gniewem Wszyscy Święci – & \textbf{fis} \\
Handel lud zalewa boży & \textbf{A} \\
Obce kupce i klienci & \textbf{E} \\
W złote wabią go obroże & \textbf{Cis7} \\
& \\
Liczy chciwy Żyd i Niemiec & \textbf{fis} \\
Dziś po ile polska czystość; & \textbf{A} \\
Kupi dusze, kupi ziemię & \textbf{E} \\
I zostawi pośmiewisko… & \textbf{Cis7} \\
& \\
\hspace*{2em}\textit{Co nam hańba, gdy talary} & \textbf{fis Cis7 fis} \\
\hspace*{2em}\textit{Mają lepszy kurs od wiary!} & \textbf{A E} \\
\hspace*{2em}\textit{Wymienimy na walutę} & \textbf{Cis7 fis} \\
\hspace*{2em}\textit{Honor i pokutę!} & \textbf{D fis Cis7} \\
& \\
Jeden naród, tyle kwestii, & \textbf{fis} \\
Wszystkich naraz – nie wysłuchasz – & \textbf{A} \\
Zadumali się Niebiescy & \textbf{E} \\
W imię Ojca, Syna, Ducha… & \textbf{Cis7} \\
& \\
\hspace*{2em}\textit{Co nam hańba, gdy talary} & \textbf{fis Cis7 fis} \\
\hspace*{2em}\textit{Mają lepszy kurs od wiary!} & \textbf{A E} \\
\hspace*{2em}\textit{Wymienimy na walutę} & \textbf{Cis7 fis} \\
\hspace*{2em}\textit{Honor i pokutę!} & \textbf{D fis Cis7} \\
\end{longtable}
\clearpage

% --- Źródło: Karabin.tex ---
\section{Karabin}
\begin{longtable}{ll}
Niech w księgach wiedzy szpera rabin, & \textbf{a E a} \\
Nauka to jest wymysł diabli. & \textbf{G E} \\
Mądrością moją jest karabin & \textbf{C G} \\
I klinga ukochanej szabli. & \textbf{E a} \\
& \\
Nie dbam o szarżę ni o gwiazdki, &  \\
Co kiedyś mi przystroją kołnierz.  & \\
Wy piszcie klechdy i powiastki,  & \\
Ja biję się, jak musi żołnierz.  & \\
& \\
Nie tęsknię do kawiarni gwarnej,  & \\
Gdzie mieszka banda dziwolągów.  & \\
Gardzę zapachem buduarów,  & \\
Gdzie para psoci wśród szezlongów  & \\
& \\
Nie nęcą mnie zaloty babin,  & \\
Kobieta zdradna, bierz ją diabli!  & \\
Kochanką moją jest karabin  & \\
I klinga ukochanej szabli.  & \\
& \\
Niejeden wróg miał na mnie chrapkę,  & \\
A teraz jęczy w piekle na dnie.  & \\
Ze śmiercią igram w ciuciubabkę,  & \\
Więc może wkrótce mnie dopadnie.  & \\
& \\
Ksiądz mnie nie grzebie ani rabin.  & \\
Żołnierza nie czepią się diabli,  & \\
Lecz w grób połóżcie mi karabin  & \\
I klingę ukochanej szabli.  & \\
& \\
& \\
\end{longtable}
\newpage
\begin{longtable}{ll}
\end{longtable}
\clearpage

% --- Źródło: Kartka_z_kalendarza.tex ---
\section{Kartka z kalendarza}
\begin{longtable}{ll}
Jesteś bitwą moją nieskończoną & \textbf{G a} \\
W której ciągle o przytułek walczę & \textbf{C D G} \\
Jesteś drzwiami które otworzyłem  & \\
A potem przycięły mi palce  & \\
& \\
\hspace*{2em}\textit{Bo jesteś kartką z kalendarza}  & \\
\hspace*{2em}\textit{Zagubioną gdzieś pomiędzy szufladami}  & \\
\hspace*{2em}\textit{(szufladami, kredensami i oknami)}  & \\
\hspace*{2em}\textit{I ulicą, na której co dzień}  & \\
\hspace*{2em}\textit{Uciekałem między latarniami}  & \\
& \\
Jesteś mgłą ogromną niezmierzoną  & \\
Ciszą w huku i łoskotem w ciszy  & \\
Jesteś piórem i wyblakłą kartką  & \\
którym i na której dzisiaj piszę  & \\
& \\
\hspace*{2em}\textit{Bo jesteś kartką z kalendarza...}  & \\
& \\
Przyszłaś do mnie a ja nie spostrzegłem  & \\
Dzisiaj tylko mogę mówić byłaś  & \\
Nie wiem czy na jawie wzięłaś mnie za rękę  & \\
Czy jak wszystko ty się tylko śniłaś  & \\
& \\
\hspace*{2em}\textit{Bo jesteś kartką z kalendarza...}  & \\
& \\
& \\
\end{longtable}
\newpage
\begin{longtable}{ll}
\end{longtable}
\clearpage

% --- Źródło: Kaszubskie_noce.tex ---
\section{\textbf{Kaszubskie noce}}
\begin{longtable}{ll}
Kaszubskie noce nad nami lśniące & \textbf{C a} \\
Harcerski krąg, ogniska blask & \textbf{d G} \\
Gdy się tu znajdziesz wtedy zrozumiesz & \textbf{C a} \\
Że wszystko to łączy nas & \textbf{d G} \\
& \\
\hspace*{2em}\textit{To szumi las, kołysze drzewa} & \textbf{C a d G} \\
\hspace*{2em}\textit{Fala jeziora brzegi zalewa} & \textbf{C a d G} \\
\hspace*{2em}\textit{Tu mały Giewont czoło pochyla} & \textbf{C a d G} \\
\hspace*{2em}\textit{A my siedzimy - harcerska brać} & \textbf{C a d G} \\
& \\
I choć Ojczyzna jest tak daleko & \textbf{C a} \\
To nasze myśli wciąż do niej mkną & \textbf{d G} \\
Tam też wśród nocy płoną ogniska & \textbf{C a} \\
Harcerskie piosnki unosi wiatr & \textbf{d G} \\
& \\
\hspace*{2em}\textit{To szumi las, kołysze drzewa...}  & \\
& \\
A gdy drużyna znowu jest razem & \textbf{C a} \\
Płonie ognisko, gitara gra & \textbf{d G} \\
Znów wspominamy pierwsze wyprawy & \textbf{C a} \\
Trudy, radości- wspaniały czas & \textbf{d G} \\
& \\
\hspace*{2em}\textit{To szumi las, kołysze drzewa...}  & \\
& \\
Czy Ty pamiętasz pierwsze ognisko & \textbf{C a} \\
Pierwsze spojrzenie, cudowna noc & \textbf{d G} \\
Gwiazdy na niebie, byłeś tak blisko & \textbf{C a} \\
Mówiłeś wtedy: „Kocham Cię” (A ja nie!) & \textbf{d G} \\
& \\
\hspace*{2em}\textit{To szumi las, kołysze drzewa...}  & \\
& \\
\end{longtable}
\clearpage

% --- Źródło: Kiler.tex ---
\section{\textbf{Kiler}}
\vspace{-\baselineskip}
\textit{Elektryczne Gitary}\\
\begin{longtable}{ll}
To, co się dzieje, naprawdę nie istnieje & \textbf{D} \\
Więc nie warto mieć niczego, tylko karmić zmysły & \textbf{e A D} \\
Będzie, co ma być, już wiem, że stąd nie zwieję & \textbf{D} \\
Poczekam i popatrzę, nie cofnę kijem Wisły & \textbf{e A D} \\
& \\
\hspace*{2em}\textit{Już tylko Kiler, o sobie tylko tyle} & \textbf{h fis e A} \\
\hspace*{2em}\textit{Wiem, co za ile, nie muszę dbać o bilet} & \textbf{h fis e A} \\
\hspace*{2em}\textit{Mam wszystko w tyle, są czasem takie chwile} & \textbf{h fis e A} \\
\hspace*{2em}\textit{Że się nie mylę, choć wcale nie wiem ile} & \textbf{h fis e A} \\
& \\
Nie kiwnąłem nawet palcem, by się znaleźć w takiej walce & \textbf{G D A} \\
Teraz w pace swe ostatnie resztki image’u tracę & \textbf{G D A} \\
& \\
Co się ze mną dzieje, naprawdę nie istnieje & \textbf{D} \\
Więc nie warto tak się bronić, tylko lecieć z wiatrem & \textbf{e A D} \\
Poczekam, popatrzę, zrozumiem więcej & \textbf{D} \\
I wtedy wreszcie sam też włączę się do akcji & \textbf{e A D} \\
& \\
\hspace*{2em}\textit{Już tylko Kiler, o sobie tylko tyle} & \textbf{h fis e A} \\
\hspace*{2em}\textit{Wiem, co za ile, nie muszę dbać o bilet} & \textbf{h fis e A} \\
\hspace*{2em}\textit{Mam wszystko w tyle, są czasem takie chwile} & \textbf{h fis e A} \\
\hspace*{2em}\textit{Że się nie mylę, choć wcale nie wiem ile} & \textbf{h fis e A} \\
& \\
Już tylko Kiler, podniosłem bilę & \textbf{h fis} \\
Wracam za chwilę, nie dbam o bagaż, nie dbam o bilet & \textbf{e A} \\
Już tylko Kiler, mówię ooo… & \textbf{h fis e A} \\
Mam wszystko w tyle, wiem, co za ile, może się mylę, & \textbf{h fis e A} \\
to chyba thriller, aj-aj-aj-aj-aj-aj-aj-aj & \textbf{h fis e A} \\
Już tylko Kiler...  & \\
& \\
& \\
\end{longtable}
\newpage
\begin{longtable}{ll}
\end{longtable}
\clearpage

% --- Źródło: King.tex ---
\section{King}
\vspace{-\baselineskip}
\textit{T.Love}\\
\begin{longtable}{ll}
Mówiono o nim King & \textbf{e} \\
W mieście świętej wieży & \textbf{e} \\
Pamiętam z podstawówki & \textbf{G} \\
Jak całował się z papieżem & \textbf{G} \\
& \\
Przejeżdżał też sekretarz & \textbf{e} \\
Gdy przecinano wstęgę & \textbf{e} \\
Kingi poszedł na wagary & \textbf{G} \\
Pomarzyć o czymś innym & \textbf{G} \\
& \\
\hspace*{2em}\textit{Był zawsze trochę z boku} & \textbf{a e} \\
\hspace*{2em}\textit{Na bakier trochę był} & \textbf{a e} \\
\hspace*{2em}\textit{W szkole nikt nie wiedział} & \textbf{a e} \\
\hspace*{2em}\textit{Czym King naprawdę żył} & \textbf{H7} \\
& \\
To było trochę później & \textbf{e} \\
Już miał przyjaciółkę Ewę & \textbf{e} \\
Mieszkali więc bez ślubu & \textbf{G} \\
I klepali słodką biedę & \textbf{G} \\
& \\
Dawali czasem czadu & \textbf{e} \\
Bo lubili lekkie dragi & \textbf{e} \\
Znajomych było wielu & \textbf{G} \\
Wieczory i poranki & \textbf{G} \\
& \\
\hspace*{2em}\textit{Uważaj na sąsiadów swych} & \textbf{a e} \\
\hspace*{2em}\textit{Bo lubią dawać cynk} & \textbf{a e} \\
\hspace*{2em}\textit{Ty wiesz, kto rządzi w mieście} & \textbf{a e} \\
\hspace*{2em}\textit{Tu biskup z komisarzem — King!} & \textbf{H7} \\
& \\
Tak mówił mu przyjaciel & \textbf{e} \\
Długi, chudy Lolo & \textbf{e} \\
Kiedy wyszli na ulicę & \textbf{G} \\
Zapalić spliffa z colą & \textbf{G} \\
& \\
Mam dosyć tego miasta & \textbf{e} \\
Czerwono-czarnej mafii & \textbf{e} \\
Czy mnie rozumiesz Lolo? & \textbf{G} \\
Czy wiesz, co mnie trapi? & \textbf{G} \\
& \\
& \\
\end{longtable}
\newpage
\begin{longtable}{ll}
\hspace*{2em}\textit{Tymczasem blada Ewa} & \textbf{a e} \\
\hspace*{2em}\textit{Wytłumaczyć pragnie wszystko} & \textbf{a e} \\
\hspace*{2em}\textit{Bo komisarz wszedł przez okno} & \textbf{a e} \\
\hspace*{2em}\textit{A spod łóżka wylazł biskup} & \textbf{H7} \\
& \\
Co masz w kieszeni King? & \textbf{e} \\
Komisarz spytał w drzwiach & \textbf{e} \\
Wy palicie wciąż to świństwo & \textbf{G} \\
Mieliśmy wiele skarg & \textbf{G} \\
& \\
A biskup łypie z boku & \textbf{e} \\
To na Kinga, to na Ewę & \textbf{e} \\
Wy żyjecie tu bezbożnie & \textbf{G} \\
Myślicie, że nic nie wiem & \textbf{G} \\
& \\
\hspace*{2em}\textit{Za posiadanie zielska} & \textbf{a e} \\
\hspace*{2em}\textit{Ty dostaniesz dziesięć latek} & \textbf{a e} \\
\hspace*{2em}\textit{Za nielegalny związek z nią} & \textbf{a e} \\
\hspace*{2em}\textit{Następnych parę kratek} & \textbf{H7} \\
& \\
Dziś Kingi siedzi w celi & \textbf{e} \\
I wspomina dobre dni & \textbf{e} \\
Napisał do papieża & \textbf{G} \\
Bardzo długi list & \textbf{G} \\
& \\
Świąteczną wysłał kartkę & \textbf{e} \\
Do samego prezydenta & \textbf{e} \\
Lecz nikt o nim już nie mówi & \textbf{G} \\
Nikt o nim nie pamięta & \textbf{G} \\
& \\
\hspace*{2em}\textit{Był zawsze trochę z boku} & \textbf{a e} \\
\hspace*{2em}\textit{Na bakier trochę był} & \textbf{a e} \\
\hspace*{2em}\textit{Właściwie nikt nie wiedział} & \textbf{a e} \\
\hspace*{2em}\textit{Czym King naprawdę żył} & \textbf{H7} \\
\end{longtable}
\clearpage

% --- Źródło: Kocham_Cię_jak_Irlandię.tex ---
\section{\textbf{Kocham Cię jak Irlandię}}
\vspace{-\baselineskip}
\textit{Kobranocka}\\
\begin{longtable}{ll}
Znikałaś gdzieś w domu nad Wisłą, & \textbf{C e} \\
Pamiętam to tak dokładnie, & \textbf{a d} \\
Twoich czarnych oczu bliskość, & \textbf{B F} \\
Wciąż kocham cię jak Irlandię. & \textbf{C G} \\
& \\
\hspace*{2em}\textit{A ty się temu nie dziwisz,} & \textbf{C e} \\
\hspace*{2em}\textit{Wiesz dobrze, co byłoby dalej,} & \textbf{a d} \\
\hspace*{2em}\textit{Jak byśmy byli szczęśliwi,} & \textbf{B F} \\
\hspace*{2em}\textit{Gdybym nie kochał cię wcale.} & \textbf{C G C} \\
& \\
Przed szczęściem żywić obawę & \textbf{C e} \\
Z nadzieją, że mi ją skradniesz, & \textbf{a d} \\
Wlokę ten ból przez Włocławek, & \textbf{B F} \\
Kochając cię jak Irlandię. & \textbf{C G} \\
& \\
\hspace*{2em}\textit{A ty się temu nie dziwisz…}  & \\
& \\
Gdzieś na ulicy Fabrycznej & \textbf{C e} \\
Spotkać nam się wypadnie, & \textbf{a d} \\
Lecz takie są widać wytyczne, & \textbf{B F} \\
By kochać cię jak Irlandię. & \textbf{C G} \\
& \\
\hspace*{2em}\textit{A ty się temu nie dziwisz…}  & \\
& \\
Czy mi to kiedyś wybaczysz? & \textbf{C e} \\
Działałem tak nieporadnie. & \textbf{a d} \\
Czy to dla ciebie coś znaczy, & \textbf{B F} \\
Że kocham cię jak Irlandię? & \textbf{C G} \\
& \\
\hspace*{2em}\textit{A ty się temu nie dziwisz…}  & \\
\end{longtable}
\clearpage

% --- Źródło: Koniec.tex ---
\section{Koniec}
\vspace{-\baselineskip}
\textit{Elektryczne Gitary}\\
\begin{longtable}{ll}
\textbf{G D e C} \\ & \\
\hspace*{2em}\textit{To już jest koniec nie ma już nic} &  \\
\hspace*{2em}\textit{Jesteśmy wolni możemy iść}  & \\
\hspace*{2em}\textit{To już jest koniec możemy iść}  & \\
\hspace*{2em}\textit{Jesteśmy wolni bo nie ma już nic}  & \\
& \\
Robaczek w swej dziurce jak docent za biurkiem  & \\
I pszczółka na kwiatkach jak kontrol w tramwajach  & \\
Tak dłubie i gmera napisze wymyśli  & \\
Obejdzie w około zabrudzi wyczyści  & \\
& \\
I krzaczek przy drodze i brat przy maszynie  & \\
Jak noga w skarpecie sprzedawca w kantynie  & \\
Kamyczek na polu i strażnik na straży  & \\
Lodówka wciąż ziębi kuchenka wciąż parzy  & \\
& \\
A po co a po co tak dłubie i dłubie  & \\
A za co a za co tak myśli i skubie  & \\
I tak się przykłada i mówi z ekranu  & \\
I bredzi latami wieczorem i rano  & \\
& \\
\hspace*{2em}\textit{To już jest koniec nie ma już nic}  & \\
\hspace*{2em}\textit{jesteśmy wolni możemy iść}  & \\
\hspace*{2em}\textit{to już jest koniec możemy iść}  & \\
\hspace*{2em}\textit{jesteśmy wolni bo nie ma już nic}  & \\
& \\
\end{longtable}
\clearpage

% --- Źródło: Krajka.tex ---
\section{\textbf{Krajka}}
\begin{longtable}{ll}
Chorałem dzwonków dzień rozkwita, & \textbf{a E} \\
Jeszcze od rosy rzęsy mokre. & \textbf{a F} \\
We mgle turkoce pierwsza bryka, & \textbf{C d} \\
Słońce wyrusza na włóczęgę. & \textbf{a E} \\
& \\
\hspace*{2em}\textit{Drogą pylistą, drogą polną,} & \textbf{a E} \\
\hspace*{2em}\textit{Jak kolorowa panny krajka,} & \textbf{a F} \\
\hspace*{2em}\textit{Słońce się wznosi nad stodołę,} & \textbf{C d} \\
\hspace*{2em}\textit{Będzie tańczyć walca...} & \textbf{a E E7} \\
& \\
\hspace*{2em}\textit{A ja mam swoją gitarę,} & \textbf{d G} \\
\hspace*{2em}\textit{Spodnie wytarte i buty stare,} & \textbf{C a} \\
\hspace*{2em}\textit{Wiatry niosą mnie… Na skrzydłach!} & \textbf{d E a A7} \\
& \\
\hspace*{2em}\textit{A ja mam swoją gitarę,} & \textbf{d G} \\
\hspace*{2em}\textit{Spodnie wytarte i buty stare,} & \textbf{C a} \\
\hspace*{2em}\textit{Wiatry niosą mnie… (nie wiadomo gdzie)} & \textbf{d E a E a} \\
& \\
Zmoknięte świerszcze stroją skrzypce, & \textbf{a E} \\
\smash{Żuraw} się wsparł o cembrowinę, & \textbf{a F} \\
Wiele nanosi wody jeszcze, & \textbf{C d} \\
Wielu się ludzi z niej napije. & \textbf{a E} \\
& \\
\hspace*{2em}\textit{Drogą pylistą, drogą polną,} & \textbf{a E} \\
\hspace*{2em}\textit{Jak kolorowa panny krajka,} & \textbf{a F} \\
\hspace*{2em}\textit{Słońce się wznosi nad stodołę,} & \textbf{C d} \\
\hspace*{2em}\textit{Będziemy tańczyć walca…} & \textbf{a E E7} \\
& \\
\hspace*{2em}\textit{A ja mam swoją gitarę,} & \textbf{d G} \\
\hspace*{2em}\textit{Spodnie wytarte i buty stare,} & \textbf{C a} \\
\hspace*{2em}\textit{Wiatry niosą mnie… Na skrzydłach!} & \textbf{d E a A7} \\
& \\
\hspace*{2em}\textit{A ja mam swoją gitarę,} & \textbf{d G} \\
\hspace*{2em}\textit{Spodnie wytarte i buty stare,} & \textbf{C a} \\
\hspace*{2em}\textit{Wiatry niosą mnie… (nie wiadomo gdzie)} & \textbf{d E a E a} \\
& \\
& \\
\end{longtable}
\newpage
\begin{longtable}{ll}
\end{longtable}
\clearpage

% --- Źródło: Lekcja_historii_klasycznej.tex ---
\section{Lekcja historii klasycznej}
\vspace{-\baselineskip}
\textit{Jacek Kaczmarski}\\
\begin{longtable}{ll}
„Gallia est omnis divisa in partes tres & \textbf{C G} \\
Quarum unam incolunt Belgae aliam Aquitani & \textbf{d E} \\
Tertiam qui ipsorum lingua Celtae nostra Galli appellantur & \textbf{a F} \\
Ave Caesar morituri te salutant!” & \textbf{F C G C} \\
& \\
Nad Europą twardy krok legionów grzmi & \textbf{C G} \\
Nieunikniony wróży koniec republiki & \textbf{d E} \\
Gniją wzgórza galijskie w pomieszanej krwi & \textbf{a F} \\
A Juliusz Cezar pisze swoje pamiętniki & \textbf{F C G C} \\
& \\
Gallia est omnis divisa in partes tres  & \\
Quarum unam incolunt Belgae aliam Aquitani  & \\
Tertiam qui ipsorum lingua Celtae nostra Galli appellantur  & \\
Ave Caesar morituri te salutant  & \\
& \\
Pozwól Cezarze gdy zdobędziemy cały świat  & \\
Gwałcić rabować sycić wszelkie pożądania  & \\
Proste prośby żołnierzy te same są od lat  & \\
A Juliusz Cezar milcząc zabaw nie zabrania  & \\
& \\
Gallia est omnis divisa in partes tres  & \\
Quarum unam incolunt Belgae aliam Aquitani  & \\
Tertiam qui ipsorum lingua Celtae nostra Galli appellantur  & \\
Ave Caesar morituri te salutant  & \\
& \\
Cywilizuje podbite narody nowy ład  & \\
Rosną krzyże przy drogach od Renu do Nilu  & \\
Skargą krzykiem i płaczem rozbrzmiewa cały świat  & \\
A Juliusz Cezar ćwiczy lapidarność stylu  & \\
& \\
Gallia est omnis divisa in partes tres  & \\
Quarum unam incolunt Belgae aliam Aquitani  & \\
Tertiam qui ipsorum lingua Celtae nostra Galli appellantur  & \\
& \\
Ave Caesar morituri te salutant  & \\
Ave Caesar morituri te salutant  & \\
Ave Caesar morituri te salutant  & \\
\end{longtable}
\clearpage

% --- Źródło: Lewe_Lewe_Loff.tex ---
\section{\textbf{Lewe Lewe Loff}}
\vspace{-\baselineskip}
\textit{Kult}\\
\begin{longtable}{ll}
Chcę ci powiedzieć jak bardzo cię cenię & \textbf{a C} \\
chcę ci powiedzieć jak bardzo cię podziwiam & \textbf{G D} \\
chcę ci powiedzieć: uważaj na te drogi  & \\
ale nie mam odwagi…  & \\
& \\
Jest czwarta w nocy. Piszę przez chwilę  & \\
to co mi się we łbie ułożyło.  & \\
Chciałbym, chociaż za oknem wiatr dmucha  & \\
Zanucić ci prosto do ucha.  & \\
& \\
\hspace*{2em}\textit{lewe lewe lewe loff loff loff loff}  & \\
\hspace*{2em}\textit{lewe lewe lewe loff loff loff loff}  & \\
\hspace*{2em}\textit{lewe lewe lewe loff loff loff loff}  & \\
\hspace*{2em}\textit{lewe lewe lewe lewe lewe lewe}  & \\
& \\
Ty masz to co ja chciałbym mieć  & \\
gdybym kilka lat mniej miał  & \\
i tylko chcę cię ostrzec:  & \\
nie wyważaj drzwi otwartych na oścież  & \\
& \\
ty masz taką mądrość głupią  & \\
niech której wszyscy od ciebie się uczą  & \\
i tylko chcę ci powiedzieć  & \\
ten pociąg nie pojedzie jeśli ty w nim nie będziesz  & \\
& \\
\hspace*{2em}\textit{lewe lewe lewe loff loff loff loff…}  & \\
& \\
Przed chwilą o tym śniłem,  & \\
że na jakimś dworcu wszystko zostawiłem.  & \\
Niewiadomy niepokój obudził mnie  & \\
Dlatego teraz siedzę i piszę,  & \\
ale żadne słowa tego nie opiszą  & \\
Co poczuć może człowiek ciemną jesienną nocą  & \\
dlatego już kończę ten list  & \\
listopad 1993  & \\
& \\
\hspace*{2em}\textit{lewe lewe lewe loff loff loff loff...}  & \\
\end{longtable}
\clearpage

% --- Źródło: Lewe_Lewe_Loff_Jak_zapomnieć.tex ---
\section{Jak zapomnieć}
\vspace{-\baselineskip}
\textit{Jeden Osiem L}\\
\begin{longtable}{ll}
\hspace*{2em}\textit{Ile dałbym, by zapomnieć Cię,} & \textbf{a C} \\
\hspace*{2em}\textit{Wszystkie chwile te, które są na nie,} & \textbf{G D} \\
\hspace*{2em}\textit{Bo chcę (bo chcę) nie myśleć o tym już,}  & \\
\hspace*{2em}\textit{Zdmuchnąć wszystkie wspomnienia niczym zaległy kurz,}  & \\
\hspace*{2em}\textit{Tak już (tak już)}  & \\
\hspace*{2em}\textit{Po prostu nie pamiętać sytuacji, w których serce klęka,}  & \\
\hspace*{2em}\textit{Wiem, nie wyrwę się, chociaż bardzo chcę,}  & \\
\hspace*{2em}\textit{Mam nadzieję, że to wiesz i Ty.}  & \\
& \\
Znowu widzę Ciebie przed swoimi oczami,  & \\
Znowu zasnąć nie mogę, owładnięty marzeniami ,  & \\
Wszystko poświęcam myśli, że byłaś kiedyś blisko,  & \\
Kiedy czułem Ciebie obok, wtedy czułem, że mam wszystko,  & \\
Tyle zostało po mnie, tylko Ty i setki wspomnień,  & \\
Ile dałbym za to, by móc o tym już zapomnieć,  & \\
Teraz nie ma Nas i nie chcę być tam gdzie Ty jesteś,  & \\
Znowu staniesz przede mną, zawsze robisz mi to we śnie,  & \\
Będę patrzył jak odchodzisz, chociaż chciałbym się odwrócić,  & \\
Będę myślał ile dałbym komuś kto by czas zawrócił,  & \\
Kto by zatrzymał wskazówki, tylko na ten jeden moment,  & \\
W chwili, w której Cię poznałem poszedłbym już w drugą stronę.  & \\
& \\
\hspace*{2em}\textit{Ile dałbym, by zapomnieć Cię...}  & \\
& \\
To był sen na jawie, gdy marzenia się spełniały,  & \\
Wszystko takie realne, chwile szybko tak mijały,  & \\
Tylko my, zamknięci w czterech ścianach, a tak wolni,  & \\
Ważna Ty byłaś obok, a ja czułem się spokojny,  & \\
Pamiętasz jeszcze? Te dni, całe miesiące,  & \\
Pamiętasz? Chcesz zapomnieć? Ja nie mogę, wiem, że błądzę,  & \\
Snute kiedyś opowiastki, ja, Ty i srebrna taca,  & \\
Kiedyś to nie przerażało, już do tego nie chcę wracać,  & \\
Aura zepsucia w powietrzu, tracisz te 50 procent,  & \\
Chcę zapomnieć o Tobie, zatrzeć w pamięci te noce,  & \\
By odeszły w niepamięć, chwile, które zwałem złotem,  & \\
Tamte chwile to tombak, bo już wiem co było potem.  & \\
& \\
\hspace*{2em}\textit{Ile dałbym, by zapomnieć Cię...}  & \\
\end{longtable}
\newpage
\begin{longtable}{ll}
Moje myśli spiętrzone wokół jednej chwili,  & \\
Kiedyś ta krótka potrafiła czas umilić,  & \\
Teraz stojąc jakby obok wciąż się przyglądam,  & \\
Już nie cieszy jak kiedyś, wspominam, myślę dokąd zdążam,  & \\
Inne cele w życiu, inne plany i pragnienia,  & \\
Muszę wszystko pozmieniać, tak jak czas wszystko zmienia,  & \\
To co było nie wróci, wiem, choć czasem mam nadzieję,  & \\
Po co mam więc pamiętać, ktoś by powiedział "stare dzieje",  & \\
Wiem to, nie mogę zapomnieć jak było dobrze,  & \\
Wiem to, skończyło się, mój własny pogrzeb,  & \\
Wiem to, i proszę Boga, nigdy więcej,  & \\
Niech nie pozwoli na to, by ktoś trafił w moje serce.  & \\
& \\
\hspace*{2em}\textit{Ile dałbym, by zapomnieć Cię...}  & \\
\end{longtable}
\clearpage

% --- Źródło: Lipka.tex ---
\section{\textbf{Lipka}}
\begin{longtable}{ll}
Z tamtej strony jeziora & \textbf{a E a} \\
Stoi lipka zielona & \textbf{C G E} \\
A na tej lipce, na tej zieloniutkiej & \textbf{a G a G} \\
Trzej ptaszkowie śpiewają & \textbf{F E a} \\
A na tej lipce, na tej zieloniutkiej & \textbf{a G a E} \\
Trzej ptaszkowie śpiewają & \textbf{F G a} \\
& \\
Nie byli to ptaszkowie  & \\
Tylko trzej braciszkowie  & \\
Co się spierali o jedną dziewczynę  & \\
Który ci ją dostanie  & \\
Co się spierali o jedną dziewczynę  & \\
Który ci ją dostanie  & \\
& \\
Jeden mówi: „Tyś moja”  & \\
Drugi mówi: „Jak Bóg da”  & \\
A trzeci mówi: „Moja najmilejsza,  & \\
Czemu Tyś mi taka smutna?”  & \\
A trzeci mówi: „Moja najmilejsza,  & \\
Czemuś Tyś mi taka smutna?”  & \\
& \\
„Jakże nie mam smutna być?  & \\
Za starego każą iść  & \\
Czasu niewiele  & \\
Jeszcze dwie niedziele  & \\
Mogę, miły, z Tobą być!”  & \\
Czasu niewiele  & \\
Jeszcze dwie niedziele  & \\
Mogę, miły, z Tobą być!”  & \\
& \\
Z tamtej strony jeziora  & \\
Stoi lipka zielona  & \\
A na tej lipce, na tej zieloniutkiej  & \\
Trzej ptaszkowie śpiewają  & \\
A na tej lipce, na tej zieloniutkiej  & \\
Trzej ptaszkowie śpiewają  & \\
& \\
& \\
\end{longtable}
\newpage
\begin{longtable}{ll}
\end{longtable}
\clearpage

% --- Źródło: List_do_Boga.tex ---
\section{List do Boga}
\begin{longtable}{ll}
Drogi Boże piszę chociaż kilka słów, & \textbf{G D} \\
innym razem napiszę więcej. & \textbf{C G} \\
Na początku życzę Ci wszystkiego dobrego & \textbf{e h} \\
i pozdrawiam Cię najgoręcej. & \textbf{C D} \\
Tak się jakoś złożyło, że nie miałam okazji & \textbf{G D} \\
podziękować za list coś mi przysłał. & \textbf{C G} \\
Miałam wiele pracy, miałam wiele nauki, & \textbf{e h} \\
także piszę dopiero dzisiaj. & \textbf{C D} \\
& \\
\hspace*{2em}\textit{U mnie wszystko jak dawniej} & \textbf{G D} \\
\hspace*{2em}\textit{tylko jeden samobójca więcej,} & \textbf{C G} \\
\hspace*{2em}\textit{tylko jedna znów rodzina rozbita,} & \textbf{e h} \\
\hspace*{2em}\textit{tylko życie pędzi coraz prędzej.} & \textbf{C D} \\
\hspace*{2em}\textit{Gdzieś obok rozbił się samolot,} & \textbf{G D} \\
\hspace*{2em}\textit{trochę dalej trzęsła się ziemia.} & \textbf{C G} \\
\hspace*{2em}\textit{Kiedy patrzę na to wszystko tak, jak dziś} & \textbf{e h} \\
\hspace*{2em}\textit{Nie wiem czy długo wytrzymam.} & \textbf{C D} \\
& \\
Tak w ogóle to przepraszam Cię bardzo  & \\
za to, że tak długo milczałam,  & \\
lecz dopiero dzisiaj zaczynam rozumieć  & \\
Biblię, którą mi przysłałeś.  & \\
Wczoraj odszedł ode mnie przyjaciel,  & \\
z którym tak wiele mnie łączyło.  & \\
I dopiero dzisiaj zaczynam doceniać  & \\
czym jest życie i prawdziwa miłość.  & \\
& \\
\hspace*{2em}\textit{U mnie wszystko jak dawniej...}  & \\
& \\
U mnie wszystko jak dawniej,  & \\
tylko świat jest mniej kolorowy,  & \\
tylko życie pędzi coraz prędzej,  & \\
tylko ludzie szybciej tracą głowy.  & \\
Gdzieś obok rozbił się samolot,  & \\
trochę dalej trzęsła się ziemia.  & \\
Kiedy patrzę na to wszystko tak jak dziś  & \\
Nie wiem czy długo wytrzymam.  & \\
& \\
\hspace*{2em}\textit{U mnie wszystko jak dawniej ...}  & \\
\end{longtable}
\clearpage

% --- Źródło: List_do_M.tex ---
\section{List do M.}
\vspace{-\baselineskip}
\textit{Dżem}\\
\begin{longtable}{ll}
Mamo piszę do Ciebie wiersz, & \textbf{a G F D2} \\
Może ostatni, na pewno pierwszy.  & \\
Jest głęboka, ciemna noc,  & \\
Siedzę w łóżku a obok śpi ona  & \\
& \\
I tak spokojnie oddycha.  & \\
Dobiega mnie jakaś muzyka,  & \\
Nie, to tylko w mej głowie szum.  & \\
Siedzę, tonę i tonę we łzach,  & \\
Bo jest mi smutno, bo jestem sam.  & \\
Dławi mnie strach !  & \\
& \\
\hspace*{2em}\textit{Samotność to taka straszna trwoga,} & \textbf{d} \\
\hspace*{2em}\textit{Ogarnia mnie, przenika mnie.} & \textbf{a} \\
\hspace*{2em}\textit{Wiesz mamo, wyobraziłem sobie, że} & \textbf{G} \\
\hspace*{2em}\textit{Że nie ma Boga, nie ma nie! Nie ma Boga, nie.} & \textbf{d G} \\
& \\
\textbf{a G F D2}  & \\
& \\
Spokojny jest tylko mój dom,  & \\
Gdzie Ty jesteś a mnie tam nie ma.  & \\
Gdzie nie wrócę już chyba, chyba nie,  & \\
Mamo bardzo Cię kocham, kocham Cię!  & \\
& \\
Myślałem, że Ty skrzywdziłaś mnie,  & \\
A to ja, skrzywdziłem Ciebie.  & \\
Szkoda, że tak późno pojąłem to  & \\
tak późno to zrozumiałem, zrozumiałem to  & \\
& \\
\hspace*{2em}\textit{Samotność to taka straszna trwoga...}  & \\
\end{longtable}
\clearpage

% --- Źródło: Lubię_mówić_z_Tobą.tex ---
\section{\textbf{Lubię mówić z Tobą}}
\vspace{-\baselineskip}
\textit{Akurat}\\
\begin{longtable}{ll}
Kiedy z serca płyną słowa & \textbf{a C} \\
Uderzają z wielką mocą & \textbf{G a} \\
Krążą blisko wśród nas ot tak & \textbf{a C} \\
Dając chętnym szczere złoto & \textbf{G a} \\
& \\
\hspace*{2em}\textit{I dlatego lubię mówić z Tobą} & \textbf{a C e a} \\
\hspace*{2em}\textit{I dlatego lubię mówić z Tobą} & \textbf{a C e a} \\
& \\
Każdy myśli to co myśli & \textbf{a C} \\
Myśli sobie moja głowa & \textbf{G a} \\
Może w końcu mi się uda & \textbf{a C} \\
Wypowiedzieć proste słowa & \textbf{G a} \\
& \\
\hspace*{2em}\textit{I dlatego lubię mówić z Tobą} & \textbf{a C e a} \\
\hspace*{2em}\textit{I dlatego lubię mówić z Tobą} & \textbf{a C e a} \\
\end{longtable}
\clearpage

% --- Źródło: Majster_Bieda.tex ---
\section{\textbf{Majster Bieda}}
\vspace{-\baselineskip}
\textit{Wolna Grupa Bukowina}\\
\begin{longtable}{ll}
Skąd przychodził kto go znał & \textbf{D G} \\
Kto mu rękę podał kiedy & \textbf{D G A} \\
Nad rowem siadał wyjmował chleb & \textbf{D A} \\
Serem przekładał i dzielił się z psem & \textbf{fis h} \\
Tyle wszystkiego co sobą miał & \textbf{A G fis e} \\
Majster Bieda & \textbf{A D} \\
& \\
& \textbf{D G fis e A D} \\
& \\
Czapkę z głowy ściągał gdy  & \\
Wiatr gałęzie chylił drzewom  & \\
Śmiał się do słońca i śpiewał do gwiazd  & \\
Drogą bez końca co przed nim szła  & \\
Znał jak pięć palców jak szeląg zły  & \\
Majster Bieda  & \\
& \\
Nikt nie pytał skąd się wziął  & \\
Gdy do ognia się przysiadał  & \\
Wtulał się w krąg ciepła jak w kożuch  & \\
Zmęczony drogą wędrowiec boży  & \\
Zasypiał długo gapiąc się w noc  & \\
Majster Bieda  & \\
& \\
Aż nastąpił taki rok & \textbf{D G} \\
Smutny rok tak widać trzeba & \textbf{D D7 G A} \\
Nie przyszedł Bieda zieloną wiosną & \textbf{D A} \\
Miejsce gdzie siadał zielskiem zarosło & \textbf{fis h} \\
I choć niejeden wytężał wzrok & \textbf{A G} \\
& \\
Choć lato pustym gościńcem przeszło & \textbf{A G} \\
Z rudymi liśćmi jesienną schedą & \textbf{A G} \\
Wiatrem niesiony popłynął w przeszłość & \textbf{A G} \\
Wiatrem niesiony popłynął w przeszłość & \textbf{A G} \\
Wiatrem niesiony popłynął w przeszłość & \textbf{A G A} \\
Majster Bieda & \textbf{D} \\
& \\
& \textbf{D G fis e A D} \\
\end{longtable}
\clearpage

% --- Źródło: Marco_Polo.tex ---
\section{Marco Polo}
\begin{longtable}{ll}
Nasz „Marco Polo” to dzielny ship & \textbf{e G D e} \\
Największe fale brał & \textbf{e G} \\
W Australii będąc widziałem go & \textbf{C e G D} \\
Gdy w porcie przy kei stał & \textbf{e D e} \\
I urzekł mnie tak urodą swą & \textbf{e G D e} \\
Że zaciągnąłem się & \textbf{e G} \\
I powiał wiatr w dali zniknął ląd & \textbf{C e G D} \\
Mój dom i Australii brzeg & \textbf{e D e} \\
& \\
\hspace*{2em}\textit{„Marco Polo”} & \textbf{e D C H7} \\
\hspace*{2em}\textit{w królewskich liniach był} & \textbf{e D e} \\
\hspace*{2em}\textit{„Marco Polo”} & \textbf{e D C H7} \\
\hspace*{2em}\textit{tysiące przebył mil} & \textbf{e D e} \\
& \\
Na jednej z wysp za korali sznur  & \\
Tubylec złoto dał  & \\
I poszli wszyscy w ten dziki kraj  & \\
Bo złoto mieć każdy chciał  & \\
I wielkie szczęście spotkało tych  & \\
Co wyszli na ten brzeg  & \\
Bo pełne złota ładownie są  & \\
I każdy bogaczem jest  & \\
& \\
\hspace*{2em}\textit{„Marco Polo”}  & \\
\hspace*{2em}\textit{w królewskich liniach był}  & \\
\hspace*{2em}\textit{„Marco Polo”}  & \\
\hspace*{2em}\textit{tysiące przebył mil}  & \\
& \\
W powrotnej drodze tak szalał sztorm  & \\
Że drzazgi poszły z rej  & \\
A statek wciąż burtą wodę brał  & \\
Do dna było coraz mniej  & \\
Ładunek cały trza było nam  & \\
Do morza wrzucić tu  & \\
Do lądu dojść i biedakiem być  & \\
Ratować choć żywot swój  & \\
& \\
\hspace*{2em}\textit{„Marco Polo”}  & \\
\hspace*{2em}\textit{w królewskich liniach był}  & \\
\hspace*{2em}\textit{„Marco Polo”}  & \\
\hspace*{2em}\textit{tysiące przebył mil}  & \\
\end{longtable}
\clearpage

% --- Źródło: Mały_Obóz.tex ---
\section{Mały Obóz}
\begin{longtable}{ll}
Kiedy rankiem ze skowronkiem & \textbf{C} \\
Powitamy nowy dzień & \textbf{e} \\
Rosy z trawy się napijesz & \textbf{A A7} \\
Pierwszy promień słońca zjesz & \textbf{d} \\
Potem wracać trzeba będzie & \textbf{F f} \\
Pożegnamy rzekę las & \textbf{C a} \\
Bądźcie zdrowi nasi bracia & \textbf{d} \\
Bądźcie zdrowi na nas czas & \textbf{G} \\
& \\
\hspace*{2em}\textit{Ustawimy mały obóz}  & \\
\hspace*{2em}\textit{Bramę zbudujemy z serc}  & \\
\hspace*{2em}\textit{A z tych dusz co tak gorące}  & \\
\hspace*{2em}\textit{Zbudujemy sobie piec} &  \\
\hspace*{2em}\textit{Rozpalimy mały ogień}  & \\
\hspace*{2em}\textit{A w tym ogniu będziesz piekł}  & \\
\hspace*{2em}\textit{Naszą przyjaźń która łączy}  & \\
\hspace*{2em}\textit{Która da ci to co chcesz}  & \\
& \\
Hej my jeszcze tu wrócimy  & \\
Nie za rok no to za dwa  & \\
Więc dlaczego płacze rzeka  & \\
Więc dlaczego szumi las  & \\
Wszak przyjaźni naszej wielkiej  & \\
Nie rozerwie piorun zła  & \\
Ona mocna jest bezczelnie  & \\
Więc my wszyscy jeszcze raz  & \\
& \\
\hspace*{2em}\textit{Ustawimy mały obóz...}  & \\
& \\
\end{longtable}
\clearpage

% --- Źródło: Mewy.tex ---
\section{Mewy}
\begin{longtable}{ll}
\textbf{F G a}  & \\
\textbf{a F G}  & \\
& \\
Mewy białe mewy wiatrem rzeźbione z pian & \textbf{a F G a} \\
Skrzydlate białe muzy okrętów odchodzących w dal & \textbf{a F G a} \\
Kto wam szybować każe za horyzontu kres & \textbf{a F G a} \\
W bezimienne oceany przez sztormów święty gniew & \textbf{a F e a} \\
& \\
\hspace*{2em}\textit{Żeglarzom wracającym z morza} & \textbf{F G a} \\
\hspace*{2em}\textit{Na pamięć przywodzicie dom} & \textbf{F G a} \\
\hspace*{2em}\textit{Rozbitkom wasze skrzydła niosą} & \textbf{F G C F} \\
\hspace*{2em}\textit{Nadzieję na zbawienny ląd} & \textbf{F e a} \\
& \\
Ptaki zapamiętane jeszcze z dziecięcych lat  & \\
Drapieżnie spadające ze skał na szary Skagerrak  & \\
Wiatr w grzywy czesał morze po falach skacząc lekko biegł  & \\
Pamiętam tamte mewy przestworzy słonych zew  & \\
& \\
\hspace*{2em}\textit{Żeglarzom wracającym z morza...}  & \\
& \\
& \\
\end{longtable}
\newpage
\begin{longtable}{ll}
\end{longtable}
\clearpage

% --- Źródło: Miasto_budzi_się.tex ---
\section{Miasto budzi się}
\vspace{-\baselineskip}
\textit{Yugopolis, Paweł Kukuz, Parni Valjak}\\
\begin{longtable}{ll}
\textbf{h D A e}  & \\
\textbf{\textbf{h D A}}  & \\
& \\
Poranek taki cichy, dzień powoli wstaje. & \textbf{h D A e} \\
Moje miasto budzi się. & \textbf{h D A} \\
Słońce purpurą już okryło czarne dachy. & \textbf{h D A e} \\
W złoto zaraz zmieni je. & \textbf{h D A} \\
& \\
Idę ulicą pustą, sławię co nad nami. & \textbf{h D A e} \\
Za tę ciszę, za ten świt. & \textbf{h D A} \\
Że jesteś obok mnie, że nie poddałaś się. & \textbf{h D A e} \\
Za tę chwilę, która jest. & \textbf{h D A} \\
& \\
Patrzę na moje miasto & \textbf{e h} \\
Kocham je & \textbf{A} \\
Ty jeszcze śnij i wyśnij dla nas sen & \textbf{e h D A} \\
& \\
\hspace*{2em}\textit{Miasto budzi się} & \textbf{D A} \\
\hspace*{2em}\textit{Z naszymi marzeniami} & \textbf{G A} \\
\hspace*{2em}\textit{Szumem ulic woła mnie} & \textbf{D A G} \\
\hspace*{2em}\textit{Miasto budzi się} & \textbf{D A} \\
\hspace*{2em}\textit{Nie jesteśmy sami} & \textbf{G A} \\
\hspace*{2em}\textit{Daj nam dzisiaj dobry dzień} & \textbf{D A G} \\
& \\
\textbf{h D A e}  & \\
\textbf{h D A}  & \\
& \\
Wieczorem gdy już cicho, zamykamy oczy & \textbf{h D A e} \\
W ciemną noc obejmę Cię & \textbf{h D A} \\
A potem tak jak zawsze & \textbf{h D A e} \\
Ja przed słońcem wstanę, & \textbf{h D A e} \\
by powitać nowy dzień & \textbf{h D A} \\
& \\
Patrzę na moje miasto & \textbf{e h} \\
Kocham je & \textbf{A} \\
Ty jeszcze śnij i wyśnij dla nas sen & \textbf{e h D A} \\
& \\
\hspace*{2em}\textit{Miasto budzi się…}  & \\
& \\
& \\
\end{longtable}
\newpage
\begin{longtable}{ll}
\end{longtable}
\clearpage

% --- Źródło: Mieć_czy_być.tex ---
\section{Mieć czy być}
\vspace{-\baselineskip}
\textit{Myslovitz}\\
\begin{longtable}{ll}
Strach przed lataniem i głód doświadczeń, & \textbf{e C G e} \\
Wstyd przed mówieniem sobie „nie wiem”. & \textbf{C G D} \\
Ogromna siła wyobrażeń. & \textbf{e C G e} \\
To nie przypadek, że jesteśmy razem. & \textbf{C G D} \\
& \\
\hspace*{2em}\textit{Już teraz wiem} & \textbf{a C} \\
\hspace*{2em}\textit{Wszystko trwa, dopóki sam tego chcesz,} & \textbf{G D a C} \\
\hspace*{2em}\textit{Wszystko trwa, sam dobrze wiesz,} & \textbf{G D} \\
\hspace*{2em}\textit{Że upadamy wtedy, gdy} & \textbf{a C} \\
\hspace*{2em}\textit{Nasze życie przestaje być codziennym zdumieniem.} & \textbf{G D a C e} \\
& \\
Kolejna strona - mieć czy być?  & \\
Czy Erich Fromm wiedział jak żyć?  & \\
W rzeczywistości ciągłej sprzedaży  & \\
Gdzie "być" przestaje cokolwiek znaczyć  & \\
& \\
\hspace*{2em}\textit{Już teraz wiem...}  & \\
& \\
Już teraz wiem...  & \\
Już teraz wiem...  & \\
\end{longtable}
\clearpage

% --- Źródło: Modlitwa_o_wschodzie_słońca.tex ---
\section{Modlitwa o wschodzie słońca}
\vspace{-\baselineskip}
\textit{Jacek Kaczmarski, Przemysław Gintrowski, Zbigniew Łapiński}\\
\begin{longtable}{ll}
Każdy Twój wyrok przyjmę twardy & \textbf{E A E A} \\
Przed mocą Twoją się ukorzę & \textbf{E H E} \\
Ale chroń mnie, Panie, od pogardy & \textbf{A H E A E} \\
Od nienawiści strzeż mnie, Boże & \textbf{A E H E} \\
& \\
Wszak Tyś jest niezmierzone dobro & \textbf{A E A E A} \\
Którego nie wyrażą słowa & \textbf{E H E A H} \\
Więc mnie od nienawiści obroń & \textbf{E A E} \\
I od pogardy mnie zachowaj & \textbf{A E H E} \\
& \\
Co postanowisz, niech się ziści & \textbf{A H E A E} \\
Niechaj się wola Twoja stanie & \textbf{A E H E} \\
Ale zbaw mnie od nienawiści & \textbf{A H E A E A} \\
Ocal mnie od pogardy Panie & \textbf{E H E} \\
& \\
Co postanowisz, niech się ziści & \textbf{A H E A E} \\
Niechaj się wola Twoja stanie & \textbf{A E H E} \\
Ale zbaw mnie od nienawiści & \textbf{A H E A E A} \\
Ocal mnie od pogardy Panie & \textbf{E H E} \\
& \\
\end{longtable}
\clearpage

% --- Źródło: Moja_Wielka_Ciężarówa.tex ---
\section{Moja Wielka Ciężarówa}
\vspace{-\baselineskip}
\textit{Kabaret Koń Polski}\\
\begin{longtable}{ll}
Czwarta rano, wstaje świt, & \textbf{C} \\
łyk herbaty w ustach ćmig, & \textbf{F C} \\
moje dłonie czule pieszczą kierownicę. & \textbf{C G} \\
Mam przed sobą drogi szmat, & \textbf{C} \\
muszę ruszać już na szlak, & \textbf{F} \\
trzysta koni czas pogonić na granicę. & \textbf{F C G} \\
& \\
\hspace*{2em}\textit{Moja wielka ciężarówka ma 48 ton,} & \textbf{F C} \\
\hspace*{2em}\textit{moja wielka ciężarówka to po prostu jest mój dom.} & \textbf{F C G} \\
\hspace*{2em}\textit{Sam już nawet nie policzę, ile razy przez granicę} & \textbf{C F C} \\
\hspace*{2em}\textit{przejeżdżałem, gdy nad cłami radził rząd.} & \textbf{C G C} \\
& \\
Zgodnie z listem przewozowym  & \\
mam na pace pięć ton bobu,  & \\
trochę słomy, gęsi smalec i krasnale.  & \\
W beczce - smalcu na trzy palce,  & \\
tak dla picu, bo pod smalcem –  & \\
sto wagonów papierosów marki „Camel”.  & \\
& \\
\hspace*{2em}\textit{Moja wielka ciężarówka ma 48 ton...}  & \\
& \\
W głębi między krasnalami  & \\
stoi pudło z żółwikami,  & \\
które przeszły przez granicę z Ukrainą.  & \\
Te krasnale - zamiast z gliny –  & \\
całe są z amfetaminy,  & \\
a te żółwie nakarmione są platyną.  & \\
& \\
\hspace*{2em}\textit{Moja wielka ciężarówka ma 48 ton...}  & \\
& \\
W osiach mam aktywny pluton,  & \\
a na osi - kryty jutą –  & \\
siedzi Rumun, który Reich kojarzy z rajem.  & \\
W tylnym moście trzech Bułgarów –  & \\
piją wódkę z samowaru –  & \\
jak wypiją, skoczą z mostu - się zabiją.  & \\
& \\
\hspace*{2em}\textit{Moja wielka ciężarówka ma 48 ton...}  & \\
& \\
Mam w szoferce zdjęcie żony,  & \\
obok wiszą dwie ikony  & \\
i proporczyk klubu „Bayer-Leverkusen”.  & \\
Celnik zajrzy mi do środka,  & \\
powiem - „Witam pana Włodka!”,  & \\
bo ten celnik to jest Włodek, żony kuzyn!  & \\
& \\
\end{longtable}
\clearpage

% --- Źródło: Mury.tex ---
\section{\textbf{Mury}}
\vspace{-\baselineskip}
\textit{Jacek Kaczmarski, Przemysław Gintrowski, Zbigniew Łapiński}\\
\begin{longtable}{ll}
On natchniony i młody był, & \textbf{e H7 e} \\
ich nie policzyłby nikt & \textbf{e H7} \\
On im dodawał pieśnią sił, & \textbf{C H7 e} \\
śpiewał że blisko już świt & \textbf{e H7 e} \\
& \\
Świec tysiące palili mu, & \textbf{e H7 e} \\
znad głów podnosił się dym & \textbf{e H7} \\
Śpiewał, że czas by runął mur, & \textbf{C H7 e} \\
oni śpiewali wraz z nim: & \textbf{e H7 e} \\
& \\
\hspace*{2em}\textit{Wyrwij murom zęby krat!} & \textbf{H7 e} \\
\hspace*{2em}\textit{Zerwij kajdany, połam bat!} & \textbf{H7 e} \\
\hspace*{2em}\textit{A mury runą, runą, runą} & \textbf{a e} \\
\hspace*{2em}\textit{I pogrzebią stary świat!} & \textbf{H7 e} \\
& \\
Wkrótce na pamięć znali pieśń & \textbf{e H7 e} \\
i sama melodia bez słów & \textbf{e H7} \\
Niosła ze sobą starą treść, & \textbf{C H7 e} \\
dreszcze na wskroś serc i głów & \textbf{e H7 e} \\
& \\
Śpiewali więc, klaskali w rytm, & \textbf{e H7 e} \\
jak wystrzał poklask ich brzmiał & \textbf{e H7} \\
I ciążył łańcuch, zwlekał świt, & \textbf{C H7 e} \\
on wciąż śpiewał i grał: & \textbf{e H7 e} \\
& \\
& \\
Aż zobaczyli ilu ich, & \textbf{e H7 e} \\
poczuli siłę i czas & \textbf{e H7} \\
I z pieśnią, że już blisko świt & \textbf{C H7 e} \\
szli ulicami miast; & \textbf{e H7 e} \\
& \\
Zwalali pomniki i rwali bruk – & \textbf{e H7 e} \\
Ten z nami! Ten przeciw nam! & \textbf{e H7} \\
Kto sam ten nasz najgorszy wróg! & \textbf{C H7 e} \\
A śpiewak także był sam & \textbf{e H7 e} \\
& \\
\hspace*{2em}\textit{Patrzył na równy tłumów marsz} & \textbf{H7 e} \\
\hspace*{2em}\textit{Milczał wsłuchany w kroków huk} & \textbf{H7 e} \\
\hspace*{2em}\textit{A mury rosły, rosły, rosły} & \textbf{a e} \\
\hspace*{2em}\textit{Łańcuch kołysał się u nóg...} & \textbf{H7 e} \\
& \\
\hspace*{2em}\textit{Patrzy na równy tłumów marsz} & \textbf{H7 e} \\
\hspace*{2em}\textit{Milczy wsłuchany w kroków huk} & \textbf{H7 e} \\
\hspace*{2em}\textit{A mury rosną, rosną, rosną} & \textbf{a e} \\
\end{longtable}
\clearpage

% --- Źródło: Muszka_plujka.tex ---
\section{Muszka plujka}
\begin{longtable}{ll}
Muszka plujka i żuczek gnojarek & \textbf{a E} \\
Stanowili dość dobraną parę & \textbf{E a} \\
Ona była plujka a on zwykły żuk & \textbf{a d} \\
I żyli szczęśliwie choć mu brakło nóg & \textbf{a E a} \\
& \\
Muszka plujka i żuczek gnojarek &  \\
CO dzień rano wokół krowiej kupki odprawiali bale  & \\
Ona była plujka a on nie miał wujka  & \\
A na dodatek jeszcze nie miał nóg ten żuk  & \\
miał jeszcze parę innych suczek  & \\
Miłosnej euforii szybko mijał czas  & \\
Lecz to trwało krótko bo zabił ich gaz  & \\
& \\
Taka to historia smutna lecz prawdziwa  & \\
Ona byłą plujka on był stary dziwak  & \\
Lecz skąd gaz pytacie cień się plujki błąka  & \\
Tak to słoń zawinił bo drań puścił bąka  & \\
Muszka plujka i gnojarek żuczek  & \\
Ona była wierna  & \\
A on miał jeszcze parę innych suczek  & \\
Miłosnej euforii szybko mijał czas  & \\
Lecz to trwało krótko bo zabił ich gaz  & \\
& \\
Taka to historia smutna lecz prawdziwa  & \\
Ona byłą plujka on był stary dziwak  & \\
Lecz skąd gaz pytacie cień się plujki błąka  & \\
Tak to słoń zawinił bo drań puścił bąka  & \\
& \\
& \\
\end{longtable}
\newpage
\begin{longtable}{ll}
\end{longtable}
\clearpage

% --- Źródło: Na_co_komu_dziś.tex ---
\section{Na co komu dziś}
\vspace{-\baselineskip}
\textit{Lady Pank}\\
\begin{longtable}{ll}
Stała pod ścianą sącząc kakao & \textbf{F a F G} \\
Kapela cięła walca na sześć & \textbf{F a C G} \\
Spytałem skromnie: “czy pójdziesz do mnie?” & \textbf{F a F G} \\
Kiwnęła głową zgadzając się & \textbf{F a F G} \\
& \\
\hspace*{2em}\textit{Trzeba zawsze żyć biegnącą chwilą} & \textbf{a G C F} \\
\hspace*{2em}\textit{Na co komu dziś wczorajszy dzień} & \textbf{a G C F} \\
& \\
Topiłem smutki w butelce wódki & \textbf{F a F G} \\
Obok Japończyk do lustra pił & \textbf{F a F G} \\
Pytam żółtego: "powiedz dlaczego & \textbf{F a F G} \\
Też jesteś smutny?" On na to mi & \textbf{F a F G} \\
& \\
\hspace*{2em}\textit{Na co komu dziś wczorajsza miłość} & \textbf{a G C F} \\
\hspace*{2em}\textit{Na co komu dziś wczorajszy sen} & \textbf{a G C F} \\
\hspace*{2em}\textit{Po co dalej pić to samo piwo} & \textbf{a G C F} \\
\hspace*{2em}\textit{Kiedy czujesz, że uleciał gaz} & \textbf{a G C F} \\
& \\
Chciałem być sobą za wielką wodą & \textbf{F a F G} \\
Na czekoladę poczułem chęć & \textbf{F a F G} \\
Była namiętna, bardzo nieletnia & \textbf{F a F G} \\
I dobrze znała refrenu sens & \textbf{F a F G} \\
& \\
\hspace*{2em}\textit{Na co komu dziś wczorajsza miłość…}  & \\
& \\
Spotkałem narzeczoną & \textbf{F G a F G} \\
Taką ze szkolnych lat & \textbf{F G a F G} \\
Próbowaliśmy mocno & \textbf{F G a F G} \\
By taniec naszych ciał & \textbf{F G a F G} \\
Rozgrzała jakaś iskra & \textbf{F G a F G} \\
& \\
\hspace*{2em}\textit{Na co komu dziś wczorajsza miłość...}  & \\
\end{longtable}
\clearpage

% --- Źródło: Na_jednej_z_dzikich_plaż.tex ---
\section{\textbf{Na jednej z dzikich plaż}}
\vspace{-\baselineskip}
\textit{Rotary}\\
\begin{longtable}{ll}
Samochód w deszczu stał & \textbf{C a} \\
Radio przestało grać & \textbf{C a} \\
Dotknąłem kolan twych & \textbf{C a} \\
Nie liczyliśmy gwiazd... & \textbf{C a} \\
& \\
\hspace*{2em}\textit{Lubiła tańczyć pełna radości tak,} & \textbf{F G} \\
\hspace*{2em}\textit{Ciągle goniła wiatr} & \textbf{e a} \\
\hspace*{2em}\textit{Spragniona życia wciąż, zawsze gubiła coś,} & \textbf{F G} \\
\hspace*{2em}\textit{Nie chciała nic} & \textbf{e a} \\
& \\
\hspace*{2em}\textit{Nie rozumiałem, kiedy mówiła mi:} & \textbf{F G} \\
\hspace*{2em}\textit{- Dzisiaj ostatni raz} & \textbf{e a} \\
\hspace*{2em}\textit{Zatańczmy proszę tak, jak gdyby umarł czas...} & \textbf{F G} \\
\hspace*{2em}\textit{Mówiła mi...} & \textbf{A a} \\
& \\
Mieliśmy wiecznie trwać  & \\
Na jednej z dzikich plaż  & \\
Chciałem ze wszystkich sił  & \\
Pozostać z tobą tam  & \\
& \\
\hspace*{2em}\textit{Lubiła tańczyć pełna radości tak...}  & \\
& \\
\end{longtable}
\clearpage

% --- Źródło: Napisz_list.tex ---
\section{Napisz list}
\begin{longtable}{ll}
Nie zwlekaj już, i napissz list & \textbf{A E} \\
Nim lęk do serca się wkradnie  & \\
Nie pozwól by samotność dziś  & \\
Była mi wiernym kompanem  & \\
& \\
\hspace*{2em}\textit{Nich z pomiędzy twych słów} & \textbf{A G C E} \\
\hspace*{2em}\textit{Ciało twe do mnie wygląda}  & \\
\hspace*{2em}\textit{Twe oczy w literach O}  & \\
\hspace*{2em}\textit{Ramiona w T jak tęsknota}  & \\
& \\
*melorecytacja* W każdym z nas cząstka  & \\
drugiej osoby w o itd.  & \\
& \\
\hspace*{2em}\textit{Nich z pomiędzy twych słów...}  & \\
& \\
Więc napisz list, niech gesty słów  & \\
Spojrzenia zdań mnie otulą  & \\
Ach napisz list jeżeli możesz  & \\
Wypełnij słowami pustkę  & \\
& \\
\hspace*{2em}\textit{Nich z pomiędzy twych słów...}  & \\
\end{longtable}
\clearpage

% --- Źródło: Nasza_klasa.tex ---
\section{Nasza klasa}
\vspace{-\baselineskip}
\textit{Jacek Kaczmarski}\\
\begin{longtable}{ll}
Co się stało z naszą klasą –  & \\
Pyta Adam w Tel Awiwie,  & \\
Ciężko sprostać takim czasom,  & \\
Ciężko w ogóle żyć uczciwie.  & \\
Co się stało z naszą klasą?  & \\
Wojtek w Szwecji, w porno klubie  & \\
Pisze – dobrze mi tu płacą  & \\
Za to, co i tak wszak lubię.  & \\
& \\
Kaśka z Piotrkiem są w Kanadzie,  & \\
Bo tam mają perspektywy,  & \\
Staszek w Stanach sobie radzi,  & \\
Paweł do Paryża przywykł.  & \\
Gośka z Przemkiem ledwie przędą,  & \\
W maju będzie trzeci bachor;  & \\
Próżno skarżą się urzędom,  & \\
Że też chcieliby na zachód.  & \\
& \\
Za to Magda jest w Madrycie  & \\
I wychodzi za Hiszpana,  & \\
Maciek w grudniu stracił życie,  & \\
Gdy chodzili po mieszkaniach,  & \\
Janusz, ten, co zawiść budził,  & \\
Że go każda fala niesie,  & \\
Jest chirurgiem, leczy ludzi,  & \\
Ale brat mu się powiesił.  & \\
& \\
Marek siedzi za odmowę,  & \\
Bo nie strzelał do Michała,  & \\
A ja piszę ich historię  & \\
I to już jest klasa cała.  & \\
Jeszcze Filip, fizyk w Moskwie –  & \\
Dziś nagrody różne zbiera,  & \\
Jeździ, kiedy chce, do Polski,  & \\
Był przyjęty przez premiera.  & \\
& \\
& \\
\end{longtable}
\newpage
\begin{longtable}{ll}
Odnalazłem klasę całą –  & \\
Na wygnaniu, w kraju, w grobie,  & \\
Ale coś się pozmieniało,  & \\
Każdy sobie żywot skrobie.  & \\
Odnalazłem całą klasę  & \\
Wyrośniętą i dojrzałą,  & \\
Rozdrapałem młodość naszą,  & \\
Lecz za bardzo nie bolało…  & \\
& \\
Już nie chłopcy, lecz mężczyźni,  & \\
Już kobiety – nie dziewczyny.  & \\
Młodość szybko się zabliźni,  & \\
Nie ma w tym niczyjej winy;  & \\
Wszyscy są odpowiedzialni,  & \\
Wszyscy mają w życiu cele,  & \\
Wszyscy w miarę są – normalni,  & \\
Ale przecież – to niewiele…  & \\
& \\
Nie wiem sam, co mi się marzy,  & \\
Jaka z gwiazd nade mną świeci,  & \\
Gdy wśród tych – nieobcych – twarzy  & \\
Szukam ciągle twarzy – dzieci.  & \\
Czemu wciąż przez ramię zerkam,  & \\
Choć nie woła nikt – kolego!  & \\
Że ktoś ze mną zagra w berka,  & \\
Lub przynajmniej w chowanego…  & \\
& \\
Własne pędy, własne liście,  & \\
Zapuszczamy – każdy sobie;  & \\
I korzenie – oczywiście –  & \\
Na wygnaniu, w kraju, w grobie;  & \\
W dół, na boki, wzwyż ku słońcu,  & \\
Na stracenie, w prawo, w lewo…  & \\
Kto pamięta, że to w końcu  & \\
Jedno i to samo drzewo…  & \\
& \\
\end{longtable}
\clearpage

% --- Źródło: Nie_jestem_święty.tex ---
\section{Nie jestem święty}
\begin{longtable}{ll}
Lepiej diabłu wlecieć w ramiona & \textbf{a d C9 G} \\
Zwłaszcza gdy diabeł jest piękną dziewczyną & \textbf{a d C9 G} \\
Niż tylko po to pokusę pokonać & \textbf{a d C9 G} \\
Żeby bez lęku iść ciemną doliną & \textbf{a d C9 G} \\
& \\
Chociaż dałem się wieść mocom ciemnym & \textbf{a d C9 G} \\
Zbiesiłem się, czarcim uległem czarom & \textbf{a d C9 G} \\
I choć kochałem też beznadziejnie & \textbf{a d C9 G} \\
To przecież miłość znam nie do wiary & \textbf{a d C9 G} \\
& \\
\hspace*{2em}\textit{A miłość daleko jest od grzechu} & \textbf{a d C9 a d C9} \\
\hspace*{2em}\textit{Miłości mi trzeba jak oddechu} & \textbf{a d C9 a d C9} \\
\hspace*{2em}\textit{Miłość daleko jest od grzechu} & \textbf{a d C9 a d C9} \\
\hspace*{2em}\textit{Miłości mi trzeba jak oddechu} & \textbf{a d C9 a d C9} \\
& \\
\textbf{g7, a7 || x2}  & \\
& \\
Nie biesów to czekają w niebiesiech & \textbf{d B9 C9 G} \\
Nie pieją ku ich wątpliwej czci hymnów & \textbf{d B9 g7 a7} \\
Ja święty nie jestem, mój aniele & \textbf{d B9 C9 G} \\
Ale ty zawsze stój przy mnie & \textbf{B9 d C9} \\
& \\
Nie biesów w niebie czekają z nadzieją & \textbf{d B9 C9 G} \\
Nie na ich cześć tam hymny pieją & \textbf{d B9 g7 a7} \\
Choć wiary nie budzę, aniele mój & \textbf{d B9 C9 G} \\
To ty jednak przy mnie stój & \textbf{B9 d C9 a} \\
& \\
Być może garnków nie lepią święci & \textbf{a d C9 G} \\
Ale kto z innej ulepiony gliny & \textbf{a d C9 G} \\
W górę nie gapię się wniebowzięcie & \textbf{a d C9 G} \\
Gdy grzeszy myślą, słowem lub czynem & \textbf{a d C9 G} \\
& \\
Nie poznałem języka aniołów & \textbf{a d C9 G} \\
A i z człowiekiem dogadać się trudno & \textbf{a d C9 G} \\
Niczego na wiarę ot tak nie przyjąłem & \textbf{a d C9 G} \\
Po co nadzieją karmić się złudną? & \textbf{a d C9 G} \\
& \\
\hspace*{2em}\textit{A miłość daleko jest od grzechu...}  & \\
& \\
\end{longtable}
\clearpage

% --- Źródło: Nie_lubię.tex ---
\section{Nie lubię}
\begin{longtable}{ll}
Nie lubię, gdy mi mówią po imieniu,  & \\
Gdy w zdaniu jest co drugie słowo – brat.  & \\
Nie lubię, gdy mnie klepią po ramieniu,  & \\
Z uśmiechem wykrzykując – kopę lat!  & \\
Nie lubię, gdy czytają moje listy,  & \\
Przez ramię odczytując treść ich kart!  & \\
Nie lubię tych, co myślą, że na wszystko  & \\
Najlepszy jest cios w pochylony kark.  & \\
& \\
Ne znoszę, gdy do czegoś ktoś mnie zmusza,  & \\
Nie znoszę, gdy na litość brać mnie chce,  & \\
Nie znoszę, gdy z butami lezą w duszę,  & \\
Tym bardziej gdy mi napluć w nią starają się!  & \\
Nie znoszę much, co żywią się krwią świeżą,  & \\
Nie znoszę psów, co szarpią mięsa strzęp!  & \\
Nie znoszę tych, co tępo w siebie wierzą,  & \\
Gdy nawet już ich dławi własny pęd!  & \\
& \\
Nie cierpię poczucia bezradności,  & \\
Z jakim zaszczute zwierzę patrzy w lufy strzelb!  & \\
Nie cierpię zbiegów złych okoliczności,  & \\
Co pojawiają się, gdy ktoś osiąga cel.  & \\
Nie cierpię więc niewyjaśnionych przyczyn,  & \\
Nie cierpię niepowetowanych strat!  & \\
Nie cierpię liczyć niespełnionych życzeń,  & \\
Nim mi ostatnie uprzejmy spełni kat.  & \\
& \\
Ja nienawidzę, gdy przerwie mi rozmowę  & \\
W słuchawce suchy metaliczny szczęk.  & \\
Ja nienawidzę strzałów w tył głowy;  & \\
Do salw w powietrze czuję tylko wstręt.  & \\
Ja nienawidzę siebie, kiedy tchórzę,  & \\
Gdy wytłumaczeń dla łajdactw szukam swych,  & \\
Kiedy uśmiecham się do tych, którym służę,  & \\
Choć z całej duszy nienawidzę ich!  & \\
\end{longtable}
\clearpage

% --- Źródło: Nie_płacz_Ewka.tex ---
\section{\textbf{Nie płacz Ewka}}
\vspace{-\baselineskip}
\textit{Perfect}\\
\begin{longtable}{ll}
Nie płacz Ewka, bo tu miejsca brak, & \textbf{A fis} \\
na twe babskie łzy & \textbf{E} \\
Po ulicy Miłość hula wiatr & \textbf{A fis} \\
wśród rozbitych szyb & \textbf{E} \\
Patrz poeci śliczni prawdy sens & \textbf{A fis} \\
roztrwonili w grach & \textbf{E} \\
W półlitrówkach pustych SOS & \textbf{A fis} \\
wysyłają w świat & \textbf{E} \\
& \\
\hspace*{2em}\textit{Żegnam was, już wiem} & \textbf{h D} \\
\hspace*{2em}\textit{Nie załatwię wszystkich pilnych spraw} & \textbf{A E fis} \\
\hspace*{2em}\textit{Idę sam, właśnie tam} & \textbf{E D} \\
\hspace*{2em}\textit{gdzie czekają mnie} & \textbf{A} \\
\hspace*{2em}\textit{Tam przyjaciół kilku mam od lat} & \textbf{h D} \\
\hspace*{2em}\textit{Dla nich zawsze śpiewam dla nich gram} & \textbf{A E fis} \\
\hspace*{2em}\textit{Jeszcze raz żegnam was,} & \textbf{E D} \\
\hspace*{2em}\textit{nie spotkamy się} & \textbf{A} \\
& \\
Proza życia to przyjaźni kat, & \textbf{A fis} \\
pęka cienka nić & \textbf{E} \\
Telewizor, meble, mały fiat, & \textbf{A fis} \\
oto marzeń szczyt & \textbf{E} \\
Hej prorocy moi z gniewnych lat, & \textbf{A fis} \\
obrastacie w tłuszcz & \textbf{E} \\
Już was w swoje szpony dopadł szmal, & \textbf{A fis} \\
zdrada płynie z ust & \textbf{E} \\
& \\
\hspace*{2em}\textit{Żegnam was, już wiem...}  & \\
& \\
\end{longtable}
\clearpage

% --- Źródło: Noc_w_Bieszczadach.tex ---
\section{Noc w Bieszczadach}
\begin{longtable}{ll}
Milkną słowa, milkną słowa, & \textbf{a C} \\
Nie potrzeba więcej ich, & \textbf{d E} \\
Gaśnie watra, gaśnie watra,  & \\
Chociaż nastrój w nas się tli.  & \\
Jeszcze chwila, jeszcze chwila,  & \\
Jeszcze tylko jeden gest,  & \\
Kilka iskier, kilka iskier,  & \\
Cierpki dym i parę łez.  & \\
& \\
\hspace*{2em}\textit{Noc w Bieszczadach, noc w Bieszczadach,}  & \\
\hspace*{2em}\textit{Dobrze, że nie jestem sam.}  & \\
\hspace*{2em}\textit{Noc w Bieszczadach, noc w Bieszczadach,}  & \\
\hspace*{2em}\textit{Najdziwniejsza, jaką znam.}  & \\
\hspace*{2em}\textit{Za jej uśmiech, za jej uśmiech,}  & \\
\hspace*{2em}\textit{Za jej lekko gorzki smak}  & \\
\hspace*{2em}\textit{Tysiąc nocy, tysiąc nocy}  & \\
\hspace*{2em}\textit{W wielkim mieście mogę dać.}  & \\
& \\
Milkną drzewa, milkną drzewa,  & \\
Nad głowami cisza trwa.  & \\
Nie przerywaj, nie przerywaj,  & \\
Lepiej nie krzycz, czekaj dnia.  & \\
Możesz zbudzić, możesz zbudzić,  & \\
Co zapadło w wielki sen,  & \\
Lepiej zaśnij, lepiej zaśnij,  & \\
Wtedy nic nie stanie się.  & \\
& \\
\hspace*{2em}\textit{Noc w Bieszczadach, noc w Bieszczadach...}  & \\
& \\
Kiedy w domu, kiedy w domu  & \\
Cztery ściany, ciepły kąt,  & \\
Drzwi zamknięte, drzwi zamknięte -  & \\
Jak bezpieczna jest ta noc.  & \\
Chciałbym wrócić, chciałbym wrócić  & \\
Chciałbym poczuć znowu lęk  & \\
Pod tym wichrem, pod tym wichrem,  & \\
Co w Bieszczadach drzemie gdzieś...  & \\
& \\
\hspace*{2em}\textit{Noc w Bieszczadach, noc w Bieszczadach...}  & \\
& \\
\end{longtable}
\clearpage

% --- Źródło: North__West_Passage.tex ---
\section{North – West Passage}
\begin{longtable}{ll}
Brnę przez kry na zachód od Davisa zimnych wrót, & \textbf{a F C G} \\
Szlakiem tych, których bogactwa wiodły na Daleki Wschód. & \textbf{A F C G C} \\
Sławę zdobyć chcieli, został po nich tylko proch, & \textbf{a F C a} \\
Białe kości popłynęły gdzieś na dno. & \textbf{F C G a} \\
& \\
\hspace*{2em}\textit{Spróbuj chociaż raz north-westowe przejście zdobyć,} & \textbf{C G F a} \\
\hspace*{2em}\textit{Znajdź miejsca gdzie zimował Franklin u Beauforta Wrót,} & \textbf{F C d F} \\
\hspace*{2em}\textit{Wykuj własny szlak przez kraj dziki i surowy,} & \textbf{C G F a} \\
\hspace*{2em}\textit{Przejdź drogą Północ-Zachód poza lód.} & \textbf{F C G C} \\
& \\
Trzy wieki przeminęły, na wyprawę ruszam znów & \textbf{a F C G} \\
Śladami dzielnych chłopców, co walczyli z furią mórz & \textbf{A F C G C} \\
Miasta z lodu wyrastają, by rozpłynąć za mną się & \textbf{a F C a} \\
Jak odkrywcom dawnym wskażą nowy brzeg  & \\
& \\
\hspace*{2em}\textit{Spróbuj chociaż raz north-westowe przejście zdobyć...}  & \\
& \\
Mile wloką się bez końca, całą noc pcham się na West. & \textbf{a F C G} \\
Tu McKenzie, David Thompson, cała reszta z nimi też, & \textbf{A F C G C} \\
Wytyczali dla mnie drogę wśród iskrzących lodem gór. & \textbf{a F C a} \\
W mroźnych wiatrach głos ich słyszę, jak ze snu.  & \\
& \\
\hspace*{2em}\textit{Spróbuj chociaż raz north-westowe przejście zdobyć...}  & \\
& \\
No i czymże ja się różnię od pionierów szlaków tych? & \textbf{a F C G} \\
Tak, jak oni, porzuciłem życie pośród bliskich mi & \textbf{A F C G C} \\
By znów odkryć North-West Passage, dla tak wielu koniec snów & \textbf{a F C a} \\
Ale marzę, by do domu wrócić znów & \textbf{F C G a} \\
& \\
\hspace*{2em}\textit{Spróbuj chociaż raz north-westowe przejście zdobyć...}  & \\
& \\
\end{longtable}
\clearpage

% --- Źródło: Obława.tex ---
\section{\textbf{Obława}}
\vspace{-\baselineskip}
\textit{Jacek Kaczmarski}\\
\begin{longtable}{ll}
Skulony w jakiejś ciemnej jamie smaczniem sobie spał & \textbf{a C G C} \\
I spały wilczki dwa, zupełnie ślepe jeszcze & \textbf{F E} \\
Wtem stary wilk przewodnik, co życie dobrze znał & \textbf{a C G C} \\
Łeb podniósł, warknął groźnie, aż mną szarpnęły dreszcze & \textbf{F E} \\
Poczułem wokół siebie nienawistną woń & \textbf{a F E a} \\
Woń, która burzy wszelki spokój, zrywa wszystkie sny & \textbf{F E} \\
Z daleka ktoś, gdzieś krzyknął krótki rozkaz: goń! & \textbf{a F E a} \\
I z czterech stron wypadły na nas cztery gończe psy! & \textbf{F E} \\
& \\
\hspace*{2em}\textit{Obława, obława na młode wilki obława} & \textbf{a C G C} \\
\hspace*{2em}\textit{Te dzikie zapalczywe, w gęstym lesie wychowane} & \textbf{F E} \\
\hspace*{2em}\textit{Krąg śniegu wydeptany, w tym kręgu plama krwawa} & \textbf{a C G C} \\
\hspace*{2em}\textit{I ciała wilcze kłami gończych psów szarpane!} & \textbf{F E a} \\
& \\
Ten, który na mnie rzucił się, niewiele szczęścia miał & \textbf{a C G C} \\
Bo wpadł prosto mi na kły i krew trysnęła z rany & \textbf{F E} \\
Gdym teraz ile w łapach sił przed siebie prosto gnał & \textbf{a C G C} \\
Ujrzałem młode wilczki na strzępy rozszarpane & \textbf{F E} \\
Zginęły ślepe ufne tak, puszyste kłębki dwa & \textbf{a F E a} \\
Bezradne na tym świecie złym nie wiedząc, kto je zdławił & \textbf{F E} \\
I zginie wilk-przewodnik, choć życie dobrze zna & \textbf{a F E a} \\
Bo z trzema naraz walczy psami i z ran trzech naraz krwawi & \textbf{F E} \\
& \\
\hspace*{2em}\textit{Obława, obława na młode wilki obława} & \textbf{a C G C} \\
\hspace*{2em}\textit{Te dzikie zapalczywe, w gęstym lesie wychowane} & \textbf{F E} \\
\hspace*{2em}\textit{Krąg śniegu wydeptany, w tym kręgu plama krwawa} & \textbf{a C G C} \\
\hspace*{2em}\textit{I ciała wilcze kłami gończych psów szarpane!} & \textbf{F E a} \\
& \\
Wypadłem na otwartą przestrzeń pianą z pyska tocząc & \textbf{a C G C} \\
Lecz tutaj też ze wszystkich stron zła mnie otacza woń & \textbf{F E} \\
A myśliwemu, co mnie dojrzał, już się śmieją oczy & \textbf{a C G C} \\
I ręka pewna niezawodna podnosi w górę broń & \textbf{F E} \\
& \\
Rzucam się w bok, na oślep gnam, aż ziemia spod łap pryska & \textbf{a F E a} \\
I wtedy pada pierwszy strzał, co kark mi rozszarpuje & \textbf{F E} \\
Pędzę, słyszę jak on klnie, krew mi płynie z pyska & \textbf{a F E a} \\
On strzela po raz drugi, lecz teraz już pudłuje & \textbf{F E} \\
\end{longtable}
\newpage
\begin{longtable}{ll}
\hspace*{2em}\textit{Obława, obława na młode wilki obława} & \textbf{a C G C} \\
\hspace*{2em}\textit{Te dzikie zapalczywe, w gęstym lesie wychowane} & \textbf{F E} \\
\hspace*{2em}\textit{Krąg śniegu wydeptany, w tym kręgu plama krwawa} & \textbf{a C G C} \\
\hspace*{2em}\textit{I ciała wilcze kłami gończych psów szarpane!} & \textbf{F E a} \\
& \\
Wyrwałem się z obławy tej, schowałem w jakiś las & \textbf{a C G C} \\
Lecz ile szczęścia miałem w tym to każdy chyba przyzna & \textbf{F E} \\
Leżałem w śniegu jak nieżywy długi, długi czas & \textbf{a C G C} \\
Po strzale zaś na zawsze mi została krwawa blizna & \textbf{F E} \\
& \\
Lecz nie skończyła się obława i nie śpią gończe psy & \textbf{a F E a} \\
I giną ciągle wilki młode na całym wielkim świecie & \textbf{F E} \\
Nie dajcie z siebie zedrzeć skór, brońcie się i wy, & \textbf{a F E a} \\
O bracia wilcy! Brońcie się nim wszyscy wyginiecie & \textbf{F E} \\
& \\
\hspace*{2em}\textit{Obława, obława na młode wilki obława} & \textbf{a C G C} \\
\hspace*{2em}\textit{Te dzikie zapalczywe, w gęstym lesie wychowane} & \textbf{F E} \\
\hspace*{2em}\textit{Krąg śniegu wydeptany, w tym kręgu plama krwawa} & \textbf{a C G C} \\
\hspace*{2em}\textit{I ciała wilcze kłami gończych psów szarpane!}  & \\
\end{longtable}
\clearpage

% --- Źródło: Obława_III_potrzaski.tex ---
\section{Obława III (potrzaski)}
\vspace{-\baselineskip}
\textit{Jacek Kaczmarski}\\
\begin{longtable}{ll}
Obławy już przeżyłem dwie, dziękuję - dosyć! & \textbf{a F0 a} \\
Zjeżona sierść, zbłąkany wzrok, zmętniała myśl! & \textbf{a F0} \\
Wciąż czuję obce wonie, obce słyszę głosy & \textbf{F0} \\
I innym wilkom nie dowierzam nie od dziś! & \textbf{F0 a} \\
& \\
Lecz jakże trudno jest polować samotnikom! & \textbf{a F0 a} \\
Z łownego zwierza oczyszczono cały las, & \textbf{a F0} \\
Poczułem łup - do ziemi głodny pysk przytykam, & \textbf{F0} \\
Wtem straszny ból i stokroć odeń gorszy trzask! & \textbf{F0 a} \\
& \\
\hspace*{2em}\textit{Strzeżcie się wilki! Strzeżcie się przynęty!} & \textbf{a} \\
\hspace*{2em}\textit{Strzeżcie się wilki! Strzeżcie ludzkiej łaski!} & \textbf{a} \\
\hspace*{2em}\textit{Zastawił na was wróg zawzięty} & \textbf{F E} \\
\hspace*{2em}\textit{Potrzaski!} & \textbf{a} \\
& \\
Już moja prawa łapa tkwi w żelaznych szczękach  & \\
I jej nie wyrwę, choćbym wszystkich użył sił,  & \\
A królik w pętli - moja zguba i przynęta  & \\
Czerwone oczy przerażone we mnie wbił!  & \\
& \\
Ale i jemu śmierć pisana - on nie winien!  & \\
Ten, co zastawił wnyki to dopiero wróg!  & \\
To z jego marnie zginę rąk, jak zwierzę ginie!  & \\
Dostanę pałką w łeb nim warknąć będę mógł!  & \\
& \\
\hspace*{2em}\textit{Strzeżcie się wilki! Strzeżcie się przynęty!}  & \\
\hspace*{2em}\textit{Strzeżcie się wilki! Strzeżcie ludzkiej łaski!}  & \\
\hspace*{2em}\textit{Zastawił na was wróg zawzięty}  & \\
\hspace*{2em}\textit{Potrzaski!}  & \\
& \\
Żałosny koniec - śmierć haniebna - nie dla wilka,  & \\
Niech królik mdleje w pętli, czeka na swój los!  & \\
Moja pieśń życia jeszcze dla mnie nie zamilkła!  & \\
Niejedna przestrzeń jeszcze mój usłyszy głos!  & \\
& \\
Na własnej łapie szczęk zaciskam straszny uchwyt  & \\
Ona nie moja już! W niewoli musi zgnić  & \\
Już pęka kość i własnej krwi mam pełne żuchwy...  & \\
Jednym szarpnięciem się uwalniam, żeby żyć!  & \\
& \\
& \\
\end{longtable}
\newpage
\begin{longtable}{ll}
\hspace*{2em}\textit{Strzeżcie się wilki! Strzeżcie się przynęty!}  & \\
\hspace*{2em}\textit{Strzeżcie się wilki! Strzeżcie ludzkiej łaski!}  & \\
\hspace*{2em}\textit{Zastawił na was wróg zawzięty}  & \\
\hspace*{2em}\textit{Potrzaski!}  & \\
& \\
Słuchajcie głosu Trójłapego, choć z daleka  & \\
Krew szybko wsiąka w ziemię, strach zabija czas!  & \\
Słuchajcie bracia! Wyje do was wilk - kaleka,  & \\
Trzeba odrzucić to co w nas zniewala nas!  & \\
& \\
I po dziś dzień naganiacz, strzelec, czy kłusownik,  & \\
Przyzwyczajony do czytania tropów map,  & \\
Przez zęby mówi - Oto jest wilk wolny...  & \\
Kiedy na śniegu ujrzy ślady trojga łap!  & \\
& \\
\hspace*{2em}\textit{Strzeżcie się wilki! Strzeżcie się przynęty!}  & \\
\hspace*{2em}\textit{Strzeżcie się wilki! Strzeżcie ludzkiej łaski!}  & \\
\hspace*{2em}\textit{Zastawił na was wróg zawzięty}  & \\
\hspace*{2em}\textit{Potrzaski!}  & \\
\end{longtable}
\clearpage

% --- Źródło: Obława_II_z_helikopterów.tex ---
\section{Obława II (z helikopterów)}
\vspace{-\baselineskip}
\textit{Jacek Kaczmarski}\\
\begin{longtable}{ll}
Obce lasy przemierzam, serce szarpie mi krtań! & \textbf{e C H7 e} \\
Nie ze strachu - z wściekłości, z rozpaczy! & \textbf{e C H7 e} \\
Ślad po wilczych gromadach mchy pokryły i darń, & \textbf{a C E a} \\
Niedobitki los cierpią sobaczy! & \textbf{a C E a} \\
Z gąszczu żaden kudłaty pysk nie wyjrzy na krok, & \textbf{H7 C} \\
W ślepiach obłęd lęk chciwość lub zdrada! & \textbf{H7 C H7} \\
Na otwartych przestrzeniach dawno znikł wilczy trop, & \textbf{a} \\
Wilk wie dobrze czym pachnie zagłada! & \textbf{a C H7} \\
& \\
\hspace*{2em}\textit{Słyszę wciąż i uszom nie wierzę,} & \textbf{e a} \\
\hspace*{2em}\textit{Lecz potwierdza co krok wszystko mi:} & \textbf{Fis H} \\
\hspace*{2em}\textit{Zwierzem jesteś i żyjesz jak zwierzę,} & \textbf{e a} \\
\hspace*{2em}\textit{Lecz nie wilki, nie wilki już wy!} & \textbf{C H7 e} \\
& \\
Myśli brat, że bezpieczny, skoro schronił się w las,  & \\
Lecz go ściga nie bóg! Ściga człowiek!  & \\
Śmigieł świst nad głowami, grad pocisków i wrzask,  & \\
Co wyrywa źrenice spod powiek!  & \\
Strzelców twarze pijane w drzew koronach znad luf,  & \\
Wrzący deszcz wystrzelonych ładunków!  & \\
To już nie polowanie, nie obława, nie łów!  & \\
To planowe niszczenie gatunku!  & \\
& \\
\hspace*{2em}\textit{Z rąk w mundurach, z helikopterów}  & \\
\hspace*{2em}\textit{Maszynowa broń wbija we łby}  & \\
\hspace*{2em}\textit{Czarne kule i wrzask oficerów:}  & \\
\hspace*{2em}\textit{Wy nie wilki, nie wilki już wy!}  & \\
& \\
Kto nie popadł w szaleństwo, kto nie poszedł pod strzał  & \\
Jeszcze biegnie klucząc po norach,  & \\
Lecz już nie ma kryjówek, które miał, które znał,  & \\
Wszędzie wściekła wywęszy go sfora!  & \\
I pomyśleć, że kiedyś ją traktował jak łup,  & \\
Który nie wart wilczych był kłów!  & \\
Dziś krewniaka swym panom zawloką do stóp,  & \\
Lub rozszarpią na rozkaz bez słów!  & \\
& \\
\hspace*{2em}\textit{Bo kto biegnie - zginie dziś w biegu!}  & \\
\hspace*{2em}\textit{A kto stanął - padnie gdzie stał!}  & \\
\hspace*{2em}\textit{Krwią w panice piszemy na śniegu:}  & \\
\hspace*{2em}\textit{My nie wilki, my mięso na strzał!}  & \\
& \\
Ten skowyczy trafiony, tamten skomli na wznak,  & \\
Cóż ja sam? Nic tu zrobić nie mogę.  & \\
Niech się zdarzy co musi się zdarzyć i tak,  & \\
Kiedy pocisk zabiegnie mi drogę!  & \\
Starą ranę na karku rozszarpuje do krwi,  & \\
Ale póki wilk krwawi - wilk żyw!  & \\
Więc to jeszcze nie śmierć! Śmierć ostrzejsze ma kły!  & \\
Nie mój tryumf, lecz zwycięstwo - nie ich!  & \\
& \\
\hspace*{2em}\textit{Na nic skowyt we wrzawie i skarga!}  & \\
\hspace*{2em}\textit{Póki w żyłach starczy mi krwi}  & \\
\hspace*{2em}\textit{Pierwszy bratu skoczę do gardła,}  & \\
\hspace*{2em}\textit{Gdy zawyje - nie wilki już my!}  & \\
\end{longtable}
\clearpage

% --- Źródło: Obława_IV.tex ---
\section{Obława IV}
\vspace{-\baselineskip}
\textit{Jacek Kaczmarski}\\
\begin{longtable}{ll}
Oto i ja, w skrzepłej posoce skrzepłej osaczony, & \textbf{e G e} \\
Ja - wilk trójłapy - wśród sfory płatnych łapsów & \textbf{e G e} \\
Staję i warczę, kaleki i bezbronny, & \textbf{e G e} \\
Szczuty, jak pies od niepamiętnych czasów. & \textbf{e G e} \\
& \\
Uciekać dalej nie będę już i nie chcę, & \textbf{a F a} \\
Więc pysk w pysk staję z myśliwym i nagonką. & \textbf{a C a} \\
Mdławy niewoli zapach nozdrza łechce & \textbf{a F a} \\
I nagła cisza unosi się nad łąką. & \textbf{F H7} \\
& \\
\hspace*{2em}\textit{Myśliwy jeszcze ma broń i trzyma smycze,} & \textbf{e a} \\
\hspace*{2em}\textit{Lecz las jest nasz i łąki też są nasze!} & \textbf{C H7 e} \\
\hspace*{2em}\textit{Wilk wolny - wyje, na smyczy pies - skowycze} & \textbf{e a} \\
\hspace*{2em}\textit{I bać się musi i swoich braci straszyć.} & \textbf{C H7 e} \\
& \\
Popatrzcie na mnie, gończe psy zziajane,  & \\
Bite za próżną pogoń za swym bratem -  & \\
Stoję przed wami, po stokroć zabijany  & \\
Z blizną na karku, z odgryzioną łapą.  & \\
& \\
Nie ufam wam, ale i nie chcę zaufania,  & \\
Swoje za sobą mam i macie wy za swoje.  & \\
Byłem ścigany, byliście oszukani  & \\
A oszukanych sfor - ja się nie boję. &  \\
& \\
\hspace*{2em}\textit{Myśliwy jeszcze ma broń i trzyma smycze,}  & \\
\hspace*{2em}\textit{Lecz las jest nasz i łąki też są nasze!}  & \\
\hspace*{2em}\textit{Wilk wolny - wyje, na smyczy pies - skowycze}  & \\
\hspace*{2em}\textit{I bać się musi i swoich braci straszyć.}  & \\
& \\
Podejdźcie do mnie wy, karmione z ręki,  & \\
Kikut i blizna to wolności cena.  & \\
Sam tylko zapach jej zaciska szczęki  & \\
Psa, w którym skomle zapomniany szczeniak.  & \\
& \\
Po lasach jeszcze wciąż żyją wilki młode,  & \\
Porozpraszane przez bezrozumne salwy,  & \\
Silne i wściekłe, i strasznej zemsty głodne  & \\
I ja je kocham i tak mi bardzo żal ich.  & \\
& \\
& \\
\end{longtable}
\newpage
\begin{longtable}{ll}
\hspace*{2em}\textit{Myśliwy jeszcze ma broń i trzyma smycze,}  & \\
\hspace*{2em}\textit{Lecz las jest nasz i łąki też są nasze!}  & \\
\hspace*{2em}\textit{Wilk wolny - wyje, na smyczy pies - skowycze}  & \\
\hspace*{2em}\textit{I bać się musi i swoich braci straszyć.}  & \\
& \\
Niejeden z was, co się na miskę łaszczy  & \\
Zapominając swoje niespokojne sny -  & \\
Wie wszak, że bije ręka, która głaszcze,  & \\
Na pierwszy objaw jedynego zewu krwi.  & \\
& \\
Skomleć o łaskę - niegodne psa, ni wilka,  & \\
Dać się tresować i na rozkazy czekać!  & \\
Nasza ma być najkrótsza życia chwilka,  & \\
I być wyborem - przyjaźń do człowieka...  & \\
& \\
\hspace*{2em}\textit{Wilk wolny - wyje, na smyczy pies - skowycze}  & \\
\hspace*{2em}\textit{I bać się musi i swoich braci straszyć.}  & \\
\hspace*{2em}\textit{Myśliwy jeszcze ma broń i trzyma smycze,}  & \\
\hspace*{2em}\textit{Lecz las jest nasz i łąki też są nasze!}  & \\
\end{longtable}
\clearpage

% --- Źródło: Obława_V.tex ---
\section{Obława V}
\vspace{-\baselineskip}
\textit{Martin Lechowicz}\\
\begin{longtable}{ll}
Żyć w lasach nie musimy nikt na nas nie poluje & \textbf{e e G D} \\
Prowadzą nas na smyczy i śledzą każdy ruch & \textbf{e e G D} \\
Bo mądry pies nie warczy bo mądry pies waruje & \textbf{e G a e} \\
Bo wie że za to zawsze mieć będzie pełny brzuch & \textbf{C a C D e} \\
Bo mądry pies nie warczy bo mądry pies waruje & \textbf{a G a e} \\
Bo wie że za to zawsze mieć będzie pełny brzuch & \textbf{C a C D e} \\
& \\
\hspace*{2em}\textit{Nikt broni nie wymierzy i nie podniesie pałki} & \textbf{C D G e} \\
\hspace*{2em}\textit{Po dzikich wolnych wilkach zostały tylko sny} & \textbf{C D G e D} \\
\hspace*{2em}\textit{Nie krzykną już „obława" nie będzie żadnej walki} & \textbf{C G a e} \\
\hspace*{2em}\textit{Bo nie ma więcej wilków zostały tylko psy} & \textbf{C D C D e} \\
& \\
Każdego dnia wstajemy by nową stoczyć walkę & \textbf{e e G D} \\
By jeszcze bezpieczniejsze wygodne życie wieść & \textbf{e e G D} \\
I o to byśmy mogli dla siebie mieć wersalkę & \textbf{a G a e} \\
I o to byśmy częściej kiełbasę mogli jeść & \textbf{C a C D e} \\
I o to byśmy mogli dla siebie mieć wersalkę & \textbf{a G a e} \\
I o to byśmy częściej kiełbasę mogli jeść & \textbf{C a C D e} \\
& \\
\hspace*{2em}\textit{Nikt broni nie wymierzy i nie podniesie pałki} & \textbf{C D G e} \\
\hspace*{2em}\textit{Po dzikich wolnych wilkach zostały tylko sny} & \textbf{C D G e D} \\
\hspace*{2em}\textit{Nie krzykną już „obława" nie będzie żadnej walki} & \textbf{C G a e} \\
\hspace*{2em}\textit{Bo nie ma więcej wilków zostały tylko psy} & \textbf{C D C D e} \\
& \\
I nie ma na nas obław nikt z nami nie chce walczyć & \textbf{e e G D} \\
I nikt nas się nie boi choć mamy jeszcze kły! & \textbf{e e G D} \\
I dławi nas bezsilność zew krwi nam w gardle charczy & \textbf{a G a e} \\
Bo sami wiemy żeśmy tylko domowe psy & \textbf{C a C D e} \\
I dławi nas bezsilność zew krwi nam w gardle charczy & \textbf{a G a e} \\
Bo sami wiemy żeśmy tylko domowe psy & \textbf{C a C D e} \\
& \\
\hspace*{2em}\textit{Nikt broni nie wymierzy i nie podniesie pałki} & \textbf{C D G e} \\
\hspace*{2em}\textit{Po dzikich wolnych wilkach zostały tylko sny} & \textbf{C D G e D} \\
\hspace*{2em}\textit{Nie krzykną już „obława" nie będzie żadnej walki} & \textbf{C G a e} \\
\hspace*{2em}\textit{Bo nie ma więcej wilków zostały tylko psy} & \textbf{C D C D e} \\
& \\
& \\
\end{longtable}
\newpage
\begin{longtable}{ll}
I gdy mi powiedzieli o starym mądrym wilku & \textbf{e e G D} \\
Co odgryzł swoją łapę by z sideł wyrwać się & \textbf{e e G D} \\
Słyszałem go jak krzyczy choć dawno temu żył tu & \textbf{a G a e} \\
Że tam jest moje miejsce gdzie serce woła mnie & \textbf{C a C D e} \\
Słyszałem go jak krzyczy choć dawno temu żył tu & \textbf{a G a e} \\
Że tam jest moje miejsce gdzie serce woła mnie & \textbf{C a C D e} \\
& \\
\hspace*{2em}\textit{Ze snu się przebudzimy gdy wezwie serca zew} & \textbf{C D G e} \\
\hspace*{2em}\textit{Łańcuchy przegryziemy i wyrąbiemy drzwi} & \textbf{C D G e D} \\
\hspace*{2em}\textit{I w lasy pobiegniemy i zawrze w żyłach krew} & \textbf{C G a e} \\
\hspace*{2em}\textit{Bo kiedy braknie wilków wilkami będą psy} & \textbf{C D C D e} \\
& \\
\hspace*{2em}\textit{Ze snu się przebudzimy gdy wezwie serca zew} & \textbf{C D G e} \\
\hspace*{2em}\textit{Łańcuchy przegryziemy i wyrąbiemy drzwi} & \textbf{C D G e D} \\
\hspace*{2em}\textit{I w lasy pobiegniemy i zawrze w żyłach krew} & \textbf{C G a e} \\
\hspace*{2em}\textit{Bo kiedy braknie wilków wilkami będą psy} & \textbf{C D C D e} \\
& \\
\end{longtable}
\clearpage

% --- Źródło: Ogień.tex ---
\section{Ogień}
\begin{longtable}{ll}
\hspace*{2em}\textit{Zwyczaj to stary jak świat} & \textbf{C d} \\
\hspace*{2em}\textit{Ogień, ogień, ogień.} & \textbf{G C} \\
\hspace*{2em}\textit{Rozpalmy blisko nas}  & \\
\hspace*{2em}\textit{Ogień, ogień, ogień.}  & \\
\hspace*{2em}\textit{Dla spóźnionego wędrowca,}  & \\
\hspace*{2em}\textit{Dla wszystkich spóźnionych w noc}  & \\
\hspace*{2em}\textit{Rozpalmy tu, rozpalmy tu Ogień, ogień, ogień.}  & \\
& \\
Pierwsza gwiazdka już wzeszła,  & \\
Czas, by ogień rozpalić.  & \\
Lipy, sosny i buki  & \\
Chylą gałęzie ku nam.  & \\
& \\
\hspace*{2em}\textit{Zwyczaj to stary jak świat...}  & \\
& \\
Najpiękniejsze ognisko  & \\
Z trzaskiem sypią się skry.  & \\
Wokół samych przyjaciół masz,  & \\
Więc śpiewaj z nami i Ty.  & \\
& \\
\hspace*{2em}\textit{Zwyczaj to stary jak świat...}  & \\
\end{longtable}
\clearpage

% --- Źródło: Ojczyzna_Jezior_Błękit.tex ---
\section{Ojczyzna (Jezior Błękit)}
\begin{longtable}{ll}
Kraina srebrnych brzóz, żeremia bobrów & \textbf{G e} \\
łosi potężny ryk, wiatr niesie w dal & \textbf{C D} \\
& \\
\hspace*{2em}\textit{Jezior błękit i groza skał to jest Ojczyzna ma (x2)} & \textbf{G e C D} \\
\hspace*{2em}\textit{Bum tiri bum bum bum…}  & \\
& \\
Kiedyś powrócę tam, zbuduje wigwam & \textbf{G e} \\
Gdzie rzeki bystry nurt, urwisty brzeg & \textbf{C D} \\
& \\
\hspace*{2em}\textit{Jezior błękit i groza skał to jest Ojczyzna ma (x2)} & \textbf{G e C D} \\
\hspace*{2em}\textit{Bum tiri bum bum bum…}  & \\
& \\
Srebrna toń wody, słońce w dolinach & \textbf{G e} \\
Kiedy zobaczę znów wierzchołki gór & \textbf{C D} \\
& \\
\hspace*{2em}\textit{Jezior błękit i groza skał to jest Ojczyzna ma (x2)} & \textbf{G e C D} \\
\hspace*{2em}\textit{Bum tiri bum bum bum…}  & \\
& \\
\end{longtable}
\clearpage

% --- Źródło: Opadły_mgły.tex ---
\section{Opadły mgły}
\vspace{-\baselineskip}
\textit{Stare Dobre Małżeństwo}\\
\begin{longtable}{ll}
Opadły mgły i miasto ze snu się budzi & \textbf{D G} \\
Górą czmycha już noc & \textbf{D A} \\
Ktoś tam cicho czeka, by ktoś powrócił; & \textbf{D G} \\
Do gwiazd jest bliżej niż krok! & \textbf{D A} \\
Pies się włóczy popod murami - bezdomny; & \textbf{D G} \\
Niesie się tęsknota czyjaś na świata cztery strony! & \textbf{D A D G D A} \\
& \\
\hspace*{2em}\textit{A ziemia toczy, toczy swój garb uroczy;} & \textbf{D G} \\
\hspace*{2em}\textit{Toczy, toczy się los!} & \textbf{D A} \\
\hspace*{2em}\textit{A ziemia toczy, toczy swój garb uroczy;} & \textbf{D G} \\
\hspace*{2em}\textit{Toczy, toczy się los!} & \textbf{D A} \\
& \\
\hspace*{2em}\textit{Ty, co płaczesz, ażeby śmiać mógł się ktoś:} & \textbf{D G} \\
\hspace*{2em}\textit{Już dość, już dość, już dość} & \textbf{D A} \\
\hspace*{2em}\textit{Odpędź czarne myśli, dość już twoich łez!} & \textbf{D G} \\
\hspace*{2em}\textit{Niech to wszystko przepadnie we mgle!} & \textbf{D A} \\
& \\
\hspace*{2em}\textit{Bo nowy dzień wstaje} & \textbf{D G} \\
\hspace*{2em}\textit{Nowy dzień!} & \textbf{D A} \\
\hspace*{2em}\textit{Bo nowy dzień wstaje} & \textbf{D G} \\
\hspace*{2em}\textit{Nowy dzień!} & \textbf{D A} \\
\hspace*{2em}\textit{Bo nowy dzień wstaje} & \textbf{D G} \\
\hspace*{2em}\textit{Nowy dzień!} & \textbf{D A} \\
& \\
Z dusznego snu już miasto tu się wynurza & \textbf{D G} \\
Słońce wschodzi gdzieś tam & \textbf{D A} \\
Tramwaj na przystanku zakwitł jak róża; & \textbf{D G} \\
Uchodzą cienie do bram! & \textbf{D A} \\
Ciągną swoje wózki - dwukółki mleczarze; & \textbf{D G} \\
Nad dachami snują się sny podlotków pełne marzeń! & \textbf{D A D G D A} \\
& \\
\hspace*{2em}\textit{A ziemia toczy, toczy swój garb uroczy...}  & \\
& \\
& \\
\end{longtable}
\newpage
\begin{longtable}{ll}
\end{longtable}
\clearpage

% --- Źródło: Oranżada.tex ---
\section{\textbf{Oranżada}}
\vspace{-\baselineskip}
\textit{Koniec Świata}\\
\begin{longtable}{ll}
Chciałbym się jeszcze powłóczyć z tobą & \textbf{h D} \\
Póki żyjemy i mam cię obok & \textbf{A G} \\
Zjechać z tobą w dół po poręczy & \textbf{h D} \\
Wspólnie się wyczołgać z nędzy & \textbf{A G} \\
& \\
Schować szczęście tu pod podłogą & \textbf{h A} \\
Zanim przyjadą i nas wywiozą & \textbf{G fis} \\
Zobaczyć razem niebo po burzy & \textbf{h D} \\
Skoczyć w kałuże żyć jak najdłużej & \textbf{A G} \\
& \\
\hspace*{2em}\textit{Siedzę na ławce i patrzę na słońce} & \textbf{h A G fis} \\
\hspace*{2em}\textit{Chyba już dzisiaj nigdzie nie zdążę}  & \\
\hspace*{2em}\textit{Chyba już nigdy nie będzie lepiej}  & \\
\hspace*{2em}\textit{Nie będzie dobrze, więc się nie spieszę}  & \\
& \\
Chciałbym się jeszcze powłóczyć z tobą  & \\
Póki żyjemy i mam cię obok  & \\
Poznać wszystkie diabły i anioły  & \\
Elfy, strzygi i upiory  & \\
& \\
Błąkać się w obrazach świętych  & \\
Spędzić dwie noce u wiedźm przeklętych  & \\
Spotkać tego co się boją  & \\
Boga ze zrudziałą brodą  & \\
& \\
\hspace*{2em}\textit{Siedzę na ławce i patrzę na słońce...}  & \\
& \\
Chciałbym się jeszcze powłóczyć z tobą  & \\
Póki żyjemy i mam cię obok  & \\
Złączyć się jednym przyjemnym dreszczem  & \\
Pochodzić razem nocą po mieście  & \\
& \\
Zimnym zachłysnąć się majem  & \\
Siedzieć i patrzeć na nasze tramwaje  & \\
Wypić z worka oranżadę  & \\
I wyprowadzić się na stałe  & \\
& \\
\hspace*{2em}\textit{Siedzę na ławce i patrzę na słońce...}  & \\
\end{longtable}
\clearpage

% --- Źródło: Ostatnia_kula.tex ---
\section{Ostatnia kula}
\vspace{-\baselineskip}
\textit{Lech Makowiecki}\\
\begin{longtable}{ll}
Zapadam w ciemność i las, & \textbf{a d} \\
To moje życie, to mój dom... & \textbf{a E7} \\
Przyjaciel-Księżyc w porę zgasł... & \textbf{a F} \\
Mateczka – Noc okryła mgłą... & \textbf{C G7} \\
& \\
Już nie mam dokąd uciec stąd  & \\
A nogi nie chcą nosić mnie  & \\
Domyka się obławy krąg...  & \\
Do rana sam wykrwawię się...  & \\
& \\
\hspace*{2em}\textit{Za to, że wolnym chciałem być} & \textbf{a F} \\
\hspace*{2em}\textit{Ścigają mnie jak sfora psów} & \textbf{C G7} \\
\hspace*{2em}\textit{Ci, co na smyczy wolą żyć} & \textbf{a F} \\
\hspace*{2em}\textit{I nie podniosą nigdy głów} & \textbf{C G7} \\
\hspace*{2em}\textit{Moja wolności, nie znał cię} & \textbf{a F} \\
\hspace*{2em}\textit{\smash{Żaden} niewolnik, żaden pies} & \textbf{C G7} \\
\hspace*{2em}\textit{Dla nich obroża - zwykła rzecz.} & \textbf{a d} \\
\hspace*{2em}\textit{Dla mojej szyi pętlą jest...} & \textbf{C G7} \\
\hspace*{2em}\textit{Dla mej wolności...} & \textbf{(a d a E7)} \\
& \\
Tylu przyjaciół przeszło już  & \\
Na drugą stronę, w lepszy świat  & \\
Garbate krzyże pokrył kurz  & \\
W dziurawych hełmach gwiżdże wiatr  & \\
& \\
Szukali Światła w mroczne dni  & \\
Przed Bogiem tylko chyląc kark  & \\
Pośród klęczących dumnie szli  & \\
W pogardzie mając każdy targ  & \\
& \\
\hspace*{2em}\textit{Za to, że wolnym chciałem być...}  & \\
& \\
Wstaje świt, opada mgła  & \\
Nadchodzi nowy, piękny dzień  & \\
Moja wolności, przy mnie trwaj  & \\
Najbardziej teraz kocham cię  & \\
& \\
Ostatni nabój... Krzyża znak...  & \\
Burzy się krew... Szczęknęła broń...  & \\
Wybaczcie, jeśli coś nie tak...  & \\
Przez chwilę lufa chłodzi skroń...  & \\
& \\
\hspace*{2em}\textit{Za to, że wolnym chciałem być...}  & \\
& \\
Nie płacz, kochanie, twe łzy, bardziej mnie ranią niźli cierń  & \\
To już nie boli... To nic... Nareszcie odnalazłem Cię...  & \\
\end{longtable}
\clearpage

% --- Źródło: Ostatnia_mapa_Polski.tex ---
\section{Ostatnia mapa Polski}
\vspace{-\baselineskip}
\textit{Jacek Kaczmarski}\\
\begin{longtable}{ll}
Zbłąkany pocisk w namiot sztabu trafił rano  & \\
I spadł na stół zasłany obrusami map.  & \\
Pergaminowy popiół czyjąś krwią schlapany  & \\
Zamiast jedynej mapy kraju ujrzał sztab.  & \\
& \\
Pędzi Naczelnik wśród wiwatujących czapek,  & \\
Stolica dobrych parę staj, a wróg – tuż, tuż;  & \\
Kraj zalał Moskal, teraz diabli wzięli mapę!  & \\
Może naprawdę Bóg zapomniał o nas już?!  & \\
& \\
Wpada na Zamek, w rozbiegane korytarze.  & \\
Pakuje kufry ktoś, papiery pali ktoś,  & \\
Wiernopoddańcze listy piszą dygnitarze  & \\
O łaskę prosząc w skrusze Jej Cesarską Mość.  & \\
& \\
– Wasza Wysokość ma ostatnią mapę kraju! –  & \\
Woła Naczelnik i królowi – bęc do stóp! –  & \\
Napiera wróg, a na nią w sztabie tam czekają!  & \\
Bez niej – masakra i dla wszystkich wspólny grób!  & \\
& \\
– Ostatniej mapy nie dam kłuć chorągiewkami,  & \\
Co oznaczają wojska, których nie mam już! –  & \\
Rzekł król i Polskę zwinął w rulon, a pergamin  & \\
Jak muszla schował w sobie szum Jej obu mórz.  & \\
& \\
Więc z niczym wybiegł wódz, o gniew wołając boży  & \\
A król, wsłuchany w znikający tupot nóg,  & \\
Pomiędzy osobiste rzeczy mapę włożył  & \\
Do sakwojaża na ostatnią ze swych dróg.  & \\
\end{longtable}
\clearpage

% --- Źródło: Ostatnia_nadzieja.tex ---
\section{Ostatnia nadzieja}
\vspace{-\baselineskip}
\textit{Dawid Podsiadło, Sanah}\\
\begin{longtable}{ll}
Miał pod nosem czarny wąs & \textbf{e} \\
Rozdawał koniaki, liczył na drobniaki & \textbf{C} \\
Ten błagający wzrok & \textbf{a} \\
Chciałby uciec stąd & \textbf{e} \\
Biegł, a wszystko to co miał  & \\
To w kieszeni pyszny trunek  & \\
I jeden kierunek  & \\
By lecieć tam gdzie ptak  & \\
A śpiewał sobie tak  & \\
& \\
\hspace*{2em}\textit{Wszystko to co mam} & \textbf{C} \\
\hspace*{2em}\textit{Wszystko to co mam} & \textbf{D} \\
\hspace*{2em}\textit{To ta nadzieja, że życie mnie poskleja} & \textbf{e G} \\
\hspace*{2em}\textit{Dziś odchodzę sam} & \textbf{C} \\
\hspace*{2em}\textit{Dziś odchodzę sam} & \textbf{D} \\
\hspace*{2em}\textit{Już nie zawrócę} & \textbf{e} \\
\hspace*{2em}\textit{To wszystko dziś porzucę} & \textbf{G} \\
\hspace*{2em}\textit{Ja się zarzekam, uciekam} & \textbf{C} \\
\hspace*{2em}\textit{Dość mam przeznaczenia} & \textbf{D} \\
\hspace*{2em}\textit{Po co zwlekać, czekać} & \textbf{e} \\
\hspace*{2em}\textit{Gdy się nic nie zmienia} & \textbf{G} \\
\hspace*{2em}\textit{Moja mama, mówiła} & \textbf{C} \\
\hspace*{2em}\textit{Ostatnia umiera nadzieja} & \textbf{D} \\
& \\
Drżał jej we włosach piękny kwiat  & \\
Ze strachu, że go zmieni  & \\
Gdy się przestanie mienić  & \\
Więc prężył się jak kot  & \\
Gdy w lustro wbiła wzrok  & \\
Strzał, lubiła trafiać tam  & \\
Skąd wypływała rzeka  & \\
Ta burgundowa rzeka  & \\
Lecz popełniła błąd  & \\
Zbyt wiele serc na stos  & \\
& \\
\hspace*{2em}\textit{Wszystko to co mam...}  & \\
\end{longtable}
\clearpage

% --- Źródło: Ostatnia_nocka.tex ---
\section{Ostatnia nocka}
\vspace{-\baselineskip}
\textit{Yugopolis}\\
\begin{longtable}{ll}
Boli mnie głowa i nie mogę spać, & \textbf{a G C} \\
chociaż dokoła wszyscy już posnęli, & \textbf{F E} \\
nie mogę leżeć a nie mogę wstać, & \textbf{a G C} \\
mija ostatnia nocka w mojej celi. & \textbf{F E} \\
& \\
\hspace*{2em}\textit{Tylko noc, noc, noc, płoną światła lamp,} & \textbf{a G C} \\
\hspace*{2em}\textit{nocny reflektor teren przeczesuje,} & \textbf{F E} \\
\hspace*{2em}\textit{owo światło to jak ja dobrze znam,} & \textbf{a G C} \\
\hspace*{2em}\textit{nigdy nie gaśnie ktoś zawsze obserwuje.} & \textbf{F E} \\
\hspace*{2em}\textit{Nie wiem czy wierzę jej czy nie wierzę,} & \textbf{a G C F E} \\
\hspace*{2em}\textit{wierzę jej czy nie wierzę.} & \textbf{a G C F E} \\
& \\
Ostatnia doba jutro będę tam,  & \\
ale na razie ciągle jestem tutaj,  & \\
nie mogę leżeć a nie mogę spać,  & \\
„gad” po „betonce” kamaszami stuka.  & \\
& \\
\hspace*{2em}\textit{Tylko noc, noc, noc, płoną światła lamp...}  & \\
& \\
Boli mnie głowa i nie mogę spać,  & \\
chociaż dokoła wszyscy już posnęli,  & \\
nie mogę leżeć a nie mogę wstać,  & \\
parę lat życia darmo diabli wzięli.  & \\
& \\
\hspace*{2em}\textit{Tylko noc, noc, noc, płoną światła lamp...}  & \\
& \\
Gdy przyjdzie ranek stanę u twych bram,  & \\
się pożegnałem bez do widzenia,  & \\
nie wiem czy będziesz tam,  & \\
nie ma znaczenia wychodzę z więzienia.  & \\
& \\
\hspace*{2em}\textit{Tylko noc, noc, noc, płoną światła lamp...}  & \\
\end{longtable}
\clearpage

% --- Źródło: Pałacyk_Michla.tex ---
\section{Pałacyk Michla}
\begin{longtable}{ll}
Pałacyk Michla, Żytnia, Wola, & \textbf{C} \\
Bronią się chłopcy od „Parasola” & \textbf{G C} \\
Choć na „tygrysy” mają visy, & \textbf{C} \\
To Warszawiaki fajne urwisy są & \textbf{G C G C} \\
& \\
\hspace*{2em}\textit{Czuwaj, wiaro, i wytężaj słuch,} & \textbf{G C} \\
\hspace*{2em}\textit{Pręż swój młody duch, pracując jak zuch!} & \textbf{G C} \\
\hspace*{2em}\textit{Czuwaj, wiaro, i wytężaj słuch,} & \textbf{G C} \\
\hspace*{2em}\textit{Pręż swój młody duch jak stal!} & \textbf{G C} \\
& \\
Każdy chłopaczek chce być ranny & \textbf{C} \\
Sanitariuszki - morowe panny, & \textbf{G C} \\
I gdy cię kula trafi jaka, & \textbf{C} \\
Poprosisz pannę - da ci buziaka, hej! & \textbf{G C G C} \\
& \\
\hspace*{2em}\textit{Czuwaj, wiaro, i wytężaj słuch...}  & \\
& \\
Z tyłu za linią dekowniki, & \textbf{C} \\
Intendentura, różne umrzyki, & \textbf{G C} \\
Gotują zupę, czarną kawę- & \textbf{C} \\
I tym sposobem walczą za sprawę, hej! & \textbf{G C G C} \\
& \\
\hspace*{2em}\textit{Czuwaj, wiaro, i wytężaj słuch...}  & \\
& \\
Za to dowództwo jest morowe, & \textbf{C} \\
Bo w pierwszej linii nadstawia głowę, & \textbf{G C} \\
A najmorowszy z przełożonych & \textbf{C} \\
To jest nasz „Miecio”, w kółko golony, hej! & \textbf{G C G C} \\
& \\
\hspace*{2em}\textit{Czuwaj, wiaro, i wytężaj słuch...}  & \\
& \\
Wiara się bije, wiara śpiewa,  & \\
szkopy się złoszczą, krew ich zalewa,  & \\
różnych sposobów się imają,  & \\
co chwila „szafę” nam posuwają, hej!  & \\
& \\
\hspace*{2em}\textit{Czuwaj, wiaro, i wytężaj słuch...}  & \\
& \\
Lecz na nic „szafa” i granaty,  & \\
za każdym razem dostają baty  & \\
i co dzień się przybliża chwila,  & \\
że zwyciężymy - i do cywila, hej!  & \\
\end{longtable}
\clearpage

% --- Źródło: Pejzaże_harasymowiczowskie.tex ---
\section{Pejzaże harasymowiczowskie}
\vspace{-\baselineskip}
\textit{Wolna Grupa Bukowina}\\
\begin{longtable}{ll}
Kiedy stałem w przedświcie, a Synaj & \textbf{G D} \\
Prawdę głosił przez trąby wiatru, & \textbf{C e} \\
Zasmreczyły się chmur igliwiem - & \textbf{G D} \\
Bure świerki o górach wsparte. & \textbf{e C D} \\
I na niebie byłem ja jeden & \textbf{G D} \\
Plotąc pieśni w warkocze bukowe & \textbf{C e} \\
I schodziłem na ziemię za kwestą & \textbf{G D} \\
Przez skrzydlącą się bramę Lackowej & \textbf{e C D} \\
& \\
\hspace*{2em}\textit{I był Beskid, i były słowa} & \textbf{G C G} \\
\hspace*{2em}\textit{Zanurzone po pępki w cerkwi baniach} & \textbf{C D} \\
\hspace*{2em}\textit{Rozłożyście złotych} & \textbf{D} \\
\hspace*{2em}\textit{Smagających się wiatrem do krwi} & \textbf{C D G} \\
& \\
Moje myśli biegały końmi  & \\
Po niebieskich mokrych połoninach  & \\
I modliłem się złożywszy dłonie  & \\
Do gór, do Madonny brunatnolicej  & \\
A gdy serce kroplami tęsknoty  & \\
Jęło spadać na góry sine  & \\
Czarodziejskim kwiatem paproci  & \\
Rozzłociła się bukowina  & \\
& \\
I był Beskid, i były słowa  & \\
Zanurzone po pępki w cerkwi baniach  & \\
Rozłożyście złotych  & \\
Smagających się wiatrem do krwi  & \\
& \\
\end{longtable}
\clearpage

% --- Źródło: Pieśń_Dawida_na_pustyni.tex ---
\section{\textbf{Pieśń Dawida na pustyni}}
\begin{longtable}{ll}
Daleko wędrowałem sam, & \textbf{e h} \\
nie miałem ani kropli wody, & \textbf{h e} \\
ktoś żywił mnie i o mnie dbał & \textbf{e h} \\
najmniejszej nie poniosłem szkody. & \textbf{h e} \\
& \\
\hspace*{2em}\textit{Bóg jest jak ogień, Bóg jest jak wiatr,} & \textbf{G D} \\
\hspace*{2em}\textit{mocny jak morze, wielki jak świat} & \textbf{e h} \\
\hspace*{2em}\textit{Bóg jest jak Bóg, Bóg jest jak Bóg, Bóg jest jak Bóg?} & \textbf{e h e h e} \\
& \\
Złoczyńcy otoczyli mnie & \textbf{e h} \\
i starliby mnie bez przeszkody, & \textbf{h e} \\
obłok i góra skryły mnie & \textbf{e h} \\
najmniejszej nie poniosłem szkody. & \textbf{h e} \\
& \\
\hspace*{2em}\textit{Bóg jest jak ogień, Bóg jest jak wiatr…}  & \\
& \\
I zapytałem siebie kto & \textbf{e h} \\
ochrania mnie od złego losu? & \textbf{h e} \\
Zabłysnął obłok, huknął grzmot & \textbf{e h} \\
nie mogłem z siebie dobyć głosu. & \textbf{h e} \\
& \\
\hspace*{2em}\textit{(Refren nucony)}  & \\
& \\
Aż odczytałem pismo gwiazd & \textbf{e h} \\
rękę co tknęła mnie poznałem & \textbf{h e} \\
pojąłem skąd mam to co mam & \textbf{e h} \\
i wtedy głos swój odzyskałem & \textbf{h e} \\
& \\
\hspace*{2em}\textit{Bóg jest jak ogień, Bóg jest jak wiatr…}  & \\
\end{longtable}
\clearpage

% --- Źródło: Pieśń_pożegnalna.tex ---
\section{Pieśń pożegnalna}
\begin{longtable}{ll}
Ogniska już dogasa blask, & \textbf{C G} \\
braterski splećmy krąg, & \textbf{C F} \\
w wieczornej ciszy, w świetle gwiazd & \textbf{C G} \\
ostatni uścisk rąk. & \textbf{C F C} \\
& \\
\hspace*{2em}\textit{Kto raz przyjaźni poznał moc,} & \textbf{C G} \\
\hspace*{2em}\textit{nie będzie trwonić słów.} & \textbf{C F} \\
\hspace*{2em}\textit{Przy innym ogniu, w inną noc} & \textbf{C G} \\
\hspace*{2em}\textit{do zobaczenia znów.} & \textbf{C F C} \\
& \\
Nie zgaśnie tej przyjaźni żar, & \textbf{C G} \\
co połączyła nas. & \textbf{C F} \\
Nie pozwolimy by ją starł & \textbf{C G} \\
nieubłagany czas. & \textbf{C F C} \\
& \\
\hspace*{2em}\textit{Kto raz przyjaźni poznał moc,} & \textbf{C G} \\
\hspace*{2em}\textit{nie będzie trwonić słów.} & \textbf{C F} \\
\hspace*{2em}\textit{Przy innym ogniu, w inną noc} & \textbf{C G} \\
\hspace*{2em}\textit{do zobaczenia znów.} & \textbf{C F C} \\
\end{longtable}
\clearpage

% --- Źródło: Pieśń_wielorybników.tex ---
\section{\textbf{Pieśń wielorybników}}
\begin{longtable}{ll}
Nasz diament prawie gotów już, w cieśninach nie ma kry & \textbf{a e a e} \\
Na kei piękne panny stoją, a w oczach błyszczą łzy & \textbf{a e d e a} \\
Kapitan w niebo wlepia wzrok, ruszamy lada dzień & \textbf{a e a e} \\
Płyniemy tam gdzie słońca blask, nie mąci nocy dzień & \textbf{a e d e a} \\
& \\
\hspace*{2em}\textit{A więc krzycz „Ahoj”} & \textbf{a E a} \\
\hspace*{2em}\textit{Odwagę w sercu miej} & \textbf{a E a} \\
\hspace*{2em}\textit{Wielorybów cielska groźne są} & \textbf{a C G} \\
\hspace*{2em}\textit{Lecz dostaniemy je} & \textbf{F E a} \\
& \\
Ej panno po co łzy, nic nie zatrzyma mnie & \textbf{a e a e} \\
Bo prędzej w lodach kwiat zakwitnie, niż wycofam się & \textbf{a e d e a} \\
No nie płacz wrócę tu, nasz los nie taki zły & \textbf{a e a e} \\
Bo da dukatów wór za tran i wielorybie kły & \textbf{a e d e a} \\
& \\
\hspace*{2em}\textit{A więc krzycz „Ahoj”} & \textbf{a E a} \\
\hspace*{2em}\textit{Odwagę w sercu miej} & \textbf{a E a} \\
\hspace*{2em}\textit{Wielorybów cielska groźne są} & \textbf{a C G} \\
\hspace*{2em}\textit{Lecz dostaniemy je} & \textbf{F E a} \\
& \\
Na deku stary wąchał wiatr, lunetę w ręku miał & \textbf{a e a e} \\
Na łodziach co zwisały już, z harpunem każdy stał & \textbf{a e d e a} \\
I dmucha tu i dmucha ta, ogromne stado w krąg & \textbf{a e a e} \\
Harpuny liny wiosła brać i ciągaj brachu ciąg & \textbf{a e d e a} \\
& \\
\hspace*{2em}\textit{A więc krzycz „Ahoj”} & \textbf{a E a} \\
\hspace*{2em}\textit{Odwagę w sercu miej} & \textbf{a E a} \\
\hspace*{2em}\textit{Wielorybów cielska groźne są} & \textbf{a C G} \\
\hspace*{2em}\textit{Lecz dostaniemy je} & \textbf{F E a} \\
& \\
I dla wieloryba już  & \\
Ostatni to dzień  & \\
Bo śmiały harpunnik  & \\
Uderza weń  & \\
& \\
\hspace*{2em}\textit{A więc krzycz „Ahoj”} & \textbf{a E a} \\
\hspace*{2em}\textit{Odwagę w sercu miej} & \textbf{a E a} \\
\hspace*{2em}\textit{Wielorybów cielska groźne są} & \textbf{a C G} \\
\hspace*{2em}\textit{Lecz dostaniemy je} & \textbf{F E a} \\
& \\
\textbf{Z dedykacją dla ……………………… (wpisz imię)}  & \\
\end{longtable}
\clearpage

% --- Źródło: Piła_tango.tex ---
\section{Piła tango}
\vspace{-\baselineskip}
\textit{Strachy na Lachy}\\
\begin{longtable}{ll}
Oto historia z kantem & \textbf{a a2 a d E} \\
Co podwójne ma dno  & \\
Gdyby napisał ją Dante  & \\
To nie tak by to szło  & \\
& \\
Grzesiek Kubiak, czyli Kuba rządził naszą podstawówką  & \\
Po lekcjach na boisku ganiał za mną z cegłówką  & \\
W Pile było jak w Chile, każdy miał czerwone ryło  & \\
Mniej lub bardziej to pamiętasz – spytaj jak to było  & \\
W czasach gdy nad Piłą jeszcze latały samoloty  & \\
Wojewoda Śliwiński kazał pomalować płoty  & \\
Potem wszystkie płoty w Pile miały kolor zieleni  & \\
Rogaczem na wieżowcu Piła witała jeleni  & \\
& \\
\hspace*{2em}\textit{Statek Piła Tango}  & \\
\hspace*{2em}\textit{Czarna bandera}  & \\
\hspace*{2em}\textit{To tylko Piła Tango}  & \\
\hspace*{2em}\textit{Tańczysz to teraz}  & \\
\hspace*{2em}\textit{Płynie statek Piła Tango}  & \\
\hspace*{2em}\textit{Czarna Bandera}  & \\
\hspace*{2em}\textit{Ukłoń się świrom}  & \\
\hspace*{2em}\textit{Żyj, nie umiera}  & \\
& \\
Gruby jak armata Szczepan błąkał się po kuli ziemskiej  & \\
Trafił do Ameryki prosto z Legii Cudzoziemskiej  & \\
Baca w Londynie z buchami się sąsiedzi  & \\
Lżej się tam halucynuje, nikt go tam nie śledzi  & \\
Karawan z Holandii, on przyjechał tutaj wreszcie  & \\
Są już Kula, Czarny Dusioł – słychać strzały na mieście  & \\
Znam jednak takie miejsca, gdzie jest lepiej chodzić z nożem  & \\
Całe Górne i Podlasie – wszyscy są za Kolejorzem  & \\
(Hej Kolejorz!)  & \\
& \\
\hspace*{2em}\textit{Statek Piła Tango...}  & \\
& \\
& \\
\end{longtable}
\newpage
\begin{longtable}{ll}
Andrzej Kozak, Mandaryn? Znana postać medialna  & \\
Tyci przy nim jest kosmos, gaśnie gwiazda polarna  & \\
Jest tu Siwy, który w rękach niebezpieczne ma narzędzie  & \\
A kiedy Siwy tańczy – znaczy mordobicie będzie  & \\
U Budzików pod tytułem chleją nawet z gór szkieły  & \\
Zbigu śpi przy stoliku, ma nieczynny przełyk  & \\
Lecz spokojnie panowie, według mej najlepszej wiedzy  & \\
Najszersze gardła tu to mają z INRI koledzy  & \\
(Polej, polej!)  & \\
& \\
\hspace*{2em}\textit{Statek Piła Tango...}  & \\
& \\
Nad rzeką, latem ferajna na grilla się zasadza  & \\
Auta z Niemiec? Sam wiem kto je tu sprowadza  & \\
Żaden spleen i cud, na ulicach nie śpią złotówki  & \\
W Pile Święta jest Rodzina i święte są żarówki  & \\
Nic nie szkodzi, że z wieczora miasto dławi się w fetorach  & \\
Ważne, że jest żużel i kiełbasy senatora!  & \\
Fajne z Wincentego Pola idą w świat dziewczyny  & \\
Po pokładzie jeździ Jojo bicyklem z Ukrainy  & \\
& \\
\hspace*{2em}\textit{Statek Piła Tango...}  & \\
& \\
Oto historia z kantem  & \\
Co podwójne ma dno  & \\
Gdyby napisał ją Dante  & \\
To nie tak by to szło  & \\
(By szło, by szło)  & \\
& \\
\end{longtable}
\clearpage

% --- Źródło: Poezja.tex ---
\section{Poezja}
\vspace{-\baselineskip}
\textit{Na Bani}\\
\begin{longtable}{ll}
Ty przychodzisz, jak noc majowa, & \textbf{a e} \\
biała noc, uśpiona w jaśminie, & \textbf{F G} \\
i jaśminem pachną twoje słowa, & \textbf{a e} \\
i księżycem sen srebrny płynie. & \textbf{F G} \\
 & \\
Płyniesz cicha przez noce bezsenne & \textbf{a e} \\
– cichą nocą tak liście szeleszczą – & \textbf{F G} \\
szepcesz sny, szepcesz słowa tajemne, & \textbf{a e} \\
w słowach cichych skąpana, jak w deszczu… & \textbf{F G} \\
 & \\
To za mało! Za mało! Za mało! & \\
Twoje słowa tumanią i kłamią! & \\
Piersiom żywych daj oddech zapału, & \\
wiew szeroki i skrzydła do ramion! & \\
 & \\
Nam te słowa ciche nie starczą. & \\
Marne słowa. I błahe. I zimne. & \\
Ty masz werbel nam zagrać do marszu! & \\
Smagać słowem! Bić pieśnią! Wznieść hymnem! & \\
 & \\
Jest gdzieś radość ludzka, zwyczajna, & \\
jest gdzieś jasne i piękne życie. – & \\
Powszedniego chleba słów daj nam & \\
i stań przy nas, i rozkaż – bić się! & \\
 & \\
Niepotrzebne nam białe westalki, & \\
noc nie zdławi świętego ognia – & \\
bądź jak sztandar rozwiany wśród walki, & \\
bądź jak w wichrze wzniesiona pochodnią! & \\
 & \\
Odmień, odmień nam słowa na wargach, & \\
naucz śpiewać płomienniej i prościej, & \\
niech nas miłość ogromna potarga, & \\
więcej bólu i więcej radości! & \\
\end{longtable}
\newpage
\begin{longtable}{ll}
Jeśli w pięści potrzebna ci harfa, & \textbf{a e} \\
jeśli harfa ma zakląć pioruny, & \textbf{F G} \\
rozkaż żyły na struny wyszarpać & \textbf{a e} \\
i naciągać, i trącać, jak struny. & \textbf{F G} \\
 & \\
Trzeba pieśnią bić aż do śmierci, & \\
trzeba głuszyć w ciemnościach syk węży. & \\
Jest gdzieś życie piękniejsze od wierszy. & \\
I jest miłość. I ona zwycięży. & \\
 & \\
Wtenczas daj nam, poezjo, najprostsze & \\
ze słów prostych i z cichych – najcichsze, & \\
a umarłych w wieczności rozpostrzyj, & \\
jak chorągwie podarte na wichrze. & \\
\end{longtable}
\clearpage

% --- Źródło: Poziomki.tex ---
\section{Poziomki}
\begin{longtable}{ll}
Piękne jest dzisiaj to niebo nad nami & \textbf{C d G C} \\
Słońca promienie się tulą do warg  & \\
Pachnie majowo, czerwcowo, lipcowo  & \\
Ptaki aż chrypną od śpiewu i skarg  & \\
& \\
\hspace*{2em}\textit{O, to tak jak gdybym całował} & \textbf{d} \\
\hspace*{2em}\textit{Czerwone poziomki zebrane do ręki} & \textbf{F C a} \\
\hspace*{2em}\textit{Kto, to tak, ten świat zaczarował} & \textbf{d} \\
\hspace*{2em}\textit{tu rzucił czerwień, tam zieleń, tam błękit} & \textbf{F C a} \\
\hspace*{2em}\textit{Dla ciebie ooo...} & \textbf{d} \\
\hspace*{2em}\textit{Dla ciebie ooo...} & \textbf{C} \\
& \\
Drzewa konary schylają w ukłonie  & \\
W takt szumu wiatru kołysze się las  & \\
Przyjdź moja miła podaj mi swe dłonie  & \\
W imieniu lata całuję cię tak  & \\
& \\
\hspace*{2em}\textit{O, to tak jak gdybym całował...}  & \\
& \\
Gdy stanę w cieniu zielonej tej bramy  & \\
Popatrzę wtedy w twe oczy bez dna  & \\
Ty mi do ucha wyszepczesz kochany  & \\
Pójdziemy razem w lipcowy ten świat.  & \\
& \\
\hspace*{2em}\textit{O, to tak jak gdybym całował...}  & \\
\end{longtable}
\clearpage

% --- Źródło: Połoniny_niebieskie.tex ---
\section{Połoniny niebieskie}
\vspace{-\baselineskip}
\textit{Adam Drąg}\\
\begin{longtable}{ll}
Gdy nie zostanie po mnie nic & \textbf{C F C F} \\
Oprócz pożółkłych fotografii & \textbf{C F C G} \\
Błękitny mnie przywita świt & \textbf{e F C G} \\
W miejscu co nie ma go na mapie & \textbf{C F C F} \\
& \\
A kiedy sypną na mnie piach &  \\
Gdy mnie okryją cztery deski  & \\
To pójdę tam gdzie wiedzie szlak  & \\
Na połoniny na niebieskie  & \\
& \\
Podwiezie mnie błękitny wóz  & \\
Ciągnięty przez błękitne konie  & \\
Przez świat błękitny będzie wiózł  & \\
Aż zaniebieszczy w dali błonie  & \\
& \\
Od zmartwień wolny i od trosk  & \\
Pójdę wygrzewać się na trawie  & \\
A czasem gdy mi przyjdzie chęć  & \\
Z góry na ziemię się pogapię  & \\
& \\
Popatrzę jak wśród smukłych malw  & \\
Wiatr w przedwieczornej ciszy kona  & \\
Trochę mi tylko będzie żal  & \\
Że trawa u was tak zielona  & \\
\end{longtable}
\clearpage

% --- Źródło: Pożegnanie_Liverpoolu.tex ---
\section{Pożegnanie Liverpoolu}
\begin{longtable}{ll}
Żegnaj nam dostojny stary porcie & \textbf{C C7 F C} \\
Rzeko Mersey żegnaj nam & \textbf{C G G7} \\
Zaciągnąłem się na rejs do Kalifornii & \textbf{C C7 F C} \\
Byłem tam już niejeden raz & \textbf{C G7 C} \\
& \\
\hspace*{2em}\textit{A więc żegnaj mi kochana ma} & \textbf{G F C} \\
\hspace*{2em}\textit{Za chwilę wypłyniemy w długi rejs} & \textbf{C G G7} \\
\hspace*{2em}\textit{Ile miesięcy Cię nie będę widział} & \textbf{C C7} \\
\hspace*{2em}\textit{Nie wiem sam} & \textbf{F C} \\
\hspace*{2em}\textit{Lecz pamiętać zawsze będę Cię} & \textbf{C G7 C} \\
& \\
Zaciągnąłem się na herbaciany kliper  & \\
Dobry statek choć sławę ma złą  & \\
A że kapitanem jest tam stary Burgess  & \\
Pływającym piekłem wszyscy go zwą  & \\
& \\
\hspace*{2em}\textit{A więc żegnaj mi kochana ma...}  & \\
& \\
Z kapitanem tym płynę już nie pierwszy raz  & \\
Znamy się od wielu wielu lat  & \\
Jeśliś dobrym żeglarzem - radę sobie dasz  & \\
Jeśli nie - toś cholernie wpadł  & \\
& \\
\hspace*{2em}\textit{A więc żegnaj mi kochana ma...}  & \\
& \\
Żegnaj nam dostojny stary porcie  & \\
Rzeko Mersey żegnaj nam  & \\
Wypływamy już na rejs do Kalifornii  & \\
Gdy wrócimy - opowiemy wam  & \\
& \\
\hspace*{2em}\textit{A więc żegnaj mi kochana ma...}  & \\
\end{longtable}
\clearpage

% --- Źródło: Preludium_dla_Leonarda.tex ---
\section{Preludium dla Leonarda}
\begin{longtable}{ll}
Na parterze w mojej chacie & \textbf{D D} \\
Mieszkał kiedyś taki facet & \textbf{G D} \\
Który dnia pewnego cicho do mnie rzekł; & \textbf{C G D D} \\
Gdy zachwycisz się dziewczyną & \textbf{C G} \\
Nie podrywaj jej na kino & \textbf{D A} \\
Ale prosto w oczy szepnij słowa te & \textbf{C G D D} \\
& \\
\hspace*{2em}\textit{Jestem taki samotny,} & \textbf{h G} \\
\hspace*{2em}\textit{jak palec albo pies} & \textbf{D A} \\
\hspace*{2em}\textit{Kocham wiersze Stachury} & \textbf{C G} \\
\hspace*{2em}\textit{i stary dobry jazz} & \textbf{D D} \\
\hspace*{2em}\textit{Szczęścia w życiu nie miałem,} & \textbf{h G} \\
\hspace*{2em}\textit{rzucały mnie dziewczyny} & \textbf{D A} \\
\hspace*{2em}\textit{Szukam cichego portu,} & \textbf{C} \\
\hspace*{2em}\textit{gdzie okręt mój zawinie} & \textbf{G D D} \\
& \\
Po tych słowach z miłosierdzia & \textbf{D D} \\
Padła już niejedna twierdza & \textbf{G D} \\
I niejedna cnota chyżo poszła w las & \textbf{C G D D} \\
Ryba bierze na robaki & \textbf{C G} \\
A panienka na tekst taki & \textbf{D A} \\
Który zawsze prosto w oczy szepcze jej; & \textbf{C G D D} \\
& \\
\hspace*{2em}\textit{Jestem taki samotny,} & \textbf{h G} \\
\hspace*{2em}\textit{jak palec albo pies} & \textbf{D A} \\
\hspace*{2em}\textit{Kocham wiersze Stachury} & \textbf{C G} \\
\hspace*{2em}\textit{i stary dobry jazz} & \textbf{D D} \\
\hspace*{2em}\textit{Szczęścia w życiu nie miałem,} & \textbf{h G} \\
\hspace*{2em}\textit{rzucały mnie dziewczyny} & \textbf{D A} \\
\hspace*{2em}\textit{Szukam cichego portu,} & \textbf{C} \\
\hspace*{2em}\textit{gdzie okręt mój zawinie} & \textbf{C G D D} \\
& \\
Gdy czas pierwszych zrywów minął & \textbf{D D} \\
Zakochałem się w dziewczynie & \textbf{G D} \\
Z którą się na całe życie zostać chce & \textbf{C G D D} \\
Chciałem rzec; będziemy razem & \textbf{C G} \\
Zrozumiała mnie od razu & \textbf{D A} \\
I jak echo powtórzyła słowa me: & \textbf{C G D D} \\
& \\
& \\
\end{longtable}
\newpage
\begin{longtable}{ll}
\hspace*{2em}\textit{Jesteś taki samotny,} & \textbf{h G} \\
\hspace*{2em}\textit{jak palec albo pies} & \textbf{D A} \\
\hspace*{2em}\textit{Kochasz wiersze Stachury} & \textbf{C G} \\
\hspace*{2em}\textit{i stary dobry jazz} & \textbf{D D} \\
\hspace*{2em}\textit{Szczęścia w życiu nie miałeś,} & \textbf{h G} \\
\hspace*{2em}\textit{rzucały cię dziewczyny} & \textbf{D A} \\
\hspace*{2em}\textit{Szukasz cichego portu,} & \textbf{C} \\
\hspace*{2em}\textit{gdzie okręt twój zawinie} & \textbf{G D D} \\
& \\
\hspace*{2em}\textit{Jestem taka samotna,} & \textbf{h G} \\
\hspace*{2em}\textit{jak bardzo stara miotła} & \textbf{D A} \\
\hspace*{2em}\textit{Kocham wiersze Leśmiana} & \textbf{C G} \\
\hspace*{2em}\textit{i szaleć aż do rana} & \textbf{D D} \\
\hspace*{2em}\textit{Szczęścia w życiu nie miałam,} & \textbf{h G} \\
\hspace*{2em}\textit{Chłopaków porzucałam} & \textbf{D A} \\
\hspace*{2em}\textit{Szukam cichego portu,} & \textbf{C} \\
\hspace*{2em}\textit{Do uprawiania sportu} & \textbf{G D D} \\
& \\
\end{longtable}
\clearpage

% --- Źródło: Przebudzenie.tex ---
\section{Przebudzenie}
\begin{longtable}{ll}
Słuchać w pełnym słońcu jak pulsuje ziemia, & \textbf{C G d a} \\
Uspokoić swoje serce, niczego już nie zmieniać  & \\
I uwierzyć w siebie porzucając sny  & \\
To Twój bunt przemija, a nie Ty.  & \\
& \\
\hspace*{2em}\textit{A Ty nie wiesz, nie wiesz, nie wiesz Nie rozumiesz nic (x2)}  & \\
& \\
Widzieć parę bobrów przytulonych nad potokiem,  & \\
Nie zabijać ich więcej, cieszyć się widokiem,  & \\
Nie wyjadać im wnętrzności, nie wchodzić im w skórę,  & \\
Stępić w sobie instynkt łowcy, wtopić się w naturę  & \\
& \\
\hspace*{2em}\textit{A Ty nie wiesz, nie wiesz, nie wiesz Nie rozumiesz nic (x2)}  & \\
& \\
Wybrać to co dobre z mądrych starych ksiąg,  & \\
Uszanować swoją godność doceniając ją,  & \\
A gdy wreszcie uda się własne złe pokonać.  & \\
Żeby zawsze mieć przy sobie czyjeś ramiona.  & \\
& \\
\hspace*{2em}\textit{A Ty nie wiesz, nie wiesz, nie wiesz Nie rozumiesz nic (x2)}  & \\
& \\
Wyczuć taką chwilę, w której kocha się życie  & \\
I móc w niej stale, na wieczność w zachwycie  & \\
W pełnym słońcu, dumnie i na własnych nogach  & \\
Może wtedy będzie można dojrzeć uśmiech Boga.  & \\
& \\
\hspace*{2em}\textit{Przejść wielką rzekę bez bólów i wyrzeczeń / x3} & \textbf{C G F G} \\
\hspace*{2em}\textit{Przejść Wielką Rzekę…}  & \\
& \\
& \\
\end{longtable}
\newpage
\begin{longtable}{ll}
\end{longtable}
\clearpage

% --- Źródło: Przechyły.tex ---
\section{\textbf{Przechyły}}
\begin{longtable}{ll}
Pierwszy raz przy pełnym takielunku & \textbf{e D e} \\
Trzymam ster i biorę kurs na wiatr! & \textbf{e D e} \\
I jest jak przy pierwszym pocałunku, & \textbf{a D e} \\
W ustach sól, gorącej wody smak & \textbf{a H7 e} \\
& \\
\hspace*{2em}\textit{O-ho-ho, przechyły i przechyły,} & \textbf{a D e} \\
\hspace*{2em}\textit{O-ho-ho, za falą fala mknie,} & \textbf{a D e} \\
\hspace*{2em}\textit{O-ho-ho, trzymajcie się dziewczyny!} & \textbf{a D e} \\
\hspace*{2em}\textit{Ale wiatr, ósemka chyba dmie} & \textbf{a H7 e} \\
& \\
Zwrot przez sztag! – O’key, zaraz zrobię! & \textbf{e D e} \\
Słyszę jak kapitan cicho klnie & \textbf{e D e} \\
Bo gubię wiatr i zamiast w niego dziobem, & \textbf{a D e} \\
To on mnie od tyłu kumple w śmiech! & \textbf{A H7 e} \\
& \\
\hspace*{2em}\textit{O-ho-ho, przechyły i przechyły,} & \textbf{a D e} \\
\hspace*{2em}\textit{O-ho-ho, za falą fala mknie,} & \textbf{a D e} \\
\hspace*{2em}\textit{O-ho-ho, trzymajcie się dziewczyny!} & \textbf{a D e} \\
\hspace*{2em}\textit{Ale wiatr, ósemka chyba dmie} & \textbf{a H7 e} \\
& \\
Hej, ty tam, za burtę wychylony! & \textbf{e D e} \\
Tu naprawdę się nie ma z czego śmiać! & \textbf{e D e} \\
Cicho siedź i lepiej proś Neptuna, & \textbf{a D e} \\
Żeby coś nie spadło ci na kark! & \textbf{a H7 e} \\
& \\
\hspace*{2em}\textit{O-ho-ho, przechyły i przechyły,} & \textbf{a D e} \\
\hspace*{2em}\textit{O-ho-ho, za falą fala mknie,} & \textbf{a D e} \\
\hspace*{2em}\textit{O-ho-ho, trzymajcie się dziewczyny!} & \textbf{a D e} \\
\hspace*{2em}\textit{Ale wiatr, ósemka chyba dmie} & \textbf{a H7 e} \\
\end{longtable}
\clearpage

% --- Źródło: Przemijanie.tex ---
\section{Przemijanie}
\begin{longtable}{ll}
Dzień kolejny minął & \textbf{a G a} \\
Dzień co nic nie przyniósł & \textbf{C G a e} \\
Jeszcze się nie skończył & \textbf{C G a e} \\
A już nowy wyrósł & \textbf{C G a} \\
& \\
\hspace*{2em}\textit{Tyle dni minęło, tyle marzeń} & \textbf{C G a e} \\
\hspace*{2em}\textit{Tyle ludzi przeszło, tyle zdarzeń} & \textbf{C G a e} \\
\hspace*{2em}\textit{Tyle marzeń sennych się nie spełniło} & \textbf{C G a e} \\
\hspace*{2em}\textit{Tyle dobrych gwiazd ubyło.} & \textbf{C G a} \\
& \\
Tyle słów powiedział  & \\
Słów, co nic nie znaczą  & \\
Może kogoś uraził  & \\
Czyjeś oczy płaczą  & \\
& \\
\hspace*{2em}\textit{Tyle dni minęło, tyle marzeń...}  & \\
& \\
Znowu czasu mijanie,  & \\
Znowu minął dzień  & \\
Komu przyniósł radość?  & \\
Komu smutek zeń  & \\
& \\
\hspace*{2em}\textit{Tyle dni minęło, tyle marzeń...}  & \\
\end{longtable}
\clearpage

% --- Źródło: Płonie_ognisko_i_szumią_knieje.tex ---
\section{\textbf{Płonie ognisko i szumią knieje}}
\begin{longtable}{ll}
Płonie ognisko i szumią knieje & \textbf{a E a} \\
Drużynowy jest wśród nas & \textbf{E E7 a} \\
Opowiada starodawne dzieje & \textbf{a E a} \\
Bohaterski wskrzesza czas & \textbf{E E7 a G} \\
& \\
O rycerstwie spod kresowych stanic & \textbf{C G} \\
O obrońcach naszych polskich granic & \textbf{E E7 a E7} \\
A ponad nami wiatr szumny wieje & \textbf{a E a} \\
I dębowy huczy las & \textbf{E E7 a} \\
& \\
Już do odwrotu głos trąbki wzywa & \textbf{a E a} \\
Alarmując ze wszech stron & \textbf{E E7 a} \\
Staje wiara w ordynku szczęśliwa & \textbf{a E a} \\
Serca biją w zgodny ton & \textbf{E E7 a G} \\
& \\
Każda twarz się uniesieniem płoni & \textbf{C G} \\
Każdy laskę krzepko dzierży w dłoni & \textbf{E E7 a E7} \\
A z młodzieńczej się piersi wyrywa & \textbf{a E a} \\
Pieśń potężna pieśń jak dzwon & \textbf{E E7 a} \\
& \\
Zgasło ognisko i szumią drzewa & \textbf{a E a} \\
Spojrzyj weń ostatni raz & \textbf{E E7 a} \\
Niech ci w duszy radośnie zaśpiewa & \textbf{a E a} \\
Że na zawsze łączą nas & \textbf{E E7 a G} \\
& \\
Wspólne troski i radości życia & \textbf{C G} \\
Serc harcerskich zjednoczone bicia & \textbf{E E7 a E7} \\
I ta przyjaźń najszczersza na świecie & \textbf{a E a} \\
Którą Bóg połączył nas & \textbf{E E7 a} \\
\end{longtable}
\clearpage

% --- Źródło: Rapsod_o_Warneńczyku.tex ---
\section{\textbf{Rapsod o Warneńczyku}}
\begin{longtable}{ll}
Lśni chorągiew pozłocista & \textbf{a G a} \\
Chrzęści zbroja szmelcowana & \textbf{C G a} \\
Jedzie, jedzie król Władysław & \textbf{d a} \\
By pokonać Bisurmana & \textbf{G a} \\
& \\
Po wąwozach grzmią cykady  & \\
Koń królewski raźno parska  & \\
Dzielny Węgier Jan Hunyady  & \\
Sprawdza szyki klnąc z madziarska  & \\
& \\
Nad wzgórzami wstają zorze  & \\
Wojsko w marszu rumor czyni  & \\
O już widać czarne morze  & \\
Rzecze Legat Cezarini  & \\
& \\
I ruszają gromić pogan  & \\
W sile szesnastu tysięcy  & \\
A ich okrzyk, tak potężny  & \\
Niczym stary dzwon co dźwięczy  & \\
& \\
Król naprędce je śniadanie  & \\
Jan Hunyady wszedł z łoskotem  & \\
Nawalili wenecjanie  & \\
Wycofują swoją flotę  & \\
& \\
Król odstawił kubek z winem  & \\
Błysk mu strzelił spod powieki  & \\
Wyruszamy za godzinę  & \\
A Wenecji wstyd na wieki  & \\
& \\
Wszędzie ruch i gwar panował,  & \\
Rycerze wsiedli na konie,  & \\
Każdy na śmierć się gotował,  & \\
W pożegnaniu wznieśli dłonie  & \\
& \\
Jeszcze Warna w dali drzemie  & \\
Jeszcze nisko stoi słońce  & \\
A pancerni strzemię w strzemię  & \\
A pancerni koncerz w koncerz  & \\
& \\
A pancerni kopia w kopię  & \\
Ku piaszczystym patrzą brzegom  & \\
No to cześć- daj pyska chłopie  & \\
Rzecze król do Hunyadego  & \\
\end{longtable}
\newpage
\begin{longtable}{ll}
I trzasnęły jednym trzaskiem  & \\
Setki przyłbic zatrzaśniętych  & \\
I błysnęły jednym blaskiem  & \\
Setki mieczy wyszarpniętych  & \\
I zadrżała ziemia święta  & \\
I huknęły dzwony w mieście  & \\
I ruszyli najpierw stępa  & \\
Potem kłusem, cwałem wreszcie  & \\
& \\
A Janczarzy zaprzedańcy  & \\
Atakują Węgrów batem  & \\
A szalony król Warneńczyk  & \\
Dla większości stał się katem  & \\
& \\
Amurat ominął skrzydło  & \\
I uderzył w Bobrzyckiego  & \\
A w burnusach dzikie bydło  & \\
Jęło bić się na całego  & \\
& \\
Sześć tysięcy oturaków  & \\
Uderzyło na Frankbana  & \\
A Słoweniec wraz z Biskupem  & \\
Rzucił wszystkich na kolana  & \\
& \\
Biskup Szymon w kontrataku  & \\
Rzucił bić się za Rozgonie,  & \\
Lecz padł ze swym wojskiem w krzyku  & \\
Pianą krwawą pluły konie.  & \\
& \\
Lech Bobrzycki zbiera wojsko  & \\
I za wrogiem się rozgląda,  & \\
Pot ociera z czoła ciężko,  & \\
Wpada jazda na wielbłądach  & \\
& \\
Konie rżą zaraz w popłochu,  & \\
Widząc z garbem stwory dziwne  & \\
Turcy jadą na Wołochów  & \\
Zagęszczając dziką bitwę  & \\
& \\
Pędzi do króla posłaniec  & \\
Cały we krwi ubabrany  & \\
Rzecze: Najjaśniejszy Panie  & \\
Lech Bobrzycki usiekany  & \\
\end{longtable}
\newpage
\begin{longtable}{ll}
Czoło marszczy się królewskie  & \\
Za kompanów tą niedolę  & \\
Władysław do wojska rzecze:  & \\
Ja od sromu śmierć dziś wolę  & \\
\ & \\
Spina swego konia w cwale  & \\
Widząc wojska swe w agonii  & \\
I w bitewnym pędząc szale  & \\
Wrzeszczy: żołnierze do broni  & \\
\\ & \\
Strzaskane padają kopie,  & \\
Miecze oraz mizykordie,  & \\
Jeszcze pięść żelazna łupie  & \\
Grając wojenną melodię  & \\
& \\
W sile pięciuset rycerstwa  & \\
Ruszył młody król na pogan  & \\
I nie tracąc wcale męstwa  & \\
Pędząc modlił się do Boga  & \\
& \\
Poszła dzielna polska jazda  & \\
Poszli Węgrzy niczym diabli  & \\
Jak stalowa ostra drzazga  & \\
Jak błyszczące ostrze szabli  & \\
& \\
I widziano jak lecieli  & \\
Pędem dzikim i szalonym  & \\
I widziano jak tonęli  & \\
W morzu Turków niezmierzonym  & \\
& \\
Król Władysław stracił konia  & \\
I z rozpędem padł na ziemię  & \\
Pod naporem wroga skonał  & \\
Wraz z rycerstwem stanął w niebie  & \\
& \\
Zbezczeszczoną głowę króla  & \\
Turcy zatknęli na pice  & \\
Węgrzy niczym pszczoły z ula  & \\
& \\
Potem z piórem siadł pod skarpą  & \\
Mnich uczony, stary skryba  & \\
Warto było czy nie warto  & \\
Odwrót lepszy byłby chyba  & \\
\end{longtable}
\newpage
\begin{longtable}{ll}
Chrzanił zacny zjadacz chleba  & \\
Czas nad nami wartko goni  & \\
I tak przecież umrzeć trzeba  & \\
To już lepiej tak jak oni  & \\
& \\
Zresztą koniec dzieło wieńczy  & \\
Mnich w klasztorze kipnął marnie  & \\
A szalony król Warneńczyk  & \\
Ma grobowiec w pięknej Warnie  & \\
& \\
I szanują go Bułgarzy  & \\
I nas dzięki niemu cenią  & \\
Więc na czarnomorskiej plaży  & \\
Składam hołd królewskim cieniom  & \\
\end{longtable}
\clearpage

% --- Źródło: Równoległe.tex ---
\section{Równoległe}
\vspace{-\baselineskip}
\textit{Słodki Całus Od Buby}\\
\begin{longtable}{ll}
A jeśli było to możliwe: & \textbf{C F G} \\
Na plaży już czekała noc, & \textbf{C F G} \\
Pod granatowym baldachimem & \textbf{C F G} \\
Z niecierpliwości gęstniał mrok... & \textbf{C F G} \\
A jeśli było to możliwe: & \textbf{C F G} \\
Nad ranem w śnieżnobiałym łóżku & \textbf{C F G} \\
Puentować noc czerwonym winem, & \textbf{C F G} \\
Pośród kradzionych pocałunków... & \textbf{C F G} \\
& \\
\hspace*{2em}\textit{Jedno wiem, jedno jest pewne,} & \textbf{F G} \\
\hspace*{2em}\textit{W nieskończoności tajemniczej} & \textbf{F G} \\
\hspace*{2em}\textit{Muszą się spotkać- na to liczę,} & \textbf{F G} \\
\hspace*{2em}\textit{Nawet równoległe.} & \textbf{C} \\
& \\
\textbf{F G C F G}  & \\
& \\
Więc może wcale nie jest głupie, & \textbf{C F G} \\
Nocą pod twoim krążyć domem, & \textbf{C F G} \\
Samego siebie oszukując, & \textbf{C F G} \\
Samemu sobie wchodząc w drogę... & \textbf{C F G} \\
Więc może wcale nie jest głupie, & \textbf{C F G} \\
Liczyć na jeszcze jedną szansę, & \textbf{C F G} \\
Bez żadnej jasnej perspektywy, & \textbf{C F G} \\
Poza przestrzenią, poza czasem... & \textbf{C F G} \\
& \\
\hspace*{2em}\textit{Jedno wiem, jedno jest pewne...}  & \\
& \\
\textbf{F G C F G}  & \\
& \\
Więc jeśli jeszcze to możliwe & \textbf{C F G} \\
Naucz mnie milczeć, naucz śpiewać, & \textbf{C F G} \\
Naucz mnie składać obietnice, & \textbf{C F G} \\
Naucz mnie już się więcej nie bać... & \textbf{C F G} \\
Więc jeśli było to możliwe, & \textbf{C F G} \\
A ja wiedziałem, ja wiedziałem, & \textbf{C F G} \\
Co ci odpowiem gdy zapytasz, & \textbf{C F G} \\
Czemu po prostu nie zostałem... & \textbf{C F G} \\
& \\
\hspace*{2em}\textit{Jedno wiem, jedno jest pewne...}  & \\
\end{longtable}
\clearpage

% --- Źródło: Róża_i_bez.tex ---
\section{Róża i bez}
\begin{longtable}{ll}
To nic, że długi jest marsz, & \textbf{C F G7 C} \\
Słońce osuszy twarz, & \textbf{e F G7} \\
Idziesz i liczysz naboje - ostatnie trzy, & \textbf{C F f C a} \\
I nie chybisz już - to wiesz. & \textbf{d7 G7 C} \\
& \\
Róża czerwono, biało kwitnie bez, & \textbf{C F C} \\
Nikt z nas nie pęka, chociaż krucho jest, & \textbf{C F C} \\
Wzgórza przejdziemy, wodą popijmy, & \textbf{a e} \\
Kuchnie polowe diabli wiedzą gdzie. & \textbf{a D7 G7} \\
Kto by się martwił, że na drodze & \textbf{C F C} \\
Kurz i śnieg i deszcz - to znamy już. & \textbf{F C d6} \\
Wzgórza przejdziemy, wodą popijemy, & \textbf{a e} \\
Woda po walce ma jak wino smak. & \textbf{a D7 G7} \\
Róża czerwono, biało kwitnie bez, & \textbf{C F C} \\
Dojdziesz bracie, choć krucho jest. & \textbf{E7 a F C} \\
& \\
Stary karabin, twój brat,  & \\
Jeszcze zadziwi świat,  & \\
Będą znów piękne dziewczyny za wojskiem szły,  & \\
A że w oczy deszcz to nic.  & \\
& \\
Róża czerwono, biało kwitnie bez,  & \\
Nikt z nas nie pęka, chociaż krucho jest,  & \\
Wzgórza przejdziemy, wodą popijemy,  & \\
Kuchnie polowe diabli wiedzą gdzie.  & \\
Kto by się martwił, że na drodze  & \\
Kurz i śnieg i deszcz - to znamy już.  & \\
Wzgórza przejdziemy, wodą popijemy,  & \\
Woda po walce ma jak wino smak.  & \\
Róża czerwono, biało kwitnie bez,  & \\
Dojdziesz bracie, choć krucho jest.  & \\
& \\
Róża czerwono, biało kwitnie bez,  & \\
Choć było krucho, teraz dobrze jest,  & \\
Wzgórza przeszliśmy, cało wróciliśmy,  & \\
Kuchnie polowe odnalazły się.  & \\
Jeszcze na twarzach mamy z drogi kurz,  & \\
Lecz dziś ten marsz za nami już.  & \\
Wzgórza przeszliśmy, cało wróciliśmy,  & \\
Czoło otrzemy, oczyścimy broń.  & \\
Róża czerwono, biało kwitnie bez,  & \\
Oto bracie wędrówki kres.  & \\
\end{longtable}
\clearpage

% --- Źródło: Scenariusz_dla_moich_sąsiadów.tex ---
\section{Scenariusz dla moich sąsiadów}
\vspace{-\baselineskip}
\textit{Myslovitz}\\
\begin{longtable}{ll}
Kiedy powrócisz już ja będę czekał & \textbf{A cis G F E} \\
Ulicą pójdę wzdłuż kupię gazetę & \textbf{A cis G F E} \\
Zabiorę z sobą psa usiądę na ławce & \textbf{A cis G F E} \\
Skończę scenariusz by gotowy był & \textbf{A cis G F E} \\
& \\
\hspace*{2em}\textit{Wieczorem wieczorem przed mym domem} & \textbf{C e} \\
\hspace*{2em}\textit{Wystawię ekran i wyświetlę film} & \textbf{F} \\
\hspace*{2em}\textit{Coś o mnie i o tobie} & \textbf{C e} \\
\hspace*{2em}\textit{Będę leczył chore sąsiadów sny} & \textbf{F C e} \\
& \\
Z nieba przyleciał mój wielki przyjaciel & \textbf{A cis G F E} \\
Kiedy lądował ja jadłem kanapkę & \textbf{A cis G F E} \\
Wyśnił że chyba jest chorym człowiekiem & \textbf{A cis G F E} \\
Usiądź wygodnie i nie martw się bo & \textbf{A cis G F E} \\
& \\
\hspace*{2em}\textit{Wieczorem wieczorem przed mym domem} & \textbf{A cis G F E} \\
\hspace*{2em}\textit{Wystawię ekran i wyświetlę film} & \textbf{A cis G F E} \\
\hspace*{2em}\textit{Coś o mnie i o tobie} & \textbf{A cis G F E} \\
\hspace*{2em}\textit{Będę leczył chore sąsiadów sny} & \textbf{A cis G F E} \\
& \\
\hspace*{2em}\textit{Wieczorem przed mym domem} & \textbf{A cis G F E} \\
\hspace*{2em}\textit{Wystawię ekran i wyświetlę film} & \textbf{A cis G F E} \\
\hspace*{2em}\textit{Coś o mnie i o tobie} & \textbf{A cis G F E} \\
\hspace*{2em}\textit{Będę leczył chore sąsiadów sny || x2} & \textbf{A cis G F E} \\
\end{longtable}
\clearpage

% --- Źródło: Sen_Katarzyny_II.tex ---
\section{Sen Katarzyny II}
\vspace{-\baselineskip}
\textit{Jacek Kaczmarski}\\
\begin{longtable}{ll}
Na smyczy trzymam filozofów Europy & \textbf{G D G} \\
Podparłam armią marmurowe Piotra stropy & \textbf{G D e} \\
Mam psy, sokoły, konie, kocham łów szalenie & \textbf{C D e} \\
A wokół same zające i jelenie & \textbf{C D G} \\
Pałace stawiam głowy ścinam & \textbf{Fis h} \\
Kiedy mi przyjdzie na to chęć & \textbf{Fis G D} \\
Mam biografów, portrecistów & \textbf{C D e} \\
I jeszcze jedno pragnę mieć... & \textbf{C D G} \\
& \\
\hspace*{2em}\textit{Stój Katarzyno! koronę carów} & \textbf{e a e a} \\
\hspace*{2em}\textit{Sen taki jak ten może ci z głowy zdjąć} & \textbf{e a C D G} \\
& \\
Kobietą jestem ponad miarę swoich czasów &  \\
Nie bawią mnie umizgi bladych lowelasów  & \\
Ich miękkich palców dotyk budzi obrzydzenie  & \\
Już wolę łowić zające i jelenie  & \\
Ze wstydu potem ten i ów  & \\
Rzekł o mnie: niewyżyta Niemra  & \\
I pod batogiem nago biegł  & \\
Po śniegu dookoła Kremla  & \\
& \\
\hspace*{2em}\textit{Stój Katarzyno! koronę carów}  & \\
\hspace*{2em}\textit{Sen taki jak ten może ci z głowy zdjąć}  & \\
& \\
Kochanka trzeba mi takiego jak imperium  & \\
Co by mnie brał tak, jak ja daję: całą pełnią  & \\
Co by i władcy i poddańca był wcieleniem  & \\
By mi zastąpił zające i jelenie  & \\
Co by rozumiał tak jak ja  & \\
Ten głupi dwór rozdanych ról  & \\
I pośród pochylonych głów  & \\
Dawał mi rozkosz albo ból  & \\
& \\
\hspace*{2em}\textit{Stój Katarzyno! koronę carów}  & \\
\hspace*{2em}\textit{Sen taki jak ten może ci z głowy zdjąć}  & \\
\hspace*{2em}\textit{Gdyby się kiedyś kochanek taki znalazł...}  & \\
\hspace*{2em}\textit{Wiem, sama wiem! Kazałabym go ściąć!}  & \\
\end{longtable}
\clearpage

% --- Źródło: Sen_o_Warszawie.tex ---
\section{Sen o Warszawie}
\vspace{-\baselineskip}
\textit{Czesław Niemen}\\
\begin{longtable}{ll}
Mam tak samo jak ty & \textbf{e D e D} \\
Miasto moje, a w nim & \textbf{e D G} \\
Najpiękniejszy mój świat & \textbf{G C D} \\
Najpiękniejsze dni & \textbf{G C D} \\
Zostawiłem tam kolorowe sny & \textbf{D e D e D} \\
& \\
Kiedyś zatrzymam czas  & \\
I na skrzydłach jak ptak  & \\
Będę leciał co sił  & \\
Tam, gdzie moje sny  & \\
I warszawski dzień  & \\
Kolorowe dni  & \\
& \\
Gdybyś ujrzeć chciał nadwiślański świt & \textbf{G} \\
Już dziś wyruszaj ze mną tam & \textbf{a e} \\
Zobaczysz jak przywita pięknie nas & \textbf{C B7 e G} \\
Warszawski dzień & \textbf{C G} \\
& \\
Mam tak samo jak ty  & \\
Miasto moje a w nim  & \\
Najpiękniejszy mój świat  & \\
Najpiękniejsze dni  & \\
Zostawiłem tam kolorowe sny  & \\
& \\
Gdybyś ujrzeć chciał nadwiślański świt  & \\
Już dziś wyruszaj ze mną tam  & \\
Zobaczysz jak przywita pięknie nas  & \\
Warszawski dzień  & \\
\end{longtable}
\clearpage

% --- Źródło: Siedem_grzechów_głównych.tex ---
\section{Siedem grzechów głównych}
\vspace{-\baselineskip}
\textit{Jacek Kaczmarski}\\
\begin{longtable}{ll}
Wielkich upadków więcej widzieliśmy niż wzlotów, & \textbf{fis G A G fis} \\
Byliśmy oczywiście na uczcie Baltazara, & \textbf{fis G A G fis} \\
Uczyliśmy się mowy zwycięskich Wizygotów & \textbf{e fis G fis e} \\
Na służbie ostatniego przepiwszy żołd Cezara. & \textbf{fis G A G fis} \\
& \\
Przeżyliśmy Rolanda, by świadczyć śmierć Karola, &  \\
Pozostałościom mocarstw nie oczekiwać łaski.  & \\
Razem z Ludwikiem Świętym widząc się w aureolach  & \\
Wyrzygiwaliśmy krew w jerozolimskie piaski.  & \\
& \\
Co było wszechpotężne – zdaje się niedorzeczne.  & \\
Gdzie słodka woń Arabii? Gdzie tajemniczy Syjam?  & \\
Religie tysiącletnie też nie są dla nas wieczne  & \\
I demokracja kwitnie, dojrzewa i przemija…  & \\
& \\
\hspace*{2em}\textit{A nas wiedzie siedem demonów, co nami się karmią:} & \textbf{G D e} \\
\hspace*{2em}\textit{Na przedzie pycha podąża z tańczącą latarnią,} & \textbf{G D a C e} \\
\hspace*{2em}\textit{Chciwość wczepiła się w siodło i grzebie po sakwach,} & \textbf{G D a C e} \\
\hspace*{2em}\textit{Broi pod zbroją lubieżność, pokusa niełatwa.} & \textbf{C G a} \\
\hspace*{2em}\textit{Bandzioch domaga się płynów i straw do przesytu,} & \textbf{G D a C e} \\
\hspace*{2em}\textit{Wabi rozkoszne lenistwo do łóż z aksamitu,} & \textbf{G D a C e} \\
\hspace*{2em}\textit{Gniew zrywa ze snu i groźbą na oślep wywija,} & \textbf{C G a} \\
\hspace*{2em}\textit{A zazdrość nie wie, co sen i po cichu zabija.} & \textbf{G D a C e} \\
& \\
Tak zbrojni w moce, na które nie ma lekarstwa & \textbf{G D e} \\
Stawiamy nadal i obalamy mocarstwa. & \textbf{G D e} \\
& \\
Każdemu więc z imperiów bezsprzecznie zasłużeni & \textbf{e fis G A e} \\
My – ludzie pióra, miecza lub zajęczego lęku & \textbf{G A h A h} \\
Jesteśmy grabarzami swych własnych dzieł stworzenia, & \textbf{D C h C a} \\
Zajęczym lękiem niszcząc je, mieczem lub piosenką. & \textbf{C h a h G} \\
& \\
Samotnie wędrujemy po dawnych bitew szlakach, & \textbf{h A G A fis} \\
Którymi dzisiaj rządzi chwast, kamień lub jaszczurka. & \textbf{A G fis G e} \\
Wierzchowiec się potyka, bo ciąży mu kulbaka & \textbf{g A G A fis} \\
I jeździec w pełnej zbroi błądzący po pagórkach. & \textbf{e fis G A e} \\
& \\
Jesteśmy jak zwierzęta – z rytmami śmierci zżyte, & \textbf{D C h C a} \\
Choć człowiek w nas – do Nowej wciąż prze Jerozolimy; & \textbf{C h a h G} \\
Więc nastawiamy ucha na echa nowych bitew, & \textbf{h A G A fis} \\
Bo wiemy, że na pewno je w końcu usłyszymy… & \textbf{a G fis G e} \\
& \\
\hspace*{2em}\textit{A nas wiedzie siedem demonów, co nami się karmią:}  & \\
& \\
Tak zbrojni w moce, na które nie ma lekarstwa & \textbf{C G a} \\
Będziemy nadal stawiać i zwalać mocarstwa. & \textbf{G D e} \\
\end{longtable}
\clearpage

% --- Źródło: Sosenka.tex ---
\section{\textbf{Sosenka}}
\begin{longtable}{ll}
Jak to dobrze być harcerzem & \textbf{a d} \\
Na obozie spędzać czas & \textbf{E a} \\
Na północy pojezierze  & \\
Na południu szumi las  & \\
& \\
\hspace*{2em}\textit{Hej las, mówię wam} & \textbf{d} \\
\hspace*{2em}\textit{szumi las, mówię wam} & \textbf{a} \\
\hspace*{2em}\textit{A w lesie, mówię wam, sosenka} & \textbf{E a} \\
\hspace*{2em}\textit{Spodobała mi się jeden raz} & \textbf{d a} \\
\hspace*{2em}\textit{Harcerka Marysieńka} & \textbf{E a} \\
& \\
Woda sama łódkę niosła  & \\
Łódkę niosła w siną dal  & \\
A on zamiast trzymać wiosła  & \\
Objął ją i śpiewał tak  & \\
& \\
\hspace*{2em}\textit{Hej las, mówię wam...}  & \\
& \\
Całuj, całuj druhu miły  & \\
Całuj, całuj póki czas  & \\
Bo gdy obóz nasz się skończy  & \\
To już nas nie będzie tam  & \\
& \\
\hspace*{2em}\textit{Hej las, mówię wam...}  & \\
\end{longtable}
\clearpage

% --- Źródło: Stacja_Warszawa.tex ---
\section{Stacja Warszawa}
\vspace{-\baselineskip}
\textit{Lady Pank}\\
\begin{longtable}{ll}
W moich snach wciąż Warszawa, pełna ulic, placów, drzew & \textbf{a F7+ G C} \\
Rzadko słyszysz tu brawa, częściej to drwiący śmiech & \textbf{a F7+ G C} \\
Twarze w metrze są obce, bo i po co się znać? & \textbf{a F7+ G C} \\
To kosztuje zbyt drogo, lepiej jechać i spać & \textbf{a F7+ G C} \\
& \\
Wszystko byłoby inne, gdybyś tu była, ja wiem & \textbf{G a C G F} \\
Nie tak trudne i dziwne, gdybyś tu była, ja wiem & \textbf{G a C G F} \\
& \\
Noce są zawsze długie, A za dnia - ciągły szum & \textbf{a F7+ G C} \\
Mało kto to zrozumie, dokąd gna zdyszany tłum & \textbf{a F7+ G C} \\
& \\
Wszystko byłoby inne, gdybyś tu była, ja wiem & \textbf{G a C G F} \\
Nie tak trudne i dziwne, gdybyś tu była, ja wiem & \textbf{G a C G F} \\
& \\
\hspace*{2em}\textit{Jeśli miłość coś znaczy to musi dać znak} & \textbf{a G} \\
\hspace*{2em}\textit{Kiedyś też to zobaczysz, powiesz mi tak:} & \textbf{C F G} \\
\hspace*{2em}\textit{Zniknie Warszawa tak jawa, jak sen} & \textbf{a G} \\
\hspace*{2em}\textit{Życie to nie zabawa - dobrze to wiem!} & \textbf{C F G} \\
& \\
W moich snach wciąż Warszawa, i do grosza wciąż grosz & \textbf{a F7+ G C} \\
Ktoś mi mówi: to sprawa, a ja chcę uciec stąd & \textbf{a F7+ G C} \\
& \\
Wszystko byłoby inne, gdybyś tu była, ja wiem & \textbf{G a C G F} \\
Nie tak trudne i dziwne, gdybyś tu była, ja wiem & \textbf{G a C G F} \\
& \\
\hspace*{2em}\textit{Jeśli miłość coś znaczy to musi dać znak… || x 2}  & \\
& \\
Wszystko byłoby inne, gdybyś tu była, ja wiem & \textbf{G a C G F} \\
Nie tak trudne i dziwne, gdybyś tu była, ja wiem & \textbf{G a C G F} \\
& \\
\hspace*{2em}\textit{Jeśli miłość coś znaczy to musi dać znak… || x 2}  & \\
\end{longtable}
\clearpage

% --- Źródło: Stokrotka.tex ---
\section{Stokrotka}
\begin{longtable}{ll}
Gdzie strumyk płynie z wolna, & \textbf{G G7 G} \\
Rozsiewa zioła maj, & \textbf{G7 G D7} \\
Stokrotka rosła polna, & \textbf{a D7 a} \\
A nad nią szumiał gaj, & \textbf{G C G} \\
Stokrotka rosła polna, & \textbf{C D G e} \\
A nad nią szumiał gaj, & \textbf{a D G} \\
Zielony gaj. & \textbf{G} \\
& \\
W tym gaju tak ponuro,  & \\
Że aż przeraża mnie,  & \\
Ptaszęta za wysoko,  & \\
A mnie samotnej źle,  & \\
Ptaszęta za wysoko,  & \\
A mnie samotnej źle,  & \\
samotnej źle.  & \\
& \\
Wtem harcerz idzie z wolna.  & \\
„Stokrotko, witam cię,  & \\
Twój urok mnie zachwyca,  & \\
Czy chcesz być mą, czy nie?”  & \\
„Twój urok mnie zachwyca,  & \\
Czy chcesz być mą, czy nie?  & \\
Czy nie, czy nie?”  & \\
& \\
Stokrotka się zgodziła  & \\
I poszli w ciemny las,  & \\
A harcerz taki gapa  & \\
Że aż w pokrzywy wlazł,  & \\
A harcerz taki gapa,  & \\
Że aż w pokrzywy wlazł,  & \\
Po pas, po pas.  & \\
& \\
A ona, ona, ona,  & \\
Cóż biedna robić ma,  & \\
Nad gapą pochylona  & \\
I śmieje się: ha, ha,  & \\
Nad gapą pochylona  & \\
I śmieje: się ha, ha,  & \\
ha, ha, ha, ha  & \\
\end{longtable}
\clearpage

% --- Źródło: Szanta_narciarska.tex ---
\section{\textbf{Szanta narciarska}}
\vspace{-\baselineskip}
\textit{Artur Andrus}\\
\begin{longtable}{ll}
Nazywają go marynarz & \textbf{d C d} \\
Bo opaskę ma na oku & \textbf{F G A} \\
Na każdym stoku dziewczyna & \textbf{B F} \\
Dziewczyna na każdym stoku & \textbf{F E d} \\
Pochodzi spod Poznania & \textbf{d C d} \\
Podobno umie wróżyć z kart & \textbf{F G A} \\
Panny rwie na wiązania & \textbf{B F} \\
Mężatki - na długość nart & \textbf{F E d} \\
& \\
\hspace*{2em}\textit{Caryco mokrego śniegu} & \textbf{A d} \\
\hspace*{2em}\textit{Ratrakiem płynę do Ciebie pod prąd (hej!)} & \textbf{A B} \\
\hspace*{2em}\textit{Dobrze, że stoisz na brzegu} & \textbf{B F} \\
\hspace*{2em}\textit{Bo ja właśnie schodzę na ląd} & \textbf{F E d} \\
& \\
Nigdy się nie lękał biedy  & \\
I się nie przejmował jutrem  & \\
A jego ratrak był kiedyś  & \\
Zwyczajnym rybackim kutrem  & \\
I woził dorsze i śledzie  & \\
Zimą i latem, okrągły rok  & \\
Teraz jak nieraz przejedzie  & \\
Rybami czuć cały stok  & \\
& \\
\hspace*{2em}\textit{Caryco mokrego śniegu...}  & \\
& \\
Wszyscy w porcie odetchnęli  & \\
Zwiał, nim się zakończył sezon  & \\
Jeszcze nam się jak żagiel bieli  & \\
Jego czarny kombinezon  & \\
Odpłynął gdzieś pod Ustrzyki  & \\
Przez baby straszne miał kłopoty  & \\
Forsę z polowań na orczyki  & \\
Przehulał na antybiotyk  & \\
& \\
\hspace*{2em}\textit{Caryco mokrego śniegu...}  & \\
& \\
Jeśli kiedyś go zobaczysz  & \\
Na ratraku w podłym świecie  & \\
To powiedz mu, że w Karpaczu  & \\
Czekają na niego dzieci  & \\
I kiedy opuszcza statek  & \\
Żeby się znowu oddać złu  & \\
Każda z dwudziestu siedmiu matek  & \\
Dzieciątku śpiewa do snu  & \\
& \\
\end{longtable}
\clearpage

% --- Źródło: Szara_lilijka.tex ---
\section{Szara lilijka}
\begin{longtable}{ll}
Gdy zakochasz się w szarej lilijce & \textbf{a d} \\
I w świetlanym harcerskim krzyżu & \textbf{E a} \\
Kiedy olśni cię blask ogniska & \textbf{a d} \\
Radę jedną ci dam & \textbf{E a} \\
& \\
\hspace*{2em}\textit{Załóż mundur i przypnij lilijkę} & \textbf{a d} \\
\hspace*{2em}\textit{Czapkę na bakier włóż} & \textbf{G C E} \\
\hspace*{2em}\textit{W szeregu stań wśród harcerzy} & \textbf{a d} \\
\hspace*{2em}\textit{I razem z nimi w świat rusz} & \textbf{E a} \\
& \\
Razem z nimi będziesz wędrował  & \\
Po Łysicy i Świętym Krzyżu  & \\
Poznasz uroki Gór Świętokrzyskich  & \\
Które powiedzą ci tak  & \\
& \\
\hspace*{2em}\textit{Załóż mundur i przypnij lilijkę...}  & \\
& \\
Gdy po latach będziesz wspominał  & \\
Stare dzieje w harcerskiej drużynie  & \\
Swemu dziecku co dorastać zaczyna  & \\
Radę jedną dasz  & \\
& \\
\hspace*{2em}\textit{Załóż mundur i przypnij lilijkę...}  & \\
\end{longtable}
\clearpage

% --- Źródło: Szuner_Im_Alone.tex ---
\section{\textbf{Szuner I’m Alone}}
\begin{longtable}{ll}
Baksztagiem pruł nasz “I'm Alone”, hen, od Meksyku bram, & \textbf{e e G D} \\
A Jankes, w dziób kopany, po piętach deptał nam. & \textbf{a7 e} \\
Tysiące beczek rumu od lockerów aż po dno, & \textbf{e G D D} \\
I nawet kabla luzu, choćbyś robił nie wiem co. & \textbf{a e} \\
& \\
\hspace*{2em}\textit{Sam Neptun śpiewał szanty, po cichu sprzyjał nam,} & \textbf{C G} \\
\hspace*{2em}\textit{Więc bił rekordy "I'm Alone", choć groził wciąż Wuj Sam.} & \textbf{a H7} \\
\hspace*{2em}\textit{Na jedną kartę wszystko, jak struna każdy bras,} & \textbf{C G H7 e} \\
\hspace*{2em}\textit{“Niech diabli porwą Coast Guard” - tak mawiał każdy z nas.} & \textbf{a e} \\
& \\
A dawniej szkuner "I'm Alone", hen, po łowiskach gnał,  & \\
Lecz w końcu ryb zabrakło i głód w oczy zajrzał nam.  & \\
Za burtę poszły sieci, bo tak krzyczał kobiet tłum,  & \\
Jankesi mają ginu dość, postawmy więc na rum.  & \\
& \\
\hspace*{2em}\textit{Sam Neptun śpiewał szanty, po cichu sprzyjał nam…}  & \\
& \\
Gdy stawialiśmy żagle, to Coast Guard wpadał w trans,  & \\
Ta banda bubków w baliach nie miała żadnych szans.  & \\
Pułapkę zastawili gnoje, choć tak dobrze szło,  & \\
Posłali dzielny “I'm Alone” z ładunkiem aż na dno.  & \\
& \\
Niejeden w Nowej Szkocji szkuner taki spotkał los,  & \\
A wszystko przez cholerny głód i wiecznie pusty trzos.  & \\
Choć jeden z nich - nasz “I'm Alone” - swe miejsce w pieśni ma  & \\
I pewnie Neptun lubi go, i w kości na nim gra.  & \\
& \\
\hspace*{2em}\textit{I nawet śpiewa szanty, po cichu sprzyja nam,}  & \\
\hspace*{2em}\textit{Choć leży na dnie “I'm Alone” i śmieje się Wuj Sam.}  & \\
\hspace*{2em}\textit{Na jedną kartę wszystko, jak struna każdy bras,}  & \\
\hspace*{2em}\textit{“Niech diabli porwą Coast Guard” - tak mawiał każdy z nas.}  & \\
& \\
A ci co pokład "I'm Alone" kochali jak swój dom,  & \\
Nie dla nich blaski sławy i nie dla nich w niebie tron.  & \\
Niech mają choć ten cichy klang, ten jeden marny dzwon,  & \\
Niech każdy do nich woła: "Hej, smugglers z "I'm Alone"!"  & \\
& \\
\hspace*{2em}\textit{Niech Neptun śpiewa szanty, po cichu sprzyja nam,}  & \\
\hspace*{2em}\textit{Rekordy bije "I'm Alone" i zamknie się Wuj Sam.}  & \\
\hspace*{2em}\textit{Na jedną kartę wszystko, jak struna każdy bras,}  & \\
\hspace*{2em}\textit{„Niech Smuggler pije tylko rum!” - tak mawia każdy z nas.}  & \\
\end{longtable}
\clearpage

% --- Źródło: Szwejkowa_ballada.tex ---
\section{Szwejkowa ballada}
\vspace{-\baselineskip}
\textit{Marek Gajdziński (Szwejk)}\\
\begin{longtable}{ll}
Jak długo będziesz druhu z Szesnastką włóczył się, & \textbf{C a d G7} \\
przez pola i przez lasy, przez miasta i przez wsie, & \textbf{C a E A7} \\
tak długo tę piosenkę na ustach będziesz miał, & \textbf{d G7 e A7} \\
tak długo będziesz ten refren znał. & \textbf{D G7 C a d G7} \\
& \\
\hspace*{2em}\textit{Piosenka jest w wędrówce jak nieodłączny cień.} & \textbf{C a d G7} \\
\hspace*{2em}\textit{Piosenka to przyjaciel, co nie opuści Cię} & \textbf{C a E A7} \\
\hspace*{2em}\textit{i będzie razem z Tobą na jawie i we śnie,} & \textbf{d G7 e A7} \\
\hspace*{2em}\textit{bo ona zamieszkuje serce Twe.} & \textbf{D G7 C a d G7} \\
& \\
Kiedy z trudem kroczysz, ramiona plecak gnie, & \textbf{C a d G7} \\
gdy pot zalewa oczy, a w ustach ślina schnie, & \textbf{C a E A7} \\
zaśpiewasz w rytm Twych kroków i iść Ci będzie lżej, & \textbf{d G7 e A7} \\
zrozumiesz właśnie wtedy refren ten. & \textbf{D G7 C a d G7} \\
& \\
\hspace*{2em}\textit{Piosenka jest w wędrówce jak nieodłączny cień.} & \textbf{C a d G7} \\
\hspace*{2em}\textit{Piosenka to przyjaciel, co nie opuści Cię} & \textbf{C a E A7} \\
\hspace*{2em}\textit{i będzie razem z Tobą na jawie i we śnie,} & \textbf{d G7 e A7} \\
\hspace*{2em}\textit{bo ona zamieszkuje serce Twe.} & \textbf{D G7 C a d G7} \\
& \\
Głęboko Ci w pamięci zaryją słowa tej & \textbf{C a d G7} \\
piosenki co w wędrówce narodziła się, & \textbf{C a E A7} \\
piosenki, której słowa szumi las i gwiżdże wiatr, & \textbf{d G7 e A7} \\
piosenki, której sens zna każdy skaut. & \textbf{D G7 C a d G7} \\
& \\
\hspace*{2em}\textit{Piosenka jest w wędrówce jak nieodłączny cień.} & \textbf{C a d G7} \\
\hspace*{2em}\textit{Piosenka to przyjaciel, co nie opuści Cię} & \textbf{C a E A7} \\
\hspace*{2em}\textit{i będzie razem z Tobą na jawie i we śnie,} & \textbf{d G7 e A7} \\
\hspace*{2em}\textit{bo ona zamieszkuje serce Twe.} & \textbf{D G7 C a d G7} \\
& \\
\end{longtable}
\clearpage

% --- Źródło: Tak_jak_ptaki.tex ---
\section{Tak jak ptaki}
\begin{longtable}{ll}
Straszny ból głośny krzyk & \textbf{d F} \\
Znowu życie utracono & \textbf{a G} \\
Zginął tak jak wielu z nich  & \\
Za swą wolność upragnioną  & \\
Miał na piersi szary krzyż  & \\
I zaledwie 10 lat  & \\
Był harcerzem tak jak ty  & \\
I tak samo kochał świat  & \\
& \\
\hspace*{2em}\textit{Dziś szybuje pośród chmur} & \textbf{d F} \\
\hspace*{2em}\textit{Tak jak ptaki na wolności} & \textbf{a G} \\
\hspace*{2em}\textit{Poszukuje w świecie tym}  & \\
\hspace*{2em}\textit{Ciepła, dobra i miłości}  & \\
& \\
Tam na wzgórzu leży on  & \\
Nad nim krzyż brzozowy stoi  & \\
Jego dusza uleciała  & \\
Ziemia ciału rany goi  & \\
Bo poświęcił młode życie  & \\
Walcząc w szarych szeregach  & \\
Wielu takich jest harcerzy  & \\
Szybujących u wrót nieba  & \\
& \\
\hspace*{2em}\textit{Dziś szybuje pośród chmur...}  & \\
& \\
Nie minęło parę lat  & \\
Nad grobami matki płaczą  & \\
Po policzkach łzy im płyną  & \\
Już ich więcej nie zobaczą  & \\
Nie zobaczą swoich dzieci  & \\
Lecz pamiętać o nich będą  & \\
O swych młodych bohaterach  & \\
Owianych smutna legendą  & \\
& \\
\hspace*{2em}\textit{Dziś szybuje pośród chmur...}  & \\
\end{longtable}
\clearpage

% --- Źródło: Testament.tex ---
\section{\textbf{Testament}}
\begin{longtable}{ll}
Idę samotny wśród gór zagubiony w świecie & \textbf{e} \\
wiatr tylko ścieżki me zna & \textbf{G} \\
Słońce niebawem za horyzont się wyniesie & \textbf{D} \\
Światem zawładnie mgła & \textbf{e} \\
Gdy wśród ciemności ognisk jasność ujrzę & \textbf{e} \\
Będzie to dobry znak & \textbf{G} \\
Nowych ludzi poznam, od nich się nauczę & \textbf{D} \\
Gawęd tak starych jak świat & \textbf{e} \\
& \\
\hspace*{2em}\textit{Niech każdy z Was się dowie} & \textbf{G D} \\
\hspace*{2em}\textit{Co traci nie będąc tu} & \textbf{e} \\
\hspace*{2em}\textit{Wiatr szumem Ci opowie} & \textbf{G D} \\
\hspace*{2em}\textit{Co zaszło przed lat stu} & \textbf{e} \\
\hspace*{2em}\textit{Strumień orzeźwi stopy} & \textbf{G D} \\
\hspace*{2em}\textit{To on kiedyś wody dał} & \textbf{e} \\
\hspace*{2em}\textit{Spragnionym w górach wędrowcom} & \textbf{G D} \\
\hspace*{2em}\textit{Co szli tędy tak jak ja} & \textbf{e} \\
& \\
Góra za górą chowa swe oblicze  & \\
Lecz te dokładnie już znam  & \\
Żadna mi nie ujdzie, wszystkie je zaliczę  & \\
Chociaż to drogi szmat   & \\
Lat wciąż przybywa i świat się starzeje  & \\
Cóż, naturalna to rzecz  & \\
 Ktoś, kto przeglądać będzie stare szpargały  & \\
Może przeczyta ten tekst  & \\
& \\
\hspace*{2em}\textit{Niech każdy z Was się dowie}  & \\
\hspace*{2em}\textit{Co traci nie będąc tu}  & \\
\hspace*{2em}\textit{Świerk szumem Ci opowie}  & \\
\hspace*{2em}\textit{Co zaszło tu przed lat stu}  & \\
\hspace*{2em}\textit{Strumień orzeźwi stopy}  & \\
\hspace*{2em}\textit{To on kiedyś wody dał}  & \\
\hspace*{2em}\textit{Spragnionym w górach wędrowcom}  & \\
\hspace*{2em}\textit{Co szli tędy tak jak ja}  & \\
\end{longtable}
\clearpage

% --- Źródło: Tyle_słońca_w_całym_mieście.tex ---
\section{Tyle słońca w całym mieście}
\vspace{-\baselineskip}
\textit{Anna Jantar}\\
\begin{longtable}{ll}
Dzień – wspomnienie lata & \textbf{a E} \\
Dzień – słoneczne ćmy (a-a) & \textbf{E7 a A7} \\
Nagle w tłumie w samym środku miasta & \textbf{d a} \\
Ty, po prostu Ty & \textbf{H7 E7} \\
& \\
Dzień (dzień) – godzina zwierzeń (zwierzeń) & \textbf{a E} \\
Dzień (dzień) – przy twarzy twarz (u-u) & \textbf{E7 a A7} \\
Szuka pamięć poplątanych ścieżek & \textbf{d a} \\
Lecz (lecz), czy znajdzie nas? & \textbf{H7 E7} \\
& \\
\hspace*{2em}\textit{Tyle słońca w całym mieście} & \textbf{a} \\
\hspace*{2em}\textit{Nie widziałeś tego jeszcze} & \textbf{a} \\
\hspace*{2em}\textit{Popatrz, o popatrz!} & \textbf{d} \\
\hspace*{2em}\textit{Szerokimi ulicami} & \textbf{E7} \\
\hspace*{2em}\textit{Niosą szczęście zakochani} & \textbf{E7} \\
\hspace*{2em}\textit{Popatrz, o popatrz!} & \textbf{a} \\
& \\
\hspace*{2em}\textit{Wiatr porywa ich spojrzenia} & \textbf{A7} \\
\hspace*{2em}\textit{Biegnie światłem w smugę cienia} & \textbf{A7} \\
\hspace*{2em}\textit{Popatrz, o popatrz!} & \textbf{d} \\
\hspace*{2em}\textit{Łączy serca, wiąże dłonie} & \textbf{E7} \\
\hspace*{2em}\textit{Może nam zawróci w głowie też!} & \textbf{E7 a} \\
& \\
La-la-la-la-la-la-la… || x 2 & \textbf{a d E7 a} \\
& \\
Dzień (dzień) – powrotna podróż (podróż) & \textbf{a E} \\
Dzień – podanie rąk (u-u) & \textbf{E7 a A7} \\
Ale niebo całe jeszcze w ogniu & \textbf{d a} \\
Chcę, zatrzymać wzrok & \textbf{H7 E7} \\
& \\
\hspace*{2em}\textit{Tyle słońca w całym mieście...}  & \\
\end{longtable}
\clearpage

% --- Źródło: Tysiąc_siedemset_osiemdziesiąt_osiem.tex ---
\section{1788}
\vspace{-\baselineskip}
\textit{Jacek Kaczmarski}\\
\begin{longtable}{ll}
Ta pierwsza morska podróż do Australii! & \textbf{F B} \\
Łotry przy burtach, prostytutki w kojach - & \textbf{F C} \\
Wszyscy się bali, łkali i rzygali & \textbf{F B} \\
W drodze do raju. Przewrotności Twoja & \textbf{F C} \\
Panie, coś w jeszcze nam nieznanych planach & \textbf{d g} \\
Miał czarne diabły strzegące wybrzeży & \textbf{d a} \\
Edenu, który przeznaczyłeś dla nas, & \textbf{B C F} \\
A w który nikt, prawdę mówiąc, nie wierzył! & \textbf{B C d} \\
& \\
Czym żeśmy, marni, zasłużyli na to?  & \\
Ten, co zawisnąć miał za kradzież płaszcza -  & \\
Płakał nad swoją niechybną zatratą;  & \\
Nie widział Ciebie w robaczywych masztach  & \\
Statku, co tylko był więzieniem nowym;  & \\
Tej co kupczyła ciałami swych dziatek -  & \\
Ani przez mgnienie nie przyszło do głowy,  & \\
Że to nadziei - nie rozpaczy statek.  & \\
& \\
Niejeden żołnierz z ponurej eskorty  & \\
(Bo czym się ich los od naszego różnił?)  & \\
Wiedział, że nigdy już nie ujrzy portu,  & \\
Gdzie go podejmą karczmarze usłużni  & \\
I płatne dziewki; że zabraknie rumu  & \\
Zanim do celu przygnasz okręt szparki.  & \\
Z marynarzami pili więc na umór  & \\
I - wbrew zakazom - grali o więźniarki.  & \\
& \\
Prawda, nie wszyscy próby Twe przetrwali,  & \\
Ale też ciężkoś nas doświadczał, Panie:  & \\
Nie oszczędzałeś nam wysokiej fali,  & \\
Za którą mnogim przyszło w oceanie  & \\
Zakończyć żywot; innym dziąsła zgniły,  & \\
Wypadły zęby, rozgorzały wrzody...  & \\
Więc znaczą nasz zielony szlak mogiły  & \\
Szkorbutu, szału, francuskiej choroby.  & \\
& \\
& \\
\end{longtable}
\newpage
\begin{longtable}{ll}
Nikt nie odnajdzie w ruchomych otchłaniach  & \\
Ciał nieszczęśników - oprócz Ciebie, Boże.  & \\
Ich żywot grzeszny epitafiów wzbrania,  & \\
Lecz - ukarani. Więc wystarczy może,  & \\
Żeś się posłużył straszliwym przykładem:  & \\
Oni naprawdę dotarli do piekieł,  & \\
A umierając nie wierzył z nich żaden,  & \\
Że w swym cierpieniu umiera – człowiekiem  & \\
& \\
Ląd nam się wydał niegościnny, dziki;  & \\
Łotr bez honoru, kobieta sprzedajna  & \\
Z dnia na dzień - jak się ma stać osadnikiem  & \\
Nieznanych światów? Bo rozpoznać Raj nam  & \\
Nie było łatwo; znaleźć w sobie siłę,  & \\
Wbrew przeciwnościom, bez słowa zachęty  & \\
By mimo wszystko żyć - nim nam odkryłeś  & \\
Kraj szczodry w zboże, złoto i diamenty.  & \\
& \\
Łajdacki pomiot, łotrowskie nasienie  & \\
Czerpiąc ze spichrza Twoich dóbr wszelakich -  & \\
Choć tyle wiemy własnym doświadczeniem:  & \\
W nas jest Raj, Piekło -  & \\
I do obu - szlaki.  & \\
\end{longtable}
\clearpage

% --- Źródło: Ułan.tex ---
\section{\textbf{Ułan}}
\begin{longtable}{ll}
Siedzi ułan na widecie, & \textbf{C} \\
a siodło go w tyłek gniecie, & \textbf{F} \\
a szkapina poczciwina & \textbf{C} \\
nie chce dalej iść || x2 & \textbf{G C \quad F C G C} \\
& \\
Siedzi ułan na okopie, & \textbf{C} \\
a śmierć pod nim dołki kopie, & \textbf{F} \\
ale ułan jak to ułan, & \textbf{C} \\
nie boi się nic. || x2 & \textbf{G C \quad  F C G C} \\
& \\
Siedzi ułan i flirtuje, & \textbf{C} \\
a śmierć nad nim przelatuje, & \textbf{F} \\
granat trzasnął, ułan wrzasnął, & \textbf{C} \\
nie ma głowy już. || x2 & \textbf{G C \quad F C G C} \\
& \\
Wiozą trumnę przez Dąbrowę, & \textbf{C} \\
w jednej ułan, w drugiej głowę, & \textbf{F} \\
pochowali, pogrzebali, & \textbf{C} \\
dobrze jemu jest || x2 & \textbf{G C \quad F C G C} \\
& \\
Rano, gdy pobudkę grali, & \textbf{C} \\
to ułana odkopali, & \textbf{F} \\
do tułowia łeb przyszyli & \textbf{C} \\
i już ułan jest. || x2 & \textbf{G C \quad F C G C} \\
& \\
Teraz ułan znowu hula, & \textbf{C} \\
niestraszna mu żadna kula, & \textbf{F} \\
bo któż może, o mój Boże & \textbf{C} \\
umrzeć drugi raz. || x2 & \textbf{G C \quad F C G C} \\
\end{longtable}
\clearpage

% --- Źródło: W_taką_ciszę.tex ---
\section{\textbf{W taką ciszę}}
\begin{longtable}{ll}
Nie o uśmiech mi chodzi, bo się śmiałaś nie raz, & \textbf{C e} \\
Ale o to co kiedyś otworzyło się w nas & \textbf{F G} \\
Coś co przyszło tak szybko i przeszło jak wiatr, & \textbf{C e} \\
Czego właśnie najbardziej mi brak & \textbf{F G} \\
& \\
\hspace*{2em}\textit{W taką, w taką ciszę} &  \\
\hspace*{2em}\textit{Wszystkie gwiazdy na niebie ja wyliczę dla ciebie,} & \textbf{C e F G} \\
\hspace*{2em}\textit{Ciebie, ciebie, ciebie wołam, ale cisza i pustka dookoła} & \textbf{C e F G} \\
Przychodziłem co wieczór posłuchać Twych płyt,  & \\
O miłości w ogóle nie mówiliśmy nic  & \\
Wyjechałaś tak nagle, cichutko jak mysz,  & \\
Zostawiłaś swój adres i list  & \\
& \\
\hspace*{2em}\textit{W taką, w taką ciszę...}  & \\
& \\
Jesteś moim aniołem, miłością bez dna,  & \\
Jesteś moją boginią, którą widzę co dnia  & \\
Jakże długo mam czekać, jak prosić Cię mam,  & \\
Każesz trwać w niepewności, więc trwam  & \\
& \\
\hspace*{2em}\textit{W taką, w taką ciszę...}  & \\
& \\
Choć dostaję Twe listy i zdjęć parę mam, & \textbf{D h G A} \\
Żyję jak grzeszny anioł w tłumie ludzi, lecz sam  & \\
Jeszcze tli się nadzieja, że spotkamy się znów,  & \\
Do księżyca się śmiejąc przywołuję Cię - wróć!  & \\
& \\
\hspace*{2em}\textit{W taką, w taką ciszę...}  & \\
\end{longtable}
\clearpage

% --- Źródło: Warszawa.tex ---
\section{Warszawa}
\vspace{-\baselineskip}
\textit{T.Love}\\
\begin{longtable}{ll}
Za oknem zimowo zaczyna się dzień & \textbf{G} \\
Zaczynam kolejny dzień życia & \textbf{a F} \\
Wyglądam przez okno, na oczach mam sen & \textbf{G} \\
A Grochów się budzi z przepicia & \textbf{a F} \\
Wypity alkohol uderza w tętnice & \textbf{G} \\
Autobus tapla się w śniegu & \textbf{a F} \\
Zza szyby oglądam betonu stolicę & \textbf{G} \\
Już jestem na drugim jej brzegu & \textbf{a F} \\
& \\
\hspace*{2em}\textit{Gdy patrzę w twe oczy, zmęczone jak moje} & \textbf{C G a F} \\
\hspace*{2em}\textit{To kocham to miasto, zmęczone jak ja} & \textbf{C G a F} \\
\hspace*{2em}\textit{Gdzie Hitler i Stalin zrobili, co swoje} & \textbf{C G a F} \\
\hspace*{2em}\textit{Gdzie wiosna spaliną oddycha} & \textbf{C G F} \\
& \\
Krakowskie Przedmieście zalane jest słońcem  & \\
Wirujesz jak obłok, wynurzasz się z bramy  & \\
A ja jestem głodny, tak bardzo głodny  & \\
Kochanie, nakarmisz mnie snami  & \\
Zielony Żoliborz, pieprzony Żoliborz  & \\
Rozkwita na drzewach, na krzewach  & \\
Ściekami z rzeki kompletnie pijany  & \\
Chcę krzyczeć, chcę ryczeć, chcę śpiewać  & \\
& \\
\hspace*{2em}\textit{Gdy patrzę w twe oczy, zmęczone jak moje… || x2}  & \\
& \\
Jesienią zawsze zaczyna się szkoła  & \\
A w knajpach zaczyna się picie  & \\
Jest tłoczno i duszno, olewa nas kelner  & \\
I tak skończymy o świcie  & \\
Jesienią zawsze myślę o latach  & \\
Tak starych, jak te kamienice  & \\
Jesienią o zmroku przechodzę z tobą  & \\
Przez pełne kasztanów ulice  & \\
& \\
\hspace*{2em}\textit{Gdy patrzę w twe oczy, zmęczone jak moje… || x2}  & \\
\end{longtable}
\clearpage

% --- Źródło: Warszawskie_Dzieci.tex ---
\section{Warszawskie Dzieci}
\begin{longtable}{ll}
Nie złamie wolnych żadna klęska, & \textbf{G} \\
Nie strwoży śmiałych żaden trud – & \textbf{E} \\
Pójdziemy razem do zwycięstwa, & \textbf{C} \\
Gdy ramię w ramię stanie lud & \textbf{A D D7} \\
& \\
\hspace*{2em}\textit{Warszawskie dzieci, pójdziemy w bój,} & \textbf{G} \\
\hspace*{2em}\textit{Za każdy kamień Twój,} & \textbf{G} \\
\hspace*{2em}\textit{Stolico, damy krew!} & \textbf{a} \\
\hspace*{2em}\textit{Warszawskie dzieci, pójdziemy w bój,} & \textbf{E a} \\
\hspace*{2em}\textit{Gdy padnie rozkaz Twój,} & \textbf{D} \\
\hspace*{2em}\textit{Poniesiem wrogom gniew!} & \textbf{D7 G} \\
& \\
Powiśle, Wola i Mokotów,  & \\
Ulica każda, każdy dom –  & \\
Gdy padnie pierwszy strzał, bądź gotów,  & \\
Jak w ręku Boga złoty grom  & \\
& \\
\hspace*{2em}\textit{Warszawskie dzieci, pójdziemy w bój...}  & \\
& \\
Od piły, dłuta, młota, kielni –  & \\
Stolico, synów swoich sław,  & \\
Że stoją wraz przy Tobie wierni,  & \\
Na straży Twych żelaznych praw  & \\
& \\
\hspace*{2em}\textit{Warszawskie dzieci, pójdziemy w bój...}  & \\
& \\
Poległym chwała, wolność żywym,  & \\
Niech płynie w niebo dumny śpiew,  & \\
Wierzymy, że nam Sprawiedliwy,  & \\
Odpłaci za przelaną krew  & \\
& \\
\hspace*{2em}\textit{Warszawskie dzieci, pójdziemy w bój...}  & \\
\end{longtable}
\clearpage

% --- Źródło: We_wtorek_w_schronisku.tex ---
\section{We wtorek w schronisku}
\vspace{-\baselineskip}
\textit{Wołosatki}\\
\begin{longtable}{ll}
Złotym kobiercem wymoszczone góry & \textbf{C F C} \\
Jesień w doliny przyszła dziś nad ranem & \textbf{C F C G} \\
Buki czerwienią zabarwiły chmury & \textbf{C F E7 a} \\
Z latem się złotym właśnie pożegnałem & \textbf{F G C G} \\
& \\
\hspace*{2em}\textit{We wtorek w schronisku po sezonie} & \textbf{C F G C} \\
\hspace*{2em}\textit{W doliny wczoraj zszedł ostatni gość} & \textbf{a D G G7} \\
\hspace*{2em}\textit{Za oknem plucha kubek parzy w dłonie} & \textbf{C F E7 a} \\
\hspace*{2em}\textit{I tej herbaty i tych gór mam dość} & \textbf{F G C} \\
& \\
Szaruga niebo powoli zasnuwa  & \\
Wiatr już gałęzie pootrząsał z liści  & \\
Pod wiatr pod górę znowu sam zasuwam  & \\
Może w schronisku spotkam kogoś z bliskich  & \\
& \\
\hspace*{2em}\textit{We wtorek w schronisku po sezonie}  & \\
\hspace*{2em}\textit{W doliny wczoraj zszedł ostatni gość}  & \\
\hspace*{2em}\textit{Za oknem plucha kubek parzy w dłonie}  & \\
\hspace*{2em}\textit{I tej herbaty i tych gór mam dość}  & \\
& \\
Ludzie tak wiele spraw muszą załatwić  & \\
A czas sobie płynie wolno panta rei  & \\
Do ciebie tylko już nie umiem trafić  & \\
Kochać to więcej siebie dać czy mniej  & \\
& \\
\hspace*{2em}\textit{We wtorek w schronisku po sezonie}  & \\
\hspace*{2em}\textit{W doliny wczoraj zszedł ostatni gość}  & \\
\hspace*{2em}\textit{Za oknem plucha kubek parzy w dłonie}  & \\
\hspace*{2em}\textit{I tej herbaty i tych gór mam dość}  & \\
\end{longtable}
\clearpage

% --- Źródło: Wehikuł_czasu.tex ---
\section{\textbf{Wehikuł czasu}}
\vspace{-\baselineskip}
\textit{Dżem}\\
\begin{longtable}{ll}
Pamiętam dobrze ideał swój & \textbf{A E fis D} \\
Marzeniami żyłem jak król & \textbf{A E D A} \\
Siódma rano to dla mnie noc & \textbf{A E fis D} \\
Pracować nie chciałem, włóczyłem się & \textbf{A E D A} \\
& \\
Za to do „puszki” zamykano mnie & \textbf{A E fis D} \\
Za to zwykle zamykano mnie & \textbf{A E D A} \\
Po knajpach grywałem za piwko i chleb & \textbf{A E fis D} \\
Na szyciu blues'a tak mijał mi dzień & \textbf{A E D A} \\
& \\
\hspace*{2em}\textit{Tylko nocą do klubu „Puls”} & \textbf{E fis D A} \\
\hspace*{2em}\textit{Jam Session do rana - tam królował blues} & \textbf{E fis D A} \\
\hspace*{2em}\textit{To już minęło, ten klimat, ten luz} & \textbf{E fis D A} \\
\hspace*{2em}\textit{Ci wspaniali ludzie nie powrócą} & \textbf{E fis D} \\
\hspace*{2em}\textit{Nie powrócą już} & \textbf{D} \\
& \\
Lecz we mnie zostało coś z tamtych lat & \textbf{A E fis D} \\
Mój mały intymny, muzyczny świat & \textbf{A E D A} \\
Gdy tak wspominam ten miniony czas & \textbf{A E fis D} \\
Jak dobrze że to nie poszło w las & \textbf{A E D A} \\
& \\
Dużo bym dałby przeżyć to znów & \textbf{A E fis D} \\
Wehikuł czasu - to byłby cud & \textbf{A E D A} \\
Mam jeszcze wiarę, odmieni się los & \textbf{A E fis D} \\
Znów kwiatek do lufy wetknie mi ktoś & \textbf{A E D A} \\
& \\
\hspace*{2em}\textit{Tylko nocą do klubu „Puls”...}  & \\
\end{longtable}
\clearpage

% --- Źródło: Wieża_spadochronowa.tex ---
\section{Wieża spadochronowa}
\begin{longtable}{ll}
W parku na ławce w zwyczajny dzień & \textbf{F C G} \\
Środa czy piątek nieważne to & \textbf{F C G} \\
Ważne że siedział tam pewien gość & \textbf{F E a} \\
Starszy pan który powiedział wprost & \textbf{F G a} \\
& \\
\hspace*{2em}\textit{Czy wy na zbiórkę idziecie chłopaki} & \textbf{F C G} \\
\hspace*{2em}\textit{Szkoda że młody nie jestem już} & \textbf{F C G} \\
\hspace*{2em}\textit{Bo moje zbiórki tak wyglądały} & \textbf{F E a} \\
\hspace*{2em}\textit{Granat pistolet na twarzy kurz} & \textbf{F G a} \\
& \\
Wam to zabawne się dzisiaj wydaje  & \\
Lecz nas prawdziwa ścinała śmierć  & \\
Jakże zazdroszczę Wam tej zabawy  & \\
I odszedł laskę dzierżąc jak miecz  & \\
& \\
\hspace*{2em}\textit{Czy wy na zbiórkę idziecie chłopaki}  & \\
\hspace*{2em}\textit{Szkoda że młody nie jestem już}  & \\
\hspace*{2em}\textit{Bo moje zbiórki tak wyglądały}  & \\
\hspace*{2em}\textit{Granat pistolet na twarzy kurz}  & \\
& \\
Wspomnienia tamtych dzielnych harcerzy  & \\
Ciągle są żywe każdy je zna  & \\
Nie ma harcerza co nie zna Wieży  & \\
To symbol który w pamięci trwa  & \\
& \\
\hspace*{2em}\textit{Czy wy na zbiórkę idziecie chłopaki}  & \\
\hspace*{2em}\textit{Szkoda że młody nie jestem już}  & \\
\hspace*{2em}\textit{Bo moje zbiórki tak wyglądały}  & \\
\hspace*{2em}\textit{Granat pistolet na twarzy kurz}  & \\
& \\
Nie wiem czy ktoś z Was zdolny dziś byłby  & \\
Na poświęcenie jak ludzie Ci  & \\
Więc niech choć będzie dobrym człowiekiem  & \\
Na świecie który jest taki zły  & \\
& \\
\hspace*{2em}\textit{Czy wy na zbiórkę idziecie chłopaki}  & \\
\hspace*{2em}\textit{Szkoda że młody nie jestem już}  & \\
\hspace*{2em}\textit{Bo moje zbiórki tak wyglądały}  & \\
\hspace*{2em}\textit{Granat pistolet na twarzy kurz}  & \\
\end{longtable}
\clearpage

% --- Źródło: Wilcza_zamieć.tex ---
\section{Wilcza zamieć}
\begin{longtable}{ll}
Na szlak moich blizn poprowadź palec & \textbf{a d F E} \\
By nasze drogi spleść gwiazdom na przekór & \textbf{a d G E} \\
Otwórz te rany, a potem zalecz & \textbf{F d F E} \\
Aż w zawiły losu ułożą się wzór & \textbf{a F E} \\
& \\
\hspace*{2em}\textit{Z moich snów uciekasz nad ranem} & \textbf{a G a} \\
\hspace*{2em}\textit{Cierpka jak agrest, słodka jak bez} & \textbf{F d F (E)} \\
\hspace*{2em}\textit{Chcę śnić czarne loki splątane} & \textbf{a G a d} \\
\hspace*{2em}\textit{Fiołkowe oczy mokre od łez} & \textbf{F d F E} \\
& \\
Za wilczym śladem podążę w zamieć &  \\
I twoje serce wytropię uparte  & \\
Przez gniew i smutek, stwardniałe w kamień  & \\
Rozpalę usta smagane wiatrem  & \\
& \\
\hspace*{2em}\textit{Z moich snów uciekasz nad ranem...}  & \\
& \\
Nie wiem, czy jesteś moim przeznaczeniem  & \\
Czy przez ślepy traf miłość nas związała?  & \\
Kiedy wyrzekłem moje życzenie  & \\
Czyś mnie wbrew sobie wtedy pokochała?  & \\
& \\
\hspace*{2em}\textit{Z moich snów uciekasz nad ranem...}  & \\
\end{longtable}
\clearpage

% --- Źródło: Wind_of_change.tex ---
\section{Wind of change}
\vspace{-\baselineskip}
\textit{Scorpions}\\
\begin{longtable}{ll}
\textbf{F d F d a7 d a7 G}  & \\
& \\
I follow the Moskva & \textbf{C d} \\
Down to Gorky Park & \textbf{C} \\
Listening to the wind of change & \textbf{d a7 G} \\
An August summer night & \textbf{C d} \\
Soldiers passing by & \textbf{C} \\
Listening to the wind of change & \textbf{d a7 G} \\
& \\
\textbf{F d F d a7 d a7 G}  & \\
& \\
The world is closing in & \textbf{C d} \\
And did you ever think & \textbf{C} \\
That we could be so close, like brothers? & \textbf{d a7 G} \\
The future's in the air & \textbf{C d} \\
I can feel it everywhere & \textbf{C} \\
Blowing with the wind of change & \textbf{d a7 G} \\
& \\
\hspace*{2em}\textit{Take me to the magic of the moment} & \textbf{C G d G} \\
\hspace*{2em}\textit{On a glory night} & \textbf{C G} \\
\hspace*{2em}\textit{Where the children of tomorrow dream away} & \textbf{d G a} \\
\hspace*{2em}\textit{In the wind of change} & \textbf{a/F G} \\
& \\
Walking down the street  & \\
Distant memories  & \\
Are buried in the past forever  & \\
I follow the Moskva  & \\
Down to Gorky Park  & \\
Listening to the wind of change  & \\
& \\
\hspace*{2em}\textit{Take me to the magic of the moment}  & \\
\hspace*{2em}\textit{On a glory night }  & \\
\hspace*{2em}\textit{Where the children of tomorrow share their dreams }  & \\
\hspace*{2em}\textit{With you and me}  & \\
& \\
\hspace*{2em}\textit{Take me  to the magic of the moment}  & \\
\hspace*{2em}\textit{On a glory night }  & \\
\hspace*{2em}\textit{Where the children of tomorrow dream away }  & \\
\hspace*{2em}\textit{In the wind of change }  & \\
& \\
& \\
\end{longtable}
\newpage
\begin{longtable}{ll}
The wind of change  & \\
Blows straight into the face of time  & \\
Like a stormwind that will ring the freedom bell  & \\
For peace of mind  & \\
Let your balalaika sing  & \\
What my guitar wants to say  & \\
& \\
\textbf{F G E a F G a F G E7 a d E}  & \\
& \\
\hspace*{2em}\textit{Take me (take me) to the magic of the moment}  & \\
\hspace*{2em}\textit{On a glory night (a glory night)}  & \\
\hspace*{2em}\textit{Where the children of tomorrow share their dreams (share their dreams)}  & \\
\hspace*{2em}\textit{With you and me (with you and me)}  & \\
& \\
\hspace*{2em}\textit{Take me (take me) to the magic of the moment}  & \\
\hspace*{2em}\textit{On a glory night (a glory night)}  & \\
\hspace*{2em}\textit{Where the children of tomorrow dream away (dream away)}  & \\
\hspace*{2em}\textit{In the wind of change (the wind of change)}  & \\
& \\
\textbf{F d F d a7 d}  & \\
& \\
\end{longtable}
\clearpage

% --- Źródło: Wojna_postu_z_karnawałem.tex ---
\section{Wojna postu z karnawałem}
\vspace{-\baselineskip}
\textit{Jacek Kaczmarski}\\
\begin{longtable}{ll}
Niecodzienne zbiegowisko na śródmiejskim rynku  & \\
W oknach, bramach i przy studni, w kościele i w szynku.  & \\
Straganiarzy, zakonników, błaznów i karzełków  & \\
Roi się pstrokate mrowie, roi się wśród zgiełku.  & \\
& \\
Praca stała się zabawą, a zabawa – pracą:  & \\
Toczą się po ziemi kości, z kart się sypią wióry,  & \\
Nic nie znaczy ten, kto nie gra, ci co grają – tracą  & \\
Ale nie odróżnić w ciżbie który z nich jest który.  & \\
& \\
W drzwiach świątyni na serwecie krzyże po trzy grosze,  & \\
Rozgrzeszeni wysypują się bocznymi drzwiami.  & \\
Klęczą jałmużnicy w prochu pomiędzy mnichami,  & \\
Nie odróżnić, który święty, a który świętoszek.  & \\
& \\
\hspace*{2em}\textit{Oszalało miasto całe,}  & \\
\hspace*{2em}\textit{Nie wie starzec ni wyrostek}  & \\
\hspace*{2em}\textit{Czy to post jest karnawałem,}  & \\
\hspace*{2em}\textit{Czy karnawał – postem!}  & \\
& \\
Dosiadł stulitrowej beczki kapral kawalarzy  & \\
Kałdun – tarczą, hełmem – rechot na rozlanej twarzy.  & \\
Zatknął na swej kopii upieczony łeb prosięcia,  & \\
Będzie żarcie, będzie picie, będzie łup do wzięcia.  & \\
& \\
Przeciw niemu – tron drewniany zaprzężony w księży,  & \\
A na tronie wychudzony tkwi apostoł postu.  & \\
Już przeprasza Pana Boga za to, że zwycięży,  & \\
A do ręki zamiast kopii wziął Piotrowe Wiosło.  & \\
& \\
Prześcigają się stronnicy w hasłach i modlitwach,  & \\
Minstrel śpiewa jak to stanął brat przeciwko bratu.  & \\
W przepełnionej karczmie gawiedź czeka rezultatu,  & \\
Dziecko macha chorągiewką – będzie wielka bitwa.  & \\
& \\
\hspace*{2em}\textit{Oszalało miasto całe...}  & \\
& \\
Siedzę w oknie, patrzę z góry, cały świat mam w oku,  & \\
Widzę co kto kradnie, gubi, czego szuka w tłoku.  & \\
Zmierzchem pójdę do kościoła, wyspowiadam grzeszki,  & \\
Nocą przejdę się po rynku i pozbieram resztki.  & \\
& \\
Z nich karnawałowo-postną ucztą jak się patrzy  & \\
Uraduję bliski sercu ludek wasz żebraczy.  & \\
Żeby w waszym towarzystwie pojąć prawdę całą:  & \\
\end{longtable}
\clearpage

% --- Źródło: Wspinaczka.tex ---
\section{Wspinaczka (czyli historia pewnej rewolucji)}
\vspace{-\baselineskip}
\textit{Lady Pank}\\
\begin{longtable}{ll}
Porwaliśmy się na zdobycie wielkich gór & \textbf{a C D} \\
Herosi z dawnych lat służyli nam za wzór & \textbf{h e} \\
Przez niebotyczną grań pięliśmy długo się & \textbf{C D} \\
Niejeden opadł tam znajdując w dole śmierć & \textbf{h e} \\
Po drodze hulał wiatr i sypał w oczy śnieg & \textbf{C D} \\
Paraliżował strach odbierał zmysły lęk & \textbf{h e} \\
Lecz nie ustawał nikt nawet w godzinie złej & \textbf{C D} \\
Zaciskał pięści i ze śmiechem wołał Hej & \textbf{h e} \\
& \\
\hspace*{2em}\textit{Przepięknie jest} & \textbf{C D} \\
\hspace*{2em}\textit{I tylko tlenu mniej} & \textbf{h e} \\
& \\
A prowadziły nas Nadzieja Wiara Złość  & \\
Bo tam na dole Zła naprawdę było dość &  \\
I warto było iść do góry wciąż się piąć &  \\
By sobą wreszcie być by przestać karki giąć  & \\
I w czas wędrówki tej był każdy z nas jak brat  & \\
Choć nie obyło się bez wiarołomnych zdrad  & \\
Lecz nie ustawał nikt nawet w godzinie złej  & \\
Zaciskał pieści i ze śmiechem wołał Hej  & \\
& \\
\hspace*{2em}\textit{Przepięknie jest}  & \\
\hspace*{2em}\textit{I tylko tlenu mniej}  & \\
& \\
Aż po tysiącach prób przez przeraźliwą biel  & \\
Opłacił się nasz trud- osiągnęliśmy cel  & \\
Czuliśmy bicie serc i pod stopami szczyt  & \\
Gdzie pewne było że przed nami nie był nikt  & \\
Lecz nie odezwał się tym razem żaden śmiech  & \\
Bo wszyscy padli tu zajadle łapiąc dech  & \\
Idziwny jakiś był zwycięstwa słodki smak  & \\
Minęło parę chwil aż ktoś wychrypiał tak  & \\
& \\
\hspace*{2em}\textit{Przepięknie jest}  & \\
\hspace*{2em}\textit{I tylko tlenu brak}  & \\
\hspace*{2em}\textit{Przepięknie jest}  & \\
\hspace*{2em}\textit{I tylko tlenu brak}  & \\
\end{longtable}
\clearpage

% --- Źródło: Wspomnienie_bumeranga.tex ---
\section{Wspomnienie bumeranga}
\begin{longtable}{ll}
Przyjdzie rozstań czas i nie będzie nas. & \textbf{F G a F G a} \\
Na polanie tylko pozostanie po ognisku ślad. & \textbf{F G a C F G a} \\
& \\
\hspace*{2em}\textit{Na na na na naj Daba daba daj daba daj da...ba da...j}  & \\
& \\
Zdartych głosów chór, źle złapany Dur,  & \\
Warty w nocy, jej niebieskie oczy nie powrócą już.  & \\
& \\
\hspace*{2em}\textit{Na na na na naj Daba daba daj daba daj da...ba da...j}  & \\
& \\
Zapomniany rajd, zarośnięty szlak,  & \\
Schronisk pustych i harcerskiej chusty, kiedyś będzie brak.  & \\
& \\
\hspace*{2em}\textit{Na na na na naj Daba daba daj daba daj da...ba da...j}  & \\
& \\
Staniesz z nami w krąg, dotkniesz innych rąk,  & \\
Będziesz śpiewał, marzył i rozlewał cały serca żal.  & \\
& \\
\hspace*{2em}\textit{Na na na na naj Daba daba daj daba daj da...ba da...j}  & \\
& \\
Chciałbyś cofnąć czas, stanąć twarzą w twarz  & \\
W cieniu drzew przyjaźń ci wyśpiewam, aż po wieczny czas  & \\
& \\
\hspace*{2em}\textit{Na na na na naj Daba daba daj daba daj da...ba da...j}  & \\
& \\
Czyjś zbłąkany głos do strumienia wpadł.  & \\
Nad górami, białymi chmurami cicho śpiewa wiatr.  & \\
& \\
\hspace*{2em}\textit{Na na na na naj Daba daba daj daba daj da...ba da...j}  & \\
& \\
Gdzieś za rok, lub dwa przyjdzie rozstań czas.  & \\
Złotych włosów, orzechowych oczu już nie będzie żal.  & \\
& \\
\hspace*{2em}\textit{Na na na na naj Daba daba daj daba daj da...ba da...j}  & \\
& \\
Gdzie ogniska blask - stanie obóz nasz.  & \\
Na polanie bratni krąg powstanie jak za dawnych lat.  & \\
\end{longtable}
\clearpage

% --- Źródło: Wędrowanie.tex ---
\section{Wędrowanie}
\begin{longtable}{ll}
Rozwichrzone nad głową sosny rosochate, & \textbf{a d E a} \\
Biegną niebem chmurki, owieczki skrzydlate, & \textbf{a d G C} \\
Senne oko jeziora, zda się, na wpół drzemie, & \textbf{a d G C} \\
Kolorowe sady słodkie niosą brzemię. & \textbf{a d E a} \\
& \\
\hspace*{2em}\textit{A nam czegóż to więcej potrzeba?} & \textbf{a E a} \\
\hspace*{2em}\textit{Powiedz nam!} & \textbf{C G C} \\
\hspace*{2em}\textit{Powiedz nam lesie i drogo piaszczysta,} & \textbf{C G C a} \\
\hspace*{2em}\textit{Powiedz nam.} & \textbf{a E a} \\
& \\
W połoniny zielone przepastne doliny,  & \\
ukwiecone łąki strojne jak dziewczyny.  & \\
Płaczka wierzba przysiadła na przydrożnym rowie,  & \\
matka żegnająca ruszających w drogę.  & \\
& \\
\hspace*{2em}\textit{A nam czegóż to więcej potrzeba...}  & \\
& \\
Przemierzamy doliny jak wędrowne ptaki,  & \\
co na niebie kluczem wyznaczają szlaki.  & \\
Dokąd, dokąd tak lecisz uskrzydlony bracie?  & \\
Pędzisz nie bez celu już we krwi to macie.  & \\
& \\
\hspace*{2em}\textit{A nam czegóż to więcej potrzeba...}  & \\
\end{longtable}
\clearpage

% --- Źródło: Wędrowiec.tex ---
\section{\textbf{Wędrowiec}}
\begin{longtable}{ll}
Nie oglądaj się za siebie, kiedy wstaje brzask, & \textbf{a C} \\
Ruszaj dalej w świat nie zatrzymuj się & \textbf{G d a} \\
Sam wybierasz swoją drogę z wiatrem czy pod wiatr & \textbf{a C} \\
Znasz tu każdy szlak przestrzeń woła cię & \textbf{G d a} \\
& \\
\hspace*{2em}\textit{Przecież wiesz, że dla ciebie każdy nowy dzień} & \textbf{C G d a} \\
\hspace*{2em}\textit{Przecież wiesz, że dla ciebie chłodny lasu cień} & \textbf{C G d a} \\
\hspace*{2em}\textit{Przecież wiesz, jak upalna bywa letnia noc} & \textbf{C G d a} \\
\hspace*{2em}\textit{Przecież wiesz, że wędrowca los to jest twój los} & \textbf{C G d a} \\
& \\
Lśni w oddali toń jeziora słyszysz ptaków krzyk & \textbf{a C} \\
Tu odpoczniesz dziś i nabierzesz sił & \textbf{G d a} \\
Ale jutro znów wyruszysz na swój stary szlak & \textbf{a C} \\
Będziesz dalej szedł tam gdzie pędzi wiatr & \textbf{G d a} \\
& \\
\hspace*{2em}\textit{Przecież wiesz, że dla ciebie każdy nowy dzień} & \textbf{C G d a} \\
\hspace*{2em}\textit{Przecież wiesz, że dla ciebie chłodny lasu cień} & \textbf{C G d a} \\
\hspace*{2em}\textit{Przecież wiesz, jak upalna bywa letnia noc} & \textbf{C G d a} \\
\hspace*{2em}\textit{Przecież wiesz, że wędrowca los to jest twój los} & \textbf{C G d a} \\
\end{longtable}
\clearpage

% --- Źródło: Wędrujemy.tex ---
\section{Wędrujemy}
\begin{longtable}{ll}
Wędruję ścieżką od ciebie do ciebie & \textbf{A2 A2} \\
Choć droga prowadzi tylko przez góry & \textbf{fis F A2} \\
Przez świat zatopiony wierzchołkami w niebie & \textbf{A2 A2} \\
Dwa światy znam lecz ten mój to który & \textbf{fis F A2} \\
& \\
Góry rozpadły się w stos fotografii & \textbf{D E7} \\
Poprzecinane wąwozami miasta & \textbf{A2 Fis F A2} \\
Ale ty mój świat ułożyć potrafisz & \textbf{D E7} \\
I świat znów zaczyna w góry się zrastać & \textbf{A2 fis F A2} \\
& \\
\hspace*{2em}\textit{Góry to nasze spiętrzone marzenia} & \textbf{C G a F} \\
\hspace*{2em}\textit{W górach ludzie jak one rosną ku niebu} & \textbf{C G a F} \\
\hspace*{2em}\textit{Morze szczytów nas w żeglarzy przemienia} & \textbf{C G a F} \\
\hspace*{2em}\textit{Sterujących coraz dalej od brzegu} & \textbf{C G a F} \\
& \\
\hspace*{2em}\textit{Góry to ludzie którzy je niosą w plecaku} & \textbf{C G a F} \\
\hspace*{2em}\textit{Ludzie są jak góry które noszą w sobie} & \textbf{C G a F} \\
\hspace*{2em}\textit{Gdzie oczy poniosą wędrujemy szlakiem} & \textbf{C G a F} \\
\hspace*{2em}\textit{A u celu i tak czeka drugi człowiek} & \textbf{C G a F} \\
& \\
Wędruję ścieżką od ciebie do ciebie & \textbf{A2 A2} \\
Choć nie ma drogi poza górami & \textbf{fis F A2} \\
Już poza tobą świata nie dostrzegam & \textbf{A2 A2} \\
Zawieszony między dwoma światami & \textbf{fis F A2} \\
& \\
Tęsknię za tobą na pustych szczytach & \textbf{D E7} \\
Lecz mój wzrok nie sięga w doliny & \textbf{A2 Fis F A2} \\
U świata krawędzi z chmur skłębionych czytam & \textbf{D E7} \\
Świat na tobie się kończy na tobie zaczyna & \textbf{A2 Fis F A2} \\
& \\
\hspace*{2em}\textit{Góry to nasze spiętrzone marzenia...}  & \\
\chord{t}{n,n,f1, p2, p2, n}{A2} & \\
\end{longtable}
\clearpage

% --- Źródło: Wędrówka.tex ---
\section{\textbf{Wędrówka}}
\vspace{-\baselineskip}
\textit{Adam Halber}\\
\begin{longtable}{ll}
Kto zliczy ile przeszliśmy już dróg, & \textbf{C F G a} \\
ile słońc paliło twarz? & \textbf{C F G} \\
Kto zna co długiej wędrówki jest trud? & \textbf{C F G a} \\
No, powiedz czy Ty to znasz. & \textbf{F G C C7} \\
& \\
\hspace*{2em}\textit{Upał i kurz i myślisz już,} & \textbf{F G C a} \\
\hspace*{2em}\textit{że to już koniec, że dno.} & \textbf{F G C C7} \\
\hspace*{2em}\textit{Ale ten świat - czar młodych lat,} & \textbf{F G C a} \\
\hspace*{2em}\textit{tak bracie to właśnie to.} & \textbf{F G C} \\
& \\
Deszcz obmył twarz z pyłu ścieżek wśród gór. & \textbf{C F G a} \\
Wiatr suszył pot czasem łzy. & \textbf{C F G} \\
Do snu kołysał nas niejeden wiatr. & \textbf{C F G a} \\
lecz teraz Ja znaczy My. & \textbf{F G C C7} \\
& \\
\hspace*{2em}\textit{Upał i kurz i myślisz już,}  & \\
\hspace*{2em}\textit{że to już koniec, że dno.}  & \\
\hspace*{2em}\textit{Ale ten świat - czar młodych lat,}  & \\
\hspace*{2em}\textit{tak bracie to właśnie to.}  & \\
\end{longtable}
\clearpage

% --- Źródło: Z_nim_będziesz_szczęśliwa.tex ---
\section{Z nim będziesz szczęśliwa}
\vspace{-\baselineskip}
\textit{Stare Dobre Małżeństwo}\\
\begin{longtable}{ll}
Zrozum to co powiem & \textbf{a E7} \\
Spróbuj to zrozumieć dobrze & \textbf{C G} \\
Jak życzenia najlepsze te urodzinowe & \textbf{F C} \\
Albo noworoczne jeszcze lepsze może & \textbf{d E7} \\
O północy gdy składane & \textbf{F C} \\
Drżącym głosem niekłamane & \textbf{E7} \\
& \\
\hspace*{2em}\textit{Z nim będziesz szczęśliwsza} & \textbf{F C} \\
\hspace*{2em}\textit{Dużo szczęśliwsza będziesz z nim} & \textbf{d E7} \\
\hspace*{2em}\textit{Ja cóż włóczęga niespokojny duch} & \textbf{F C} \\
\hspace*{2em}\textit{Ze mną można tylko} & \textbf{d} \\
\hspace*{2em}\textit{Pójść na wrzosowisko} & \textbf{G} \\
\hspace*{2em}\textit{I zapomnieć wszystko} & \textbf{a} \\
\hspace*{2em}\textit{Jaka epoka jaki wiek} & \textbf{F C D} \\
\hspace*{2em}\textit{Jaki rok jaki miesiąc jaki dzień} & \textbf{C d F C} \\
\hspace*{2em}\textit{I jaka godzina} & \textbf{d} \\
\hspace*{2em}\textit{Kończy się} & \textbf{F} \\
\hspace*{2em}\textit{A jaka zaczyna} & \textbf{a} \\
& \\
Nie myśl że nie kocham  & \\
Lub że tylko trochę  & \\
Jak cię kocham nie powiem no bo nie wypowiem  & \\
Tak ogromnie bardzo jeszcze więcej może  & \\
I dlatego właśnie żegnaj  & \\
Zrozum dobrze żegnaj  & \\
& \\
\hspace*{2em}\textit{Z nim będziesz szczęśliwsza...}  & \\
& \\
Ze mną można tylko & \textbf{d} \\
W dali znikać cicho & \textbf{F a} \\
\end{longtable}
\clearpage

% --- Źródło: Zamki_w_piasku.tex ---
\section{Zamki na piasku}
\vspace{-\baselineskip}
\textit{Lady Pank}\\
\begin{longtable}{ll}
Jesteś idolem & \textbf{b G A F} \\
Wielbi Cię tłum & \textbf{F G} \\
Gdzie się pojawisz słychać & \textbf{b G A F} \\
Zdumionych głosów szum & \textbf{F G} \\
& \\
W porannej prasie widzisz & \textbf{b G A F} \\
Codziennie swoją twarz & \textbf{F G} \\
Z możnymi tego świata & \textbf{b G A F} \\
O wielkie stawki grasz & \textbf{F G} \\
& \\
\hspace*{2em}\textit{Zamki na piasku} & \textbf{b G A F} \\
\hspace*{2em}\textit{Gdy pełno w szkle} & \textbf{F G} \\
\hspace*{2em}\textit{Poranna witaj zmiano} & \textbf{b G A F} \\
\hspace*{2em}\textit{To życie Twe} & \textbf{F G} \\
& \\
Idziesz ulicą  & \\
Uśmiechasz się  & \\
Skonstruowałeś bombę  & \\
Skondensowaną śmierć  & \\
& \\
Znasz datę i godzinę  & \\
Gdy świat się zacznie bać  & \\
Policja wszystkich krajów  & \\
Rysopis Twój chce znać  & \\
& \\
Zamki na piasku  & \\
Gdy pełno w szkle  & \\
Poranna witaj zmiano  & \\
To życie Twe  & \\
& \\
Taśma kręci się  & \\
Ty stoisz przy niej  & \\
Jesteś pionkiem w grze  & \\
Kółkiem w maszynie  & \\
& \\
Żyjesz w zamkach pośród chmur  & \\
Na ich wieżach  & \\
Nie chcąc wiedzieć, ani czuć  & \\
Dokąd zmierzasz  & \\
& \\
\hspace*{2em}\textit{Zamki na piasku}  & \\
\hspace*{2em}\textit{Gdy pełno w szkle}  & \\
\hspace*{2em}\textit{Poranna witaj zmiano}  & \\
\hspace*{2em}\textit{To życie Twe || x2}  & \\
\end{longtable}
\clearpage

% --- Źródło: Zanim_pójdę.tex ---
\section{Zanim pójdę}
\vspace{-\baselineskip}
\textit{Happysad}\\
\begin{longtable}{ll}
Ile jestem Ci winien? & \textbf{a d e} \\
Ile policzyłaś mi za swą przyjaźń? & \textbf{a d e} \\
Ale kiedy wszystko już oddam, czy? & \textbf{a d e} \\
Będziesz szczęśliwa i wolna, czy? & \textbf{a d e} \\
Będziesz szczęśliwa i wolna, czy? & \textbf{a d e} \\
& \\
Ale zanim pójdę, & \textbf{a d e} \\
Ale zanim pójdę. & \textbf{a d e} \\
Ale zanim pójdę, & \textbf{a} \\
Chciałbym powiedzieć Ci, że: & \textbf{d e} \\
& \\
\hspace*{2em}\textit{Miłość, to nie pluszowy miś ani kwiaty.} & \textbf{a d e a} \\
\hspace*{2em}\textit{To też nie diabeł rogaty.} & \textbf{d e} \\
\hspace*{2em}\textit{Ani miłość, kiedy jedno płacze,} & \textbf{a d G} \\
\hspace*{2em}\textit{A drugie po nim skacze.} & \textbf{C F} \\
\hspace*{2em}\textit{Miłość, to żaden film w żadnym kinie,} & \textbf{a d e} \\
\hspace*{2em}\textit{Ani róże, ani całusy małe, duże.} & \textbf{a d e} \\
\hspace*{2em}\textit{Ale miłość - kiedy jedno spada w dół} & \textbf{a d G} \\
\hspace*{2em}\textit{Drugie ciągnie je ku górze.} & \textbf{C F} \\
& \\
Ile jestem Ci winien?  & \\
Ile policzyłaś mi za swą przyjaźń?  & \\
Ile były warte nasze słowa,  & \\
Kiedy próbowaliśmy wszystko od nowa?  & \\
Kiedy próbowaliśmy wszystko od nowa?  & \\
& \\
Ale zanim pójdę,  & \\
Ale zanim pójdę.  & \\
Ale zanim pójdę,  & \\
Chciałbym powiedzieć Ci, że:  & \\
& \\
\hspace*{2em}\textit{Miłość, to nie pluszowy miś ani kwiaty...}  & \\
& \\
Ale zanim pójdę,  & \\
Ale zanim pójdę.  & \\
Ale zanim pójdę,  & \\
Chciałbym powiedzieć Ci, że:  & \\
& \\
\hspace*{2em}\textit{Miłość, to nie pluszowy miś ani kwiaty... ||x2}  & \\
\end{longtable}
\clearpage

% --- Źródło: Zapach_róży.tex ---
\section{\textbf{Zapach róży} \faHeart[regular]}
\begin{longtable}{ll}
Wiem dobrze, co czujesz, te moje nastroje & \textbf{a F C G} \\
Szybka zmiana zdania, zamknięte pokoje & \textbf{a F d G} \\
Może to zabawne, ale ja nie żartuję  & \\
Chcę być sprawiedliwym, przecież ja też czuję  & \\
& \\
\hspace*{2em}\textit{Też jestem człowiekiem, też serce posiadam}  & \\
\hspace*{2em}\textit{Także innych kocham, także innych zdradzam}  & \\
\hspace*{2em}\textit{Też mnie nienawidzą, kochają niektórzy}  & \\
\hspace*{2em}\textit{Też prawie jak, każdy (Tomek) lubię zapach róży}  & \\
& \\
Też jestem samotny, czuję zapomnienie  & \\
Dobrze wiesz, że miłość to moje zbawienie  & \\
Niszczy mnie samotność, tak bardzo powoli  & \\
Choć już zrozumiałem, że nienawiść boli  & \\
& \\
\hspace*{2em}\textit{Też jestem człowiekiem, też serce posiadam...}  & \\
& \\
Chociaż nie znalazłem duszy sobie bratniej  & \\
Chcę tu jeszcze zostać do chwili ostatniej  & \\
Trochę smutno było, trochę się zawiodłem  & \\
Wtedy już wiedziałem, zostać tu nie mogę  & \\
& \\
\hspace*{2em}\textit{Też jestem człowiekiem, też serce posiadam...}  & \\
& \\
Teraz, gdy odszedłem kwiatom się spowiadam  & \\
Z tego, że Cię kocham, że serce posiadam  & \\
Z tego, że żałuję, że brak mi sumienia  & \\
Które mi wskazuje drogę do zbawienia  & \\
& \\
\hspace*{2em}\textit{Też jestem człowiekiem, też serce posiadam…}  & \\
& \\
& \\
\textbf{Z dedykacją dla Tomasza D}  & \\
\end{longtable}
\clearpage

% --- Źródło: Zawrat.tex ---
\section{\textbf{Zawrat}}
\vspace{-\baselineskip}
\textit{Andrzej Kossakowski}\\
\begin{longtable}{ll}
Już ciemna noc nad nami rozpięła swój płaszcz, & \textbf{e D a e H7} \\
Przyszedłeś, aby z nami pomarzyć jeszcze raz. & \textbf{e D a e H7} \\
I jesteś już w gromadzie, przyjaciół wszystkich znasz, & \textbf{e a H7 e} \\
Więc usiądź przy ognisku i śpiewaj z nami wraz. & \textbf{e a H7 e} \\
& \\
\hspace*{2em}\textit{Zły humor odrzuć precz,} & \textbf{a} \\
\hspace*{2em}\textit{Troskami podziel się,} & \textbf{e} \\
\hspace*{2em}\textit{Bo w naszej bandzie to} & \textbf{H7} \\
\hspace*{2em}\textit{Wszystko nieważne jest.} & \textbf{e} \\
\hspace*{2em}\textit{A swoją smutną twarz} & \textbf{a} \\
\hspace*{2em}\textit{Skieruj na ognia blask.} & \textbf{e} \\
\hspace*{2em}\textit{Posłuchaj szumu drzew.} & \textbf{H7} \\
\hspace*{2em}\textit{O, teraz dobrze jest…} & \textbf{e} \\
& \\
A kiedy po ognisku już wszyscy pójdą spać, & \textbf{e D a e H7} \\
Ty zostań, popatrz w ogień, przeżyjesz jeszcze raz: & \textbf{e D a e H7} \\
Czar wspomnień, dawne chwile i piękno górskich dróg, & \textbf{e a H7 e} \\
A wtedy na wakacje pojedziesz w Tatry znów. & \textbf{e a H7 e} \\
& \\
\hspace*{2em}\textit{Zły humor odrzuć precz,} & \textbf{a} \\
\hspace*{2em}\textit{Troskami podziel się,} & \textbf{e} \\
\hspace*{2em}\textit{Bo w naszej bandzie to} & \textbf{H7} \\
\hspace*{2em}\textit{Wszystko nieważne jest.} & \textbf{e} \\
\hspace*{2em}\textit{A swoją smutną twarz} & \textbf{a} \\
\hspace*{2em}\textit{Skieruj na ognia blask.} & \textbf{e} \\
\hspace*{2em}\textit{Posłuchaj szumu drzew.} & \textbf{H7} \\
\hspace*{2em}\textit{O, teraz dobrze jest..} & \textbf{e} \\
& \\
\end{longtable}
\clearpage

% --- Źródło: Zawsze_tam_gdzie_Ty.tex ---
\section{Zawsze tam gdzie Ty}
\vspace{-\baselineskip}
\textit{Lady Pank}\\\\
\begin{longtable}{ll}
Zamienię każdy oddech w niespokojny wiatr & \textbf{C a G G7} \\
By zabrał mnie z powrotem, tam gdzie masz swój świat & \textbf{C a G G7} \\
Poskładam wszystkie szepty w jeden ciepły krzyk & \textbf{C a G G7} \\
Żeby znalazł Cię aż tam, gdzie pochowałaś sny & \textbf{C a G G7} \\
 & \\
\hspace*{2em}\textit{Już teraz wiem, że dni są tylko po to} & \textbf{F G} \\
\hspace*{2em}\textit{By do Ciebie wracać każdą nocą złotą} & \textbf{C a} \\
\hspace*{2em}\textit{Nie znam słów, co mają jakiś większy sens} & \textbf{F G} \\
\hspace*{2em}\textit{Jeśli tylko jedno, jedno tylko wiem:} & \textbf{C a} \\
\hspace*{2em}\textit{Być tam, zawsze tam, gdzie Ty} & \textbf{F G} \\
 & \\
Nie pytaj mnie o jutro, to za tysiąc lat & \textbf{C a G G7} \\
Płyniemy białą łódką w niezbadany czas & \textbf{C a G G7} \\
Poskładam nasze szepty w jeden ciepły krzyk & \textbf{C a G G7} \\
By już nie uciekły nam, by wysuszyły łzy & \textbf{C a G G7} \\
 & \\
\hspace*{2em}\textit{Już teraz wiem, że dni są tylko po to} & \textbf{F G} \\
\hspace*{2em}\textit{By do Ciebie wracać każdą nocą złotą} & \textbf{C a} \\
\hspace*{2em}\textit{Nie znam słów, co mają jakiś większy sens} & \textbf{F G} \\
\hspace*{2em}\textit{Jeśli tylko jedno, jedno tylko wiem:} & \textbf{C a} \\
\hspace*{2em}\textit{Być tam, zawsze tam, gdzie Ty} & \textbf{F G} \\
 & \\
\hspace*{2em}\textit{Budzić się i chodzić spać we własnym niebie} & \textbf{C a} \\
\hspace*{2em}\textit{Być tam, zawsze tam, gdzie Ty} & \textbf{F G} \\
\hspace*{2em}\textit{Żegnać się co świt i wracać znów do Ciebie} & \textbf{C a} \\
\hspace*{2em}\textit{Być tam, zawsze tam, gdzie Ty} & \textbf{F G} \\
\hspace*{2em}\textit{Budzić się i chodzić spać we własnym niebie} & \textbf{C a} \\
\hspace*{2em}\textit{Być tam, zawsze tam, gdzie Ty, jeeee} & \textbf{F G C} \\
\end{longtable}
\clearpage

% --- Źródło: Zbroja.tex ---
\section{\textbf{Zbroja}}
\vspace{-\baselineskip}
\textit{Jacek Kaczmarski}\\
\begin{longtable}{ll}
Dałeś mi Panie zbroję, dawny kuł płatnerz ją & \textbf{e D e D e} \\
W wielu pogięta bojach, w wielu ochrzczona krwią & \textbf{e D e D e} \\
W wykutej dla giganta potykam się co krok & \textbf{G e A H7} \\
Bo jak sumienia szantaż uciska lewy bok & \textbf{G Fis F e H e (D)} \\
& \\
\hspace*{2em}\textit{Lecz choć zaginął hełm i miecz} & \textbf{G D} \\
\hspace*{2em}\textit{Dla ciała żadna w niej ostoja} & \textbf{a G} \\
\hspace*{2em}\textit{To przecież w końcu ważna rzecz} & \textbf{a H7 e a} \\
\hspace*{2em}\textit{Zbroja} & \textbf{e H7 e} \\
& \\
Magicznych na niej rytów dziś nie odczyta nikt & \textbf{e D e D e} \\
Ale wykuta z mitów i wieczna jest jak mit & \textbf{e D e D e} \\
Do ciała mi przywarła, nie daje żyć i spać & \textbf{G e A H7} \\
A tłum się cieszy z karła, co chce giganta grać & \textbf{G Fis F e H e (D)} \\
& \\
\hspace*{2em}\textit{Lecz choć zaginął hełm i miecz} & \textbf{G D} \\
\hspace*{2em}\textit{Dla ciała żadna w niej ostoja} & \textbf{a G} \\
\hspace*{2em}\textit{Bo przecież w końcu ważna rzecz} & \textbf{a H7 e a} \\
\hspace*{2em}\textit{Zbroja} & \textbf{e H7 e} \\
& \\
A taka w niej powaga dawno zaschniętej krwi & \textbf{e D e D e} \\
Że czuję jak wymaga i każe rosnąć mi & \textbf{e D e D e} \\
Być może nadaremnie, lecz stanę w niej za stu & \textbf{G e A H7} \\
Zdejmij ją Panie ze mnie, jeśli umrę podczas snu & \textbf{G Fis F e H e (D)} \\
& \\
\hspace*{2em}\textit{Bo choć zaginął hełm i miecz} & \textbf{G D} \\
\hspace*{2em}\textit{Dla ciała żadna w niej ostoja} & \textbf{a G} \\
\hspace*{2em}\textit{To w końcu życia warta rzecz} & \textbf{a H7 e a} \\
\hspace*{2em}\textit{Zbroja} & \textbf{e H7 e} \\
& \\
Wrzasnęli hasło „wojna”, zbudzili hufce hord & \textbf{e D e D e} \\
Zgwałcona noc spokojna ogląda pierwszy mord & \textbf{e D e D e} \\
Goreją świeże rany, hańbiona płonie twarz & \textbf{G e A H7} \\
Lecz nam do obrony dany pamięci pancerz nasz & \textbf{G Fis F e H e (D)} \\
& \\
\hspace*{2em}\textit{Choć, choć za ciosem pada cios} & \textbf{G D} \\
\hspace*{2em}\textit{I wróg posiłki śle w konwojach} & \textbf{a G} \\
\hspace*{2em}\textit{Nas przed upadkiem chroni wciąż} & \textbf{a H7 e a} \\
\hspace*{2em}\textit{Zbroja} & \textbf{e H7 e} \\
& \\
& \\
\end{longtable}
\newpage
\begin{longtable}{ll}
Wywlekli pudła z blachy, natkali kul do luf & \textbf{e D e D e} \\
I straszą sami w strachu, strzelają do ciał i słów & \textbf{e D e D e} \\
Zabrońcie żyć wystrzałem, niech zatryumfuje gwałt & \textbf{G e A H7} \\
Nad każdym wzejdzie ciałem pamięci żywej kształt & \textbf{G Fis F e H e (D)} \\
& \\
\hspace*{2em}\textit{Choć słońce skrył bojowy gaz} & \textbf{G D} \\
\hspace*{2em}\textit{I żołdak pławi się w rozbojach} & \textbf{a G} \\
\hspace*{2em}\textit{Wciąż przed upadkiem chroni nas} & \textbf{a H7 e a} \\
\hspace*{2em}\textit{Zbroja} & \textbf{e H7 e} \\
& \\
Wytresowali świnie, kupili sobie psy & \textbf{e D e D e} \\
I w pustych słów świątyni stawiają ołtarz krwi & \textbf{e D e D e} \\
Zawodzi przed bałwanem półślepy kapłan-łgarz & \textbf{G e A H7} \\
I każdym nowym zdaniem hartuje pancerz nasz & \textbf{G Fis F e H e (D)} \\
& \\
\hspace*{2em}\textit{Choć krwią zachłysnął się nasz czas} & \textbf{G D} \\
\hspace*{2em}\textit{Choć myśli toną w paranojach} & \textbf{a G} \\
\hspace*{2em}\textit{Jak zawsze chronić będzie nas} & \textbf{a H7 e a} \\
\hspace*{2em}\textit{Zbroja} & \textbf{e H7 e} \\
\end{longtable}
\clearpage

% --- Źródło: Zbroja_Ballada_o_Bełcie.tex ---
\section{Ballada o Bełcie}
\begin{longtable}{ll}
Dałeś mi Panie bełta, dawny kucharz robił go & \textbf{e D e D e} \\
Idealne proporcje jeden do jednego & \textbf{e D e D e} \\
Lecz problem pewien mamy, bo podła kucharka & \textbf{G e A H7} \\
Zabrania nam robić przepysznego bełta & \textbf{G Fis F e H e (D)} \\
& \\
\hspace*{2em}\textit{I choć pompkami straszą nas} & \textbf{G D} \\
\hspace*{2em}\textit{To my się tego nie boimy} & \textbf{a G} \\
\hspace*{2em}\textit{Choć brzydzi to pięknie dziewczyny} & \textbf{a H7 e a} \\
\hspace*{2em}\textit{Bełt!, Bełt, Bełt!} & \textbf{e H7 e} \\
& \\
Kazali myć nam gary, pracować cały dzień  & \\
Lecz my się nie poddamy, choć kryjemy się w cień  & \\
Gdy Gocha nas ochrzania stawiamy sztandar nasz  & \\
I z każdym nowym krzykiem rzucamy bełtem w twarz  & \\
& \\
\hspace*{2em}\textit{I choć pompkami straszą nas...}  & \\
& \\
Chcą zmienić wiek historii w przeciągu kilku dni  & \\
Do przepysznego bełta spuszczając naszej krwi  & \\
Zapewne na daremne zmieniają nazwę mu  & \\
Czerwone wody Nilu zagłuszy bełtów chór  & \\
& \\
\hspace*{2em}\textit{I choć pompkami straszą nas...}  & \\
& \\
Krzyknął Michał: do Gara! Uciemiężyli nas  & \\
Lecz nic nie stłamsi wolności sulimowych mas  & \\
Powstają nowe nazwy i Gosze płonie twarz  & \\
Lecz w sercach naszych pewność - nie powstrzymają nas!  & \\
& \\
\hspace*{2em}\textit{Choć w zgrupie nienawidzą nas}  & \\
\hspace*{2em}\textit{Choć kadra stoi po ich stronie}  & \\
\hspace*{2em}\textit{Bierzemy boski napój w dłonie}  & \\
\hspace*{2em}\textit{Bełt!, Bełt, Bełt!}  & \\
& \\
\end{longtable}
\clearpage

% --- Źródło: Zegarek.tex ---
\section{Zegarek}
\vspace{-\baselineskip}
\textit{Wojtek Szumański}\\
\begin{longtable}{ll}
Jak wiadomo na początku & \textbf{a E} \\
To proste zdaje się & \textbf{a E} \\
Wszytko było w porządku & \textbf{d a} \\
Miałem Ciebie a ty mnie & \textbf{E} \\
Nie ma o czym więcej marzyć & \textbf{a E} \\
Być z miłością za pan brat & \textbf{a E} \\
Uśmiech gościł na twej twarzy & \textbf{d a} \\
A u stóp mieliśmy świat & \textbf{E a} \\
& \\
Oto najpiękniejsza z bajek & \textbf{F C} \\
Się zdarzyła właśnie mi & \textbf{G a} \\
Kiedy dałaś mi zegarek & \textbf{F C} \\
By nam liczył dobre dni & \textbf{G a} \\
O ironio kto by wierzył & \textbf{F C} \\
Że coś poróżni nas & \textbf{E a} \\
Gdy zegarek nam odmierzał wspólny czas & \textbf{F E a E} \\
& \\
A po wszystkim rozpaczałem & \textbf{a E} \\
W rytm wskazówek "Tik i tak" & \textbf{a E} \\
Bo z zegarkiem sam zostałem & \textbf{d a} \\
Do kompletu ciebie brak & \textbf{E} \\
I tak żyłem własną zgubą & \textbf{a E} \\
Próżno czekać na twój ruch & \textbf{a E} \\
Mój przyjaciel stracił lubą & \textbf{d a} \\
Więc cierpieliśmy we dwóch & \textbf{E a} \\
& \\
Przykra sprawa to niestety & \textbf{F C} \\
Że na palcu sygnet miał & \textbf{G a} \\
Bo go dostał od kobiety & \textbf{F C} \\
Której w zamian serce dał & \textbf{G a} \\
Obaj mając artefakty & \textbf{F C} \\
Definiujące nasz stan & \textbf{E a} \\
Łącząc fakty uknuliśmy nasz plan & \textbf{F E a E} \\
& \\
Więc oddałem mu zegarek & \textbf{a E} \\
Choć nie było z tym przyjemnie & \textbf{a	E} \\
Mówię, "Schowaj jak najdalej" & \textbf{d a} \\
Jak najdalej ode mnie & \textbf{E} \\
Te zdarzenia też mnie smucą & \textbf{a E} \\
Lecz swój sygnet oddaj mnie & \textbf{a E} \\
A gdy one do nas wrócą & \textbf{d a} \\
Znowu wymienimy się! & \textbf{E a} \\
Nadzieja pozwalała & \textbf{F C} \\
Snuć powroty w wyobraźni & \textbf{G a} \\
Wspólna rozpacz się stała & \textbf{F C} \\
Fundamentem tej przyjaźni & \textbf{G a} \\
Gdy oznajmił dnia pewnego & \textbf{F C} \\
Piękna rzecz się zdarzyła & \textbf{E a} \\
Bo ona do niego wróciła & \textbf{F E a E} \\
& \\
Coś się później wydarzyło & \textbf{a E} \\
Domyśliło by się dziecię & \textbf{a E} \\
On z radością że aż miło & \textbf{d a} \\
Mi przypomniał o sygnecie & \textbf{E} \\
Więc czym prędzej mu go dałem & \textbf{a E} \\
A przyjaciel do mnie w słowa & \textbf{a E} \\
„Ja zachowam twój zegarek” & \textbf{d a} \\
„Wszak umowa to umowa!” & \textbf{E a} \\
& \\
Piszę słowa tej piosenki & \textbf{F C} \\
Bo go wciąż nie odzyskałem & \textbf{G a} \\
Nic nie zdobi mojej ręki & \textbf{F C} \\
Choć już swoje wycierpiałem & \textbf{G a} \\
On wyrzucił twój podarek & \textbf{F C} \\
Do pudełka, lub do lasu & \textbf{E a} \\
Tak straciłem mój zegarek & \textbf{F} \\
A wraz z nim poczucie czasu! & \textbf{E} \\
& \\
Lecz o zwrot nie będę prosić & \textbf{a E} \\
Bo zbyt podle się z tym czuję & \textbf{a E} \\
Próbowałem inne nosić & \textbf{a E} \\
Ale żaden nie pasuje & \textbf{a E} \\
Z samym sobą w wiecznej walce & \textbf{d a} \\
Zmagam się z tą przypadłościa & \textbf{E a} \\
Że ucieka czas przez palce & \textbf{F} \\
A ja żyję wciąż przeszłością & \textbf{E} \\
& \\
& \\
\end{longtable}
\newpage
\begin{longtable}{ll}
Nie wiem gdzie szukać pomocy & \textbf{a E} \\
Na moje niewyspanie & \textbf{a E} \\
Bo nie mogę zasnąć w nocy & \textbf{a E} \\
Gdy nie koi mnie tykanie & \textbf{a E} \\
Pojawiła się zawiłość & \textbf{d a} \\
W ciemnym serca zakamarku & \textbf{E a} \\
Chciałbym komuś dać mą miłość & \textbf{F} \\
Lecz utknęła w tym zegarku & \textbf{E} \\
& \\
A więc jestem pełen złości & \textbf{a E} \\
Ulga od niej będzie złotem & \textbf{a E} \\
Nie chcę mieć twojej litości & \textbf{a E} \\
Chcę zegarek mieć z powrotem & \textbf{a E} \\
Nie wiem ile już minęło & \textbf{F C} \\
Czas mi zatrzymała strata & \textbf{G a} \\
Ile godzin upłynęło & \textbf{F} \\
Czy to dni, czy to już lata & \textbf{E} \\
& \\
To zbyt wiele mnie kosztuje & \textbf{a E} \\
Znowu prosić byś wracała & \textbf{a E} \\
Chociaż gdzieś pod skórą czuję & \textbf{a E} \\
Że zegarek nadal działa & \textbf{a E} \\
Oto najstraszniejsza z bajek & \textbf{F C} \\
Się zdarzyła właśnie mi & \textbf{G a} \\
Kiedy dałaś mi zegarek & \textbf{F} \\
By nam liczył dobre dni & \textbf{E} \\
\end{longtable}
\clearpage

% --- Źródło: Zegarmistrz_świata.tex ---
\section{Zegarmistrz światła}
\vspace{-\baselineskip}
\textit{Tadeusz Woźniak}\\
\begin{longtable}{ll}
\hspace*{2em}\textit{A kiedy przyjdzie także po mnie} & \textbf{a G} \\
\hspace*{2em}\textit{Zegarmistrz światła purpurowy} & \textbf{D a} \\
\hspace*{2em}\textit{By mi zabełtać błękit w głowie} & \textbf{C G} \\
\hspace*{2em}\textit{To będę jasny i gotowy} & \textbf{D a} \\
& \\
Spłyną przeze mnie dni na przestrzał & \textbf{C G} \\
Zgasną podłogi i powietrza & \textbf{D a} \\
Na wszystko jeszcze raz popatrzę & \textbf{C G} \\
I pójdę nie wiem gdzie - na zawsze. & \textbf{D a} \\
& \\
\hspace*{2em}\textit{A kiedy przyjdzie także po mnie}  & \\
\hspace*{2em}\textit{Zegarmistrz światła purpurowy}  & \\
\hspace*{2em}\textit{By mi zabełtać błękit w głowie}  & \\
\hspace*{2em}\textit{To będę jasny i gotowy}  & \\
& \\
Spłyną przeze mnie dni na przestrzał  & \\
Zgasną podłogi i powietrza  & \\
Na wszystko jeszcze raz popatrzę  & \\
I pójdę nie wiem gdzie - na zawsze.  & \\
& \\
\hspace*{2em}\textit{A kiedy przyjdzie po mnie}  & \\
\hspace*{2em}\textit{Zegarmistrz światła purpurowy}  & \\
\hspace*{2em}\textit{By mi zabełtać błękit w głowie}  & \\
\hspace*{2em}\textit{To będę jasny i gotowy}  & \\
& \\
Spłyną przeze mnie dni na przestrzał  & \\
Zgasną podłogi i powietrza  & \\
Na wszystko jeszcze raz popatrzę  & \\
I pójdę nie wiem gdzie - na zawsze  & \\
& \\
\hspace*{2em}\textit{A kiedy przyjdzie także po mnie}  & \\
\hspace*{2em}\textit{Zegarmistrz światła purpurowy}  & \\
\hspace*{2em}\textit{By mi zabełtać błękit w głowie}  & \\
\hspace*{2em}\textit{To będę jasny i gotowy}  & \\
& \\
Spłyną przeze mnie dni na przestrzał  & \\
Zgasną podłogi i powietrza  & \\
Na wszystko jeszcze raz popatrzę  & \\
I pójdę nie wiem gdzie - na zawsze  & \\
\end{longtable}
\clearpage

% --- Źródło: Zielony_Płomień.tex ---
\section{Zielony Płomień}
\begin{longtable}{ll}
W dąbrowy gęstym listowiu & \textbf{a G a G} \\
błyska zielona skra & \textbf{C E a} \\
Trzepoce z wiatrem jak płomień & \textbf{a G a G} \\
mundur harcerski nasz & \textbf{C G7 C G} \\
Czapka troszeczkę na bakier & \textbf{C G7 a G} \\
dusza rogata w niej & \textbf{C G7 E} \\
Wiatr polny w uszach i ptaki & \textbf{a G a G} \\
w pachnących włosach drzew & \textbf{C E a} \\
& \\
Tam gdzie się kończy horyzont  & \\
leży nieznany ląd &  \\
Ziemia jest trochę garbata  & \\
więc go nie widać stąd  & \\
Kreską przebiega błękitną  & \\
strzępioną pasmem gór  & \\
Żeglują ku tej granicy  & \\
białe okręty chmur  & \\
& \\
Gdzie niskie niebo usypia  & \\
na rosochatych pniach  & \\
Gdziekolwiek namiot rozpinam  & \\
będzie kraina ta  & \\
Zieleń o zmroku wilgotna  & \\
z niebieską plamką dnia  & \\
Cisza jak gwiazda ogromna  & \\
w grzywie złocistych traw  & \\
& \\
W dąbrowy gęstym listowiu  & \\
błyska zielona skra  & \\
Trzepoce ogień zielony  & \\
mundur harcerski nasz  & \\
Czapka troszeczkę na  & \\
bakier lecz nie poprawiaj jej  & \\
Polny za uchem masz  & \\
kwiatek duszy rogatej lżej  & \\
& \\
\end{longtable}
\clearpage

% --- Źródło: Znowu_pada_deszcz.tex ---
\section{Znowu pada deszcz}
\vspace{-\baselineskip}
\textit{Lady Pank}\\
\begin{longtable}{ll}
Pada deszcz – tak już było wczoraj & \textbf{F g B C} \\
„Znowu Ty” - to są Twoje słowa & \textbf{F g B C} \\
W moich snach nic się nie zmieniło & \textbf{F g B C} \\
Dzień jak dzień – tak już przecież było & \textbf{F g B C} \\
& \\
\hspace*{2em}\textit{Nie chcę wiedzieć, jak i co} & \textbf{D h G} \\
\hspace*{2em}\textit{Po co mówić – to nie to} & \textbf{A D} \\
\hspace*{2em}\textit{Nie umiałem kochać wprost} & \textbf{h G} \\
\hspace*{2em}\textit{Znowu zakpił los} & \textbf{A} \\
& \\
\hspace*{2em}\textit{Jeszcze wczoraj chciałem zmienić w sobie coś} & \textbf{B C F A} \\
\hspace*{2em}\textit{I odlecieć byle gdzie} & \textbf{B C F A} \\
\hspace*{2em}\textit{Jeszcze wczoraj mogłem być daleko stąd} & \textbf{B C F A} \\
\hspace*{2em}\textit{Dzisiaj znowu pada deszcz} & \textbf{B C D} \\
& \\
Wołasz mnie – słyszę swoje imię  & \\
Płynę więc – przecież wszystko płynie  & \\
Mówisz, że nic się nie zmieniło  & \\
Tak już jest – tak już przecież było  & \\
& \\
\hspace*{2em}\textit{Nie chcę wiedzieć, jak i co...}  & \\
& \\
\hspace*{2em}\textit{Jeszcze wczoraj chciałem zmienić w sobie coś...}  & \\
& \\
\end{longtable}
\clearpage

% --- Źródło: Zombie.tex ---
\section{Zombie}
\vspace{-\baselineskip}
\textit{The Cranberries}\\
\begin{longtable}{ll}
Another head hangs lowly, & \textbf{e C} \\
Child is slowly taken. & \textbf{G D} \\
And the violence caused such silence, & \textbf{e C} \\
Who are we mistaken? & \textbf{G D} \\
& \\
But you see, it's not me, it's not my family. & \textbf{e C} \\
In your head, in your head, they are fighting. & \textbf{G D} \\
& \\
\hspace*{2em}\textit{With their tanks and their bombs,} & \textbf{e} \\
\hspace*{2em}\textit{And their bombs, and their guns.} & \textbf{C} \\
\hspace*{2em}\textit{In your head, in your head, they are crying.} & \textbf{G D} \\
& \\
\hspace*{2em}\textit{In your head, in your head,} & \textbf{e C} \\
\hspace*{2em}\textit{Zombie, zombie, zombie-ie-ie} & \textbf{G D} \\
\hspace*{2em}\textit{What's in your head?} & \textbf{e} \\
\hspace*{2em}\textit{In your head,} & \textbf{C} \\
\hspace*{2em}\textit{Zombie, zombie, zombie-ie-ie} & \textbf{G D} \\
& \\
Another mother's breakin', & \textbf{e C} \\
Heart is taking over. & \textbf{G D} \\
When the violence causes silence, & \textbf{e C} \\
We must be mistaken. & \textbf{G D} \\
& \\
It's the same old theme, since 1916. & \textbf{e C} \\
In your head, in your head, they're still fighting, & \textbf{G D} \\
& \\
\hspace*{2em}\textit{With their tanks and their bombs...}  & \\
& \\
\hspace*{2em}\textit{In your head, in your head...}  & \\
& \\
\hspace*{2em}\textit{Oh, oh, oh, oh, oh, oh, oh,}  & \\
\hspace*{2em}\textit{hey, oh, ya, ya-a...}  & \\
& \\
\end{longtable}
\clearpage

% --- Źródło: Zostanie_tyle_gór.tex ---
\section{\textbf{Zostanie tyle gór}}
\begin{longtable}{ll}
\hspace*{2em}\textit{Zostanie tyle Gór ile udźwignąłem na plecach} & \textbf{e C} \\
\hspace*{2em}\textit{Zostanie tyle drzew ile narysowało pióro || x2} & \textbf{G D} \\
& \\
Tak gotowym trzeba być & \textbf{e} \\
Do każdej ludzkiej podróży & \textbf{C} \\
Tak zdecydują w niebie & \textbf{G} \\
Lub serce nie zechce już służyć & \textbf{D} \\
& \\
Ja tylko zniknę wtedy  & \\
W starym lesie bukowym  & \\
Tak jakbym wrócił do siebie  & \\
Po prostu wrócę do domu  & \\
& \\
\hspace*{2em}\textit{Zostanie tyle Gór ile udźwignąłem na plecach} & \textbf{e C} \\
\hspace*{2em}\textit{Zostanie tyle drzew ile narysowało pióro || x2} & \textbf{G D} \\
& \\
I wszystko tam będzie jak w życiu  & \\
I stół i krzesła i buty  & \\
Te same nieporuszone  & \\
Na niebie zostaną góry  & \\
& \\
Tylko ludzi nie będzie  & \\
Tych co najbardziej kocham  & \\
Czasem we śnie ukradkiem  & \\
Zamienią ze mną dwa słowa  & \\
& \\
\hspace*{2em}\textit{Zostanie tyle Gór ile udźwignąłem na plecach} & \textbf{e C} \\
\hspace*{2em}\textit{Zostanie tyle drzew ile narysowało pióro || x2} & \textbf{G D} \\
& \\
Będą leciały stadem liście  & \\
Duszyczki i szepty ich w lesie  & \\
Będzie tak wielki i świsty  & \\
Rok cały będzie tam jesień  & \\
& \\
\hspace*{2em}\textit{Zostanie tyle Gór ile udźwignąłem na plecach} & \textbf{e C} \\
\hspace*{2em}\textit{Zostanie tyle drzew ile narysowało pióro || x2} &  \\
& \\
& \\
\end{longtable}
\newpage
\begin{longtable}{ll}
\end{longtable}
\clearpage

% --- Źródło: Zostawcie_Titanica.tex ---
\section{Zostawcie Titanica}
\vspace{-\baselineskip}
\textit{Lady Pank}\\
\begin{longtable}{ll}
Ja wierzę – oni tańczą wciąż & \textbf{G e} \\
Oni żyją swoim życiem & \textbf{C} \\
I dopłyną tam gdzie chcą & \textbf{D} \\
I wierzę w ich podwodny świat & \textbf{G e} \\
A orkiestra, która grała & \textbf{C} \\
Do tej chwili gra & \textbf{D} \\
& \\
Piękna aktorka & \textbf{e C} \\
Mierzy kolię swą & \textbf{D} \\
Kelner się potknął & \textbf{e C} \\
Pada twarzą w tort & \textbf{D} \\
& \\
Nieprawda, że już nie ma ich & \textbf{G e} \\
Oni płyną, tylko wolniej & \textbf{C} \\
Tak jak wolno płyną sny & \textbf{D} \\
I nieprawda, nie znajdziecie ich G  & \\
Tyle mil już przepłynęli & \textbf{C} \\
Tyle przetańczyli dni & \textbf{D} \\
& \\
Młody milioner & \textbf{e C} \\
Dziś zakochał się & \textbf{D} \\
Bal już się zaczął & \textbf{e C} \\
Wszyscy tańczą, więc… & \textbf{D} \\
Nie przerywajcie tego! & \textbf{D} \\
& \\
\hspace*{2em}\textit{Zostawcie Titanica!} & \textbf{G D e C} \\
\hspace*{2em}\textit{Nie wyciągajcie go!} & \textbf{G D e C} \\
\hspace*{2em}\textit{Tam ciągle gra muzyka} & \textbf{G D e C} \\
\hspace*{2em}\textit{I oni tańczą wciąż} & \textbf{G D e C} \\
& \\
Piękna aktorka mruży oko i… & \textbf{e C D} \\
Młody milioner puka do jej drzwi & \textbf{e C D} \\
Pozwólcie im śnić! & \textbf{D} \\
& \\
& \\
\end{longtable}
\newpage
\begin{longtable}{ll}
\hspace*{2em}\textit{Zostawcie Titanica!} & \textbf{G D e C} \\
\hspace*{2em}\textit{Nie wyciągajcie go!} & \textbf{G D e C} \\
\hspace*{2em}\textit{Tam ciągle gra muzyka} & \textbf{G D e C} \\
\hspace*{2em}\textit{A oni w tańcu śnią} & \textbf{G D e C} \\
& \\
Niezatapialnie śnią & \textbf{F G} \\
Nieosiągalnie śnią & \textbf{F G} \\
Nieosiągalny, niezatapialny sen & \textbf{F G} \\
& \\
\hspace*{2em}\textit{Zostawcie Titanica!}  & \\
\hspace*{2em}\textit{Nie wyciągajcie go!}  & \\
\hspace*{2em}\textit{Tam ciągle gra muzyka}  & \\
\hspace*{2em}\textit{A oni w tańcu śnią}  & \\
& \\
\hspace*{2em}\textit{Zostawcie Titanica!}  & \\
\hspace*{2em}\textit{Nie wyciągajcie go!}  & \\
\hspace*{2em}\textit{Tam ciągle gra muzyka}  & \\
\hspace*{2em}\textit{A oni w tańcu śnią}  & \\
\hspace*{2em}\textit{Niezatapialny sen…}  & \\
\end{longtable}
\clearpage

% --- Źródło: Łabędzie.tex ---
\section{Łabędzie}
\vspace{-\baselineskip}
\textit{Lej Mi Pół}\\
\begin{longtable}{ll}
Piękna dziewczyna nad przepięknym stawem & \textbf{F C} \\
Karmiła łabędzie upalnym latem & \textbf{d B} \\
Jej włosy na wietrze i piegi na słońcu & \textbf{F C} \\
Sukienka w kolorze jabłoni & \textbf{d B} \\
Spojrzała na mnie, gdy szedłem w jej stronę & \textbf{F C} \\
Zakochała się we mnie, to był moment & \textbf{d B} \\
Spytała się, czy też karmię zwierzęta & \textbf{F C} \\
Usiądź ze mną na kocu i oglądaj ten pejzaż & \textbf{d B} \\
& \\
\textbf{F C}  & \\
\textbf{d B}  & \\
\textbf{F C}  & \\
\textbf{d B}  & \\
\textbf{F C}  & \\
\textbf{d B}  & \\
\textbf{F C}  & \\
\textbf{d B}  & \\
& \\
Bardzo Cię przepraszam mój drogi Kolego & \textbf{F C} \\
Lecz przepisy są niczym, gdy zwierzęta cierpią & \textbf{d B} \\
Łabędzie chcą chleba, łabędzie chcą jeść & \textbf{F C} \\
Ale nikt z urzędu im tego nie da & \textbf{d B} \\
Bo ludzie myślą, że w Polsce jest bieda & \textbf{F C} \\
I łabędziom nie wolno dawać chleba & \textbf{d B} \\
A ja myślę inaczej i mam w tym podstawy & \textbf{F C} \\
Chwyć mnie za rękę, omińmy ustawy & \textbf{d B} \\
& \\
\end{longtable}
\clearpage

% --- Źródło: ŻŻŻŻkoniec.tex ---
\end{document}