\section{Stokrotka}
\begin{longtable}{ll}
Gdzie strumyk płynie z wolna, & \textbf{G G7 G} \\
Rozsiewa zioła maj, & \textbf{G7 G D7} \\
Stokrotka rosła polna, & \textbf{a D7 a} \\
A nad nią szumiał gaj, & \textbf{G C G} \\
Stokrotka rosła polna, & \textbf{C D G e} \\
A nad nią szumiał gaj, & \textbf{a D G} \\
Zielony gaj. & \textbf{G} \\
& \\
W tym gaju tak ponuro,  & \\
Że aż przeraża mnie,  & \\
Ptaszęta za wysoko,  & \\
A mnie samotnej źle,  & \\
Ptaszęta za wysoko,  & \\
A mnie samotnej źle,  & \\
samotnej źle.  & \\
& \\
Wtem harcerz idzie z wolna.  & \\
„Stokrotko, witam cię,  & \\
Twój urok mnie zachwyca,  & \\
Czy chcesz być mą, czy nie?”  & \\
„Twój urok mnie zachwyca,  & \\
Czy chcesz być mą, czy nie?  & \\
Czy nie, czy nie?”  & \\
& \\
Stokrotka się zgodziła  & \\
I poszli w ciemny las,  & \\
A harcerz taki gapa  & \\
Że aż w pokrzywy wlazł,  & \\
A harcerz taki gapa,  & \\
Że aż w pokrzywy wlazł,  & \\
Po pas, po pas.  & \\
& \\
A ona, ona, ona,  & \\
Cóż biedna robić ma,  & \\
Nad gapą pochylona  & \\
I śmieje się: ha, ha,  & \\
Nad gapą pochylona  & \\
I śmieje: się ha, ha,  & \\
ha, ha, ha, ha  & \\
\end{longtable}