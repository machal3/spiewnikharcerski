\section{Ballada o dziewczynie, co piła gorące mleko}
\begin{longtable}{ll}
Są małe stacje wielkich kolei & \textbf{C D} \\
Nieznane jak obce imiona, & \textbf{e} \\
Są małe stacje wielkich kolei & \textbf{C D} \\
Jakiś napis i lampa zielona. & \textbf{e} \\
& \\
Na takiej stacji dawno już temu  & \\
Z daleka jadąc, z daleka  & \\
Widziałem dziewczynę w niebieskim szaliku,  & \\
Jak piła gorące mleko.  & \\
& \\
Teraz tamtędy już nigdy nie jeżdżę, & \textbf{F C} \\
A miasto moje daleko. & \textbf{F C} \\
Lecz myślę czasem o tamtej dziewczynie, & \textbf{F C} \\
Jak piła gorące mleko. & \textbf{G C} \\
& \\
\hspace*{2em}\textit{I nieraz chciałbym aby tu była,} & \textbf{c9 D} \\
\hspace*{2em}\textit{A może to miałoby sens.} & \textbf{G e} \\
\hspace*{2em}\textit{Jak ona śmiesznie to mleko piła,} & \textbf{c9 D} \\
\hspace*{2em}\textit{Gapiąc się na mnie spod rzęs.} & \textbf{G e} \\
& \\
Mam swoje sprawy, inne podróże  & \\
I nie tamtędy mi droga.  & \\
Lubię ulice wesołe i duże  & \\
I kolorowe światła na rogach.  & \\
& \\
Pewnie ma chłopca tamta dziewczyna,  & \\
A może wybrała się w świat.  & \\
Albo po prostu może jest głupia  & \\
Jak jej siedemnaście lat.  & \\
& \\
Zresztą to przecież nie ma znaczenia,  & \\
Mieszkam naprawdę daleko.  & \\
Lecz myślę czasem o tamtej dziewczynie,  & \\
Jak piła gorące mleko.  & \\
& \\
\hspace*{2em}\textit{I nieraz chciałbym aby tu była,}  & \\
\hspace*{2em}\textit{A może to miałoby sens.}  & \\
\hspace*{2em}\textit{Jak ona śmiesznie to mleko piła,}  & \\
\hspace*{2em}\textit{Gapiąc się na mnie spod rzęs.}  & \\
& \\
& \\
& \\
\end{longtable}