\section{\textbf{Płonie ognisko i szumią knieje}}
\begin{longtable}{ll}
Płonie ognisko i szumią knieje & \textbf{a E a} \\
Drużynowy jest wśród nas & \textbf{E E7 a} \\
Opowiada starodawne dzieje & \textbf{a E a} \\
Bohaterski wskrzesza czas & \textbf{E E7 a G} \\
& \\
O rycerstwie spod kresowych stanic & \textbf{C G} \\
O obrońcach naszych polskich granic & \textbf{E E7 a E7} \\
A ponad nami wiatr szumny wieje & \textbf{a E a} \\
I dębowy huczy las & \textbf{E E7 a} \\
& \\
Już do odwrotu głos trąbki wzywa & \textbf{a E a} \\
Alarmując ze wszech stron & \textbf{E E7 a} \\
Staje wiara w ordynku szczęśliwa & \textbf{a E a} \\
Serca biją w zgodny ton & \textbf{E E7 a G} \\
& \\
Każda twarz się uniesieniem płoni & \textbf{C G} \\
Każdy laskę krzepko dzierży w dłoni & \textbf{E E7 a E7} \\
A z młodzieńczej się piersi wyrywa & \textbf{a E a} \\
Pieśń potężna pieśń jak dzwon & \textbf{E E7 a} \\
& \\
Zgasło ognisko i szumią drzewa & \textbf{a E a} \\
Spojrzyj weń ostatni raz & \textbf{E E7 a} \\
Niech ci w duszy radośnie zaśpiewa & \textbf{a E a} \\
Że na zawsze łączą nas & \textbf{E E7 a G} \\
& \\
Wspólne troski i radości życia & \textbf{C G} \\
Serc harcerskich zjednoczone bicia & \textbf{E E7 a E7} \\
I ta przyjaźń najszczersza na świecie & \textbf{a E a} \\
Którą Bóg połączył nas & \textbf{E E7 a} \\
\end{longtable}