\section{Jagnięcym futerkiem wałek pokryty 2}
\begin{longtable}{ll}
\hspace*{2em}\textit{Jagnięcym futerkiem wałek pokryty} & \textbf{d d4 d C d} \\
\hspace*{2em}\textit{Na metalowym leży regale} & \textbf{d F a d} \\
\hspace*{2em}\textit{Trochę widoczny, choć trochę skryty} & \textbf{d d4 d C d} \\
\hspace*{2em}\textit{Przywodzi na myśl tatrzańskie hale} & \textbf{d F a d} \\
& \\
Czyją to była owa owieczka & \textbf{d F C d} \\
Której futerkiem malujesz krużganek? & \textbf{d F C d} \\
Hasała po łąkach jak tancereczka & \textbf{d F C d} \\
A teraz po niej pozostał ten wałek… & \textbf{d F C d} \\
& \\
Już nie zabeczy głosikiem drżącym  & \\
Skacząc po trawce na dworze  & \\
Pozostał po niej wałek kapiący  & \\
Farbą w zielonym kolorze…  & \\
& \\
Po cóż by miała skakać po dworze?  & \\
Jeszcze by dachówki zbiła  & \\
Toć po polu hasała, nieboże  & \\
Zanim na wałku skończyła  & \\
& \\
\hspace*{2em}\textit{Jagnięcym futerkiem wałek pokryty} & \textbf{d d4 d C d} \\
\hspace*{2em}\textit{W dodatku w przystępnej cenie} & \textbf{d F a d} \\
\hspace*{2em}\textit{Oto co zostało – myślisz przybity} & \textbf{d d4 d C d} \\
\hspace*{2em}\textit{Ten wałek i jagniąt milczenie…} & \textbf{d F a d} \\
& \\
\end{longtable}