\section{Lekcja historii klasycznej}
\vspace{-\baselineskip}
\textit{Jacek Kaczmarski}\\
\begin{longtable}{ll}
„Gallia est omnis divisa in partes tres & \textbf{C G} \\
Quarum unam incolunt Belgae aliam Aquitani & \textbf{d E} \\
Tertiam qui ipsorum lingua Celtae nostra Galli appellantur & \textbf{a F} \\
Ave Caesar morituri te salutant!” & \textbf{F C G C} \\
& \\
Nad Europą twardy krok legionów grzmi & \textbf{C G} \\
Nieunikniony wróży koniec republiki & \textbf{d E} \\
Gniją wzgórza galijskie w pomieszanej krwi & \textbf{a F} \\
A Juliusz Cezar pisze swoje pamiętniki & \textbf{F C G C} \\
& \\
Gallia est omnis divisa in partes tres  & \\
Quarum unam incolunt Belgae aliam Aquitani  & \\
Tertiam qui ipsorum lingua Celtae nostra Galli appellantur  & \\
Ave Caesar morituri te salutant  & \\
& \\
Pozwól Cezarze gdy zdobędziemy cały świat  & \\
Gwałcić rabować sycić wszelkie pożądania  & \\
Proste prośby żołnierzy te same są od lat  & \\
A Juliusz Cezar milcząc zabaw nie zabrania  & \\
& \\
Gallia est omnis divisa in partes tres  & \\
Quarum unam incolunt Belgae aliam Aquitani  & \\
Tertiam qui ipsorum lingua Celtae nostra Galli appellantur  & \\
Ave Caesar morituri te salutant  & \\
& \\
Cywilizuje podbite narody nowy ład  & \\
Rosną krzyże przy drogach od Renu do Nilu  & \\
Skargą krzykiem i płaczem rozbrzmiewa cały świat  & \\
A Juliusz Cezar ćwiczy lapidarność stylu  & \\
& \\
Gallia est omnis divisa in partes tres  & \\
Quarum unam incolunt Belgae aliam Aquitani  & \\
Tertiam qui ipsorum lingua Celtae nostra Galli appellantur  & \\
& \\
Ave Caesar morituri te salutant  & \\
Ave Caesar morituri te salutant  & \\
Ave Caesar morituri te salutant  & \\
\end{longtable}