\section{Ostatnia mapa Polski}
\vspace{-\baselineskip}
\textit{Jacek Kaczmarski}\\
\begin{longtable}{ll}
Zbłąkany pocisk w namiot sztabu trafił rano  & \\
I spadł na stół zasłany obrusami map.  & \\
Pergaminowy popiół czyjąś krwią schlapany  & \\
Zamiast jedynej mapy kraju ujrzał sztab.  & \\
& \\
Pędzi Naczelnik wśród wiwatujących czapek,  & \\
Stolica dobrych parę staj, a wróg – tuż, tuż;  & \\
Kraj zalał Moskal, teraz diabli wzięli mapę!  & \\
Może naprawdę Bóg zapomniał o nas już?!  & \\
& \\
Wpada na Zamek, w rozbiegane korytarze.  & \\
Pakuje kufry ktoś, papiery pali ktoś,  & \\
Wiernopoddańcze listy piszą dygnitarze  & \\
O łaskę prosząc w skrusze Jej Cesarską Mość.  & \\
& \\
– Wasza Wysokość ma ostatnią mapę kraju! –  & \\
Woła Naczelnik i królowi – bęc do stóp! –  & \\
Napiera wróg, a na nią w sztabie tam czekają!  & \\
Bez niej – masakra i dla wszystkich wspólny grób!  & \\
& \\
– Ostatniej mapy nie dam kłuć chorągiewkami,  & \\
Co oznaczają wojska, których nie mam już! –  & \\
Rzekł król i Polskę zwinął w rulon, a pergamin  & \\
Jak muszla schował w sobie szum Jej obu mórz.  & \\
& \\
Więc z niczym wybiegł wódz, o gniew wołając boży  & \\
A król, wsłuchany w znikający tupot nóg,  & \\
Pomiędzy osobiste rzeczy mapę włożył  & \\
Do sakwojaża na ostatnią ze swych dróg.  & \\
\end{longtable}