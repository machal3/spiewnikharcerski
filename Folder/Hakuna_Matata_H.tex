\section{Hakuna Matata}
\begin{longtable}{ll}
\hspace*{2em}\textit{Hakuna matata, jak cudownie to brzmi} & \textbf{F C} \\
\hspace*{2em}\textit{Hakuna matata, to nie byle bzik!} & \textbf{F D G} \\
\hspace*{2em}\textit{Już się nie martw, aż do końca twych dni!} & \textbf{a F D7} \\
\hspace*{2em}\textit{Naucz się tych dwóch radosnych słów} & \textbf{C G7} \\
\hspace*{2em}\textit{Hakuna Matata!} & \textbf{C} \\
& \\
Otóż, gdy był z niego mały wieprz & \textbf{B F C} \\
Gdy był ze mnie mały wieprz & \textbf{B F C} \\
(No pięknie - Dzięki)  & \\
& \\
Woń przykrą rozsiewał kiedy kończył jeść, & \textbf{Dis F} \\
innym jego kąpanie ciężko było znieść & \textbf{C G} \\
Mówią o mnie żem cham, jam subtelny gość & \textbf{B F C} \\
Przykre że, przy mnie ktoś wciąż zatykał nos & \textbf{Dis F G7} \\
& \\
Och! Co za wstyd! -było mu wstyd! & \textbf{C} \\
Cały świat mi zbrzydł! - był brzydszy niż ty! & \textbf{G7} \\
Czułem się paskudnie! - a może cudnie? & \textbf{B7} \\
& \\
Zawsze gdy chciałem...  & \\
(No no Pumba nie przy dzieciach...)  & \\
(O sorry)  & \\
& \\
\hspace*{2em}\textit{Hakuna matata, jak cudownie to brzmi...}  & \\
& \\
I już się nie martw, aż do końca twych dni! & \textbf{a F D} \\
Naucz się tych dwóch radosnych słów! & \textbf{C G7} \\
Hakuna matata & \textbf{a F} \\
Hakuna matata & \textbf{G} \\
Hakuna matata & \textbf{a F} \\
Hakuuuna matata & \textbf{G} \\
Matata ha ha ha  & \\
\end{longtable}