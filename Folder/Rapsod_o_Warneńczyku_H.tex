\section{\textbf{Rapsod o Warneńczyku}}
\begin{longtable}{ll}
Lśni chorągiew pozłocista & \textbf{a G a} \\
Chrzęści zbroja szmelcowana & \textbf{C G a} \\
Jedzie, jedzie król Władysław & \textbf{d a} \\
By pokonać Bisurmana & \textbf{G a} \\
& \\
Po wąwozach grzmią cykady  & \\
Koń królewski raźno parska  & \\
Dzielny Węgier Jan Hunyady  & \\
Sprawdza szyki klnąc z madziarska  & \\
& \\
Nad wzgórzami wstają zorze  & \\
Wojsko w marszu rumor czyni  & \\
O już widać czarne morze  & \\
Rzecze Legat Cezarini  & \\
& \\
I ruszają gromić pogan  & \\
W sile szesnastu tysięcy  & \\
A ich okrzyk, tak potężny  & \\
Niczym stary dzwon co dźwięczy  & \\
& \\
Król naprędce je śniadanie  & \\
Jan Hunyady wszedł z łoskotem  & \\
Nawalili wenecjanie  & \\
Wycofują swoją flotę  & \\
& \\
Król odstawił kubek z winem  & \\
Błysk mu strzelił spod powieki  & \\
Wyruszamy za godzinę  & \\
A Wenecji wstyd na wieki  & \\
& \\
Wszędzie ruch i gwar panował,  & \\
Rycerze wsiedli na konie,  & \\
Każdy na śmierć się gotował,  & \\
W pożegnaniu wznieśli dłonie  & \\
& \\
Jeszcze Warna w dali drzemie  & \\
Jeszcze nisko stoi słońce  & \\
A pancerni strzemię w strzemię  & \\
A pancerni koncerz w koncerz  & \\
& \\
A pancerni kopia w kopię  & \\
Ku piaszczystym patrzą brzegom  & \\
No to cześć- daj pyska chłopie  & \\
Rzecze król do Hunyadego  & \\
\end{longtable}
\newpage
\begin{longtable}{ll}
I trzasnęły jednym trzaskiem  & \\
Setki przyłbic zatrzaśniętych  & \\
I błysnęły jednym blaskiem  & \\
Setki mieczy wyszarpniętych  & \\
 & \\
I zadrżała ziemia święta  & \\
I huknęły dzwony w mieście  & \\
I ruszyli najpierw stępa  & \\
Potem kłusem, cwałem wreszcie  & \\
& \\
A Janczarzy zaprzedańcy  & \\
Atakują Węgrów batem  & \\
A szalony król Warneńczyk  & \\
Dla większości stał się katem  & \\
& \\
Amurat ominął skrzydło  & \\
I uderzył w Bobrzyckiego  & \\
A w burnusach dzikie bydło  & \\
Jęło bić się na całego  & \\
& \\
Sześć tysięcy oturaków  & \\
Uderzyło na Frankbana  & \\
A Słoweniec wraz z Biskupem  & \\
Rzucił wszystkich na kolana  & \\
& \\
Biskup Szymon w kontrataku  & \\
Rzucił bić się za Rozgonie,  & \\
Lecz padł ze swym wojskiem w krzyku  & \\
Pianą krwawą pluły konie.  & \\
& \\
Lech Bobrzycki zbiera wojsko  & \\
I za wrogiem się rozgląda,  & \\
Pot ociera z czoła ciężko,  & \\
Wpada jazda na wielbłądach  & \\
& \\
Konie rżą zaraz w popłochu,  & \\
Widząc z garbem stwory dziwne  & \\
Turcy jadą na Wołochów  & \\
Zagęszczając dziką bitwę  & \\
& \\
Pędzi do króla posłaniec  & \\
Cały we krwi ubabrany  & \\
Rzecze: Najjaśniejszy Panie  & \\
Lech Bobrzycki usiekany  & \\
\end{longtable}
\newpage
\begin{longtable}{ll}
Czoło marszczy się królewskie  & \\
Za kompanów tą niedolę  & \\
Władysław do wojska rzecze:  & \\
Ja od sromu śmierć dziś wolę  & \\
\ & \\
Spina swego konia w cwale  & \\
Widząc wojska swe w agonii  & \\
I w bitewnym pędząc szale  & \\
Wrzeszczy: żołnierze do broni  & \\
\\ & \\
Strzaskane padają kopie,  & \\
Miecze oraz mizykordie,  & \\
Jeszcze pięść żelazna łupie  & \\
Grając wojenną melodię  & \\
& \\
W sile pięciuset rycerstwa  & \\
Ruszył młody król na pogan  & \\
I nie tracąc wcale męstwa  & \\
Pędząc modlił się do Boga  & \\
& \\
Poszła dzielna polska jazda  & \\
Poszli Węgrzy niczym diabli  & \\
Jak stalowa ostra drzazga  & \\
Jak błyszczące ostrze szabli  & \\
& \\
I widziano jak lecieli  & \\
Pędem dzikim i szalonym  & \\
I widziano jak tonęli  & \\
W morzu Turków niezmierzonym  & \\
& \\
Król Władysław stracił konia  & \\
I z rozpędem padł na ziemię  & \\
Pod naporem wroga skonał  & \\
Wraz z rycerstwem stanął w niebie  & \\
& \\
Zbezczeszczoną głowę króla  & \\
Turcy zatknęli na pice  & \\
Węgrzy niczym pszczoły z ula  & \\
& \\
Potem z piórem siadł pod skarpą  & \\
Mnich uczony, stary skryba  & \\
Warto było czy nie warto  & \\
Odwrót lepszy byłby chyba  & \\
\end{longtable}
\newpage
\begin{longtable}{ll}
Chrzanił zacny zjadacz chleba  & \\
Czas nad nami wartko goni  & \\
I tak przecież umrzeć trzeba  & \\
To już lepiej tak jak oni  & \\
& \\
Zresztą koniec dzieło wieńczy  & \\
Mnich w klasztorze kipnął marnie  & \\
A szalony król Warneńczyk  & \\
Ma grobowiec w pięknej Warnie  & \\
& \\
I szanują go Bułgarzy  & \\
I nas dzięki niemu cenią  & \\
Więc na czarnomorskiej plaży  & \\
Składam hołd królewskim cieniom  & \\
\end{longtable}