\section{\textbf{Szanta narciarska}}
\begin{longtable}{ll}
Nazywają go marynarz & \textbf{d C d} \\
Bo opaskę ma na oku & \textbf{F G A} \\
Na każdym stoku dziewczyna & \textbf{B F} \\
Dziewczyna na każdym stoku & \textbf{F E d} \\
Pochodzi spod Poznania & \textbf{d C d} \\
Podobno umie wróżyć z kart & \textbf{F G A} \\
Panny rwie na wiązania & \textbf{B F} \\
Mężatki - na długość nart & \textbf{F E d} \\
& \\
\hspace*{2em}\textit{Caryco mokrego śniegu} & \textbf{A d} \\
\hspace*{2em}\textit{Ratrakiem płynę do Ciebie pod prąd (hej!)} & \textbf{A B} \\
\hspace*{2em}\textit{Dobrze, że stoisz na brzegu} & \textbf{B F} \\
\hspace*{2em}\textit{Bo ja właśnie schodzę na ląd} & \textbf{F E d} \\
& \\
Nigdy się nie lękał biedy  & \\
I się nie przejmował jutrem  & \\
A jego ratrak był kiedyś  & \\
Zwyczajnym rybackim kutrem  & \\
I woził dorsze i śledzie  & \\
Zimą i latem, okrągły rok  & \\
Teraz jak nieraz przejedzie  & \\
Rybami czuć cały stok  & \\
& \\
\hspace*{2em}\textit{Caryco mokrego śniegu...}  & \\
& \\
Wszyscy w porcie odetchnęli  & \\
Zwiał, nim się zakończył sezon  & \\
Jeszcze nam się jak żagiel bieli  & \\
Jego czarny kombinezon  & \\
Odpłynął gdzieś pod Ustrzyki  & \\
Przez baby straszne miał kłopoty  & \\
Forsę z polowań na orczyki  & \\
Przehulał na antybiotyk  & \\
& \\
\hspace*{2em}\textit{Caryco mokrego śniegu...}  & \\
& \\
Jeśli kiedyś go zobaczysz  & \\
Na ratraku w podłym świecie  & \\
To powiedz mu, że w Karpaczu  & \\
Czekają na niego dzieci  & \\
I kiedy opuszcza statek  & \\
Żeby się znowu oddać złu  & \\
Każda z dwudziestu siedmiu matek  & \\
Dzieciątku śpiewa do snu  & \\
& \\
\end{longtable}