\section{Jagnięcym futerkiem wałek pokryty}
\begin{longtable}{ll}
\hspace*{2em}\textit{Jagnięcym futerkiem wałek pokryty} & \textbf{d d4 d C d} \\
\hspace*{2em}\textit{Na metalowym leży regale} & \textbf{d F a d} \\
\hspace*{2em}\textit{Trochę widoczny, choć trochę skryty} & \textbf{d d4 d C d} \\
\hspace*{2em}\textit{Przywodzi na myśl tatrzańskie hale} & \textbf{d F a d} \\
& \\
Czyją to była owa owieczka & \textbf{d F C d} \\
Której futerkiem malujesz krużganek? & \textbf{d F C d} \\
Hasała po łąkach jak tancereczka & \textbf{d F C d} \\
A teraz po niej pozostał ten wałek… & \textbf{d F C d} \\
& \\
\hspace*{2em}\textit{Jagnięcym futerkiem wałek pokryty…}  & \\
& \\
Owieczko pogodna, gdzie masz Pasterza? & \textbf{d F C d} \\
czy nie wiesz, że wilk przerobi Cię w wałek? & \textbf{d F C d} \\
Pasterz nakarmi, gdy będziesz głodna, & \textbf{d F C d} \\
nie sprzeda nikomu – On Cię ocali. & \textbf{d F C d} \\
& \\
Włoży w Twe usta słowa kwieciste,  & \\
przestaniesz beczeć tym głosem baranim…  & \\
Zaśpiewasz wtedy pieśni wieczyste,  & \\
na chmurce odtańczysz niebiański balet.  & \\
& \\
\hspace*{2em}\textit{Jagnięcym futerkiem wałek pokryty…}  & \\
\end{longtable}