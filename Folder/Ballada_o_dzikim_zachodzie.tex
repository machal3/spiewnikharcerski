\section{Ballada o dzikim zachodzie}
\begin{longtable}{ll}
Potwierdzają to setne przykłady, & \textbf{D G D G D} \\
Że westerny wciąż jeszcze są w modzie, & \textbf{h A7 D G D} \\
Wysłuchajcie więc, proszę, ballady & \textbf{G D G D} \\
O tak zwanym najdzikszym zachodzie & \textbf{h A7 D G D7} \\
& \\
Miasto było tam, jakich tysiące, & \textbf{G g D D7} \\
Wokół preria i skały naprzeciw, & \textbf{G D} \\
Jak gdzie indziej, świeciło tam słońce, & \textbf{G D G D} \\
Marli starcy, rodziły się dzieci, & \textbf{h A7 D G D7} \\
& \\
\hspace*{2em}\textit{I tym tylko od innych różni się ta ballada,} & \textbf{G D A7 D} \\
\hspace*{2em}\textit{Że w tym mieście gdzieś na prerii krańcach} & \textbf{G D D7} \\
\hspace*{2em}\textit{Na jednego mieszkańca jeden szeryf przypadał,} & \textbf{G D A7 D} \\
\hspace*{2em}\textit{Jeden szeryf na jednego mieszkańca} & \textbf{h A7 D G D A7} \\
& \\
Konsekwencje ten fakt miał ogromne,  & \\
Bo nikt w mieście za spluwę nie chwytał,  & \\
I od dawna już każdy zapomniał,  & \\
Jak wygląda prawdziwy bandyta.  & \\
& \\
Choć finanse poniekąd leżały,  & \\
Gospodarka i przemysł był na nic,  & \\
Ale każdy, czy duży czy mały,  & \\
Czuł się za to bezpieczny bez granic  & \\
& \\
\hspace*{2em}\textit{I tym tylko od innych różni się ta ballada...}  & \\
& \\
Jeśli państwa historia ta nudzi,  & \\
To pocieszcie się tym, że nareszcie  & \\
Którejś nocy krzyk ludzi obudził,  & \\
Bank rozbity! bandyci są w mieście  & \\
& \\
Dobrzy ludzie, na próżno wołacie,  & \\
Nikt nie wstanie, za spluwę nie chwyci,  & \\
Skoro każdy świadomość zatracił,  & \\
Czym się różnią od ludzi bandyci,  & \\
& \\
\hspace*{2em}\textit{I tym tylko od innych różni się ta ballada...}  & \\
\end{longtable}
\newpage
\begin{longtable}{ll}
Potwierdzają to setne przykłady, & \textbf{D G D G D} \\
Że westerny wciąż jeszcze są w modzie, & \textbf{h A7 D G D} \\
Wysłuchaliście, państwo, ballady & \textbf{G D G D} \\
O tzw. najdzikszym zachodzie, & \textbf{h A7 D G D7} \\
& \\
Miasto było tam, jakich tysiące, & \textbf{G g D D7} \\
Ludzkie w nim krzyżowały się drogi, & \textbf{G D} \\
Lecz nie wszystkim świeciło tam słońce, & \textbf{G D G D} \\
Bo bandyci krążyli bez trwogi & \textbf{h A7 D G D7} \\
& \\
\hspace*{2em}\textit{Wyciągnijmy więc morał w tej balladzie ukryty,} & \textbf{G D A7 D} \\
\hspace*{2em}\textit{Gdy nie grozi nam żadne riffifi,} & \textbf{G D D7} \\
\hspace*{2em}\textit{Że czasami najtrudniej jest rozpoznać bandytę,} & \textbf{G D A7 D} \\
\hspace*{2em}\textit{Gdy dokoła są sami szeryfi} & \textbf{h A7 D G D A7} \\
& \\
& \\
& \\
\end{longtable}