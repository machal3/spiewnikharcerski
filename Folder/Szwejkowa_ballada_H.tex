\section{Szwejkowa ballada}
\vspace{-\baselineskip}
\textit{Marek Gajdziński (Szwejk)}\\
\begin{longtable}{ll}
Jak długo będziesz druhu z Szesnastką włóczył się, & \textbf{C a d G7} \\
przez pola i przez lasy, przez miasta i przez wsie, & \textbf{C a E A7} \\
tak długo tę piosenkę na ustach będziesz miał, & \textbf{d G7 e A7} \\
tak długo będziesz ten refren znał. & \textbf{D G7 C a d G7} \\
& \\
\hspace*{2em}\textit{Piosenka jest w wędrówce jak nieodłączny cień.} & \textbf{C a d G7} \\
\hspace*{2em}\textit{Piosenka to przyjaciel, co nie opuści Cię} & \textbf{C a E A7} \\
\hspace*{2em}\textit{i będzie razem z Tobą na jawie i we śnie,} & \textbf{d G7 e A7} \\
\hspace*{2em}\textit{bo ona zamieszkuje serce Twe.} & \textbf{D G7 C a d G7} \\
& \\
Kiedy z trudem kroczysz, ramiona plecak gnie, & \textbf{C a d G7} \\
gdy pot zalewa oczy, a w ustach ślina schnie, & \textbf{C a E A7} \\
zaśpiewasz w rytm Twych kroków i iść Ci będzie lżej, & \textbf{d G7 e A7} \\
zrozumiesz właśnie wtedy refren ten. & \textbf{D G7 C a d G7} \\
& \\
\hspace*{2em}\textit{Piosenka jest w wędrówce jak nieodłączny cień.} & \textbf{C a d G7} \\
\hspace*{2em}\textit{Piosenka to przyjaciel, co nie opuści Cię} & \textbf{C a E A7} \\
\hspace*{2em}\textit{i będzie razem z Tobą na jawie i we śnie,} & \textbf{d G7 e A7} \\
\hspace*{2em}\textit{bo ona zamieszkuje serce Twe.} & \textbf{D G7 C a d G7} \\
& \\
Głęboko Ci w pamięci zaryją słowa tej & \textbf{C a d G7} \\
piosenki co w wędrówce narodziła się, & \textbf{C a E A7} \\
piosenki, której słowa szumi las i gwiżdże wiatr, & \textbf{d G7 e A7} \\
piosenki, której sens zna każdy skaut. & \textbf{D G7 C a d G7} \\
& \\
\hspace*{2em}\textit{Piosenka jest w wędrówce jak nieodłączny cień.} & \textbf{C a d G7} \\
\hspace*{2em}\textit{Piosenka to przyjaciel, co nie opuści Cię} & \textbf{C a E A7} \\
\hspace*{2em}\textit{i będzie razem z Tobą na jawie i we śnie,} & \textbf{d G7 e A7} \\
\hspace*{2em}\textit{bo ona zamieszkuje serce Twe.} & \textbf{D G7 C a d G7} \\
& \\
\end{longtable}