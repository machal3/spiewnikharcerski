\section{\textbf{Czarny chleb i czarna kawa}}
\begin{longtable}{ll}
Jedzie pociąg, złe wagony, & \textbf{a C G a} \\
Do więzienia wiozą mnie.  & \\
Świat ma tylko cztery strony,  & \\
A w tym świecie nie ma mnie.  & \\
& \\
Gdy swe oczy otworzyłem  & \\
Wielki żal ogarnął mnie.  & \\
Po policzkach łzy spłynęły,  & \\
Zrozumiałem wtedy, że...  & \\
& \\
\hspace*{2em}\textit{Czarny chleb i czarna kawa,}  & \\
\hspace*{2em}\textit{Opętani samotnością,}  & \\
\hspace*{2em}\textit{Myślą swą szukają szczęścia,}  & \\
\hspace*{2em}\textit{Które zwie się wolnością... (x2)}  & \\
& \\
Młodsza siostra zapytała:  & \\
„Mamo, gdzie braciszek mój?”  & \\
Brat Twój w ciemnej celi siedzi!  & \\
Odsiaduje wyrok swój.  & \\
& \\
\hspace*{2em}\textit{Czarny chleb i czarna kawa...}  & \\
& \\
Wtem do celi klawisz wpada,  & \\
I zaczyna więźnia bić.  & \\
Młody więzień na twarz pada,  & \\
Serce mu przestaje bić.  & \\
& \\
I nadejdzie chwila błoga  & \\
Śmierć zabierze oddech mój,  & \\
Moje ciało stąd wyniosą  & \\
A pod celą będą znów  & \\
& \\
Inny czarny chleb i czarna kawa,  & \\
Opętani samotnością,  & \\
Myślą swą szukają szczęścia,  & \\
Które zwie się wolnością...  & \\
& \\
\hspace*{2em}\textit{Czarny chleb i czarna kawa...}  & \\
\end{longtable}