\section{Ballada Wrześniowa}
\vspace{-\baselineskip}
\textit{Jacek Kaczmarski}\\
\begin{longtable}{ll}
Długośmy na ten dzień czekali  & \\
Z nadzieją niecierpliwą w duszy,  & \\
Kiedy bez słów Towarzysz Stalin  & \\
Na mapie fajką strzałki ruszy.  & \\
& \\
Krzyk jeden pomknął wzdłuż granicy  & \\
I zanim zmilkł, zagrzmiały działa –  & \\
To w bój z szybkością nawałnicy  & \\
Armia Czerwona wyruszała.  & \\
& \\
– A cóż to za historia nowa? –  & \\
Zdumiona spyta Europa.  & \\
– Jak to? To chłopcy Mołotowa  & \\
I sojusznicy Ribbentropa.  & \\
& \\
Zwycięstw się szlak ich serią znaczy,  & \\
Sztandar wolności okrył chwałą;  & \\
Głowami polskich posiadaczy  & \\
Brukują Ukrainę całą.  & \\
& \\
Pada Podole, w hołdach Wołyń,  & \\
Lud pieśnią wita ustrój nowy,  & \\
Płoną majątki i kościoły  & \\
I Chrystus – z kulą w tyle głowy.  & \\
& \\
Nad polem bitwy dłonie wzniosą  & \\
We wspólną pięść, co dech zapiera –  & \\
Nieprzeliczone dzieci Soso,  & \\
Niezwyciężony miot Hitlera.  & \\
& \\
Już starty z map wersalski bękart,  & \\
Już wolny Żyd i Białorusin,  & \\
Już nigdy więcej polska ręka  & \\
Ich do niczego nie przymusi.  & \\
& \\
Nową im wolność głosi „Prawda”,  & \\
Świat cały wieść obiega w lot,  & \\
Że jeden odtąd łączy sztandar  & \\
Gwiazdę, sierp, hakenkreuz i młot.  & \\
& \\
Tych dni historia nie zapomni,  & \\
Gdy stary ląd w zdumieniu zastygł  & \\
I święcić będą nam potomni  & \\
Po pierwszym września – siedemnasty.  & \\
\end{longtable}