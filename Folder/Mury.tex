\section{\textbf{Mury}}
\begin{longtable}{ll}
On natchniony i młody był, & \textbf{e H7 e} \\
ich nie policzyłby nikt & \textbf{e H7} \\
On im dodawał pieśnią sił, & \textbf{C H7 e} \\
śpiewał że blisko już świt & \textbf{e H7 e} \\
& \\
Świec tysiące palili mu, & \textbf{e H7 e} \\
znad głów podnosił się dym & \textbf{e H7} \\
Śpiewał, że czas by runął mur, & \textbf{C H7 e} \\
oni śpiewali wraz z nim: & \textbf{e H7 e} \\
& \\
\hspace*{2em}\textit{Wyrwij murom zęby krat!} & \textbf{H7 e} \\
\hspace*{2em}\textit{Zerwij kajdany, połam bat!} & \textbf{H7 e} \\
\hspace*{2em}\textit{A mury runą, runą, runą} & \textbf{a e} \\
\hspace*{2em}\textit{I pogrzebią stary świat!} & \textbf{H7 e} \\
& \\
Wkrótce na pamięć znali pieśń & \textbf{e H7 e} \\
i sama melodia bez słów & \textbf{e H7} \\
Niosła ze sobą starą treść, & \textbf{C H7 e} \\
dreszcze na wskroś serc i głów & \textbf{e H7 e} \\
& \\
Śpiewali więc, klaskali w rytm, & \textbf{e H7 e} \\
jak wystrzał poklask ich brzmiał & \textbf{e H7} \\
I ciążył łańcuch, zwlekał świt, & \textbf{C H7 e} \\
on wciąż śpiewał i grał: & \textbf{e H7 e} \\
& \\
& \\
Aż zobaczyli ilu ich, & \textbf{e H7 e} \\
poczuli siłę i czas & \textbf{e H7} \\
I z pieśnią, że już blisko świt & \textbf{C H7 e} \\
szli ulicami miast; & \textbf{e H7 e} \\
& \\
Zwalali pomniki i rwali bruk – & \textbf{e H7 e} \\
Ten z nami! Ten przeciw nam! & \textbf{e H7} \\
Kto sam ten nasz najgorszy wróg! & \textbf{C H7 e} \\
A śpiewak także był sam & \textbf{e H7 e} \\
& \\
\hspace*{2em}\textit{Patrzył na równy tłumów marsz} & \textbf{H7 e} \\
\hspace*{2em}\textit{Milczał wsłuchany w kroków huk} & \textbf{H7 e} \\
\hspace*{2em}\textit{A mury rosły, rosły, rosły} & \textbf{a e} \\
\hspace*{2em}\textit{Łańcuch kołysał się u nóg...} & \textbf{H7 e} \\
& \\
\hspace*{2em}\textit{Patrzy na równy tłumów marsz} & \textbf{H7 e} \\
\hspace*{2em}\textit{Milczy wsłuchany w kroków huk} & \textbf{H7 e} \\
\hspace*{2em}\textit{A mury rosną, rosną, rosną} & \textbf{a e} \\
\end{longtable}