\section{Ballada rajdowa}
\begin{longtable}{ll}
Właśnie tu, na tej ziemi, młody harcerz meldował & \textbf{G D} \\
Swą gotowość umierać za Polskę & \textbf{C G} \\
Tak jak ty niesiesz plecak on niósł w ręku karabin & \textbf{G D} \\
W sercu miłość nadzieję i troskę & \textbf{C D G} \\
Może tu w Nowej Słupi Daleszycach Bielicach & \textbf{G D} \\
Brzozowymi krzyżami znaczone & \textbf{C G} \\
Swą dziewczynę pożegnał nic nie wiedząc że tylko & \textbf{G D} \\
Kilka dni życia mu przeznaczonych & \textbf{C G} \\
& \\
\hspace*{2em}\textit{Naszej ziemi śpiewajmy ziemi pokłon składajmy} & \textbf{G D} \\
\hspace*{2em}\textit{Taki prosty serdeczny harcerski} & \textbf{C D G} \\
\hspace*{2em}\textit{Niechaj echo poniesie tę balladę rajdową} & \textbf{G D} \\
\hspace*{2em}\textit{W nowe jutro i przyszłość nową} & \textbf{C D G} \\
& \\
Na pomniku wyryto że szesnaście miał wiosen & \textbf{G D} \\
Że był śmiały odważny radosny & \textbf{C G} \\
Kiedy padał płakała cała puszcza jodłowa & \textbf{G D} \\
Nie doczekał czekanej tak wiosny & \textbf{C G} \\
I choć on nie doczekał to nie zginął tak sobie & \textbf{G D} \\
Przetarł szlak którym dzisiaj wędrujesz & \textbf{C G} \\
Kiedy tak przy ognisku śpiewasz sobię balladę & \textbf{G D} \\
W sercu tak jak on ojczyznę czujesz & \textbf{C G} \\
& \\
\hspace*{2em}\textit{Naszej ziemi śpiewajmy ziemi pokłon składajmy...}  & \\
& \\
& \\
& \\
\end{longtable}