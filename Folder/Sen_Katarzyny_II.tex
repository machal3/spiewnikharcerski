\section{Sen Katarzyny II}
\vspace{-\baselineskip}
\textit{Jacek Kaczmarski}\\
\begin{longtable}{ll}
Na smyczy trzymam filozofów Europy & \textbf{G D G} \\
Podparłam armią marmurowe Piotra stropy & \textbf{G D e} \\
Mam psy, sokoły, konie, kocham łów szalenie & \textbf{C D e} \\
A wokół same zające i jelenie & \textbf{C D G} \\
Pałace stawiam głowy ścinam & \textbf{Fis h} \\
Kiedy mi przyjdzie na to chęć & \textbf{Fis G D} \\
Mam biografów, portrecistów & \textbf{C D e} \\
I jeszcze jedno pragnę mieć... & \textbf{C D G} \\
& \\
\hspace*{2em}\textit{Stój Katarzyno! koronę carów} & \textbf{e a e a} \\
\hspace*{2em}\textit{Sen taki jak ten może ci z głowy zdjąć} & \textbf{e a C D G} \\
& \\
Kobietą jestem ponad miarę swoich czasów &  \\
Nie bawią mnie umizgi bladych lowelasów  & \\
Ich miękkich palców dotyk budzi obrzydzenie  & \\
Już wolę łowić zające i jelenie  & \\
Ze wstydu potem ten i ów  & \\
Rzekł o mnie: niewyżyta Niemra  & \\
I pod batogiem nago biegł  & \\
Po śniegu dookoła Kremla  & \\
& \\
\hspace*{2em}\textit{Stój Katarzyno! koronę carów}  & \\
\hspace*{2em}\textit{Sen taki jak ten może ci z głowy zdjąć}  & \\
& \\
Kochanka trzeba mi takiego jak imperium  & \\
Co by mnie brał tak, jak ja daję: całą pełnią  & \\
Co by i władcy i poddańca był wcieleniem  & \\
By mi zastąpił zające i jelenie  & \\
Co by rozumiał tak jak ja  & \\
Ten głupi dwór rozdanych ról  & \\
I pośród pochylonych głów  & \\
Dawał mi rozkosz albo ból  & \\
& \\
\hspace*{2em}\textit{Stój Katarzyno! koronę carów}  & \\
\hspace*{2em}\textit{Sen taki jak ten może ci z głowy zdjąć}  & \\
\hspace*{2em}\textit{Gdyby się kiedyś kochanek taki znalazł...}  & \\
\hspace*{2em}\textit{Wiem, sama wiem! Kazałabym go ściąć!}  & \\
\end{longtable}