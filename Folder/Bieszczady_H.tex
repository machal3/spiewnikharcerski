\section{\textbf{Bieszczady}}
\begin{longtable}{ll}
Tu w dolinach wstaje mgłą wilgotny dzień, & \textbf{e a} \\
Szczyty ogniem płoną, stoki kryje cień, & \textbf{D7 G H7} \\
Mokre rosą trawy wypatrują dnia, & \textbf{e a} \\
Ciepła, które pierwszy słońca promień da. & \textbf{D7 G H7} \\
& \\
\hspace*{2em}\textit{Cicho potok gada (nanana), gwarzy pośród skał} & \textbf{G C D G} \\
\hspace*{2em}\textit{O tym deszczu, co chmury trochę wody dał,} & \textbf{G C D G} \\
\hspace*{2em}\textit{\smash{Świerki} zapatrzone w horyzontu kres} & \textbf{G C D G} \\
\hspace*{2em}\textit{Głowy pragną wysoko, jak najwyżej wznieść.} & \textbf{G C D G} \\
& \\
Tęczą kwiatów barwny połoniny łan, & \textbf{e a} \\
Słońcem wypełniony jagodowy dzban, & \textbf{D7 G H7} \\
Pachnie świeżym sianem pokos pysznych traw, & \textbf{e a} \\
Owies dzwoneczkami cisza niebu gra. & \textbf{D7 G H7} \\
& \\
\hspace*{2em}\textit{ Cicho potok gada (nanana), gwarzy pośród skał...}  & \\
& \\
Serenadą świerszczy, kaskadami gwiazd & \textbf{e a} \\
Noc w zadumie kroczy mroku ścieląc płaszcz, & \textbf{D7 G H7} \\
Wielkim wozem księżyc rusza na swój szlak, & \textbf{e a} \\
Pozłocistym sierpem gasi lampy dnia. & \textbf{D7 G H7} \\
& \\
\hspace*{2em}\textit{ Cicho potok gada (nanana), gwarzy pośród skał...}  & \\
\end{longtable}