\section{Bolero}
\begin{longtable}{ll}
W małym miasteczku & \textbf{a} \\
Gdzieś na krańcach Hiszpanii & \textbf{G} \\
Stary krawiec Augusto & \textbf{F} \\
Szył bolera najtaniej & \textbf{E} \\
I czy pan był bogaty & \textbf{a} \\
Pan był biedny czy kmieć & \textbf{G} \\
Każdy takie bolero & \textbf{F} \\
Chciał mieć & \textbf{E} \\
& \\
\hspace*{2em}\textit{To bolero} & \textbf{a} \\
\hspace*{2em}\textit{Dla bogatych cavaleros} & \textbf{G} \\
\hspace*{2em}\textit{W tym bolero będziesz senior} & \textbf{G} \\
\hspace*{2em}\textit{Prezentował się jak struś} & \textbf{E} \\
\hspace*{2em}\textit{Na bolero cavaleros ty się skuś} & \textbf{F E} \\
& \\
Jakie chcesz pan bolero & \textbf{a} \\
Białe, czarne, różowe & \textbf{G} \\
Zapinane od przodu & \textbf{F} \\
Czy wkładane przez głowę & \textbf{E} \\
Z przodu czarne guziki & \textbf{a} \\
Z tyłu patka czy nie & \textbf{G} \\
Jakie chcesz pan bolero OLE! & \textbf{F E} \\
& \\
\hspace*{2em}\textit{To bolero…}  & \\
& \\
Na corridę gdy pójdziesz & \textbf{a} \\
Tym bolero okryty & \textbf{G} \\
O biust karter zabije & \textbf{F} \\
Serce twej seniority & \textbf{E} \\
No i ona zemdlona & \textbf{a} \\
Na twe łono bez sił & \textbf{G} \\
Padnie, szepcąc „Amigo! & \textbf{F} \\
Kto to szył?” & \textbf{E} \\
& \\
\hspace*{2em}\textit{To bolero…}  & \\
\end{longtable}