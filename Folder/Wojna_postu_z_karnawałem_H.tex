\section{Wojna postu z karnawałem}
\vspace{-\baselineskip}
\textit{Jacek Kaczmarski}\\
\begin{longtable}{ll}
Niecodzienne zbiegowisko na śródmiejskim rynku  & \\
W oknach, bramach i przy studni, w kościele i w szynku.  & \\
Straganiarzy, zakonników, błaznów i karzełków  & \\
Roi się pstrokate mrowie, roi się wśród zgiełku.  & \\
& \\
Praca stała się zabawą, a zabawa – pracą:  & \\
Toczą się po ziemi kości, z kart się sypią wióry,  & \\
Nic nie znaczy ten, kto nie gra, ci co grają – tracą  & \\
Ale nie odróżnić w ciżbie który z nich jest który.  & \\
& \\
W drzwiach świątyni na serwecie krzyże po trzy grosze,  & \\
Rozgrzeszeni wysypują się bocznymi drzwiami.  & \\
Klęczą jałmużnicy w prochu pomiędzy mnichami,  & \\
Nie odróżnić, który święty, a który świętoszek.  & \\
& \\
\hspace*{2em}\textit{Oszalało miasto całe,}  & \\
\hspace*{2em}\textit{Nie wie starzec ni wyrostek}  & \\
\hspace*{2em}\textit{Czy to post jest karnawałem,}  & \\
\hspace*{2em}\textit{Czy karnawał – postem!}  & \\
& \\
Dosiadł stulitrowej beczki kapral kawalarzy  & \\
Kałdun – tarczą, hełmem – rechot na rozlanej twarzy.  & \\
Zatknął na swej kopii upieczony łeb prosięcia,  & \\
Będzie żarcie, będzie picie, będzie łup do wzięcia.  & \\
& \\
Przeciw niemu – tron drewniany zaprzężony w księży,  & \\
A na tronie wychudzony tkwi apostoł postu.  & \\
Już przeprasza Pana Boga za to, że zwycięży,  & \\
A do ręki zamiast kopii wziął Piotrowe Wiosło.  & \\
& \\
Prześcigają się stronnicy w hasłach i modlitwach,  & \\
Minstrel śpiewa jak to stanął brat przeciwko bratu.  & \\
W przepełnionej karczmie gawiedź czeka rezultatu,  & \\
Dziecko macha chorągiewką – będzie wielka bitwa.  & \\
& \\
\hspace*{2em}\textit{Oszalało miasto całe...}  & \\
& \\
Siedzę w oknie, patrzę z góry, cały świat mam w oku,  & \\
Widzę co kto kradnie, gubi, czego szuka w tłoku.  & \\
Zmierzchem pójdę do kościoła, wyspowiadam grzeszki,  & \\
Nocą przejdę się po rynku i pozbieram resztki.  & \\
& \\
Z nich karnawałowo-postną ucztą jak się patrzy  & \\
Uraduję bliski sercu ludek wasz żebraczy.  & \\
Żeby w waszym towarzystwie pojąć prawdę całą:  & \\
\end{longtable}