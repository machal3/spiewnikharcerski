\section{Ballada o Bełcie}
\begin{longtable}{ll}
Dałeś mi Panie bełta, dawny kucharz robił go & \textbf{e D e D e} \\
Idealne proporcje jeden do jednego & \textbf{e D e D e} \\
Lecz problem pewien mamy, bo podła kucharka & \textbf{G e A H7} \\
Zabrania nam robić przepysznego bełta & \textbf{G Fis F e H e (D)} \\
& \\
\hspace*{2em}\textit{I choć pompkami straszą nas} & \textbf{G D} \\
\hspace*{2em}\textit{To my się tego nie boimy} & \textbf{a G} \\
\hspace*{2em}\textit{Choć brzydzi to pięknie dziewczyny} & \textbf{a H7 e a} \\
\hspace*{2em}\textit{Bełt!, Bełt, Bełt!} & \textbf{e H7 e} \\
& \\
Kazali myć nam gary, pracować cały dzień  & \\
Lecz my się nie poddamy, choć kryjemy się w cień  & \\
Gdy Gocha nas ochrzania stawiamy sztandar nasz  & \\
I z każdym nowym krzykiem rzucamy bełtem w twarz  & \\
& \\
\hspace*{2em}\textit{I choć pompkami straszą nas...}  & \\
& \\
Chcą zmienić wiek historii w przeciągu kilku dni  & \\
Do przepysznego bełta spuszczając naszej krwi  & \\
Zapewne na daremne zmieniają nazwę mu  & \\
Czerwone wody Nilu zagłuszy bełtów chór  & \\
& \\
\hspace*{2em}\textit{I choć pompkami straszą nas...}  & \\
& \\
Krzyknął Michał: do Gara! Uciemiężyli nas  & \\
Lecz nic nie stłamsi wolności sulimowych mas  & \\
Powstają nowe nazwy i Gosze płonie twarz  & \\
Lecz w sercach naszych pewność - nie powstrzymają nas!  & \\
& \\
\hspace*{2em}\textit{Choć w zgrupie nienawidzą nas}  & \\
\hspace*{2em}\textit{Choć kadra stoi po ich stronie}  & \\
\hspace*{2em}\textit{Bierzemy boski napój w dłonie}  & \\
\hspace*{2em}\textit{Bełt!, Bełt, Bełt!}  & \\
& \\
\end{longtable}