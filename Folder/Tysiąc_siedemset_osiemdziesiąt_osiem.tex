\section{1788}
\vspace{-\baselineskip}
\textit{Jacek Kaczmarski}\\
\begin{longtable}{ll}
Ta pierwsza morska podróż do Australii! & \textbf{F B} \\
Łotry przy burtach, prostytutki w kojach - & \textbf{F C} \\
Wszyscy się bali, łkali i rzygali & \textbf{F B} \\
W drodze do raju. Przewrotności Twoja & \textbf{F C} \\
Panie, coś w jeszcze nam nieznanych planach & \textbf{d g} \\
Miał czarne diabły strzegące wybrzeży & \textbf{d a} \\
Edenu, który przeznaczyłeś dla nas, & \textbf{B C F} \\
A w który nikt, prawdę mówiąc, nie wierzył! & \textbf{B C d} \\
& \\
Czym żeśmy, marni, zasłużyli na to?  & \\
Ten, co zawisnąć miał za kradzież płaszcza -  & \\
Płakał nad swoją niechybną zatratą;  & \\
Nie widział Ciebie w robaczywych masztach  & \\
Statku, co tylko był więzieniem nowym;  & \\
Tej co kupczyła ciałami swych dziatek -  & \\
Ani przez mgnienie nie przyszło do głowy,  & \\
Że to nadziei - nie rozpaczy statek.  & \\
& \\
Niejeden żołnierz z ponurej eskorty  & \\
(Bo czym się ich los od naszego różnił?)  & \\
Wiedział, że nigdy już nie ujrzy portu,  & \\
Gdzie go podejmą karczmarze usłużni  & \\
I płatne dziewki; że zabraknie rumu  & \\
Zanim do celu przygnasz okręt szparki.  & \\
Z marynarzami pili więc na umór  & \\
I - wbrew zakazom - grali o więźniarki.  & \\
& \\
Prawda, nie wszyscy próby Twe przetrwali,  & \\
Ale też ciężkoś nas doświadczał, Panie:  & \\
Nie oszczędzałeś nam wysokiej fali,  & \\
Za którą mnogim przyszło w oceanie  & \\
Zakończyć żywot; innym dziąsła zgniły,  & \\
Wypadły zęby, rozgorzały wrzody...  & \\
Więc znaczą nasz zielony szlak mogiły  & \\
Szkorbutu, szału, francuskiej choroby.  & \\
& \\
& \\
\end{longtable}
\newpage
\begin{longtable}{ll}
Nikt nie odnajdzie w ruchomych otchłaniach  & \\
Ciał nieszczęśników - oprócz Ciebie, Boże.  & \\
Ich żywot grzeszny epitafiów wzbrania,  & \\
Lecz - ukarani. Więc wystarczy może,  & \\
Żeś się posłużył straszliwym przykładem:  & \\
Oni naprawdę dotarli do piekieł,  & \\
A umierając nie wierzył z nich żaden,  & \\
Że w swym cierpieniu umiera – człowiekiem  & \\
& \\
Ląd nam się wydał niegościnny, dziki;  & \\
Łotr bez honoru, kobieta sprzedajna  & \\
Z dnia na dzień - jak się ma stać osadnikiem  & \\
Nieznanych światów? Bo rozpoznać Raj nam  & \\
Nie było łatwo; znaleźć w sobie siłę,  & \\
Wbrew przeciwnościom, bez słowa zachęty  & \\
By mimo wszystko żyć - nim nam odkryłeś  & \\
Kraj szczodry w zboże, złoto i diamenty.  & \\
& \\
Łajdacki pomiot, łotrowskie nasienie  & \\
Czerpiąc ze spichrza Twoich dóbr wszelakich -  & \\
Choć tyle wiemy własnym doświadczeniem:  & \\
W nas jest Raj, Piekło -  & \\
I do obu - szlaki.  & \\
\end{longtable}