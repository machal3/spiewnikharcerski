\section{Poezja}
\vspace{-\baselineskip}
\textit{Na Bani}\\
\begin{longtable}{ll}
Ty przychodzisz, jak noc majowa, & \textbf{a e} \\
biała noc, uśpiona w jaśminie, & \textbf{F G} \\
i jaśminem pachną twoje słowa, & \textbf{a e} \\
i księżycem sen srebrny płynie. & \textbf{F G} \\
 & \\
Płyniesz cicha przez noce bezsenne & \textbf{a e} \\
– cichą nocą tak liście szeleszczą – & \textbf{F G} \\
szepcesz sny, szepcesz słowa tajemne, & \textbf{a e} \\
w słowach cichych skąpana, jak w deszczu… & \textbf{F G} \\
 & \\
To za mało! Za mało! Za mało! & \\
Twoje słowa tumanią i kłamią! & \\
Piersiom żywych daj oddech zapału, & \\
wiew szeroki i skrzydła do ramion! & \\
 & \\
Nam te słowa ciche nie starczą. & \\
Marne słowa. I błahe. I zimne. & \\
Ty masz werbel nam zagrać do marszu! & \\
Smagać słowem! Bić pieśnią! Wznieść hymnem! & \\
 & \\
Jest gdzieś radość ludzka, zwyczajna, & \\
jest gdzieś jasne i piękne życie. – & \\
Powszedniego chleba słów daj nam & \\
i stań przy nas, i rozkaż – bić się! & \\
 & \\
Niepotrzebne nam białe westalki, & \\
noc nie zdławi świętego ognia – & \\
bądź jak sztandar rozwiany wśród walki, & \\
bądź jak w wichrze wzniesiona pochodnią! & \\
 & \\
Odmień, odmień nam słowa na wargach, & \\
naucz śpiewać płomienniej i prościej, & \\
niech nas miłość ogromna potarga, & \\
więcej bólu i więcej radości! & \\
\end{longtable}
\newpage
\begin{longtable}{ll}
Jeśli w pięści potrzebna ci harfa, & \textbf{a e} \\
jeśli harfa ma zakląć pioruny, & \textbf{F G} \\
rozkaż żyły na struny wyszarpać & \textbf{a e} \\
i naciągać, i trącać, jak struny. & \textbf{F G} \\
 & \\
Trzeba pieśnią bić aż do śmierci, & \\
trzeba głuszyć w ciemnościach syk węży. & \\
Jest gdzieś życie piękniejsze od wierszy. & \\
I jest miłość. I ona zwycięży. & \\
 & \\
Wtenczas daj nam, poezjo, najprostsze & \\
ze słów prostych i z cichych – najcichsze, & \\
a umarłych w wieczności rozpostrzyj, & \\
jak chorągwie podarte na wichrze. & \\
\end{longtable}