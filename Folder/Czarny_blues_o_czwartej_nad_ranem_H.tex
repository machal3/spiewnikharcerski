\section{\textbf{Czarny blues o czwartej nad ranem}}
\vspace{-\baselineskip}
\textit{Stare Dobre Małżeństwo}\\
\begin{longtable}{ll}
Czwarta nad ranem - może sen przyjdzie & \textbf{A cis} \\
Może mnie odwiedzisz & \textbf{D A} \\
Czwarta nad ranem - może sen przyjdzie & \textbf{E fis} \\
Może mnie odwiedzisz & \textbf{D E A} \\
& \\
Czemu cię nie ma na odległość ręki? & \textbf{A E} \\
Czemu mówimy do siebie listami? & \textbf{fis cis} \\
Gdy ci to śpiewam, u mnie pełnia lata & \textbf{D A} \\
Gdy to usłyszysz, będzie środek zimy & \textbf{D E} \\
& \\
Czemu się budzę o czwartej nad ranem & \textbf{A E} \\
I włosy twoje próbuję ugłaskać & \textbf{fis cis} \\
Lecz nigdzie nie ma twoich włosów & \textbf{D A} \\
Jest tylko blada nocna lampka & \textbf{D E} \\
Łysa śpiewaczka & \textbf{fis} \\
& \\
Śpiewamy bluesa, bo czwarta nad ranem & \textbf{A E} \\
Tak cicho, żeby nie zbudzić sąsiadów & \textbf{fis cis} \\
Czajnik z gwizdkiem świruje na gazie & \textbf{D A} \\
Myślałby kto, że rodem z Manhattanu & \textbf{D E} \\
& \\
Czwarta nad ranem - może sen przyjdzie & \textbf{A cis} \\
Może mnie odwiedzisz & \textbf{D A} \\
Czwarta nad ranem - może sen przyjdzie & \textbf{E fis} \\
Może mnie odwiedzisz & \textbf{D E A} \\
& \\
Herbata czarna myśli rozjaśnia & \textbf{A E} \\
A list twój sam się czyta & \textbf{fis cis} \\
\smash{Że} można go śpiewać - za oknem mruczą bluesa & \textbf{D A} \\
Topole z Krupniczej & \textbf{D E} \\
& \\
I jeszcze strażak wszedł na solo & \textbf{A E} \\
Ten z Mariackiej Wieży & \textbf{fis cis} \\
Jego trąbka jak księżyc  błyszczy nad topolą & \textbf{D A} \\
Nigdzie się jej nie spieszy & \textbf{D E} \\
& \\
Już piąta - może sen przyjdzie  & \\
Może mnie odwiedzisz  & \\
Już piąta - może sen przyjdzie  & \\
Może mnie odwiedzisz  & \\
\end{longtable}