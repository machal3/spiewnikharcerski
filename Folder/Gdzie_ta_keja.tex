\section{\textbf{Gdzie ta keja}}
\begin{longtable}{ll}
Gdyby tak ktoś przyszedł i powiedział: & \textbf{a} \\
– Stary, czy masz czas? & \textbf{G a} \\
Potrzebuję do załogi jakąś nową twarz, & \textbf{C G G7 C} \\
Amazonka, Wielka Rafa, oceany trzy, & \textbf{C C7 F d} \\
Rejs na całość, rok, dwa lata – to powiedziałbym: & \textbf{a E E7 a} \\
& \\
\hspace*{2em}\textit{Gdzie ta keja, a przy niej ten jacht?} & \textbf{a E7 a} \\
\hspace*{2em}\textit{Gdzie ta koja wymarzona w snach?} & \textbf{C G C} \\
\hspace*{2em}\textit{Gdzie te wszystkie sznurki od tych szmat?} & \textbf{G A7 d A7 d} \\
\hspace*{2em}\textit{Gdzie ta brama na szeroki świat?} & \textbf{a E7 a} \\
& \\
Gdzie ta keja, a przy niej ten jacht? & \textbf{a E7 a} \\
Gdzie ta koja wymarzona w snach? & \textbf{C G C} \\
W każdej chwili płynę w taki rejs, & \textbf{G A7 d A7 d} \\
Tylko gdzie to jest? No gdzie to jest? & \textbf{a E7 a} \\
& \\
Gdzieś na dnie wielkiej szafy, leży ostry nóż, &  \\
Stare dżinsy wystrzępione impregnuje kurz, &  \\
W kompasie igła zardzewiała, lecz kierunek znam,  & \\
Biorę wór na plecy i przed siebie gnam.  & \\
& \\
\hspace*{2em}\textit{Gdzie ta keja, a przy niej ten jacht...}  & \\
& \\
Przeszły lata zapyziałe, rzęsą porósł staw,  & \\
Na przystani czółno stało – kolorowy paw.  & \\
Zaokrągliły się marzenia, wyjałowiał step,  & \\
Lecz wciąż marzy o załodze ten samotny łeb.  & \\
& \\
\hspace*{2em}\textit{Gdzie ta keja, a przy niej ten jacht...}  & \\
\end{longtable}