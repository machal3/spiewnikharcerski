\section{Na co komu dziś}
\begin{longtable}{ll}
Stała pod ścianą sącząc kakao & \textbf{F a F G} \\
Kapela cięła walca na sześć & \textbf{F a C G} \\
Spytałem skromnie: “czy pójdziesz do mnie?” & \textbf{F a F G} \\
Kiwnęła głową zgadzając się & \textbf{F a F G} \\
& \\
\hspace*{2em}\textit{Trzeba zawsze żyć biegnącą chwilą} & \textbf{a G C F} \\
\hspace*{2em}\textit{Na co komu dziś wczorajszy dzień} & \textbf{a G C F} \\
& \\
Topiłem smutki w butelce wódki & \textbf{F a F G} \\
Obok Japończyk do lustra pił & \textbf{F a F G} \\
Pytam żółtego: "powiedz dlaczego & \textbf{F a F G} \\
Też jesteś smutny?" On na to mi & \textbf{F a F G} \\
& \\
\hspace*{2em}\textit{Na co komu dziś wczorajsza miłość} & \textbf{a G C F} \\
\hspace*{2em}\textit{Na co komu dziś wczorajszy sen} & \textbf{a G C F} \\
\hspace*{2em}\textit{Po co dalej pić to samo piwo} & \textbf{a G C F} \\
\hspace*{2em}\textit{Kiedy czujesz, że uleciał gaz} & \textbf{a G C F} \\
& \\
Chciałem być sobą za wielką wodą & \textbf{F a F G} \\
Na czekoladę poczułem chęć & \textbf{F a F G} \\
Była namiętna, bardzo nieletnia & \textbf{F a F G} \\
I dobrze znała refrenu sens & \textbf{F a F G} \\
& \\
\hspace*{2em}\textit{Na co komu dziś wczorajsza miłość…}  & \\
& \\
Spotkałem narzeczoną & \textbf{F G a F G} \\
Taką ze szkolnych lat & \textbf{F G a F G} \\
Próbowaliśmy mocno & \textbf{F G a F G} \\
By taniec naszych ciał & \textbf{F G a F G} \\
Rozgrzała jakaś iskra & \textbf{F G a F G} \\
& \\
\hspace*{2em}\textit{Na co komu dziś wczorajsza miłość...}  & \\
\end{longtable}