\section{\textbf{Bieszczadzkie anioły}}
\begin{longtable}{ll}
Anioły są takie ciche, zwłaszcza te w Bieszczadach & \textbf{a G} \\
Gdy spotkasz takiego w górach, wiele z nim nie pogadasz & \textbf{a e} \\
Najwyżej na ucho ci powie, gdy będzie w dobrym humorze & \textbf{C G C F} \\
Że skrzydła nosi w plecaku, nawet przy dobrej pogodzie & \textbf{C G a e a} \\
& \\
Anioły są całe zielone, zwłaszcza te w Bieszczadach & \textbf{a G} \\
Łatwo w trawie się kryją, i w opuszczonych sadach & \textbf{a e} \\
W zielone grają ukradkiem, nawet karty mają zielone & \textbf{C G C F} \\
Zielone mają pojęcie, a nawet zielony kielonek & \textbf{C G a e a} \\
& \\
\hspace*{2em}\textit{Anioły bieszczadzkie, bieszczadzkie anioły} & \textbf{C G a} \\
\hspace*{2em}\textit{Dużo w was radości i dobrej pogody} & \textbf{C G a} \\
\hspace*{2em}\textit{Bieszczadzkie anioły, anioły bieszczadzkie} & \textbf{C G a} \\
\hspace*{2em}\textit{Gdy skrzydłem cię dotkną już jesteś ich bratem} & \textbf{C G a} \\
& \\
Anioły są całkiem samotne, zwłaszcza te w Bieszczadach & \textbf{a G} \\
W kapliczkach zimą drzemią, choć może im nie wypada & \textbf{a e} \\
Czasem taki anioł samotny, zapomni dokąd ma lecieć & \textbf{C G C F} \\
I wtedy całe Bieszczady, mają szaloną uciechę & \textbf{C G a e a} \\
& \\
\hspace*{2em}\textit{Anioły bieszczadzkie, bieszczadzkie anioły...}  & \\
& \\
Anioły są wiecznie ulotne, zwłaszcza te w Bieszczadach & \textbf{a G} \\
Nas też czasami nosi, po ich anielskich śladach & \textbf{a e} \\
One nam przyzwalają i skrzydłem wskazują drogę & \textbf{C G C F} \\
I wtedy w nas się zapala, wieczny bieszczadzki ogień & \textbf{C G a e a} \\
& \\
\hspace*{2em}\textit{Anioły bieszczadzkie, bieszczadzkie anioły...}  & \\
\end{longtable}