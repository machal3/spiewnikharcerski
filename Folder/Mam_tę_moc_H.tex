\section{Mam tę moc}
\begin{longtable}{ll}
Na zboczach gór biały śnieg nocą lśni & \textbf{e C} \\
I nietknięty stopą trwa & \textbf{D A4 a} \\
Królestwo samotnej duszy & \textbf{e C} \\
A królową jestem ja & \textbf{D A4 a} \\
Posępny wiatr na strunach burzy w sercu gra & \textbf{e C D a} \\
Choć opieram się to się na nic zda & \textbf{e D A} \\
& \\
Niech nie wie nikt, nie zdradzaj nic & \textbf{D C} \\
Żadnych uczuć, od teraz tak masz żyć & \textbf{C D} \\
Bez słów, bez snów, łzom nie dać się & \textbf{D C} \\
Lecz świat już wie! & \textbf{C} \\
& \\
Mam tę moc! Mam tę moc! & \textbf{G D} \\
Rozpalę to co się tli & \textbf{e C} \\
Mam tę moc! Mam tę moc! & \textbf{G D} \\
Wyjdę i zatrzasnę drzwi! & \textbf{e C2} \\
Wszystkim wbrew, na ten gest mnie stać! & \textbf{G D e7 C2} \\
Co tam burzy gniew? & \textbf{h7 B} \\
Od lat coś w objęcia chłodu mnie pcha & \textbf{e C} \\
& \\
Z oddali to co wielkie, swój ogrom traci w mig & \textbf{e C D a} \\
Dawny strach co ściskał gardło, na zawsze wreszcie znikł & \textbf{e D A} \\
Zobaczę dziś czy sił mam dość & \textbf{D C} \\
By wejść na szczyt, odmienić los & \textbf{C D} \\
I wyjść zza krat, jak wolny ptak & \textbf{D C} \\
O tak! & \textbf{C} \\
& \\
Mam tę moc! Mam tę moc! & \textbf{G D} \\
Mój jest wiatr, okiełznam śnieg & \textbf{e C} \\
Mam tę moc! Mam tę moc! & \textbf{G D} \\
I zamiast łez jest śmiech! & \textbf{e C} \\
Wreszcie ja, & \textbf{G D} \\
zostawię ślad! & \textbf{e C} \\
Co tam burzy gniew? & \textbf{G D} \\
& \\
& \\
\end{longtable}
\newpage
\begin{longtable}{ll}
Moc mojej władzy lodem spada dziś na świat & \textbf{D C} \\
A duszę moją w mroźnych skrach ku górze niesie wiatr & \textbf{D C} \\
I myśl powietrze tnie jak kryształowy miecz & \textbf{D C} \\
Nie zrobię kroku w tył… & \textbf{D C} \\
Nie spojrzę nigdy wstecz & \textbf{D C} \\
& \\
\hspace*{2em}\textit{Mam tę moc! Mam tę moc!} & \textbf{G D} \\
\hspace*{2em}\textit{Z nową zorzą zbudzę się} & \textbf{e C} \\
\hspace*{2em}\textit{Mam tę moc! Mam tę moc!} & \textbf{G D} \\
\hspace*{2em}\textit{Już nie ma tamtej mnie} & \textbf{e C} \\
\hspace*{2em}\textit{Oto ja!} & \textbf{G D} \\
\hspace*{2em}\textit{Stanę w słońcu dnia} & \textbf{e C} \\
\hspace*{2em}\textit{Co tam burzy gniew?} & \textbf{G D} \\
\hspace*{2em}\textit{Od lat coś w objęcia chłodu mnie pcha} & \textbf{e C} \\
& \\
\end{longtable}