\section{Ballada o jednej (pani) Wisniewskiej}
\begin{longtable}{ll}
Żyli w pałacu hrabia z hrabiną, & \textbf{a} \\
On zwał się Rodryg, ona Francesca, & \textbf{d a} \\
A w drugim domku za ich meliną & \textbf{d a} \\
Mieszkała sobie jedna Wiśniewska. & \textbf{a E a} \\
& \\
Niewinne serce miała hrabina  & \\
I takąż duszę pieską, niebieską,  & \\
A on był gałgan i straszna świnia,  & \\
Bo pitigrilił się z tą Wiśniewską.  & \\
& \\
Biedna hrabina łzami płakała,  & \\
Z ciągłej żałoby wyschła na deskę  & \\
I na kolanach męża błagała:  & \\
Odczep się, draniu, od tej Wiśniewskiej.  & \\
& \\
Próżno chodziła z hrabią na udry,  & \\
Na próżno klęła swą dolę pieską,  & \\
On ciągle ganiał do tej łachudry  & \\
I szeptał czule: "O, ty Wiśniewsko!"  & \\
& \\
Aż raz hrabina miecz zdjęła z ściany,  & \\
Zmierzyła hrabię okiem królewskim.  & \\
Siedź tu, powiada, ty - w herb drapany,  & \\
Dzisiaj nie pójdziesz do tej Wiśniewskiej.  & \\
& \\
On zaś będący pod alkoholem,  & \\
Czyli, jak mówią - zalany w pestkę,  & \\
Wyrżnął hrabinę łbem w antresolę  & \\
I dawaj, gazu! Do tej Wiśniewskiej,  & \\
& \\
Biedna hrabina padła na dywan,  & \\
Cała zalała się krwią niebieską,  & \\
A gdy poczuła, że dogorywa,  & \\
Rzekła: poczekaj, o ty, Wiśniewsko.  & \\
& \\
& \\
\end{longtable}
\newpage
\begin{longtable}{ll}
Potem się odbył pogrzeb wspaniały, & \textbf{a} \\
Hrabia nad grobem uronił łezkę, & \textbf{d a} \\
Strasznie się martwił przez dzionek cały, & \textbf{d a} \\
A na noc poszedł ... do tej Wiśniewskiej. & \textbf{a E a} \\
& \\
Wtedy hrabina z mogiły wstała,  & \\
Wyrwała z trumny sękatą deskę,  & \\
Poszła za hrabią, na śmierć go sprała  & \\
I rozwaliła łeb tej Wiśniewskiej.  & \\
& \\
Chociaż lebiegi grzeszyli tyle  & \\
I na nich w końcu też przyszła kreska.  & \\
Dziś sobie leżą w jednej mogile:  & \\
Hrabia, hrabina i... ta Wiśniewska.  & \\
\end{longtable}