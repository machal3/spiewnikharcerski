\section{\textbf{Obława}}
\begin{longtable}{ll}
Skulony w jakiejś ciemnej jamie smaczniem sobie spał & \textbf{a C G C} \\
I spały wilczki dwa, zupełnie ślepe jeszcze & \textbf{F E} \\
Wtem stary wilk przewodnik, co życie dobrze znał & \textbf{a C G C} \\
Łeb podniósł, warknął groźnie, aż mną szarpnęły dreszcze & \textbf{F E} \\
Poczułem wokół siebie nienawistną woń & \textbf{a F E a} \\
Woń, która burzy wszelki spokój, zrywa wszystkie sny & \textbf{F E} \\
Z daleka ktoś, gdzieś krzyknął krótki rozkaz: goń! & \textbf{a F E a} \\
I z czterech stron wypadły na nas cztery gończe psy! & \textbf{F E} \\
& \\
\hspace*{2em}\textit{Obława, obława na młode wilki obława} & \textbf{a C G C} \\
\hspace*{2em}\textit{Te dzikie zapalczywe, w gęstym lesie wychowane} & \textbf{F E} \\
\hspace*{2em}\textit{Krąg śniegu wydeptany, w tym kręgu plama krwawa} & \textbf{a C G C} \\
\hspace*{2em}\textit{I ciała wilcze kłami gończych psów szarpane!} & \textbf{F E a} \\
& \\
Ten, który na mnie rzucił się, niewiele szczęścia miał & \textbf{a C G C} \\
Bo wpadł prosto mi na kły i krew trysnęła z rany & \textbf{F E} \\
Gdym teraz ile w łapach sił przed siebie prosto gnał & \textbf{a C G C} \\
Ujrzałem młode wilczki na strzępy rozszarpane & \textbf{F E} \\
Zginęły ślepe ufne tak, puszyste kłębki dwa & \textbf{a F E a} \\
Bezradne na tym świecie złym nie wiedząc, kto je zdławił & \textbf{F E} \\
I zginie wilk-przewodnik, choć życie dobrze zna & \textbf{a F E a} \\
Bo z trzema naraz walczy psami i z ran trzech naraz krwawi & \textbf{F E} \\
& \\
\hspace*{2em}\textit{Obława, obława na młode wilki obława} & \textbf{a C G C} \\
\hspace*{2em}\textit{Te dzikie zapalczywe, w gęstym lesie wychowane} & \textbf{F E} \\
\hspace*{2em}\textit{Krąg śniegu wydeptany, w tym kręgu plama krwawa} & \textbf{a C G C} \\
\hspace*{2em}\textit{I ciała wilcze kłami gończych psów szarpane!} & \textbf{F E a} \\
& \\
Wypadłem na otwartą przestrzeń pianą z pyska tocząc & \textbf{a C G C} \\
Lecz tutaj też ze wszystkich stron zła mnie otacza woń & \textbf{F E} \\
A myśliwemu, co mnie dojrzał, już się śmieją oczy & \textbf{a C G C} \\
I ręka pewna niezawodna podnosi w górę broń & \textbf{F E} \\
& \\
Rzucam się w bok, na oślep gnam, aż ziemia spod łap pryska & \textbf{a F E a} \\
I wtedy pada pierwszy strzał, co kark mi rozszarpuje & \textbf{F E} \\
Pędzę, słyszę jak on klnie, krew mi płynie z pyska & \textbf{a F E a} \\
On strzela po raz drugi, lecz teraz już pudłuje & \textbf{F E} \\
\end{longtable}
\newpage
\begin{longtable}{ll}
\hspace*{2em}\textit{Obława, obława na młode wilki obława} & \textbf{a C G C} \\
\hspace*{2em}\textit{Te dzikie zapalczywe, w gęstym lesie wychowane} & \textbf{F E} \\
\hspace*{2em}\textit{Krąg śniegu wydeptany, w tym kręgu plama krwawa} & \textbf{a C G C} \\
\hspace*{2em}\textit{I ciała wilcze kłami gończych psów szarpane!} & \textbf{F E a} \\
& \\
Wyrwałem się z obławy tej, schowałem w jakiś las & \textbf{a C G C} \\
Lecz ile szczęścia miałem w tym to każdy chyba przyzna & \textbf{F E} \\
Leżałem w śniegu jak nieżywy długi, długi czas & \textbf{a C G C} \\
Po strzale zaś na zawsze mi została krwawa blizna & \textbf{F E} \\
& \\
Lecz nie skończyła się obława i nie śpią gończe psy & \textbf{a F E a} \\
I giną ciągle wilki młode na całym wielkim świecie & \textbf{F E} \\
Nie dajcie z siebie zedrzeć skór, brońcie się i wy, & \textbf{a F E a} \\
O bracia wilcy! Brońcie się nim wszyscy wyginiecie & \textbf{F E} \\
& \\
\hspace*{2em}\textit{Obława, obława na młode wilki obława} & \textbf{a C G C} \\
\hspace*{2em}\textit{Te dzikie zapalczywe, w gęstym lesie wychowane} & \textbf{F E} \\
\hspace*{2em}\textit{Krąg śniegu wydeptany, w tym kręgu plama krwawa} & \textbf{a C G C} \\
\hspace*{2em}\textit{I ciała wilcze kłami gończych psów szarpane!}  & \\
\end{longtable}