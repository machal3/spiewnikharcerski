\section{Epitafium dla S. Jesienina}
\vspace{-\baselineskip}
\textit{Jacek Kaczmarski, Przemysław Gintroski, Zbigniew Łapiński}\\
\begin{longtable}{ll}
Wściekła się wielka niedźwiedzica & \textbf{d C} \\
I lecą z pyska płaty piany & \textbf{A7 d} \\
Samotny szczeniak do księżyca & \textbf{d C} \\
Zanosi jazgot obłąkany & \textbf{A7 d} \\
& \\
Po mlecznej drodze wóz się toczy & \textbf{c Bb} \\
Turkocze za nim piąte koło & \textbf{G7 c} \\
Sponad zodiaku o północy  & \\
Ze mną ptak ryba i dziwoląg  & \\
& \\
I Ruś cerkiewna Ruś dawnej wiary  & \\
Szumi pomoriem w riazańskiej duszy  & \\
Huczy po karczmach krwawym pożarem  & \\
I Twoje brzozy pali Sergiuszu  & \\
& \\
Brzozo wędrowna czemu się śnisz  & \\
I tak cię zrąbią dla mnie na krzyż  & \\
Chato rodzinna płyń tam gdzie kres  & \\
Niech się dzwoneczek śmieje do łez  & \\
& \\
Ryczy lew ranny ponad głową  & \\
Bliźnięta płyną rzeką modrą  & \\
Byk galopuje łąką płową  & \\
I pręży się na wadze skorpion  & \\
& \\
Tętni przez łąki koziorożec  & \\
Strzelec go tropi nieustannie  & \\
I płynie wodnik po jeziorze  & \\
Po utopionej płacze pannie  & \\
& \\
I Ruś jak panna niech płacze po nim  & \\
Święta Łagoda zejdzie do ziemi  & \\
Na bruku Moskwy klon oszroniony  & \\
Biblijny prorok Sergiusz Jesienin  & \\
& \\
Riazańska matko skąd w oczach łzy  & \\
Karczemne szczęście samogon dym  & \\
Moskwo karczemna płyń za mną płyń  & \\
Pokochał zodiak riazański syn  & \\
& \\
Otwórzcie mi stróże anieli  & \\
Błękitne podwoje dni  & \\
O północy anioł w bieli  & \\
Z moim wiernym koniem znikł  & \\
& \\
Koń mój Bogu niepotrzebny  & \\
Koń mój siła ma i moc  & \\
Słyszę gryzie łańcuch srebrny  & \\
Rży żałośnie w głuchą noc  & \\
& \\
Widzę pędzi wśród zamieci  & \\
Targa gniewnie gruby sznur  & \\
Jak z miesiąca z niego leci  & \\
Sierść bułana w kłęby chmur  & \\
& \\
Tutaj Jesienin w najśmieszniejszej z gier  & \\
Wybiegał w błękit zza karczemnej Moskwy  & \\
Riazańską łąką zakwitł w Angleterre  & \\
Sen pożegnalny ostatni oktostych  & \\
& \\
Pomódl się pomódl za Jesienina  & \\
Przeżegnaj wszystkie dalekie drogi  & \\
Wychyl wieczorem czareczkę wina  & \\
Ostatnie grosze rozdaj ubogim  & \\
& \\
Nie pragnął krzyku odpoczynek snił  & \\
W sekundzie się przeżywał od nowa  & \\
A potem długo waliła do drzwi  & \\
Zniecierpliwiona służba hotelowa  & \\
& \\
Podróżną sakwę zarzuć na ramię  & \\
Wyjdź na gościniec do bramy nieba  & \\
Słyszysz jak tętni przez śnieżną zamieć  & \\
Księżyc na nowiu bułany źrebak  & \\
& \\
Weszli krzyknęli a jeden się bał  & \\
Bo spoza okien milczące niebiosa  & \\
A tam na stole gdzie Jesienin stał  & \\
Snuł się powoli dymek z papierosa  & \\
& \\
Słyszysz jak woła każdego ranka  & \\
Wiatr myszkujący po połoninach  & \\
Leci w dal z wiatrem rżenie bułanka  & \\
Pomódl się pomódl za Jesienina  & \\
\end{longtable}