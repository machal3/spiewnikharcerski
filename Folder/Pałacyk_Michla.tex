\section{Pałacyk Michla}
\begin{longtable}{ll}
Pałacyk Michla, Żytnia, Wola, & \textbf{C} \\
Bronią się chłopcy od „Parasola” & \textbf{G C} \\
Choć na „tygrysy” mają visy, & \textbf{C} \\
To Warszawiaki fajne urwisy są & \textbf{G C G C} \\
& \\
\hspace*{2em}\textit{Czuwaj, wiaro, i wytężaj słuch,} & \textbf{G C} \\
\hspace*{2em}\textit{Pręż swój młody duch, pracując jak zuch!} & \textbf{G C} \\
\hspace*{2em}\textit{Czuwaj, wiaro, i wytężaj słuch,} & \textbf{G C} \\
\hspace*{2em}\textit{Pręż swój młody duch jak stal!} & \textbf{G C} \\
& \\
Każdy chłopaczek chce być ranny & \textbf{C} \\
Sanitariuszki - morowe panny, & \textbf{G C} \\
I gdy cię kula trafi jaka, & \textbf{C} \\
Poprosisz pannę - da ci buziaka, hej! & \textbf{G C G C} \\
& \\
\hspace*{2em}\textit{Czuwaj, wiaro, i wytężaj słuch...}  & \\
& \\
Z tyłu za linią dekowniki, & \textbf{C} \\
Intendentura, różne umrzyki, & \textbf{G C} \\
Gotują zupę, czarną kawę- & \textbf{C} \\
I tym sposobem walczą za sprawę, hej! & \textbf{G C G C} \\
& \\
\hspace*{2em}\textit{Czuwaj, wiaro, i wytężaj słuch...}  & \\
& \\
Za to dowództwo jest morowe, & \textbf{C} \\
Bo w pierwszej linii nadstawia głowę, & \textbf{G C} \\
A najmorowszy z przełożonych & \textbf{C} \\
To jest nasz „Miecio”, w kółko golony, hej! & \textbf{G C G C} \\
& \\
\hspace*{2em}\textit{Czuwaj, wiaro, i wytężaj słuch...}  & \\
& \\
Wiara się bije, wiara śpiewa,  & \\
szkopy się złoszczą, krew ich zalewa,  & \\
różnych sposobów się imają,  & \\
co chwila „szafę” nam posuwają, hej!  & \\
& \\
\hspace*{2em}\textit{Czuwaj, wiaro, i wytężaj słuch...}  & \\
& \\
Lecz na nic „szafa” i granaty,  & \\
za każdym razem dostają baty  & \\
i co dzień się przybliża chwila,  & \\
że zwyciężymy - i do cywila, hej!  & \\
\end{longtable}