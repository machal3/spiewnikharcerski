\section{\textbf{Ballada o zamku}}
\begin{longtable}{ll}
Stał kiedyś zamek, który twierdzą niezdobytą był & \textbf{e e G D e} \\
Potężne mury, czarna fosa, stalą kute drzwi &  \\
łuczników wielka, zbrojna moc strzegła zamku dzień i noc &  \\
By żaden nieproszony człek przez bramy nie-nie mógł przejść &  \\
& \\
A żył tam pewien stary mag co włosy białe miał jak śnieg &  \\
I pewien bardzo młody bard co z myśli splatał pieśń &  \\
I chociaż czas rozdzielił ich dekada długich lat &  \\
Z jednego dzbana pili wciąż: młody bard i stary mag &  \\
& \\
Najechał kiedyś zamek ten pewien bardzo możny pan &  \\
Najemnych setki zebrał dwie by wznieść podwoje bram &  \\
I rozgorzał wielski boj, z ran toczyła się krew &  \\
Na wieży biało-włosy mag słał swój magi-magiczny zew &  \\
& \\
I spłynął z rozwścieczonych chmur pan huraganów – wiatrów król &  \\
I na lawinie ludzkich ciał błyskawic swoje armie słał &  \\
I wygrał bitwę zamku pan, ostały się podwoje bram &  \\
A ten kto żyw ku wieży biegł, aby magowi pokłon nieść &  \\
& \\
Lecz przerwał tupot kroków ich czerwonej strzały wściekły świst &  \\
Mag zachwiał się a potem zbladł, szepcząc zaklęcie z wieży spadł  & \\
Nim rozbił się o kamieni brzeg zaklęcia błysk rozjaśnił dzień &  \\
Wieczornym niebem przemknął ptak ciągnąc za sobą nocy cień  & \\
& \\
Zostało po nim kilka ksiąg, niedopitego miodu dzban &  \\
Nikt nigdy już nie widział go i młody bard pozostał sam &  \\
Gdy nucił smutnej pieśni ton, gdy skuwał mróz gałęzie drzew  & \\
To blady świt rozjaśniał mrok, białego ptaka niosąc śpiew.  & \\
& \\
& \\
\end{longtable}