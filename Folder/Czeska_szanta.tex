\section{Czeska szanta}
\begin{longtable}{ll}
Śmiali się ze mnie sąsiedzi i śmiała się ze mnie matka & \textbf{d C d F G A} \\
Że chcę być marynarzem, co pływa na czeskich statkach & \textbf{B F E d} \\
Lecz uczy tego historia, śpiewają o tym rybitwy & \textbf{d C d F G A} \\
Że czeska marynarka nie przegrała żadnej bitwy! & \textbf{B F E d} \\
& \\
\hspace*{2em}\textit{Niech czeski naród powstanie, chłopaki liny w dłoń!} & \textbf{A d A} \\
\hspace*{2em}\textit{Jesteśmy morskim krajem, mówimy do siebie ahoj! || x2} & \textbf{B F F E d} \\
& \\
Znalazłem czeską załogę, brakuje czeskiego portu  & \\
Nie chciałem by moi ludzie byli gorszego sortu  & \\
Jest ze mną Karel i Jożin, myśleliśmy cały rok  & \\
Aż wpadliśmy na pomysł – nazywa się "suchy dok"  & \\
& \\
\hspace*{2em}\textit{Niech czeski naród powstanie, chłopaki liny w dłoń...  || x2}  & \\
& \\
Oprócz czeskiego portu przyda się czeskie morze  & \\
Ale tu już sam Pan Bóg nam raczej nie pomoże  & \\
Ale nam to nie przeszkadza i zaczyna się przygoda  & \\
Za statek będzie nam służyć napędzana żaglem skoda  & \\
& \\
\hspace*{2em}\textit{Niech czeski naród powstanie, chłopaki liny w dłoń...  || x2}  & \\
& \\
Tak z wiatrem do Bałtyku, pędzimy ile sił  & \\
Chciałem być marynarzem, stworzyłem czeski film  & \\
I biorąc za przykład Putina, wciągamy banderę w przestworza  & \\
Tak czeska republika zyskała dostęp do morza!  & \\
& \\
\hspace*{2em}\textit{Niech czeski naród powstanie, chłopaki liny w dłoń...  || x2}  & \\
\end{longtable}