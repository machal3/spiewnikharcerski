\section{\textbf{Ballada o krzyżowcu}}
\begin{longtable}{ll}
Wolniej, wolniej, wstrzymaj konia! & \textbf{e} \\
Dokąd pędzisz w stal odziany? & \textbf{A} \\
Pewnie tam, gdzie w słońcu błyszczą & \textbf{C} \\
Jeruzalem białe ściany. & \textbf{D} \\
& \\
Pewnie myślisz, że w świątyni & \textbf{e} \\
Zniewolony pan twój czeka, & \textbf{A} \\
Abyś przybył go ocalić, & \textbf{C} \\
Abyś przybył doń z daleka. & \textbf{D} \\
& \\
\hspace*{2em}\textit{(Na na na naj…)} & \textbf{e A C D} \\
& \\
Wolniej, wolniej, wstrzymaj konia! & \textbf{e} \\
Byłem wczoraj w Jeruzalem, & \textbf{A} \\
przemierzałem puste sale - & \textbf{C} \\
Pana twego nie widziałem. & \textbf{D} \\
& \\
Pan opuścił święte miasto & \textbf{e} \\
Przed minutą, przed godziną, & \textbf{A} \\
W chłodnym gaju na pustyni & \textbf{C} \\
Z Mahometem pije wino. & \textbf{D} \\
& \\
\hspace*{2em}\textit{(Na na na naj…)}  & \\
& \\
Wolniej, wolniej, wstrzymaj konia! & \textbf{e} \\
Chcesz oblegać Jeruzalem? & \textbf{A} \\
Strzegą go wysokie wieże, & \textbf{C} \\
Strzegą go mahometanie. & \textbf{D} \\
& \\
Pan opuścił Święte Miasto, & \textbf{e} \\
Na nic poświęcenie twoje, & \textbf{A} \\
Po co niszczyć białe mury, & \textbf{C} \\
Po co ludzi niepokoić. & \textbf{D} \\
& \\
\hspace*{2em}\textit{(Na na na naj…)}  & \\
& \\
Wolniej, wolniej, wstrzymaj konia & \textbf{e} \\
Porzuć walkę niepotrzebną, & \textbf{A} \\
Porzuć miecz i włócznię swoją & \textbf{C} \\
I jedź ze mną, i jedź ze mną. & \textbf{D} \\
& \\
Bo, gdy szlakiem ku północy & \textbf{e} \\
Podążają hufce ludne, & \textbf{A} \\
Ja unoszę dumnie głowę & \textbf{C} \\
I odjeżdżam na południe.  & \\
\end{longtable}