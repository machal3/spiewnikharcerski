\section{Epitafium dla Wysockiego}
\vspace{-\baselineskip}
\textit{Jacek Kaczmarski}\\
\begin{longtable}{ll}
To moja droga z piekła do piekła  & \\
W dół na złamanie karku gnam!  & \\
Nikt mnie nie trzyma, nikt nie prześwietla  & \\
Nie zrywa mostów, nie stawia bram!  & \\
& \\
Po grani! Po grani!  & \\
Nad przepaścią bez łańcuchów, bez wahania!  & \\
Tu na trzeźwo diabli wezmą  & \\
Zdradzi mnie rozsądek - drań  & \\
W wilczy dół wspomnienia zmienią  & \\
Ostrą grań!  & \\
& \\
Po grani! Po grani! Po grani!  & \\
Tu mi drogi nie zastąpią pokonani!  & \\
Tylko łapią mnie za nogi  & \\
Krzyczą - nie idź! Krzyczą - stań!  & \\
Ci, co w pół stanęli drogi  & \\
I zębami, pazurami kruszą grań!  & \\
& \\
To moja droga z piekła do piekła  & \\
W przepaść na łeb na szyję skok!  & \\
„Boskiej Komedii” nowy przekład  & \\
I w pierwszy krąg piekła mój pierwszy krok!  & \\
& \\
Tu do mnie! Tu do mnie!  & \\
Ruda chwyta mnie dziewczyna swymi dłońmi  & \\
I do końskiej grzywy wiąże  & \\
Szarpię grzywę - rumak rży!  & \\
Ona - co ci jest mój książę? -  & \\
Szepce mi...  & \\
& \\
Do piekła! Do piekła! Do piekła!  & \\
Nie mam czasu na przejażdżki wiedźmo wściekła!  & \\
- Nie wiesz ty co cię tam czeka -  & \\
Mówi sine tocząc łzy  & \\
- Piekło też jest dla człowieka!  & \\
Nie strasz, nie kuś i odchodząc zabierz sny!  & \\
& \\
To moja droga z piekła do piekła  & \\
Wokół postaci bladych tłok  & \\
Koń mnie nad nimi unosi z lekka  & \\
I w drugi krąg kieruje krok!  & \\
\end{longtable}
\newpage
\begin{longtable}{ll}
Zesłani! Zesłani!  & \\
Naznaczeni, potępieni i sprzedani!  & \\
Co robicie w piekła sztolniach  & \\
Brodząc w błocie, depcząc lód!  & \\
Czy śmierć daje ludzi wolnych  & \\
Znów pod knut!?  & \\
- To nie tak! To nie tak! To nie tak!  & \\
Nie użalaj się nad nami - tyś poeta!  & \\
Myśmy raju znieść nie mogli  & \\
Tu nasz żywioł, tu nasz dom!  & \\
Tu nie wejdą ludzie podli  & \\
Tutaj żaden nas nie zdziesiątkuje grom!  & \\
& \\
- Pani bagien, mokradeł i śnieżnych pól  & \\
Rozpal w łaźni kamienie na biel!  & \\
Z ciał rozgrzanych niech się wytopi ból  & \\
Tatuaże weźmiemy na cel!  & \\
Bo na sercu, po lewej, tam Stalin drży  & \\
Pot zalewa mu oczy i wąs!  & \\
Jego profil specjalnie tam kłuli my  & \\
Żeby słyszał jak serca się rwą!  & \\
& \\
To moja droga z piekła do piekła  & \\
Lampy naftowe wabią wzrok  & \\
Podmiejska chata, mała izdebka  & \\
I w trzeci krąg kieruję krok:  & \\
& \\
- Wchodź śmiało! Wchodź śmiało!  & \\
Nie wiem jak ci trafić tutaj się udało!  & \\
Ot jak raz samowar kipi, pij herbatę  & \\
Synu, pij!  & \\
Samogonu z nami wypij!  & \\
Zdrowy żyj!  & \\
& \\
Nam znośnie! Nam znośnie!  & \\
Tak żyjemy niewidocznie i bezgłośnie!  & \\
Pożyjemy i pomrzemy  & \\
Nie usłyszy o nas świat  & \\
A po śmierci wypijemy  & \\
Za przeżytych w dobrej wierze parę lat!  & \\
& \\
To moja droga z piekła do piekła  & \\
Miasto a w Mieście przy bloku blok  & \\
Wciągam powietrze i chwiejny z lekka  & \\
Już w czwarty krąg kieruję krok!  & \\
& \\
Do cyrku! Do cyrku! Do kina!  & \\
Telewizor włączyć - bajka się zaczyna!  & \\
Mama w sklepie, tata w barze  & \\
Syn z pepeszy tnie aż gra!  & \\
Na pionierskiej chuście marzeń  & \\
Gwiazdę ma!  & \\
& \\
Na mecze! Na mecze! Na wiece!  & \\
Swoje znać, nie rzucać w oczy się bezpiece!  & \\
Sąsiad - owszem, wypić można  & \\
Lecz to sąsiad, brat - to brat  & \\
Jak świat światem do ostrożnych  & \\
Zwykł należeć i uśmiechać się ten świat!  & \\
& \\
To moja droga z piekła do piekła  & \\
Na scenie Hamlet, skłuty bok  & \\
Z którego właśnie krew wyciekła -  & \\
To w piąty krąg kolejny krok!  & \\
& \\
O Matko! O Matko!  & \\
Jakże mogłaś jemu sprzedać się tak łatwo!  & \\
Wszak on męża twego zabił  & \\
Zgładzi mnie, splugawi tron  & \\
Zniszczy Danię, lud ograbi  & \\
Bijcie w dzwon!  & \\
& \\
Na trwogę! Na trwogę! Na trwogę!  & \\
Nie wybieraj między żądzą swą a Bogiem!  & \\
Póki czas naprawić błędy  & \\
Matko, nie rób tego - stój!  & \\
Cenzor z dziewiątego rzędu:  & \\
- Nie, w tej formie to nie może wcale pójść!  & \\
& \\
To moja droga z piekła do piekła  & \\
Wódka i piwo, koniak, grog  & \\
Najlepszych z nas ostatnia Mekka  & \\
I w szósty krąg kolejny krok!  & \\
& \\
Na górze! Na górze! Na górze!  & \\
Chciałoby się żyć najpełniej i najdłużej!  & \\
O to warto się postarać!  & \\
To jest nałóg, zrozum to!  & \\
Tam się żyje jak za cara!  & \\
I ot co!  & \\
& \\
Na dole, na dole, na dole  & \\
Szklanka wódki i razowy chleb na stole!  & \\
I my wszyscy tam - i tutaj  & \\
Tłum rozdartych dusz na pół  & \\
Po huśtawce mdłość i smutek  & \\
Choćbyś nawet co dzień walił głową w stół!  & \\
& \\
To moja droga z piekła do piekła  & \\
Z wolna zapada nade mną mrok  & \\
Więc biesów szpaler szlak mi oświetla  & \\
Bo w siódmy krąg kieruję krok!  & \\
Tam milczą i siedzą  & \\
I na moją twarz nie spojrzą - wszystko wiedzą  & \\
Siedzą, ale nie gadają  & \\
Mętny wzrok spod powiek lśni  & \\
Żują coś, bo im wypadły  & \\
Dawno kły!  & \\
& \\
Więc stoję! Więc stoję! Więc stoję!  & \\
A przed nimi leży w teczce życie moje!  & \\
Nie czytają, nie pytają -  & \\
Milczą, siedzą - kaszle ktoś  & \\
A za oknem werble grają -  & \\
Znów parada, święto albo jeszcze coś...  & \\
& \\
I pojąłem co chcą ze mną zrobić tu  & \\
I za gardło porywa mnie strach!  & \\
Koń mój zniknął a wy siedmiu kręgów tłum  & \\
Macie w uszach i w oczach piach!  & \\
Po mnie nikt nie wyciągnie okrutnych rąk  & \\
Mnie nie będą katować i strzyc!  & \\
Dla mnie mają tu jeszcze ósmy krąg!  & \\
Ósmy krąg, w którym nie ma już nic  & \\
& \\
Pamiętajcie wy o mnie co sił! Co sił!  & \\
Choć przemknąłem przed wami jak cień!  & \\
Palcie w łaźni, aż kamień się zmieni w pył -  & \\
Przecież wrócę, gdy zacznie się dzień!  & \\
\end{longtable}