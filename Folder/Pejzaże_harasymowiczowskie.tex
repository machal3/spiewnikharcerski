\section{Pejzaże harasymowiczowskie}
\vspace{-\baselineskip}
\textit{Wolna Grupa Bukowina}\\
\begin{longtable}{ll}
Kiedy stałem w przedświcie, a Synaj & \textbf{G D} \\
Prawdę głosił przez trąby wiatru, & \textbf{C e} \\
Zasmreczyły się chmur igliwiem - & \textbf{G D} \\
Bure świerki o górach wsparte. & \textbf{e C D} \\
I na niebie byłem ja jeden & \textbf{G D} \\
Plotąc pieśni w warkocze bukowe & \textbf{C e} \\
I schodziłem na ziemię za kwestą & \textbf{G D} \\
Przez skrzydlącą się bramę Lackowej & \textbf{e C D} \\
& \\
\hspace*{2em}\textit{I był Beskid, i były słowa} & \textbf{G C G} \\
\hspace*{2em}\textit{Zanurzone po pępki w cerkwi baniach} & \textbf{C D} \\
\hspace*{2em}\textit{Rozłożyście złotych} & \textbf{D} \\
\hspace*{2em}\textit{Smagających się wiatrem do krwi} & \textbf{C D G} \\
& \\
Moje myśli biegały końmi  & \\
Po niebieskich mokrych połoninach  & \\
I modliłem się złożywszy dłonie  & \\
Do gór, do Madonny brunatnolicej  & \\
A gdy serce kroplami tęsknoty  & \\
Jęło spadać na góry sine  & \\
Czarodziejskim kwiatem paproci  & \\
Rozzłociła się bukowina  & \\
& \\
I był Beskid, i były słowa  & \\
Zanurzone po pępki w cerkwi baniach  & \\
Rozłożyście złotych  & \\
Smagających się wiatrem do krwi  & \\
& \\
\end{longtable}